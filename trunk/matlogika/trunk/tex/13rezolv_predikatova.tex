\section {Metóda rezolvent pre logiku 1. rádu}

\begin{definicia}[Spojenie]
%%% {{{
    Nech $C$ je klauzula, ktorá obsahuje dva alebo viac literálov 
    (a tie pozostávajú z rovnakého predikátu len s inými parametrami).
    Ak tieto literály majú najvšeobecnejší unifikátor $\sigma$, 
    tak $C\sigma$ sa nazývame spojením $C$. 
    
    Ak $C\sigma$ je jednotková klauzula, tak $C\sigma$ nazývame tiež
    jednotkovým spojením $C$.
%%% }}}
\end{definicia}

\begin{priklad}
    Uvažujme klauzulu $C$, ktorá vyzerá nasledovne: 
    $C = \{ P(x) \lor P(f(y)) \lor \neg Q(x)\}$. Zoberme literály
    $P(x)$ a $P(f(y))$. Ich najvšeobecnejší unifikátor je
    $\sigma=\{ f(y) / x \}$. Potom spojenie je 
    $C\sigma = \{P(f(y)) \lor \neg Q(x) \}$.
\end{priklad}

\begin{definicia}[binárna rezolventa]
%%% {{{
    Nech $C_1$ a $C_2$ sú dve klauzuly (budeme ich nazývať predpoklady), 
    ktoré nemajú spoločné premenné. Nech $L_1 \in C_1$ a $L_2 \in C_2$ 
    sú dva literály. Ak $L_1$ a $\neg L_2$ majú 
    najvšeobecnejší unifikátor $\sigma$, tak výraz
    \begin{equation*}
        (C_1\sigma - L_1\sigma) \union (C_2\sigma - L_2\sigma)
    \end{equation*}
    sa nazýva binárnou rezolventou.\footnote{Za pozornosť stojí fakt, že
    vo všeobecnosti $(C\sigma - L\sigma) \ne (C-L)\sigma$.} 
    Literály $L_1$ a $L_2$ môžeme vynechať a nazývame ich nadbytočné.
%%% }}}
\end{definicia}

\begin{priklad}
    Majme $C_1 = P(x) \lor Q(x)$ a $C_2 = \neg P(a) \lor R(x)$, 
    čo budú predpoklady. Na to, aby sme mohli previesť operáciu
    binárnej rezolventy, musíme najskôr premenovať premenné v druhom
    výraze, aby boli rôzne od tých v prvom. Máme teda
    $C_2' = \neg P(a) \lor R(y)$.

    Uvažujme teraz klauzuly $L_1 = P(x)$ a $L_2 = \neg P(a)$.
    Ich najvšeobecnejší unifikátor je $\sigma = \{a/x\}$.

    Binárna rezolventa $C_1$ a $C_2$ je
    \begin{equation*}
    \begin{split}
        (C_1\sigma - L_1\sigma) \union (C_2\sigma - L_2\sigma) 
        &= (\{P(a),\ Q(a))\} - \{P(a)\}) \union
            (\{\neg P(a),\ R(y)\}-\{\neg P(a)\}) \\
        &= Q(a)\lor R(y)
    \end{split}
    \end{equation*}
    Nadbytočné literály sú $P(x), \neg P(a)$.
\end{priklad}

\begin{definicia}[Rezolventa logiky 1. rádu] 
%%% {{{
    Rezolventou z predpokladov $C_1$ a $C_2$ definujeme ako jednu z
    nasledujúcich binárnych rezolvent:
    \begin{enumerate}
        \item Binárna rezolventa $C_1$ a $C_2$
        \item Binárna rezolventa $C_1$ a spojenia $C_2$
        \item Binárna rezolventa spojenia $C_1$ a $C_2$
        \item Binárna rezolventa spojenia $C_1$ a spojenia $C_2$
    \end{enumerate}
%%% }}}
\end{definicia}

\begin{priklad}
    Uvažujme $C_1 = P(x) \lor P(f(y))\lor R(g(y))$,
             $C_2 = \neg P(f(g(a)) \lor Q(b)$.

    Spojenie $C_1$ vyzerá je $C_1': P(f(y)) \lor R(g(y))$.
    Binárna rezolventa $C_1'$ a $C_2$ a tým pádom rezolventa $C_1,C_2$
    bude $R(g(g(a))) \lor Q(b)$.
\end{priklad}

Ešte predtým, než sa plne vrhneme do dokazovania úplnosti metódy
rezolvent, uvedieme si príklady na osvieženie pamäti.

\begin{priklad}
    \label{prikl:rezolv_pred}
    Majem množinu klauzúl 
    $S=\{P,\ \neg P\lor Q,\ \neg P \lor \neg Q\}$.
    Prislúchajúca herbrandovská báza je $\{P, Q\}$.
    Množine klauzúl $S$ zodpovedá úplný sémantický strom $T$
    naznačený na obrázku \ref{fig:rezolv_pred_prikl}. 
    
    K úplnému stromu $T$ môžeme zostrojiť uzavretý sémantický strom v ktorom
    každá vetva sa končí odmietajúcim vrcholom, teda odmieta niektorú
    z klauzúl z $S$.
    Vieme, že žiadna z tých interpretácií, ktoré končia v listoch, 
    nie je splniteľná. 

    Následne, strom $T'$ môžeme postupne pravidlom rezolventy
    upravovať -- klauzuly $\neg P \lor \neg Q$,
    $\neg P \lor Q$ obsahujú kontrárnu dvojicu $\neg Q,Q$.
    Ich rezolventa je $\neg P$. Množinu $S$ teda môžeme obohatiť a
    dostaneme $S''=S \union \{ \neg P \}$. Môžeme tiež upraviť aj
    strom $T'$ a dostávame $T''$.

    Pokračujme ďalej: $P, \neg P$ sú kontrárna dvojica a ich
    rezolventou je $\eps$. Dostávame strom $T'''$.

    Dospeli sme teda k tomu, že množina klauzúl $S$ je nesplniteľná.
    Zároveň si môžeme všimnúť, že každým aplikovaním pravidla
    rezolventy zmenšujeme strom.
    \begin{figure}
        \centering
        %%% {{{
        \subfigure[$T$]{
            \imageontop{
                \includegraphics{img/13/priklad.1.mps}
            }
        }
        \subfigure[$T'$]{
            \imageontop{
                \includegraphics{img/13/priklad.2.mps}
            }
        }
        \vskip 1.5cm
        \subfigure[$T''$]{
            \imageontop{
                \includegraphics{img/13/priklad.3.mps}
            }
        }
        \subfigure[$T'''$]{
            \hskip 2cm
            \imageontop{
                \includegraphics{img/13/priklad.4.mps}
            }
        }
        %%% }}}
        \caption{Stromy z príkladu \ref{prikl:rezolv_pred}}
        \label{fig:rezolv_pred_prikl}
    \end{figure}
\end{priklad}

\begin{priklad}[Opakovanie z úvodu k substitúcii]
%%% {{{
    Uvažujme dve klauzuly $C_1=P(x) \lor Q(x)$, 
    $C_2 = \neg P(f(x)) \lor R(x)$.

    V nich neexistuje žiadna kontrárna dvojica. Nahradením
    $x$ za $f(a)$ resp. $a$ 
    dostaneme základné inštancie
    $C_1'=P(f(a)) \lor Q(f(a))$, 
    $C_2'=\neg P(f(a)) \lor R(a)$.
    Teraz môžeme vypočítať rezolventu $Q(f(a)) \lor R(a)$.

    Mohli by sme postupovať aj všeobecnejšie -- nahraďme $x$ za $f(x)$ v
    prvej klauzule a dostávame
    $C_1^*= P(f(x)) \lor Q(f(x))$, $C_2^*= \neg P(f(x)) \lor R(x)$.

    Rezolventa potom bude $C^*= Q(f(x)) \lor R(x)$.
    Vidíme teda, že sme získali 2 rôzne rezolventy, jednu viac všeobecnú
    ako druhú.
%%% }}}
\end{priklad}

\subsubsection{Úplnosť metódy rezolvent}
Metódu rezolvent zaviedol roku 1965 Robinson.
Táto metóda je omnoho efektívnejšia ako pravidlá, ktoré zaviedli
Davis a Putnam. Veľmi dôležitým aspektom je hlavne to, že metóda
je úplná:
Ak množina klauzúl nie je splniteľná, potom metódou rezolvent z nej vždy
môžeme dostať prázdnu klauzulu 
(a teda formula nie je splniteľná v žiadnej interpretácii). 

\begin{lema}
    Nech $C_1'$ a $C_2'$ sú inštancie klauzúl $C_1$ resp. $C_2$.
    Ak $C'$ je rezolventa $C_1'$ a $C_2'$,
    tak potom existuje rezolventa\footnote{Na prednáške to bolo
        prezentované takto. Je však evidentné, že to malo byť
        formulované ``existuje rezolventa $C$ taká, že pre ľubovoľné
        inštancie $C_1',C_2'$ je $C'$ inštancia $C$''. Inak povedané,
        $C$ bude najvšeobecnejšia rezolventa}    
    $C$ klauzúl $C_1$ a $C_2$, 
    že $C'$ je inštancia $C$. 
\end{lema}

\begin{dokaz}
%%% {{{
    Ak je treba, ako prvý krok premenujeme premenné v $C_1$ a $C_2$
    aby boli rôzne (samozrejme, rovnaké premenovanie spravíme aj v
    inštanciách $C_1',C_2'$).
    Nech teraz $L_1'$ a $L_2'$ sú literály v $C_1',C_2'$, 
    ktoré môžeme vynechať (sú nadbytočné).
    Zoberme ich najvšeobecnejší unifikátor $\nu$ a binárna rezolventa
    $C'$ bude
    \begin{equation*}
        C' = (C_1' \nu - L_1'\nu) \union ( C_2'\nu - L_2'\nu)
    \end{equation*}

    $C_1'$, $C_2'$ sú inštancie $C_1$ a $C_2$ a teda existuje
    substitúcia\footnote{Poznamenávame, že $C_1$ a $C_2$ majú rôzne
    premenné a teda túto substitúciu získame ako zloženie
    individuálnych substitúcii pre jednotlivé klauzuly} 
    $\Theta$ taká, že platí:
    \begin{align*}
        C_1' &= C_1 \Theta \\
        C_2' &= C_2 \Theta
    \end{align*}

    Označme si teraz literály z $C_i$, ktoré
    zodpovedajú po substituovaní substitúciou $\Theta$ literálu $L_i'$ ako
    $L^1_i, L^2_i, \ldots, L^{r_i}_i$ kde $i=1,2$.
    Teda platí
    \begin{equation*}
        L^1_i \Theta = L^2_i \Theta = \cdots = L^{r_i}_i\Theta=L'_i
    \end{equation*}
    Ďalej si označme najvšeobecnejší unifikátor pre 
    $\{ L^1_i, L^2_i, \dots, L^{r_i}_i\}$ ako $\lambda_i$.
    Platí
    \begin{equation*}
        L^1_i \lambda_i = L^2_i \lambda_i = \cdots 
        =L^{r_i}_i\lambda_i=L_i
        .\footnote{Pozor, zmizla nám čiarka z $L_i$ oproti
        predchádzajúcej rovnici}
    \end{equation*}    
    Pretože $\lambda_i$ je najvšeobecnejší unifikátor, 
    pre vhodnú substitúciu $\xi$ platí:
    \begin{equation*}
        L_i' = L^j_i \Theta = L^j_i (\lambda_i \circ \xi) = 
        (L^j_i\lambda_i)\xi = L_i \xi
    \end{equation*}

    Pre pohodlnosť označme $\lambda = \lambda_1 \union \lambda_2$.
    Z predpokladov vety je jasné, že $L_1', \neg L_2'$ sú
    unifikovateľné. Označme ich najvšeobecnejší unifikátor 
    ako $\sigma$.
    Teraz pozor, zamerajme sa na spojenie $C_1 \lambda_1$ a $C_2
    \lambda_2$. Hlaďanú najvšeobecnejšiu rezolventu zostrojíme práve
    pomocou nich:
    \begin{equation*}
    \begin{split}
    C &= ((C_1\lambda)\sigma - L_1\sigma) \union 
            ((C_2\lambda_2)\sigma-L_2\sigma) \\
      &= ((C_1\lambda)\sigma - (\{L^1_1, L^2_1, \dots, L^{r_1}_1 \}
        \lambda )\sigma) \union
        ((C_2\lambda)\sigma - (\{L^1_2, L^2_2, \dots, L^{r_2}_2 \}
        \lambda )\sigma) \\
      &= (C_1 (\lambda \circ \sigma) -
            \{ L^1_1, \dots, L^{r_1}_1 \} (\lambda\circ\sigma)) \union
         (C_2  (\lambda\circ\sigma) - 
            \{L^1_2, \dots, L^{r_2}_1\}(\lambda\circ\sigma))
    \end{split}
    \end{equation*}

    Na záver ešte potrebujeme overiť, že $C'$ je inštancia $C$.
    \begin{equation*}
    \begin{split}
        C' &= (C_1' \nu - L_1' \nu) \cup (C_2'\nu - L_2'\nu) \\
          &= ((C_1\Theta)\nu - 
                (\{L^1_1, \dots ,L^{r_1}_1\}\Theta)\nu) \union
             ((C_2\Theta)\nu - 
                (\{L^1_2, L^2_2, \ldots, L^{r_2}_2\}\Theta )\nu) \\
          &= (C_1(\Theta\circ\nu) - 
                \{L^1_1, \dots, L^{r_1}_1\} (\Theta\circ\nu)) \union
             (C_2(\Theta\circ\nu) - 
                \{L^1_2, \dots, L^{r_2}_2\} (\Theta\circ\nu))
    \end{split}
    \end{equation*}

    Lenže vieme, že $\lambda \circ \sigma$ je všeobecnejšie
    ako $\theta \circ \nu$, pretože
    $\lambda$ je všeobecnejší unifikátor ako $\Theta$ a 
    $\sigma$ je všebecnejší unifikátor ako $\nu$.
    Tým pádom $C'$ je naozaj inštancia $C$.\footnote{Ešte by sme
        mohli mať zlé svedomie z toho, čo na to povie množinové mínus,
        na ktoré sme v minulosti upozorňovali. Čitateľ sa môže sám
        presvedčiť, že v tomto prípade je to naozaj v poriadku}
%%% }}}
\end{dokaz}

\begin{veta}[Úplnosť metódy rezolvent] 
    Množina klauzúl $S$ nie je splniteľná $\iff$ 
    existuje odvodenie prázdnej klauzuly $\eps$ z $S$.
\end{veta}

\begin{dokaz}
\noindent
    \begin{itemize}
    \item[$\Leftarrow:$]
        Budeme ukazovať sporom.
        Predpokladajme, že v $S$ existuje odvodenie prázdnej klauzuly $\eps$. 
        Teda existuje postupnosť rezolvent $R_1, R_2, \dots, R_n$ 
        (medzi nimi niekde bude aj $\eps$, zrejme môžeme predpokladať
        $R_n = \eps$).

        Kvôli sporu predpokladáme, že $S$ je splniteľná, 
        teda existuje jej model $\mathcal{M}$, ktorý
        vyhovuje všetkým klauzuliam z $S$. Ako sme predtým dokázali,
        pravidlo rezolventy je korektné pravidlo a teda, ak sú
        $C_1$, $C_2$ ľubovoľné klauzuly z $S$ a $C$ je ich rezolventa,
        zo splniteľnosti $C_1$, $C_2$ vyplýva aj splniteľnosť $C$.
        Ak teda prejdeme celé rezolvenčné odvodenie $R_1,R_2,\dots,R_n$,
        postupne ukážeme, že $R_n=\eps$ je splniteľná klauzula. Spor.

\startFIXME
    \item[$\Rightarrow:$] Ak
        predpokladáme, že $S$ nie je splniteľná, potom podľa Herbrandovej vety (1.
        variant), nie je splniteľná práve vtedy, keď je možné jej priradiť konečný
        uzavretý sémantický strom. 

        \par Môže as stať, že strom $T$ pozostáva jedine z koreňa -- odmieta jedinú
        klauzulu a v tomto prípade veta platí. Teraz predpokladajme, že je konečný a má
        viac ako 1 vrchol. V tomto prípade, tak má aspoň jeden akceptujúci vrchol. Potom
        $i_v$ je čiastočná interpretácia končiaca v tom vrchole. Ďalej, každý
        nasledovník je odmietajúci. Vrchol je akceptujúci, ak čiastočná interpretácia v
        ňom existuje, a každý nasledujúci vrchol je odmietajúci.

        \par Predpokladajme, že by tento strom nema akceptujúci vrchol. Potom každý
        vrchol obsahuje nasledovníka, ktorý nie je odmietajúci. Týmto pádom by sme
        vytvorili nekonečne dlhú vetvu, čo je spor (strom je konečný).

        \par Ideme pracovať s akceptujúcim vrhcholom. Nech $v$ je akceptujúci vrchol
        stromu $T$ a $v_1$, $v_2$ sú odmietajúci nasledovníci $v$. $I(v)$ (čiastočná
        interpretácia končiaca vo vrchole $v$) vyzerá nasledovne:

        \begin{align*}
            I(v)    &= \{ m_1, m_2, \ldots, m_n \} \\
            I(v_1)  &= \{ m_1, m_2, \ldots, m_n, m_{n+1} \}  \\
            I(v_2)  &= \{ m_1, m_2, \ldots, m_n, \neg m_{n+1} \} 
        \end{align*}

        $C_1'$ a $C_2'$ sú dve základne inštancie klauzúl $C_1$ a $C_2$ -- $C_1'$ a
        $C_2'$ neplatia v $I(v_1)$ a $I(v_2)$. $C_1'$ a $C_2'$ sa neodmietajú v $I(v)$.
        $C_1'$ musí obsahovať $\neg m_{n+1}$ a $C_2'$ musí obsahovať $m_{n+1}$. $L_1' =
        \neg m_{n+1}$ a $L_2' = m_{n+1}$. AAk vynecháme $L_1'$ a $L_2'$, dostaneme
        rezolventu. $C'$ je rezolventa $C_1$ a $C_2$. $C' = (C_1' - L_1') \cup (C_2' =
        L_2')$. $C'$ -- musí byť nepravdivá v $I(v)$. Podľa predchádzajúcej lemy musí
        existovať rezolventa $C$ taká, že $C'$ je základná inštancia $C$.

        \par Vezmime si $T''$ -- uzavretý sémantický strom, $C \cup \{C\}$. ... (niečo
        ďalej?)
    \end{itemize}
\end{dokaz}

\begin{priklad}
    Majme nasledujúce formuly:
    \begin{align*}
        F_1: &(\forall x) (C(x) \implies (W(x) \land R(x))\\
        F_2: &(\exists x) (C(x) \land Q(x)) \\
        G:   &(\exists x) (Q(x) \land R(x))
    \end{align*}
    Ukážte, že $G$ je logickým dôsledkom $F_1$ a $F_2$.

    Riešenie: Budeme ukazovať nesplniteľnosť $F_1 \land F_2 \land \neg G$.
    Pre $F_1$, $F_2$ a $\neg G$ vytvoríme štandardné formy.
    $F_1$ si upravíme takto:
    \begin{equation*}
        (\forall x) (C(x) \implies (W(x) \land R(x)) \iff 
        (\forall x)(\neg C(x) \lor (W(x)\land R(x)) \iff 
        (\forall x) ((\neg C(x) \lor W(x)) \land (\neg C(x) \lor R(x)))
    \end{equation*}
    Štandardná forma je teda 
    $\{\neg C(x) \lor W(x),\ \neg C(x) \lor R(x)\}$.
    Štandardná forma $F_2$ je $\{C(a),\ Q(a)\}$.
    Pre $\neg G$ dostávame
    \begin{equation*}
        \neg G \iff \neg (\exists x)(Q(x)\land R(x)) \iff 
            (\forall x) (\neg Q(x) \lor \neg R(x))
    \end{equation*}
    Štandardná formula pre túto formulu je $\neg Q(x) \lor \neg R(x)$

    Dostávame nasledujúcich 5 klauzúl:
    \begin{itemize}
        \item[1] $\neg C(x) \lor W(x)$ -- $F_1$
        \item[2] $\neg C(x) \lor R(x)$ -- $F_1$
        \item[3] $C(a)$ -- $F_2$
        \item[4] $Q(a)$ -- $F_2$
        \item[5] $\neg Q(x) \lor \neg R(x)$ -- $G$
    \end{itemize}

    Teraz budeme postupne robiť rezolventy:
    \begin{itemize}
        \item[6] $R(a)$ -- rezolventa 2, 3
        \item[7] $\neg R(a)$ -- rezolventa 4, 5
        \item[8] $\eps$ -- rezolventa 6, 7
    \end{itemize}

    Záver: $G$ je logickýkm dôsledkom $F_1$ a $F_2$.
\end{priklad}

