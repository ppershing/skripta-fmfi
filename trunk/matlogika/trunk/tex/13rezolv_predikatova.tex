\section {Metóda rezolvent pre logiku 1. rádu}

\begin{definicia}[Spojenie]
    Nech $C$ je klauzula, ktorá obsahuje dva alebo viac literálov 
    (a tie pozostávajú z rovnakého predikátu len s inými parametrami).
    Ak tieto literály majú najvšeobecnejší unifikátor $\sigma$, 
    tak $C\sigma$ sa nazývame spojením $C$. 
    
    Ak $C\sigma$ je jednotková klauzula, tak $C\sigma$ nazývame tiež
    jednotkovým spojením $C$.
\end{definicia}

\begin{priklad}
    Uvažujme klauzulu $C$, ktorá vyzerá nasledovne: 
    $C = \{ P(x) \lor P(f(y)) \lor \neg Q(x)\}$. Zoberme literály
    $P(x)$ a $P(f(y))$. Ich najvšeobecnejší unifikátor je
    $\sigma=\{ f(y) / x \}$. Potom spojenie je 
    $C\sigma = \{P(f(y)) \lor \neg Q(x) \}$.
\end{priklad}

\begin{definicia}[binárna rezolventa]
    Nech $C_1$ a $C_2$ sú dve klauzuly (budeme ich nazývať predpoklady), 
    ktoré nemajú spoločné premenné. Nech $L_1 \in C_1$ a $L_2 \in C_2$ 
    sú dva literály. Ak $L_1$ a $\neg L_2$ majú 
    najvšeobecnejší unifikátor $\sigma$, tak výraz
    \begin{equation*}
        (C_1\sigma - L_1\sigma) \union (C_2\sigma - L_2\sigma)
    \end{equation*}
    sa nazýva binárnou rezolventou.\footnote{Za pozornosť stojí fakt, že
    vo všeobecnosti $(C\sigma - L\sigma) \ne (C-L)\sigma$.} 
    Literály $L_1$ a $L_2$ môžeme vynechať a nazývame ich nadbytočné.
\end{definicia}

\begin{priklad}
    Majme $C_1 = P(x) \lor Q(x)$ a $C_2 = \neg P(a) \lor R(x)$, 
    čo budú predpoklady. Na to, aby sme mohli previesť operáciu
    binárnej rezolventy, musíme najskôr premenovať premenné v druhom
    výraze, aby boli rôzne od tých v prvom. Máme teda
    $C_2' = \neg P(a) \lor R(y)$.

    Uvažujme teraz klauzuly $L_1 = P(x)$ a $L_2 = \neg P(a)$.
    Ich najvšeobecnejší unifikátor je $\sigma = \{a/x\}$.

    Binárna rezolventa $C_1$ a $C_2$ je
    \begin{equation*}
    \begin{split}
        (C_1\sigma - L_1\sigma) \union (C_2\sigma - L_2\sigma) 
        &= (\{P(a),\ Q(a))\} - \{P(a)\}) \union
            (\{\neg P(a),\ R(y)\}-\{\neg P(a)\}) \\
        &= Q(a)\lor R(y)
    \end{split}
    \end{equation*}
    Nadbytočné literály sú $P(x), \neg P(a)$.
\end{priklad}

\begin{definicia}[Rezolventa logiky 1. rádu] 
    Rezolventou z predpokladov $C_1$ a $C_2$ definujeme ako jednu z
    nasledujúcich binárnych rezolvent:
    \begin{enumerate}
        \item Binárna rezolventa $C_1$ a $C_2$
        \item Binárna rezolventa $C_1$ a spojenia $C_2$
        \item Binárna rezolventa spojenia $C_1$ a $C_2$
        \item Binárna rezolventa spojenia $C_1$ a spojenia $C_2$
    \end{enumerate}
\end{definicia}

\startFIXME


\paragraph{Príklad} $$C_1 = P(x) \lor P(f(y))\lor Rg(y))$$
$$ C_2 = \neg P(f(g(a)) \lor Q(b)$$
Spojenie pre $C_1$ vyzerá ako: $C_1': P(f(y)) \lor R(g(y))$.
Binárna rezolventa $C_1' a C_2$ bude vyzerať takto: $R(g(g(a))) \lor Q(b)$ --
rezolventa $C_1$ a $C_2$.


\section{Opakovanie}
	Ak množina klauzúl nie je splniteľná, potom metódou rezolvent z nej vždy
	môžeme dostať prázdnu klauzulu (a ak sa táto dostane množiny klauzúl,
	tak formula nie je splniteľná v žiadnej interpretácii). Ak máme nejakú
	klauzulu $C$, $C\sigma$ sme nazývali spojením klauzuly $C$. Definícia
	binárnej rezolventy. 

\par  Metódu rezolvent zaviedol roku 1965 Robinson, je efektívnejšia ako obe
varianty Herbrandovej metódy. 

\par \{ sleep...\}

\paragraph{Úplnosť metódy rezolvent}

\paragraph{Príklad} Majem množinu klauzúl $S$:
\begin{enumerate}
	\item $P$
	\item $\neg P\lor Q$
	\item $\neg P \lor \neg Q$
\end{enumerate}
Tejto množine klauzúl zodpoveda uzavretý sémantický strom.
Prislúchajúca herbrandovská báza je $\{P, Q\}$ (na tabuľu sa kreslí sémantický
strom pre $P$ a $Q$, usilovný čitateľ si ho isto domyslí). Každá vetva sa končí
odmietajúcim vrcholom, žiadna z tých interpretácií, ktoré končia v listoch, nie
je splniteľná. Tomuto stromu môžeme priradiť uzavretý podstrom (označíme ho
$T'$,  má odseknuté vetvy na miestach, kde sú podstromy odmietajúce)


\begin{verbatim}
            T
            /\
         P /  \ \neg P
       Q /\ nQ Q/\ nQ 
\end{verbatim}

\begin{verbatim}
            T'
            (1)
            /\
      (2)P /  \ \neg P
   (4) Q /\ nQ x (5)
       x   x
\end{verbatim}

\par $\neg P$ -- rezolventa $(4)$, $(5)$, $\neg P \cup S$.  $S\cup \{ \neg P
\}\cup \{ \square \}$.

\par Vznikli nám teda klauzuly.
$$(4) \neg P\qquad (2) (3)$$
$$(5) \Box\qquad (4)(1)$$

\par Strom sa po každej aplikácii pravidla postupne skracuje. 

\paragraph{Lema} Nech $C_1'$ a $C_2'$ sú inštancie $C_1$ resp. $C_2$ (v uvedenom
poradí). Ak $C'$ je rezolventa $C_1'$ a $C_2'$, tak potom existuje rezolventa
$C$ klauzúl $C_1$ a $C_2$, že $C'$ je inštancia $C$. 

\paragraph{Dôkaz} Ak je treba, premenujeme premenné $C_1$ a $C_2$. Nech $L_1'$ a
$L_2'$ sú literály, ktoré môžeme vynechať (sú nadbytočné). ďalej nech platí:
$$C' = (C_1' \nu - L_1'\nu) \cup ( C_2'\nu - L_2'\nu)$$
Pričom $\nu$ je najvšeobecnejší unifikátor pre $L_1'$ a $\neg L_2'$. $C_1'$,
$C_2'$ sú inštancie $C_1$ a $C_2$, a teda existuje substitúcia $\Theta$ taká, že
platí:

$$ C_1' = C_1 \Theta $$
$$ C_2' = C_2 \Theta $$

(Pozn.: $C_1$ a $C_2$ nemajú spoločné premenné). $L^1_i, L^2_i, \ldots, L^{r_i}_i,
i=1,2$ sú literály, ktoré v $C_1$ zodpovedajú $L_i'$, teda $L^1_i \Theta = L^2_i
\Theta = \cdots = L^{r_i}_i\Theta = L'_i (i=1,2)$. $D_i > 1$ dostaneme
najvšeobecnejší kvantifikátor $\lambda_i (i=1,2)$ pre $\{ L^1_i, L^2_i, \ldots,
L^{r_i}_i\}, L_i = L^1_i \lambda_i (i=1,2)$. $\lambda_i$ je najvšeobecnejší
unifikátor, tak pre vhodnú substitúciu $\xi$ platí:
$$ L_i' = L^1_i \Theta = L^1_i (lambda_i \circ \xi) = (L^1_i\lambda_i)\xi =
L_i \xi$$

$$L_i\xi = L_i'$$

$L_i$ .. spojení $C_i\lambda_i$, pre $C_i$, ak $r_i = 1$, ak $r_i = 1$, potom
$\lambda_i = \Sigma$, $L_i = L_i^1\lambda_i$.

$$ \lambda = \lambda_1 \cup \lambda_2$$
$$L_i' = \mbox{...} L_i$$

$L'_i, \neg L_2'$ -- unifikovateľné.
$L-1, \neg L_2$ -- unifikovateľné.

Označme $\sigma$ najvšeobecnejší unifikátor pre $L_1'$ a $\neg L_2'$.

$C= ((C_1\lambda)\sigma = L_1\sigma) \cup
((C_2\lambda_2)\sigma-L_2\sigma) = ((c_1\lambda)\sigma - (\{L^1_+, L^2_1,
\ldots, L^{r_i}_1 ...
= C_1(\lambda\circ \sigma) = \{ L^1_1, \ldots L^{r_1}_1 \} (\lambda\circ\sigma)
\cup C_2)(\lambda\circ\sigma) - \{L^1_2, \ldots
L^{r_i}_1\}(\lambda\circ\sigma)$

\par $C$ -- rezolventa $C_1$ a$C_2$, $C'$ je substitúcia $C$:
$C = (C_1' \nu = L_1' \nu) \cup (C_2'\nu = L_2'\nu) = (C_1\Theta)\nu -
(\{L^1_1, \ldots L^{r_i}_1\}\Theta)\nu)\cup ((C_2\Theta)\nu - \{L^1_2, L^2_2,
\ldots, L^{r_i}_2\}\Theta )\nu) = 
C_1(\Theta\circ\nu) - \{L^1_1, \ldots, L^{r_i}_1\} \Theta\circ\nu) \cup
(C_2(\Theta\circ\nu) - \{L^1_2, \ldots L^{r_i}_2\} \Theta\circ\nu)
$

$\lambda \circ \sigma$ jke všeobecnejšia ako $\theta \circ \nu$.


\paragraph{Veta (úplnosť metódy rezolvent)} Množina klauzúl $S$ nie je
splniteľná práve vtedy, keď existuje odvodenie prázdnej klauzuly $\Box$ z $S$.

\paragraph{Dôkaz} Predpokladajme, že z $S$ existuje odvodenie prázdnej klauzuly
$\square$. $R_1, R_2, \ldots R_n$ sú všetky rezolventy v odvodení (medzi nimi
niekde bude aj $\square$). Zoberiem $C_1$, $C_2$ -- ľubovoľné klauzuly z $S$ a
$C_1$ a $C_2$ bude príslúchať nejaká rezolventa. Ak sú klauzuly splniteľné, je
splniteľná aj rezolventa. To znamená, že $C_1$ a $C_2$ nemôžu byt splniteľné a
teda nemôže byť množina klauzúl (klauzuly, ktorých rezolventou je prázdna
klauzula, nebudú splniteľné nikdy). 

\subparagraph{Obrátené tvrdenie} Prepokladajme, že množina klauzúl $S$ nie je
splniteľná (máme ukázať, že ako rezolventa sa tam ukáže prázdna klauzula). Ak
predpokladáme, že $S$ nie je splniteľná, potom podľa Herbrandovej vety (1.
variant), nie je splniteľná práve vtedy, keď je možné jej priradiť konečný
uzavretý sémantický strom. 

\par Môže as stať, že strom $T$ pozostáva jedine z koreňa -- odmieta jedinú
klauzulu a v tomto prípade veta platí. Teraz predpokladajme, že je konečný a má
viac ako 1 vrchol. V tomto prípade, tak má aspoň jeden akceptujúci vrchol. Potom
$i_v$ je čiastočná interpretácia končiaca v tom vrchole. Ďalej, každý
nasledovník je odmietajúci. Vrchol je akceptujúci, ak čiastočná interpretácia v
ňom existuje, a každý nasledujúci vrchol je odmietajúci.

\par Predpokladajme, že by tento strom nema akceptujúci vrchol. Potom každý
vrchol obsahuje nasledovníka, ktorý nie je odmietajúci. Týmto pádom by sme
vytvorili nekonečne dlhú vetvu, čo je spor (strom je konečný).

\par Ideme pracovať s akceptujúcim vrhcholom. Nech $v$ je akceptujúci vrchol
stromu $T$ a $v_1$, $v_2$ sú odmietajúci nasledovníci $v$. $I(v)$ (čiastočná
interpretácia končiaca vo vrchole $v$) vyzerá nasledovne:

\begin{align*}
    I(v)    &= \{ m_1, m_2, \ldots, m_n \} \\
    I(v_1)  &= \{ m_1, m_2, \ldots, m_n, m_{n+1} \}  \\
    I(v_2)  &= \{ m_1, m_2, \ldots, m_n, \neg m_{n+1} \} 
\end{align*}

$C_1'$ a $C_2'$ sú dve základne inštancie klauzúl $C_1$ a $C_2$ -- $C_1'$ a
$C_2'$ neplatia v $I(v_1)$ a $I(v_2)$. $C_1'$ a $C_2'$ sa neodmietajú v $I(v)$.
$C_1'$ musí obsahovať $\neg m_{n+1}$ a $C_2'$ musí obsahovať $m_{n+1}$. $L_1' =
\neg m_{n+1}$ a $L_2' = m_{n+1}$. AAk vynecháme $L_1'$ a $L_2'$, dostaneme
rezolventu. $C'$ je rezolventa $C_1$ a $C_2$. $C' = (C_1' - L_1') \cup (C_2' =
L_2')$. $C'$ -- musí byť nepravdivá v $I(v)$. Podľa predchádzajúcej lemy musí
existovať rezolventa $C$ taká, že $C'$ je základná inštancia $C$.

\par Vezmime si $T''$ -- uzavretý sémantický strom, $C \cup \{C\}$. ... (niečo
ďalej?)



\paragraph{Príklad} Majme množinu formúl $F_1: (\forall x) (C(x) \implies (W(x)
\land R(x))$, $F_2: (\exists x)(C(x) \land Q(x))$ $G: (\exists x) (Q(x) \land
R(x))$. Ukážte, že $G$ je logickým dôsledkom $F_1$ a $F_2$.

\paragraph{Riešenie} Pre $F_1$, $F_2$ a $\neg G$ vytvoríme štandardné formy.
Dostávame nasledujúcich 5 klauzúl:
\begin{enumerate}
	\item $(\forall x) (C(x) \implies (W(x) \land R(x)) \iff (\forall
	x)(\neg C(x) \lor (W(x)\land R(x)) \iff (\forall x) ((\neg C(x) \lor
	W(x)) \land (\neg C(x) \lor R(x)))$
	\par (1) $\neg C(x) \lor W(x)$ -- $F_1$
	\par (2) $\neg C(x) \lor R(x)$ -- $F_1$
	\par $C(a)$ -- $F_2$
	\par $Q(a)$ -- $F_2$
\end{enumerate}
$$\neg G \iff \neg (\exists x)(Q(x)\land R(x)) \iff (\forall x) (\neg Q(x) \lor
\neg R(x)) $$. Štandardná formula pre túto formulu je:
\par (5) $\neg Q(x) \lor \neg R(x)$ -- $G$.

\par Rezolventy: 
\par (6) $R(a)$ -- rezolventa (2), (3)
\par (7) $\neg R(a)$ ($\sigma = \{a / x \})$ -- rezolventa (5), (4)
\par (8) $\square$ -- rezolventa (6), (7)

\par Záver: $G$ je logickýkm dôsledkom $F_1$ a $F_2$

