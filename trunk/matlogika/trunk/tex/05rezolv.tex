\section{Metóda rezolvent}

Celú túto kapitolu sa budeme venovať sémantike formúl logiky prvého
rádu.

Keď uvažujeme výrokovú logiku, interpretujeme formulu $A$ funkciou
$\bar{v}$ -- valuáciou.
Formulu sme nazvali tautológiou, ak $\bar{v}(A)$ je pre každú
interpretáciu pravda -- treba preskúmať $2^n$ interpretácii.
V predikátovej logike je to (ako uvidíme) ešte horšie.

Zopakujme si, čo už vieme. V predikátovej logike máme funkčné a
predikátové symboly. Univerzum budem označovať netradične ako $D$.
Znakom $M$ budeme totiž označovať jadro(maticu) formuly v prenexnom
tvare.

Funkcia $f$ jr realizovaná ako $n$-árna funkcia 
$f(x_1,\dots,x_n): D^n \rightarrow D$.
Predikát $P$ jr realizovaný ako $n$-árna funckia
$P(x_1,\dots,x_n): D^n \rightarrow \{0,1\}$.

Vo formule rozlišujeme voľné a viazané premenné.
Ak má formula voľné premenné, nevieme určiť jej pravdivostnú hodnotu,
iba ak za všetky voľné premenné dosadíme konštanty.

Ďalej, ku každej formule $A$ vieme zostrojiť takú,
ktorá je v prefixovom (prenexnom) tvare -- t.j. je tvaru
prefix + jadro: $(Q_1 x_1)\cdots(Q_n x_n)M$.

Na upravenie formuly do tohoto tvaru používame prenexné operácie.
Skutočnosť, že premenná $x$ má voľný výskyt vo formule $A$ budeme
značiť ako $A(x)$, v opačnom prípade budeme písať jednoducho $A$.

\medskip
Prenexné operácie:
\begin{align*}
    (Qx)A(x) \lor B &\equiv (Qx)(A(x) \lor B) \tag{1a}\\
    (Qx)A(x) \land B &\equiv (Qx)(A(x) \land B) \tag{1b}\\
    \neg (\forall x) A(x) &\equiv (\exists x) \neg A(x) \tag{2a}\\
    \neg (\exists x) A(x) &\equiv (\forall x) \neg A(x) \tag{2b}\\
    (\forall x) A(x) \land (\forall x) B(x) &\equiv 
        (\forall x) (A(x) \land B(x)) \tag{3a}\\
    (\exists x) A(x) \lor (\exists x) B(x) &\equiv
        (\exists x) (A(x) \lor B(x)) \tag{3b}
\end{align*}
\begin{poznamka}
    S predchádzajúcimi operáciami 3a,3b treba pracovať pozorne. Im veľmi
    podobné úpravy totiž neplatia:

    \begin{align*}
      (\forall x) A(x) \lor (\forall x) B(x)  & \not \equiv
        (\forall x) (A(x) \lor B(x)) \tag{x1}\\
      (\exists x) A(x) \land (\exists x) B(x) & \not \equiv
        (\exists x) (A(x) \land B(x)) \tag{x2}
    \end{align*}
    Zoberme si napríklad $D=\{1,2\}$. Ak položíme
    $A(1)=1, A(2)=0, B(1)=0, B(1)=1$, dostaneme, že pravá strana x1 bude
    platiť, zatiaľ čo ľavá strana nie. Pre x2 naopak.
\end{poznamka}

Pretože môžeme substituovať za premenné --
    $(\forall x) B(x) \equiv (\forall z) B(z)$, môžeme predchádzajúce
prenexné operácie zhrnúť do nasledujúcich všeobecných transformácií.
\begin{align*}
    (Q_1 x)A(x) \lor (Q_2 x)B(x) \equiv 
        (Q_1 x)(Q_2 z)(A(x) \lor B(z)) \tag{4a} \\
    (Q_3 x)A(x) \land (Q_4 x)B(x) \equiv
        (Q_3 x)(Q_4 z)(A(x) \land B(z)) \tag{4b}
\end{align*} 
kde v oboch prípadoch $z$ je premenná, ktorá sa nevyskytuje voľne v $A$
(a ani v pôvodnom $B(x)$).
Tento najvšeobecnejší tvar ale zbytočne pridáva premenné a preto je
vhodný iba v prípadoch, keď nefungujú operácie 1 až 3.

%%%%%
\subsection{Algoritmus na zostrojenie prenexného tvaru}

\begin{enumerate}
\item Odstránenie implikácií a ekvivalencií:
    \begin{align*}
        A \lequiv B &\equiv (A \implies B) \land (B \implies A) \\
        A \implies B &\equiv \neg A \lor B
    \end{align*}

\item Odstránenie dvojitej negácie a presun negácie k formule.
    \begin{align*}
        \neg (\neg A) &\equiv A \\
        \neg (A \lor B) &\equiv (\neg A \land \neg B) \\
        \neg (A \land B) &\equiv (\neg A \lor \neg B) \\
        \neg (\forall x) A(x) &\equiv (\exists x) \neg A(x) \\
        \neg (\exists x) A(x) &\equiv (\forall x) \neg A(x)
    \end{align*}

\item  Premenovanie viazaných premenných, ak je to potrebné.

\item Použijeme zákony:
    \begin{align*}
        (Q x) A(x) \lor B &\equiv (Q x)(A(x) \lor B) \\
        (Q x)A(x) \land B &\equiv (Q x)(A(x) \land B) \\
        (\forall x) A(x) \land (\forall x)B(x) &\equiv 
            (\forall x)(A(x) \land B(x)) \\
        (\exists x)A(x) \lor (\exists x)B(x) &\equiv 
            (\exists x) (A(x)\lor B(x)) \\
        (Q_1 x) A(x) \lor (Q_2 x)B(x) &\equiv
            (Q_1 x)(Q_2 z) (A(x)\lor B(z)) \mbox{ kde $z$ je nová
            premenná} \\
        (Q_3 x) A(x) \land (Q_4 x)(B(X) &\equiv 
            (Q_3 x)(Q_4 z)(A(x) \land (B(z)) \mbox{ kde $z$ je nová
            premenná}
    \end{align*}
\end{enumerate}

\begin{priklad}
    \begin{align*}
        (\forall x)(\forall y) \left[ (\exists z) \left(P(x,z) \land
            P(y,z)\right) \implies (\exists u) Q(x,y,u) \right] &\equiv \\
        (\forall x)(\forall y)\left[\neg (\exists z)(P(x,z)\land 
            P(y,z))\lor(\exists u)Q(x,y,u)\right] &\equiv \\
        (\forall x)(\forall y)\left[(\forall z)(\neg P(x,z)\lor 
            \neg P(y,z))\lor(\exists u)Q(x,y,u)\right] &\equiv \\
        (\forall x)(\forall y)(\forall z)(\exists u) \left[ \neg P(x,z) \lor 
            \neg P(y,z) \lor Q(x,y,u) \right] &
    \end{align*}
\end{priklad}
