\section{Veta o úplnosti}
\startFIXME

G\"odelova veta, ktorú si teraz vyslovíme a dokážeme, má 2 varianty.

\begin{veta}{G\"odel, 1}
    Nech $T$ je teória v jazyku $L$ a ak $A$ je
    ľubovoľná formula jazyka $L$, potom $T \provable A \iff T \models A$,
    čiže je splnená v každom modeli teórie $T$.
\end{veta}

\begin{veta}{G\"odel, 2}
    Teória $T$ je bezosporná práve vtedy, ked $T$ má model.
\end{veta}

\begin{poznamka}
    Varianta 1 G\"odelovej vety vyplýva z varianty 2.

    Veta o dedukcii mala nasledovný dôsledok:

    Ak máme teóriu $T$, $A$ je jej formula a $A'$ je uzáver formuly $A$,
    tak $T \provable A \iff T \union \{ \neg A' \}$ je sporná teória,
    t.j. podľa vety 2 $T \union \{ \neg A' \}$ nemá model.

    Toto znamená, že v každom modeli teórie $T$ je pravdivý uzáver $A'$.
    $T \models A' \Rightarrow T \models A$. Z platnosti vety 2 vyplýva veta 1.
\end{poznamka}

\paragraph{Dôkaz:}

\par Budeme sa snažiť zostrojiť model pre teóriu, ktorá je bezosporná. Majme
bezospornú teóriu v jazyku $L$. Potrebujeme relácie, zobrazenia a $M$ --
univerzum. K dispozícii máme len syntaktické prostriedky teórie. Do úvahy
prichádzajú iba termy bez premenných -- majú jednoznačnú realizáciu (sami sebe
budú realizáciou).

$$A[e] \iff T \provable A $$.

Možné problémy pri konštrukcii modelu:
\begin{enumerate}
    \item Jazyk $L$ neobsahuje žiadne konštanty (a teda žiadne termy bez
            premenných).
    \item Ak jazyk $L$ bude jazyk s rovnosťou, môže sa stať, že v teórii $T$
            bude $T \provable t=s$, ale $t$ a $s$ sú rôzne termy bez
            premenných (rôzne konštanty).
    \item Nech $m$ je ľubovoľná realizácia jazyka $L$ a $A$ je uzavretá
            formula jazyka $L$. Potom práve jedna z formúl $A$, $\neg A$ je
            pravdivá, ale žiadna z nich nemusí byť dokázateľná v $T$.
    \item Môže sa stať, že formula $(\exists x)B$ je dokázateľná v teórii
            $T$, ale pre žiaden term $t$ bez premenných formula $B_x[t]$ nie
            je dokázateľná v $T$. To znamená, že $(\exists x)B$ 
            podľa Tarského definície pravdivosti $(\exists x)B$ je
            nepravdivá, čo je spor s vetou o korektnosti.
\end{enumerate}

Ako odstránime tieto nedostatky:
Odstránenie bodu 2 je jednoduché -- riešime vhodnou faktorizáciou, čiže
zavedieme si množinu $\tau$, čo je množina všetkých termov bez premenných
a na nej zavedieme reláciu ekvivalencie.

Body 1 a 4 riešime rozšírením o konštanty.

Tretí bod sa rieši tzv. konzervatívnym rošírením teórie (Henkinovským).
Budú to tzv.  konzervatívne teórie (na pôvodnom jazyku nezískame žiadne nové
teorémy ani nestratíme žiadne).

\begin{definicia} % uplna teoria
%%% {{{
    Hovoríme, že teória $T$ s jazykom $L$ je \emph{úplná}, ak $T$ je
    bezosporná teória a pre ľubovoľnú uzavretú formulu $A$ na jazyku
    $L$ buď $A$ alebo $\neg A$ je dokázateľná v $T$.
%%% }}}
\end{definicia}

\par Nech $T_h(m)$ je množina všetkých pravdivých uzavreté formuly $T$. 
Potom $T_h$ je úplná (nemusíme brať otvorené formuly 
-- napr. $x=0$ v aritmetike nemusí byť pravdivé, 
lebo závisí od ohodnotenia $x$).

\begin{poznamka}
 V úplnej teórii (a teda špeciálne v $T_h$) nemôže nastať problém 3,
 ktorý sme spomínali.
\end{poznamka}

\begin{definicia}
%%% {{{ Henkinova teoria
    Hovoríme, že teória $T$ s jazykom $L$ je \emph{Henkinova}, ak pre
    ľubovoľnú uzavretú formulu $(\exists x)B$ jazyka $L$ platí
    \begin{equation}
        T \provable (\exists x)B \implies B_x[c]
    \end{equation}
    pre nejaké $c$ -- konštantu.
%%% }}}
\end{definicia}

\begin{poznamka}
    Ak je teória Henkinova, tak sme vyriešili problémy 1,4.
\end{poznamka}

\begin{lema}
    Ak $T$ je úplná a Henkinova teória, tak potom $T$ má model.
\end{lema}
\begin{dokaz}
    Nech $L$ je jazyk teórie $T$, $\tau$ je množina všetkých
    termov jazyka $L$ bez premenných. Na množine $\tau$ definujem reláciu
    ekvivalencie nasledovne:
    \begin{equation}
        \forall t_1, t_2 \in \tau :
            t_1 \equiv t_2 \iff T \provable t_1 = t_2
    \end{equation}
    Rovnosť je reflexívna, symetrická, tranzitívna, teda týmto
    spôsobom definovaná relácia je relácia ekvivalencie a rozdeľuje
    nám množinu $\tau$ na triedy ekvivalencie:
    \begin{equation*}
        [t] = \{ s \in \tau: t \equiv s\}
    \end{equation*}
    Univerzum budú tvoriť termy bez premenných.
    Relačná štruktúra $M$ bude mnomžina všetkých termov bez premenných.
    Nech $f$ je ľubovoľný $n$-árny funkčný symbol a nech
     $[t_1], [t_2], \ldots, [t_n] \in M$. Definujeme funkciu $f$ v relačnej
     štruktúre $M$ nasledovne:
     \begin{equation*}
      f_m([t_1], \ldots, [t_n]) = [f(t_1, \ldots t_n)]
     \end{equation*}
    Ešte treba ukázať, že táto definícia je konzistentná, čize
    $f_m([t_1], \ldots, [t_n])=f_m([s_1], \ldots, [s_n])$ ak
    $s_i \equiv t_i$, čiže nezáleží na výbere reprezentantov.
    Taktiež je dobré si uvedomiť, že
    $[t_{x_1,\dots,x_n}[t_1,\dots,t_n]]=t[e], e(x_i/t_i)$.

    Podobne, nech $P$ je $n$-árny predikátový symbol rôzny od $=$ ($=$ sme
    si už zaviedli). V tom prípade definujeme
    \begin{equation*}
     ([t_1], \ldots [t_n]) \in P_m \iff T \provable P(t_1, \ldots, t_n)
    \end{equation*}
\end{dokaz}

\paragraph{Poznámka:} Doteraz sme pracovali s $[\ldots]$. Výsledok sa nezmení,
ak zoberiem $s_1 \in [t_1], \ldots s_n \in [t_n]$. Nezávisí to od reprezentanta
danej triedy.
\par
$$ s_1, \ldots s_n $$
$$ s_i \in [t_i], i = 1, \ldots n $$
$$ [t_1], \ldots, [t_n]) \in P_m \iff T \provable P(s_1, \dots s_n) $$

\par

\fixme{zarovnanie na ...}
\begin{align*}
    x_1, x_2, &\ldots,& x_n \mbox{-- premenné}.	\\
    e(x_i) &=& t_i, i=1,\ldots,N	\\
    t[e] &=& [t_{x_1,\ldots,x_n}][t_1,\ldots, t_n]	\\
    t[e] &=& [t]
\end{align*}

% \begin{unknown}
$A$ -- ľubovoľná uzavretá formula z $L$, potom $m \models A \iff T \provable A$.
$P(t_1, t_2, \ldots t_n)$ -- atomická formula,  uzavretá, $t_1. \ldots t_n$ sú
termy bez premenných.

$m \models A \iff ([t_1], \ldots, [t_n]) \in P_m \iff T \provable P(t_1,\ldots
t_n)$ pre ľubovoľné ohodnotenie.

\par $A: t_1 = t_2$. $m \models t_1 = t_2 \iff [t_1] = [t_2] \iff T \provable t_1 =
t_2$.

\par $A: \neg B$. Indukčný predpoklad: pre $B$ bolo tvrdenie dokázané ($m\models
B \iff T \provable B)$. Formula $A$ je pravdivá v $m$ práve vtedy, keď formula $B$
nie je pravdivá $B$ nie je pravdivá, $T \unprovable B, T \vdash \neg B$.

\par $A. B \implies C$, pričom $B$ a $Cc$ sú dokázané podľa indukčného
predpokladu. Ak platí $m \models B \implies C$, potom $m \models \neg B \lor m
\models C$. $T \provable \neg B \lor T \vdash C$.

\begin{align*}
    T \provable B \implies C \\
    \provable C \implies (B \implies C)
\end{align*}

Opačná implikácia:
\begin{equation*}
    T\provable B\implies C\Rightarrow m \models B \implies C
\end{equation*}

Teória $T$ je úplná teória a $B \implies C$ je uzavretá. Pre každú uzavretú
formulu platí, jedna z možností $T \provable \neg B$ (potom $m \models \neg B$)
alebo $T \provable B$, a z toho vyplýva, že $T \vdash C$, a teda $m \models C$.
Tvrdenie je teda dokázané.

\begin{equation}
    m \models A \iff T \provable A
\end{equation}

\par Predpokladajme, že formula $A$ je v tvare $A: (\forall x) B$. Pre každú
inštanciu formuly $B$ tvrdenie platí, $\provable (\forall x) B \implies B_x[t]$
(axióma špecifikácie). Potom aj $T \provable B_x[t]$ (lebo teória je Henkinova). $m \models B[e(x/t)]$.

\par Ak je pravdivá negácia, potom $T \provable \neg A$, čo znamená, že $(\exists
x) \neg B$. $T \provable (\exists x) \neg B \implies B_x[c]$, a to znamená, že $T
\provable \neg B_x[c]$.

\par Nech $A$ je ľubovoľná formula z teórie $T$. Pre ňu platí, že $T \provable A$
(predpoklad). Potrebujem dokázať, že $T \provable A \iff m \models A$. Zoberiem si
$A'$ -- uzáver formuly $A$. Na $A'$ sa vzťahuje tvrdenie, ktoré sme už dokázali,
a teda $m \models A' \iff T \provable A'$. Podľa vety o uzávere:
$m \models A \iff  m\models A' T \provable A' \iff T \vdash A$

\par Pre ľubovoľnú teóriu, ktorá je idealizovaná (Henkinova a úplná), viem
zostrojiť model.

\stopFIXME
