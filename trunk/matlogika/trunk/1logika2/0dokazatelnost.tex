\section{Pravdivosť a dokázateľnosť}
\subsection{Veta o úplnosti}

\begin{definicia}{Logická platnosť formuly}
    Nech $L$ je jazyk prvého rádu a $A$ je formula jazyka
    $L$. Hovoríme, že formula $A$ je \emph{logicky platná},
    označujeme $\models A$,
    ak je splnená v ľubovoľnej realizácii $m$ jazyka $L$.

    \begin{align*}
        m & \models & A[e] \\
        m & \models & A \\
        & \models & A
    \end{align*}
\end{definicia}

\begin{poznamka}
    Formula $A$ je logicky platná, práve vtedy, keď je pravdivá
    bez ohľadu na realizáciu symbolov jazyka $L$.
\end{poznamka}

\begin{definicia}{Teória}
    Nech $L$ je jazyk, prvého rádu a $T$ je množina formúl
    jazyka $L$. Hovoríme, že $T$ je teória 1. rádu predikátovej logiky
    s jazykom $L$ (t.j. množina formúl $T$ je množina axióm teórie).
\end{definicia}

\begin{definicia}{Model teórie}
    Nech $T$ je teória v jazyku $L$, $m$ je realizácia jazyka $L$.
    Hovoríme, že $m$ je modelom teórie $T$, ak pre každú formulu $A$
    patriacu $T$ platí $\models A$.

    Hovoríme, že formula $A$ je
    sémantickým/tautologickým dôsledkom (vetou teórie) množiny formúl $T$,
    resp. $A$ je $T$-platná, ak $A$ je splnená.
    v každom modeli teórie $T$.
    Túto skutočnosť označujeme $T \models A$.
\end{definicia}

\fixme{nasledujuce veci}
\begin{priklad}{Teória čiastočného usporiadania}
    Majme \fixme{predikát <} na množine $N$ - fixme
    \begin{itemize}
    \item[1] $(\forall x)(\forall y) ((x < y) \implies \neg (y < x))$ - fixme
    \item[2] $(\forall x)(\forall y)(\forall z) (((x<y) \land (y<z)) \implies
        (x<z))$ - fixme
    \item[3] $(\forall x)(\forall y)( (x\not=y) \implies ((x<y) \lor (y<x)))$ -
     trichotomičnosť
    \end{itemize}
\end{priklad}

\begin{priklad}{Elementárna aritmetika}
Špeciálne symboly - 0 - konštanta (nulárny funkčný symbol),
    S - nasledovník (unárny funkčný symbol),
    +,*
    S(x) = x+1

Axiómy elementárnej aritmetiky:
\begin{itemize}
    \item[1] $\neg S(x) = 0$
    \item[2] $S(x) = S(y) \implies x=y$
    \item[3] $x+0 = x$
    \item[4] $x+s(y) = s(x) + y$
    \item[5] $x * 0 = 0$
    \item[6] $x * S(y) = (x*y)+x$
\end{itemize}
Realizácia
\fixme{n caliigraphic}
$n=<N^+,0,S,+,*>$ - je to realizácia aj model. Nazýva sa aj štandardný
model. (Hovorí sa jej aj Robinsonova aritmetika)
Ak pridáme axiómu indukcie, dostaneme Peanovu aritmetiku.
\end{priklad}

\begin{poznamka}
    Ak si nebudeme všímať axiómy, ale iba relačnú štruktúru
    $n=<N^+,0,S,+,.>$. Zoberme namiesto $S$ konštantu $1$.
    Ciže $n'=<N^+,0,1,+,.>$. N' realizuje jazyk teórie telies.
    Lenže $n'$ nie je modelom tohoto jazyka. \fixme{preco?}
\end{poznamka}

\begin{priklad}[Usporiadania]
   \fixme{co je toto?:} $\varphi, x, y, z, <$
    \begin{enumerate}
        \item $(x,y) \in \varphi  \implies (y,x) \notin  \varphi$
        \item $(x,y) \in \varphi \land (y,z) \in \varphi \implies (x,z) \in \varphi$
        \item $x \neq y \implies (x,y) \in \varphi \lor (y,x) \in \varphi$
    \end{enumerate}

je modelom pre teóriu ostrého usporiadania.
Ďalšie príklady: neostré usporiadanie: antisymetrickosť, tranzitívnosť,
reflexívnosť, trichotomockosť.
\end{priklad}


\begin{priklad}[Teória grúp]
    \begin{enumerate}
            \item $((x+y)+z) = (x+(y+z))$
            \item $(x+0) = (0+x) = x$
            \item $x, -x, x+(-x) = 0 = (-x)+x$
    \end{enumerate}
\end{priklad}

\fixme{toto som uz mal}
\begin{priklad}[Elementárna aritmetika]

    $+, \cdot, 0, S(x) = x+1, N^+ $
    \begin{enumerate}
	\item $\neg S(x) = 0$
	\item $S(x) = S(y) \implies x=y$
	\item $x+0 = x$
	\item $x+S(y) = S(x+y)$
	\item $x \cdot 0 = 0$
	\item $x\cdot S(y) = ((x\cdot y) + x)$
    \end{enumerate}

    Pätica $<N^+, +, \cdot, S, 0>$ je
    \emph{štandardný model} elementárnej aritmetiky.
\end{priklad}

\par
Pojem platnej formuly je prirodzené zovšobecnenie pojmu \emph{logicky platnej
formuly}.

Ďalším cieľom je stotožniť dokázateľné formuly s tautológiami.

\begin{veta}[o korektnosti]
    Ak $T$ je teória v jazyku $L$ a ak formula $A$ je taká,
    že $T \provable A$, potom $T \models A$.
\end{veta}

\begin{dokaz}
    Nech $A_1, A_2, \ldots A_n$ je odvodenie (dôkaz) formuly $A$
    z predpokladov $T$ (v teórii $T$). $A_1, A_2, \ldots A_n(A)$. \fixme{}
    \par $m$ nech je ľubovoľný model teórie $T$.
    Ukážeme (indukciou podľa dĺžky dôkazu), že platí $m \models A_i$
    pri predpoklade, že pre $A_j, j < i$ platí $m \models A_j$.

    Do dôkazu sa $A_i$ môže dostať niekoľkými spôsobmi:
    \begin{enumerate}
	\item $A_i \in T$, m -- model $T$, potom $m \models A_i$ a teda
                $T \models A_i$
	\item $A_i$ sa dostane do dôkazu ako axióma predikátovej logiky:
	\begin{enumerate}
            \item $A_i$ je axioma výrokovej logiky -- je poskladaná z
            atomických formúl a logických spojok $\neg$ a $\implies$. Potom
            $A_i$ je tautológia výrokovej logiky (ak formula je tautológia,
            jej pravdivostná hodnota nezávisi od ohodnotenia premenných, a
            $m \models A_i$, teda $m \models A_i[e]$).
            
            \item $A_i$ je axioma špecifikácie, teda je tvaru $A_i: (\forall
            x) B \implies B_x[t]$, $t$ je substitúcia za $x$ do $B$. $m
            \models A_i$.
            \par
            Zaujíma nás prípad, kedy $(\forall x) B$ je pravdivý.\footnote{
                v opačnom prípade implikácia triviálne platí}
            To znamená, že pre ľubovoľné o\fixme{individuum m}hodnotenie platí $B[e(x/m)]$, teda
            $e(x)=m$. Tvrdenie zo zimného semestra: $$m \models A_{x_1,
            \ldots x_n}[t_1, \ldots t_n][e] \Leftrightarrow m \models
            A[e(x_1/m_1, \ldots x_n/m_n)]$$ a $$t_i[e] = m_i.$$ 

            Ak položím tvrdenie $B_x[t][e]$, zmení sa to na $B[e(x/m)]$ a
            táto formula je pravdivá v $m \models B[e(x/m)]$.

            \item $A_i: (\forall x) (B \implies C) \implies (B \implies
            (\forall x) C)$ a $x$ nie je voľná v $B$. Mali by sme dokázať,
            že platí $m \models A_i$. Formula  $m \models (B \implies
            C)[e(x/m)]$ platí, pozeráme sa na $(B \implies C)$. To je
            ekvivalentné s $\neg B \lor C$. Dôležitý je tiež predpoklad, že
            $x$ nie je voľná v $B$, a teda nezávisí od ohodnotenia viazanej
            premennej. Keď je ľavá časť axiomy pravdivá, musí byť aj pravá
            \todo{spresniť!} $m \models A_i$.
	\end{enumerate}
	\item $A_i$ je niektorá axioma rovnosti:
	\begin{enumerate}
            \item $A_i: x=x$, potom $m\models x=x$, $A_i[e]$ a $m=m$.
            \item $A_i: x_1 = y_1 \implies x_2 = y_2 \implies \ldots \implies
            f(x_1, \ldots, x_n) = f(y_1, \ldots, y_n).$. Zaujíma nás prípad,
            keď  $e(x_i) = e(y_i)$, teda
            $e(x_i/m_i)=e(y_i/m_i)$.\fixme{mi?}
            \fixme{Dostavame m1=m1 -> m2=m2 ... fm(fm,...)=f()}
            Získavame potom zobrazenie na
            $f_m(m_1, \ldots m_n)$ a rovnosť platí.
            \item $A_i: x_1 = y_1 \implies x_2 = y_2 \implies \ldots \implies
            P(x_1, \ldots, x_n) = P(y_1, \ldots, y_n).$. Zaujíma nás opäť
            prípad, keď $x_i$ aj $y_i$ reprezentujeme rovnakým indivíduom:
            $e(x_i/m_i)$ a $e(y_i/m_i)$.  $P_m(m_1, \ldots, m_n)$, $(m_1,
            \ldots m_n) \in P$ -- buď je v relácii, alebo nie je.
	\end{enumerate}
	\item Odvodzovacie pravidlá, $A_i: A_{j_1}, A_{j_2}$, pričom $j_1 < i$ a
	$j_2 < i$.
	\begin{enumerate}
            \item $A_{j_2}. A_{j_1} \implies A_j$, $A_{j_1}$ a $m \models
            A_{j_1}$, $m \models A_{j_2}$, potom $m \models A_i$.
            \item $A_i: \forall A_j$, potom $m \models A_j[e(x/m)]$ a $m
            \models A_j$.
	\end{enumerate}

\end{enumerate}

\end{dokaz}

\begin{priklad}
    Uvažujme elementárnu aritmetiku, ktorá má svoj štandardný
    model. Uvažujme formulu $x=0$ v $N$. 

    \begin{itemize}
        \item Nech $e(x) = m \neq 0$. Potom formula $A: x=0$ nie je
        splnená pre ohodnotenie $e$ a teda $m \notmodels A[e]$.
        \item Nech $e(c) = 0$. Potom formula $A': \neg x=0$ je je
        splnená v ohodnotení $e$, t.j. $m \notmodels A'[e]$.
    \end{itemize}
    To ale znemaná, že formula $A$ ani jej negácia
    nie sú dokázateľné v elementárnej aritmetike.
\end{priklad}


\paragraph{Poznámka:} Vetu o dedukcii v predikátovej logike nemožno vysloviť pre
ľubovoľnú formulu.

$$ 
\begin{array}{ll}
	A: & \neg x=0 \\
	A: & \neg y=0 \\
\end{array}
$$

$$
	\neg x=0 \provable \neg y=0
$$

Môžem podľa vety o dedukcii napísať, že $ \provable \neg x=0 \implies y=0$?

$$
\begin{array}{l}
	A \implies B \\
	e:	\\
	e(x)=0	\\
	e(x)= m \neq 0	\\
	m \notmodels (A \implies B)[e]	\\
\end{array}
$$

\paragraph{Dôsledok:} Ak teória $T$ v jazyku $L$ má model $m$, potom $T$ je
bezposporná.
\paragraph{Dôkaz:} Nech $m$ je model $T$ a $A$ je ľubovoľná uzavretá formula
jazyka $L$. Potom práve jedna z formúl $A$, $\neg A$ je pravdivá v modeli $m$.
Tá, ktorá nie je pravdivá, nie je ani dokázateľná (podľa vety o korektnosti).

\paragraph{}
\par Tento výsledok hovorí, ze ak máme vyšetriť bezospornosť nejakej teórie,
treba nájsť jej model. Keď si zoberieme predikátovú logiku:

\par Každá realizácia jazyka predikátovej logiky je modelom (neskôr nám bude
stačiť zobrať model, ktorý má jedno jediné indivíduum).  

\par
Syntaktický prístup ku dôkazu bezospornosti: Teóriu, ktorej bezospornosť chceme
dokázať, transformuje pomocou istých pravidiel do teórie, o ktorej vieme, že je
bezsosporná  (my budeme predikátovú logiku transformovať do výrokovej logiky).

\par $L$ -- jazyk prvého rádu, $c$ -- konštanta taká, že $c \notin L$. $L'$
vznikne z $L$ pridaním $c$. Ľubovoľný term nahradíme konštantou $c$, z formuly
vynechávame všetky kvantifikátory, vynechávame premenné bezprostredne spojené s
kvantifikátorom.

\par Každej formuli $A$ na jazyku $L$ priradíme formulu $A^*$ na jazyku $L'$:
\begin{enumerate}
	\item ak A je tvaru $A: P(t_1, \cdots t_n)$, tak $A^*: P(c, c, \cdots,c)$.
	\item $A: B \square C$, potom $A^*: B^* \square C^*$.
	\item $A: \neg B$, potom $A^*: \neg B^*$
	\item $A: (Qx) B$, potom $A^*: B^*$.
\end{enumerate}

Ak $L$ je jazyk bez rovnosti a $\provable^L A$, potom $A^*$ je tautológia. Ak $L$
je jazyk s rovnosťou a $\provable^L A$, potom $A^*$ je tautologický dôsledok $c=c$.

\par Tvrdíme, že pre žiadnu formulu $A$ nie je $\provable A$ aj $\vdash \neg A$. Ak
by to platilo, dostali by sme sa do sporu, že vo výrokovej logike je $\provable
A^*$ aj $\provable \neg A^*$.
