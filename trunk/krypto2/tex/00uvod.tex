\section{0. prednáška - Ako (ne)šifrovať disky}

V decembri 2009 bola nájdená bezpečnostná chyba
v niektorých šifrovaných USB diskoch
(Kingston DataTraveler BlackBox, SanDisk Cruzer Enterprise FIPS Edition a
Verbatim Corporate Secure FIPS Edition).
Všetci traja výrobcovia uvádzajú, že disky spĺňajú bezpečnostný 
štandard FIPS 140-2.

Zistilo sa, že disky vnútorne používajú úplne rovnaký systém zabezpečenia,
ktorý vyzerá nasledovne:
\begin{itemize}
    \item Používateľ zadá heslo k disku.
    \item Heslo za pretransformuje cez MD5 hash a prvá polovica
        výslednej hash hodnoty sa použije ako kľúč $K$.
    \item Následne sa pomocou AES-256 a kľúča $K$ odšifruje
        prvých 32 bajtov z disku (označme ich $X$).
        Obslužný program potom zistí, či $D_K(X)=C$,
        kde $C$ je pevne známa konštanta
        (dokonca u všetkých výrobcov rovnaká).
        Ak overenie sedí, tak sa disk odomkne a dáta sa sprístupnia.
        Ak nie, tak sa požiadavka zamietne.
        Za pozornosť stojí najmä fakt, že dešifrovanie ostatných dát
        \emph{nezávisí} od hesla.
\end{itemize}

Útok na tento systém je vcelku jednoduchý.
Stačí v pamäti prepísať výsledok dešifrovacej transformácie tesne
predým, ako sa overí či je zhodná so známou konštantou.

Pre čitateľov, ktorý by sa chceli viac ponoriť do tejto tematiky
ponúkame odkaz na krátku správu o tomto bezpečnostnom probléme
\cite{drive_enc} a výrazne dlhší report, ktorý zohľadňuje rôzne útoky
na usb flash disky
\cite{usb_flash_enc}.


\begin{figure}[htp]
    \centering
    \includegraphics[scale=1]{img/00/extern-drive-encryption.1.mps}
    \label{fig:extern_drive_encryption}
    \caption{``Šifrovanie'' externého disku}
\end{figure}

