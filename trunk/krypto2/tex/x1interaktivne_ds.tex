\section{Interaktívne dokazovacie systémy}

V tejto časti sa budeme venovať dokazovacím systémom. Pôjde o akýsi
typ spoločného výpočtu dvoch účastníkov - jedného výpočtovo
neobmedzeného provera $P$ a výpočtovo obmedzeného overovateľa $V$.
Cieľom provera je akýmsi spôsobom presvedčiť overovateľa o znalosti
nejakého faktu.
Formálne,
interaktívnym dokazovacím systémom (IDS) nazveme dvojicu
$\langle P,V \rangle$, kde $P$ je pravdepodobnostný TS
s neobmedzenou výpočtovou silou,
$V$ je pravdepodobnostný TS pracujúci v polynomiálnom čase.
Oba stroje zdieľajú spoločný vstup $x$, môžu počas svojho výpočtu
komunikovať a o akceptovaní resp. zamietaní vstupu $x$ rozhoduje iba
$V$.
IDS pre jazyk $L$ je dvojica $\langle P,V \rangle$ pre ktorú platí
\begin{itemize}
\item {\bf úplnosť} -- $\forall x \in L:
    Pr[V\textit{ akceptuje } x \textit{ v systéme }
        \langle P,V \rangle ] \ge 2/3$
\item {\bf korektnosť} -- $\forall P^*: \forall x \not \in L:
    Pr[V\textit{ akceptuje } x \textit{ v systéme }
        \langle P^*,V \rangle ] \le 1/3$
\end{itemize}
Prvá podmienka hovorí o tom, že ak $x\in L$, dokazovateľ s veľkou
pravdepodobnosťou presvedčí overovateľa o správnosti.
Naopak, korektnosť tvrdí, že ľubovoľný (podvodný) dokazovateľ
presvedčí overovateľa na zlom vstupe len s nízkou pravdepodobnosťou.

\begin{poznamka}
    Pre $L \in P$ je jednoduché navrhnúť IDS. Overovateľ bude ignorovať
    komunikáciu a môže si vypočítať príslušnosť slova sám.
    Pre $L \in NP$ je jednoduché navrhnúť IDS posielajúci práve jednu
    správu -- konkrétny dôkaz, či výpočet NTS pre problém L.
\end{poznamka}

\begin{priklad}
    Uvažujme problém $GNI \not \in NP$ -- problém grafového neizomorfizmu.
    Vstup pozostáva zo zápisu dvoch grafov $G_0, G_1$, akceptovať chceme, keď
    dané dva grafy nie sú izomorfné. Môžeme použiť nasledovný protokol
    pri dôkaze: Uvažujme $k$ kôl, v každom z nich prebehne takáto
    komunikácia:
    %% FIXME: preco 'P \send V' je vizualne dlhsie ako 'V \send P'?
    \begin{itemize}
        \item $V$ si zvolí $i \inr \{0,1\}$ a permutáciu
         $\pi \inr perm(|G_0|)$
        \item $V \send P: H = \pi(G_i)$.
        \item $P \send V: i'$ reprezentujúce graf $G$, s ktorým je $H$
        izomorfný ($V$ je neobedzene výpočtovo silný).
        \item $V$ zamietne vstup ak $i \not = i'$.
    \end{itemize}
    Po $k$ úspešných kolách $V$ akceptuje.

    Ak $G_0 \isomorph G_1$, tak $P$ má v každom kole šancu 50\% na
    uhádnutie indexu $i$, ktorý si vymyslí $V$. 
    Pravdepodobnosť akceptovania po $k$ kolách je teda $2^{-k}$.
    Naopak, ak $G_0 \not \isomorph G_1$, tak čestný dokazovateľ vie
    vždy odlíšiť permutáciu $G_0$ a $G_1$,
    čize akceptujeme s pravdepodobnosťou 1.
\end{priklad}

Pre interaktívne dokazovacie systémy sa dá dokázať mnoho zaujímavých
vlastností. Napríklad, že $IP=PSPACE$,
čiže inak povedané, čokoľvek čo
vieme robiť v polynomiálnom priestore vieme robiť interaktívnym
dokazovacím systémom. O tom, ako na to je písané napr. v
\cite{ip-pspace}. Aby sme neostali staromódni, nedávnym dôkazom
(August 2009) bolo $QIP=PSPACE$ \cite{qip-pspace}, čiže to,
že kvantové počítače nepomôžu sile interaktívnych dôkazov.
Dôkaz je veľmi technický a využíva viacero rôznych redukcií a známych
rovností tried zložitosti. Odporúčame si ho prečítať hlavne ak má
čitateľ pocit, že kvantovým výpočtom aspoň trochu chápe.

Ďalšou možnosťou, kam môžeme rozvíjať interaktívne dokazovacie systémy
sú takzvané MIP -- multiprover IP. Pri týchto akoby overovaťel mohol
krížovo vyslúchať dokazovateľov, ktorí so sebou nemôžu komunikovať.
