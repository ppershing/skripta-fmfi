\section{Dúhové tabuľky}

\begin{poznamka}
    Túto kapitolu písal USAmec z vlastných znalostí a zdrojov
    nezávisle od prednášky docenta Staneka.
\end{poznamka}

Hacker Ivan Ivanovič získal prístup k zašifrovanému heslu šéfa FBI.
Bolo uskladnené v tvare $H(password)$, kde $H$ je nejaká hašovacia funkcia.
Ako správny hacker sa ho rozhodol prelomiť a získať tak prístup
ku všetkým možným i nemožným tajným údajom.
Postupne vygeneroval všetky heslá dĺžky 1, potom dĺžky 2, \dots a
zahešoval ich.
Ak predpokladáme, že heslo malo $n$ bitov,
tak mu to zabralo nakoniec čas $O(2^n)$,
čo bol asi rok. Nakoniec mu ale bolo nanič, lebo pôvodné heslo bolo už
dávno zmenené.

Ale pamäť neuvyužil skoro vôbec.
Sklamaný svojím neúspechom sa rozhodol, že túto chybu napraví.
Rozhodol sa teda predrátať si zahešovanú hodnotu hesla
pre všetky rôzne heslá do dĺžky $n$ v nádeji,
že keď následne uloví nejaké heslo, tak ho rozšifruje skoro hneď.

Nuže sa pustil do rátania. Čoskoro sa mu ale zaplnil prvý disk svojou tabuľkou.
Tak si kúpil ďalších 10. A potom ďalších 100. A ďalšie a ďalšie.
Takto čoskoro vykúpil disky z celého Ruska
a stále nemal uložené všetko čo treba.

Pozorný čitateľ si isto všimol, že keď dostaneme nejaké heslo na rozbitie,
tak čas bude $O(1)$, ale pamäť máme $O(2^n)$.
Prirodzená otázka znie (ak nechceme dopadnúť ako Ivan),
či neexistuje nejaký kompromis medzi týmito dvoma extrémami
(obetujeme trochu času, aby sme mohli zabrať menej pamäte).
Ukazuje sa, že áno. Ako na to sa dozviete v ďalšom texte.

\subsection{Jednoduchý time-memory tradeoff}

Toto riešenie navrhol pôvodne Martin Hellman (\cite{hellman1980}).
Uvažujme dvojice $\langle x_i, H(x_i) \rangle$.
Našim cieľom je ich uskladniť tak, aby sme si nemuseli
pamätať každú hodnotu z každej dvojice. 
Základná myšlienka bude zostrojiť niečo ako ``reťaz'' hodnôt, ktorá
bude deterministicky určená a bude nám stačiť si zapamätať poslednú
hodnotu.

Zatiaľ ale máme len funkciu, ktorá nám z hodnoty $x$ vyrobí $H(x)$.
Chcelo by to ešte funkciu, ktorá z hodnoty $H(x)$ vyrobí $y$,
ktoré môžeme opäť hašovať.
Toto môže byť veľmi jednoduchá funkcia,
ktorá zobrazí množinu hashov na množinu hesiel (alebo iných pôvodných hodnôt).
Napríklad keď chceme iba heslá z písmen malej anglickej abecedy,
môžeme hash zapísať v sústave so základom 26 a dostaneme tak nového
kandidáta na heslo.
Označme túto funkciu ako $R$ (redukčná funkcia).
Pomocou nej a pôvodnej hashovacej funkcie vieme vyrobiť nasledovnú reťaz:
\begin{equation*}
    x_{1,0} \toa{H} H(x_{1,0}) \toa{R} 
    x_{1,1} \toa{H} H(x_{1,1}) \toa{R} \dots \toa{R} 
    x_{1,t-1} \toa{H} H(x_{1,t-1})
\end{equation*}

\noindent
Jedna reťaz nám ale nebude stačiť a preto si ich vyrobíme rovno $m$.
\begin{equation*}
    x_{j,0} \toa{H} H(x_{j,0}) \toa{R}
    x_{j,1} \toa{H} H(x_{j,1}) \toa{R} \dots \toa{R}
    x_{j,t-1} \toa{H} H(x_{j,t-1})
    \quad \forall j \in \{1,\dots,m\}
\end{equation*}

To ako zvoliť vhodné $m, t$ si ukážeme neskôr.
Dôležité je, že z každej reťaze si zapamätáme jej začiatok a koniec,
teda dvojice
$\langle x_{j,0}, H(x_{j,t-1})\rangle$ pre $j \in \{1,\dots,m\}$.
Tieto dvojice následne utriedime podľa koncov,
aby sme podľa nich vedeli rýchlo nájsť patričný začiatok.

\medskip
Teraz, keď nám príde otázka \quoteme{Nájdi $y$ také, že $H(y)=h$}.
Najprv sa pozrieme, či priamo nie je $h$ medzi koncami.
Ak áno, tak zoberieme začiatok $z_0$ a postupne počítame hodnoty
$z_{i+1} = R(H(z_i))$.
A pozrieme sa, či hodnota $z_{t-1}$ je vyhovujúca.
Ak áno, máme výsledok. Ak nie máme falošný poplach.

V prípade, že $h$ nebolo medzi koncami, skúsime $H(R(h))$
a opäť sa pozrieme do tabuľky koncov.
A opakujeme ten istý postup akurát sa pozeráme na hodnotu $z_{t-2}$.
A takto opäť posúvame hodnotu $h$ a skúšame.

Samozrejme môže sa stať, že vôbec vhodnú hodnoty nenájdeme.
Predtým než ale spočítame šancu na úspech sa pozrime
na niektoré vlastnosti takejto tabuľky.
To, čo určite môže nastať sú kolízie medzi reťazami.
Dokonca aj také tie nepríjené, keď napr: $x_{27} = x_{42}$.
Tieto ani nevieme detekovať podľa zhodnosti koncov reťazcov.
Navyše môže nastať aj situácia kedy sa nám reťaz zacyklí. 

Teraz poďme spočítať pravdepodobnosť úspechu.
Nech máme $N$ rôznych môžných hashov a predpokladajme, že
funkcia $R(H(\cdot))$ sa správa ako náhodné orákulum.
Potom šanca na úspech je $Pr[uspech] = \frac{C}{X}$,
kde $C$ je počet všetkých rôznych haší v tabuľke.
Zjavne $E \leq mt$.
Ale tento odhad je dosť voľný a pri rozumne veľkých hodnotách je $C$ 
o dosť menšie ako $mt$. 
Poďme spraviť nejaký odhad zdola.
Spočítame očakávanú hodnotu $E(C)$.
\begin{align*}
    C &= \sum_{i=1}^m \sum_{j=0}^{t-1} [x_{i,j} ~\text{je nové}] \\
    E(C) &=\sum_{i=1}^m \sum_{j=0}^{t-1} Pr[x_{i,j} ~\text{je nové}] 
\end{align*}

Spočítať presne šancu nato, že $x_{i,j}$ je nový prvok 
(teda predtým sme ho nemali) je dosť netriviálne,
už len preto, lebo to napríklad zavisí aj od toho, či $x_{i,j-1}$ je nové.
Ešte si ako $X_{i,j}$ označíme \quoteme{$x_{i,j} ~\text{je nové}$}.
Teraz spravíme dolný odhad 
$Pr[X_{i,j}] \geq Pr[X_{i,0}\land X_{i,1} \land \dots \land X_{i,j}]$
a na tento výraz poštveme podmienenú pravdepodobnosť:
\begin{equation*}
    Pr[X_{i,0} \land X_{i,1} \land \dots \land X_{i,j}] = 
    Pr[X_{i,j} | X_{i,0} \land X_{i,1} \land \dots \land X_{i,j-1} ]
    \cdot
    Pr[X_{i,0} \land X_{i,1} \land \dots \land X_{i,j-1} ]
\end{equation*}
A toto zopakujeme patričný počet krát a máme:
\begin{equation*}
\begin{split}
    Pr[X_{i,0} \land X_{i,1} \land \dots \land X_{i,j}] =&    
    Pr[X_{i,0}] \cdot Pr[X_{i,1} | X_{i,0}] \cdot 
        Pr[X_{i,2} | X_{i,0} \land X_{i,1}] \cdot \\ 
        & \ldots \cdot
        Pr[X_{i,j} | X_{i,0} \land X_{i,1} \land \dots \land X_{i,j-1}]
\end{split}    
\end{equation*}

A teraz môžeme povedať, že
$Pr[X_{i,j} | X_{i,0} \land X_{i,1} \land \dots \land X_{i,j-1}] =
    \frac{N - (i-1)t - j}{N}$
(teraz už vyberáme náhodnú hodnotu z $N - (i-1)t - j$ voľných). 

Keď toto celé dáme dokopy, máme:
\begin{equation*}
    Pr[uspech] \geq \frac{\displaystyle
        \sum_{i=1}^m \sum_{j=0}^{t-1}
            \frac{N - (i-1)t}{N} \cdot \frac{N - (i-1)t - 1}{N} \cdot \dots
            \cdot \frac{N - (i-1)t - j}{N}
        }{N}
\end{equation*}

Aby sme tam nemali taký odpudivý výraz,
tak spravíme ďalší odhad zdola:
\begin{equation*}
    Pr[uspech] \geq \frac{1}{N}
        \sum_{i=1}^m \sum_{j=0}^{t-1} \left ( \frac{N - it}{N} \right )^{j+1}
        =
        \sum_{i=1}^m  \frac{N-i t}{N} \left(1 -
        \left(\frac{N-it}{N}\right)^t\right)
\end{equation*}

Dá sa ukázať, že keď pokiaľ tlačíme $mt^2$ nad $N$, tak už veľa nezískame
(väčšina sčítancov bude veľmi malých).
Naopak pokiaľ $mt^2 \ll N$, tak zase sa dá ukázať,
že každy sčítanec je dosť blízko $1$ a teda 
$Pr[uspech] = \Theta\left(\frac{mt}{N}\right)$.

Dobré hodnoty sú preto napríklad $m=t=\sqrt[3]{N}$,
kedy dostaneme (po numerickom výpočte) približne 
$Pr[uspech] \approx \frac{0.8}{\sqrt[3]{N}}$.
Toto je v bežnej situácii dosť nanič, ale zase nič nám nebráni
použiť viacero tabuliek (napr. $n = 4 \sqrt[3]{N}$). 
Samozrejme každá tabuľka potrebuje inú redukčnú funkciu, aby boli
rozdielne.
Potom bude šanca na úspech približne
$1 - (1 - \frac{0.8}{\sqrt[3]{N}})^{4 \sqrt[3]{N}} 
  \approx 1-e^{-0.8*4} \approx 96\%$.

Čo sa týka zložitosti, tak pamäťové nároky sú $O(mn) = O(\sqrt[3]{N^2})$.
Časové nároky sú $O(tn) = O(\sqrt[3]{N^2})$,
keďže v každej tabuľke musíme spraviť $t$ krokov 
(tuná trochu zavádzame, keďže ignorujeme čas strávený na falošných poplachoch).
Nie je to síce ideálny tradeoff, kde súčin časovej a pamäťovej zložitosti 
by bol $O(N)$, ale stále lepšie ako triviálne extrémne riešenia.

\subsection{Tabuľky s vyznačnými bodmi}

Pôvodná Hellmanova verzia má jednu dosť podstatnú chybu.
Môžu nám vzniknúť napríklad takéto reťaze:
$x_0, x_1, x_2, \dots, x_t$ a $x_1, x_2, \dots, x_t, x_{t+1}$ 
(posunuté o 1 prvok). 
Takýmto neserióznym situáciam by sme chceli zabrániť.
Možné vylepšenie navrhol Rivest:
Skúsme napríklad reťaz neukončiť po $t$ prvkoch,
ale vtedy, keď prvok bude niečím vyznačný, napríklad bude končiť na
$k$ núl.
Vďaka tomu vieme, že keď dve reťaze skolidujú,
tak kolíziu vieme dosť jasne detekovať, pretože skončia v rovnakom bode.
Navyše očakávaná dĺzka reťaze je $2^k$,
takže pokiaľ sme napríklad po $10 \cdot 2^k$ krokoch nenašli koniec,
môžeme si povedať, že reťaz zahodíme, lebo sa možno cyklíme.
Ďalšou výhodou je, že pri výhľadavaní nemusíme pozerať do uloženej tabuľky
v každom kroku, ale iba keď nájdeme význačný bod.

Na druhej strane analýza tohoto typu tabuliek je veľmi nepríjemná vec,
preto sa jej nebudeme venovať. 
Asympotické výsledky sú ale rovnaké ako pri predchadzajúcom type tabuliek
-- $\sqrt[3]{N}$ tabuliek s $\sqrt[3]{N}$
reťazami a význačný bod by mal mať $\lg \sqrt[3]{N}$ núl na konci.

\subsection{Rainbow tabuľky}
Budeme pokračovať vo vylepšovaní.
Opäť chceme rozumne riešiť kolízie.
Skúsime to spraviť tak, aby keď dve reťaze skolidovali,
tak skolidujú na rovnakom mieste (vďaka tomu nebudú posunuté
a môžeme detekovať kolíziu). 
Toto riešenie navrhol Oechslin (\cite{Oechslin2003Making}).
Zoberme si sadu redukčných funkcií ${\color{red}R_1}, {\color{blue}R_2}, \dots, {\color{green}R_{t-1}}$.
A v každom kroku odvodenia reťaze použijeme inú.
Dostaneme nasledovnú reťaz:
\begin{equation*}
    x_{1,0} \toa{H} H(x_{1,0}) {\color{red} \toa{R_1} }
    x_{1,1} \toa{H} H(x_{1,1}) {\color{blue} \toa{R_2} }
    x_{1,2} \toa{H} \dots {\color{green} \toa{R_{t-1}} }
    x_{1, t-1} \toa{H} H(x_{1,t-1})
\end{equation*}

\begin{poznamka}
	Dúfam, že teraz každý chápe odkiaľ sa vzal názov dúhové tabuľky :)
\end{poznamka}

Takýchto reťazí si opäť zostavíme $m$. Ako bude prebiehať vyhľadávanie? 
Už nemôžeme použiť bežnú iteráciu typu $R(H(x))$,
keďže máme rôzne redukčné funkcie.
Takže budeme postupne skúšať hľadať stĺpec, v ktorom je naše heslo.
Ako prvý skúsime stĺpec $t-1$.
V tomto prípade stačí pozrieť rovno na konce reťazí.
Ak tam máme hľadaný hash, tak sa dopracujeme k heslu od začiatku
reťaze.
Pokiaľ si chceme vyskúšať, či naše heslo nie je v stĺpci $j$,
tak najprv spravíme $H(R_{t-1}(H(R_{t-2}(\dots H(R_j(x))\dots))))$,
aby sme sa dostali na koniec reťaze a potom, ak sme našli zhodu,
môžeme hľadať odzačiatku.
Pokiaľ zanedbáme čas prehľadávania tabuľky koncov,
tak máme celkový čas hľadania $O(t^2)$. 

To na prvý pohľad vyzerá celkom zle, ale ukáže sa,
že si môžeme dovoliť oveľa viac riadkov v tabuľke.

Poďme sa pozrieť na vlastnosti tejto tabuľky.
Keďže používame každú chvíľu inú redukčnú funkciu, tak sa nám
reťaze určite nezacyklia. Navyše pokiaľ nastane kolízia typu 
$x_{i,a} = x_{j,b}$, kde $a\neq b$, tak sa nám
reťaze zhodli iba na tomto mieste a pokračujú nezávisle.
Samozrejme pokiaľ nastane prípad $x_{i,a} = x_{j,a}$, tak
zbytok reťazí máme zhodný. Ale tento prípad vieme ľahko detekovať podľa toho,
že koncové body sú zhodné.

A teraz zostáva ešte úloha spočítať šancu na úspech.
Pozrieme sa na to z trochu iného uhla pohľadu.
Pre každý stĺpec si spočítame koľko tam máme rôznych bodov.
A potom len spočítame šancu, že zadaná hodnota sa nachádza aspoň v
jedom stĺpci.
Toto si môžeme dovoliť, keďže
kolízie v rôznych stĺpcoch nás vôbec netrápia. 
Označme $m_i$ ako počet rôznych hodnôt v $i$ stĺpci.
Čo sa týka hodnôt $x_{1,0}, x_{2,0}, \dots, x_{m,0}$,
tak tam vieme vygenerovať $m$ rôznych hodnôt.
Takže $m_0 = m$. Ešte určíme $m_{i+1}$ z $m_i$.

Máme $N$ priehradok a $m_i$ náhodných hodov.
Počet zaplnených priehradok bude $m_{i+1}$. 
Dá sa ukázať, že:
\begin{equation*}
    m_{i+1} = N \left ( 1 - \left(1 - \frac{1}{N}\right)^{m_i} \right ) 
        \approx N e^{-\frac{m_i}{N}}
\end{equation*}

A teraz ešte spočítať šancu na to, že vstup sa nachádza aspoň v 1 stĺpci:
\begin{equation*}
    Pr[uspech] = 1 - \prod_{i=0}^{t-1} \left (1 - \frac{m_i}{N} \right)
\end{equation*}

Po analýze tohoto výsledku zistíme, že si môžeme dovoliť
$t \approx \sqrt[3]{N}$ a $m \approx \sqrt[3]{N^2}$.
Dolný odhad šance na úspech je pre $N=2^{60}$ približne $50\%$.
Tu nám samozrejme opäť nikto nebráni použiť viac tabuliek na zvýšenie šancí.

\begin{poznamka}
    Spomínali sme tu možnosť zahadzovať reťaze pokiaľ skolidujú.
    Výpočet šance na úspech by bol jednoduchý 
    (každý stĺpec by mal rovnako veľa prvkov).
    A navyše vieme vypočítať aj koľko reťazí by nám ostalo 
    po vyhodení kolidujúcich.
    Bude to presne $m_{t-1}$ (toľko máme presne rôznych koncov).
\end{poznamka}

\subsection{Možná obrana}

Existuje viacero spôsobov, ako zabrániť útoku pomocou dúhových
tabuliek. Prvá z nich je používanie soli (salt-u). Ide o to, že spolu
s heslom sa zahashuje aj náhodná hodnota uložená v databáze. Táto
hodnota je iná pre každého užívateľa a tým pádom vieme zabrániť
masovému zlomeniu viacerých hesiel (a napríklad ako bonus máme aj
nemožnosť overiť zdohu dvoch hesiel). Na to, aby bol salt použiteľný,
musí ale byť dostatočne veľký. Napríklad, staré verzie operačného
systému Unix používali iba 12-bitový salt. To znamená, že útočník si
namiesto jednej tabuľky vygeneroval 4096 tabuliek a bol za vodou.
Odporúčaná veľkosť saltu sa dnes pohybuje niekde od 48 po 128 bitov.

Iná možnosť je úmyselné spomalenie výpočtu hash hodnoty. Napríklad,
keď máme salt a heslo, namiesto jedného výpočtu hash hodnoty ju
niekoľkokrát potočíme -- vždy vypočítame hash zreťazenia predchádzajúcej
hodnoty, soli a hesla. Ak spravíme takto napríklad 1000 iterácii, pre
užívateľa to bude mať stále nebadateľný efekt, pre útočníka to ale
bude znamenať 1000 krát väčšie výpočtové nároky.

Nakoniec si ešte uveďme príklad, že dúhové tabuľky nie sú iba
teoretický nástroj. Staršie verzie operačného systému Windows
používali hashovaciu funkciu zvanú LM hash (od Lan Manager hash).
Problém s jej bezpečnosťou bol hlavne v znakovej sade -- napríklad
malé písmená sa automaticky konvertovali na veľké, čím sa efektívne
redukovala veľkosť abecedy. Ďalej, heslá dlhšie ako 7 znakov sa
rozdelili na dve 7-znakové bloky a každý sa zahashoval osobitne.
Na LM hash v súčasnosti existujú rainbow tabuľky, ktoré (pri veľkosti
DVD) umožnujú zlomiť všetky heslá kratšie ako 15
znakov\footnote{Dlhšie heslá sa hashujú iným spôsobom} za pár minút.
