\section{Bezpečnosť trojitého šifrovania}

Uvažujme trojité šifrovanie pomocou dvoch kľúčov v EDE móde.
Čiže $TE_{k_1,k_2} (x) = E_{k_1}(D_{k_2}(E_{k_1}(x)))$.
Predpokladajme, že oba kľúče $k_1, k_2$ majú rovnakú veľkosť a to
menovite $n$.
V krypto I sme hovorili o tom, že takéto šifrovanie je odolné voči
útoku so znalosťou otvoreného textu (KPA).
Ukazuje sa však, že voľba iba 2 kľúčov má nepríjemne vlastnosti pri
CPA -- útokom s možnosťou voľby otvoreného textu.

Jeden taký možný útok si ukážeme:
\begin{itemize}
    \item Predpripravme si hashovaciu tabuľku
        $\langle m \mapsto \{k^{(1)}, k^{(2)}, \dots \} \rangle$,
        kde správe $m$ priradíme také hodnoty kľúčov $k^{(i)}$,
        aby platilo $E_{k^{(i)}}(m)=0$. Táto tabuľka nám bude slúžiť
        na hľadanie kľúčov $k_2$.
        Tabuľku môžeme predpripraviť napríklad tak, že pre každý možný
        kľúč $k$ vypočítame $m=D_k(0)$. Predpríprava bude zaberať čas
        $O(2^n)$.
    \item Pointou celého útoku bude, že sa začneme venovať takým
        dvojiciam otvoreného a šifrového textu, pre ktoré po prvom kroku
        trojitého šifrovania dostaneme číslo 0.
        \begin{figure}[h]
            \centering
            \includegraphics{img/08/triple.1.mps}
            \caption{Trojité šifrovanie}
            \label{fig:triple}
        \end{figure}
        Preberajme teraz všetkých kandidátov $k_1$ na prvý kľúč.
        Vypočítajme plaintex ako $p=D_{k_1}(0)$.
        Následne vypočítame šifrový text ako $c=TE_{k_1,k_2}(p)$.
        Pretože vieme $k_1$, vráťme sa o 1 krok vo výpočte šifrového
        textu, dostávame $z=D_{k_1}(c)$.
        Teraz nám ale nič nebráni pozrieť sa do našej hashovacej
        tabuľky -- chceme nájsť kľúč $k_2$, taký, že
        $D_{k_2}(0) = z$, čiže pozrieme zoznam kľúčov v hashovacej
        tabuľke pri hodnote $z$.
        Zložitosť tejto časti útoku je teda tiež $O(2^n)$ (za
        predpokladu, že lookup hashu je v $O(1)$).
        Výstupom je teda množina dvojíc potenciálnych kľúčov
        $(k_1,k_2)$, ktorými sa môže šifrovať.
        Jej očakávaná veľkosť bude $O(2^n)$.
        \begin{figure}[h]
            \centering
            \includegraphics{img/08/triple.2.mps}
            \caption{Útok na trojité šifrovanie}
            \label{fig:triple-attack}
        \end{figure}
\end{itemize}
Máme teda útok, ktorého časová zložitosť je $O(2^n)$, čo je menej ako
očakávaných $O(2^{2n})$. V praxi sa samozrejme daný útok nedá
realizovať, pokiaľ nám naša obeť nie je ochotná zašifrovať zhruba
$O(2^n)$ otvorených textov. Morálnym poučením teda bude, že trojité
šifrovanie môže byť síce jednoduchá a pomerne úspešná forma zosilnenia
šifrovania pre praktické účely, pre teoretické účely je to ale
nevhodná konštrukcia.
