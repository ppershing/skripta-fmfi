\section{Prerekvizity a označenia}

\todo{odkaz na skripta z krypto I}

V zvyšnom texte budeme dodržiavať (až na občasné výnimky) nasledujúce
označenia:
\begin{itemize}
\item $A,B$ - účastníci komunikácie, $E$ - útočník, $E(A)$ - útočník
            tváriaci sa ako účastník $A$.
\item $E(p,k), E_k(p)$ - zašifrovanie otvoreného textu $p$ pomocou kľúča $k$
\item $D(c,k), D_k(c)$ - odšifrovanie šifrového textu $c$ pomocou kľúča $k$
\item $E_A(m)$ - zašifrovanie správy $m$ pomocou verejného kľúča účastníka $A$
\item $D_A(c)$ - odšifrovanie správy $c$ pomocou súkromného kľúča účastníka $A$
\item $H(t)$ - spracovanie textu $t$ pomocou hashovacej funkcie $H$
\item $x \inr M$ - $x$ je \emph{náhodne zvolený} prvok množiny $M$
\item $\exists !$ - existuje práve jeden
\item $p(A)$ - pravdepodobnosť javu $A$
\item $p(A|B)$ - podmienená pravdepodobnosť, t.j. aká je pravdepodobnosť javu $A$, ak platí $B$
\end{itemize}
