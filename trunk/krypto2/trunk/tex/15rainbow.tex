\section{Dúhové tabuľky}

Hacker Ivan Ivanovič získal prístup k zašifrovanému heslu šéfa FBI.
Bolo uskladnené v tvare $H(password)$, kde $H$ je nejaká hašovacia funkcia.
Ako správny hacker sa ho rozhodol prelomiť a získať tak prístup
ku všetkým možným i nemožným tajným údajom.
Postupne vygeneroval všetky heslá dĺžky 1, potom dĺžky 2, \dots a
zahashoval ich.
Keďže heslo malo $n$ bitov, tak mu to zabralo nakoniec čas $O(2^n)$,
čo bol asi rok. Nakoniec mu ale bolo nanič, lebo pôvodné heslo bolo už
dávno zmenené.

Ale pamäť neuvyužil skoro vôbec.
Sklamaný svojím neúspechom sa rozhodol, že túto chybu napraví.
Rozhodol sa teda predrátať si zahašovanú hodnotu hesla
pre všetky rôzne heslá do dĺžky $n$ v nádeji,
že keď následne uloví nejaké heslo, tak ho rozšifruje skoro hneď.

Nuže sa pustil do rátania. Čoskoro sa mu ale zaplnil prvý disk svojou tabuľkou.
Tak si kúpil ďalších 10. A potom ďalších 100. A ďalšie a ďalšie.
Takto čoskoro vykúpil disky z celého Ruska
a stále nemal uložené všetko čo treba.

Pozorný čitateľ si isto všimol, že keď dostaneme na nejaké heslo na rozbitie,
tak čas bude $O(1)$, ale pamäť máme $O(2^n)$.
Prirodzená otázka znie (ak nechceme dopadnúť ako Ivan),
či neexistuje nejaký kompromis medzi týmito dvoma extrémami
(obetujeme trochu času, aby sme mohli zabrať menej pamäte).
Ukazuje sa, že áno. Ako na to sa dozviete v ďalšom texte.

\subsection{Jednoduchý time-memory tradeoff}

Uvažujme dvojice $(x_i, H(x_i))$.
Našim cieľom je ich uskladniť tak, aby sme si nemuseli
pamätať každú hodnotu z každej dvojice. 
Základná myšlienka bude zostrojiť niečo ako ``reťaz'' hodnôt, ktorá
bude deterministicky určená a bude nám stačiť si zapamätať poslednú
hodnotu.

Zatiaľ ale máme len funkciu, ktorá nám z hodnoty $x$ vyrobí $H(x)$.
Chcelo by to ešte funkciu, ktorá z hodnoty $H(x)$ vyrobí $y$,
ktorú môžeme opäť hašovať.
Toto môže byť veľmi jednoduchá funkcia (napríklad keď chceme heslá iba
z písmen malej abecedy, tak môžeme hash zapísať v sústave so základom 26),
ktorá zobrazí množinu hashov na množinu hesiel (alebo iných pôvodných hodnôt).
Označme ju $R$ (ako redukčná funkcia).
Následne vieme vyrobiť nasledovnú reťaz:
\begin{equation*}
    x_{1,0} \toa{H} H(x_{1,0}) \toa{R} 
    x_{1,1} \toa{H} H(x_{1,1}) \toa{R} \dots \toa{R} 
    x_{1,t-1} \toa{H} H(x_{1,t-1})
\end{equation*}

\noindent
Jedna reťaz nám ale nebude stačiť a preto si ich vyrobíme rovno $m$.
\begin{equation*}
    x_{j,0} \toa{H} H(x_{j,0}) \toa{R}
    x_{j,1} \toa{H} H(x_{j,1}) \toa{R} \dots \toa{R}
    x_{j,t-1} \toa{H} H(x_{j,t-1})
    \quad \forall j \in \{1,\dots,m\}
\end{equation*}

To ako zvoliť vhodné $m, t$ si ukážeme neskôr.

Dôležité je, že z každej reťaze si zapamätáme jej začiatok a koniec,
teda dvojice
$\langle x_{j,0}, H(x_{j,t-1})\rangle$ pre $j \in \{1,\dots,m\}$.
Tieto dvojice následne utriedime podľa koncov,
aby sme podľa nich vedeli rýchlo nájsť patričný začiatok.

\medskip
Teraz nám príde otázka: \clqq Nájdi $y$ také, že $H(y)=h$.\crqq
Najprv sa pozrieme, či priamo nie je $h$ medzi koncami.
Ak áno, tak zoberieme začiatok $z_0$ a postupne počítame hodnoty
$z_{i+1} = R(H(z_i))$.
A pozrieme sa, či hodnota $z_{t-1}$ je vyhovujúca.
Ak áno, máme výsledok. Ak nie máme falošný poplach.

V prípade, že $h$ nebolo medzi koncami, skúsime $H(R(h))$
a opäť sa pozrieme do tabuľky koncov.
A opakujeme ten istý postup akurát sa pozeráme na hodnotu $z_{t-2}$.
A takto opäť posúvame hodnotu $h$ a skúšame.

Samozrejme môže sa stať, že vôbec vhodnú hodnoty nenájdeme.
Predtým než ale spočítame šancu na úspech sa pozrime
na niektoré vlastnosti takejto tabuľky.
To, čo určite môže nastať sú kolízie medzi reťazami.
Dokonca aj také tie nepríjené, keď napr: $x_{27} = x_{42}$.
Tieto ani nevieme detekovať podľa zhodnosti koncov reťazcov.
Navyše môže nastať aj situácia kedy sa nám reťaz zacyklí. 

Teraz poďme spočítať pravdepodobnosť úspechu.
Nech máme $N$ rôznych môžných hashov a predpokladajme, že
funkcia $R(H(\cdot))$ sa správa ako náhodné orákulum.
Potom šanca na úspech je $Pr[uspech] = \frac{E}{X}$,
kde $E$ je počet všetkých haší čo tabuľka pokrýva.
Zjavne $E \leq mt$.
Ale tento odhad je dosť voľný a pri rozumne veľkých hodnotách je $E$ 
o dosť menšie ako $mt$. 
Poďme spraviť nejaký odhad zdola.
Spočítame očakávanú hodnotu $E$.
\fixme{Pr?}
\begin{equation*}
    E = \sum_{i=1}^m \sum_{j=0}^{t-1} [x_{i,j} ~\text{je nové}] = 
        \sum_{i=1}^m \sum_{j=0}^{t-1} Pr[x_{i,j} ~\text{je nové}] 
\end{equation*}

Spočítať presne šancu nato, že $x_{i,j}$ je nový prvok 
(teda predtým sme ho nemali) je dosť netriviálne,
už len preto, lebo to napríklad zavisí aj od toho, či $x_{i,j-1}$ je nové.
Ešte si ako $X_{ij}$ označíme $x_{ij} ~\text{je nové}$.
Preto spravíme dolný odhad:
$Pr[X_{i,j}] \geq Pr[X_{i,0}\land X_{i,1} \land \dots \land X_{i,j}]$.
Na tento výraz poštveme podmienenú pravdepodobnosť:
\begin{equation*}
    Pr[X_{i,0} \land X_{i,1} \land \dots \land X_{i,j}] = 
    Pr[X_{i,j} | X_{i,0} \land X_{i,1} \land \dots \land X_{i,j-1} ]
    \cdot
    Pr[X_{i,0} \land X_{i,1} \land \dots \land X_{i,j-1} ]
\end{equation*}
A toto zopakujeme patričný počet krát a máme:
\begin{equation*}
    Pr[X_{i,0} \land X_{i,1} \land \dots \land X_{i,j}] =
    Pr[X_{i,0}] \cdot Pr[X_{i,1} | X_{i,0}] 
    \cdot Pr[X_{i,2} | X_{i,0} \land X_{i,1}] \dots
    Pr[X_{i,j} | X_{i,0} \land X_{i,1} \land \dots \land X_{i,j-1}]
\end{equation*}

A teraz môžeme povedať, že
$Pr[X_{i,j} | X_{i,0} \land X_{i,1} \land \dots \land X_{i,j-1}] =
    \frac{N - (i-1)t - j}{N}$
(teraz už fakt vyberáme náhodnú hodnotu z $N - (i-1)t - j$ voľných). 

A toto keďže celé dáme dokopy tak máme:
\begin{equation*}
    Pr[uspech] \geq \frac{\displaystyle
        \sum_{i=1}^m \sum_{j=0}^{t-1}
            \frac{N - (i-1)t}{N} \cdot \frac{N - (i-1)t - 1}{N} \cdot \dots
            \cdot \frac{N - (i-1)t - j}{N}
        }{N}
\end{equation*}

Aby sme tam nemali taký odpudivý výraz, tak si to ešte trochu odhadneme:
\begin{equation*}
    Pr[uspech] \geq \frac{\displaystyle
        \sum_{i=1}^m \sum_{j=0}^{t-1} \left ( \frac{N - it}{N} \right )^{j+1}
        }{N}
\end{equation*}

Dá sa ukázať, že keď pokiaľ tlačíme $mt^2$ nad $N$, tak už veľa nezískame
(väčšina sčítancov bude veľmi malých).
Naopak pokiaľ $mt^2 \ll N$, tak zase sa dá ukázať,
že každy sčítanec je dosť blízko $1$ a teda 
$Pr[uspech] = \Theta\left(\frac{mt}{N}\right)$.

Preto keď použijeme $m=t=\sqrt[3]{N}$ dostaneme približne 
$Pr[uspech] = \frac{0.8}{\sqrt[3]{N}}$.
Toto je v bežnej situácii dosť nanič, ale zase nič nám nebráni
použiť viacero tabuliek (napr. $n = \sqrt[3]{N}$).
Potom bude šanca na úspech približne
$1 - (1 - \frac{1}{\sqrt[3]{N}})^{\sqrt[3]{N}} = 1-e^{-1} \doteq 70\%$.
