\section{Podpisové schémy s určeným overovateľom}

V tejto kapitole sa budeme baviť o podpisových schémach, ktoré fungujú
tak trochu podobne ako zero knowledge dôkazy -- chceli by sme vedieť
podpísať správu špeciálne pre jedného overovateľa. Navyše, úplne
ideálne by bolo, ak by nikto iný okrem neho nebol schopný overiť náš
podpis a dokonca aby overovateľ nebol schopný dokázať niekomu inému,
že sme to podpísali my.

Pretože myšlienka určeného overovateľa (designated verifier) je veľmi
blízka zero knowledge dôkazom, budú aj definície veľmi podobné.

\begin{definicia}[Designated verifier]
    Uvažujme dve komunikujúce strany -- Alicu $A$ a Boba $B$. Nech
    $P(A,B)$ je protokol pre Alicu na presvedčenie
    Boba o pravdivosti nejakého výroku $\Omega$. Môžeme hovoriť, že
    Bob je určený overovateľ, ak vie produkovať komunikácie, ktorých
    distribúcia je identická s distrubúciou komunikácii protokolu
    $P(A,B)$.
\end{definicia}

Niekedy ale ani táto situácia nestačí. Predstavme si Evu, ktorá sa
snaží zistiť niečo viac o podpise. Ak je presvedčená, že
\begin{enumerate}
    \item Bob sa zdá byť čestný, je nepravdepodobné, že by on
    vygeneroval falošný podpis.

    \item Ak je Eva predvedčená, že Bob ešte nevidel daný podpis, je
    tiež evidentné, že podpisovať musela Alica.
\end{enumerate}

V tomto prípade môžeme zaviesť striktnejšie predpoklady.

\begin{definicia}[Strong designated verifier]
    Nech $P(A,B)$ je porokol pre Alicu a Boba na ukázanie pravdivosti
    nejakého výroku $\Omega$. Hovoríme, že $P(A,B)$ je protokol so
    silne určeným overovateľom, ak ktokoľvek vie produkovať
    komunikácie, ktoré sú nerozlíšiteľné od komunikácii protokolu
    $P(A,B)$ pre všetkých s výnikou Boba (pretože Bob musí byť schopný
    overiť pravosť podpisu).
\end{definicia}

Samozrejme, celú túto definíciu môžeme zaviesť aj omnoho formálnejšie,
ako sme robili pri ostatných podpisových schémach

\begin{definicia}[Designated verifier signature scheme]
    Nech $M$ je priestor všetkých možných správ (štandardne
    $M=\{0,1\}^*$). Podpisová schéma s určeným overovateľom je
    štvorica algoritmov
    $\langle Gen, Sig, Verify, Simulate \rangle$ polynomiálnych
    algoritmov kde
    \begin{itemize}
        \item $Gen(1^n)$ je pravdepodobnostný algoritmus ktorý zo
            vstupu $1^n$ vyrobí štvoricu kľúčov
            $\langle pk_A,sk_A,pk_B,sk_B \rangle$.
            Štandardne sa kľúče pre $A$ a $B$ generujú nezávisle a
            preto je možné tento algoritmus rozdeliť na dva rôzne
            algoritmy (resp. dvakrát ten istý algoritmus).

        \item $Sig(m,sk_A,pk_B)$ je pravdepodobnostný algoritmus,
            ktorý na vstupe dostane správu, privátny kľúč Alice a
            verejný kľúč Boba a na výstupe vegeneruje podpis $\sigma$
            správy $m$.

        \item $Verify(m,\sigma,sk_B,pk_A)$ je deterministický
            algoritmus, ktorý zo správy, jej podpisu, verejného kľúča
            Alice a súkromného kľúča Boba vie rozhodnúť, či bol podpis
            $\sigma$ správy $m$ podpísaný Alicou a určený pre Boba.
            Algoritmus vypíše na vstup jeden bit $b$, ktorý je 1 práve
            vtedy, ak bol podpis korektný.

        \item $Simulate(m,pk_A,sk_B)$ je pravdepodobnostný algoritmus
            produkujúci podpisy nerozlíšiteľné od podpisov algoritmu
            $Sig(m,sk_A,pk_B)$.
            V prípade, že uvažujeme silnú verziu schémy, $Simulate$ na
            vstupe nedostáva $sk_B$ ale $pk_B$.
    \end{itemize}
    Navyše, pre každú štvoricu $\langle pk_A,sk_A,pk_B,sk_B \rangle$
    vygenerovaný algoritmom $Gen(1^n)$ a pre každú správu $m\in M$
    musí platiť
    \begin{equation*}
        Verify(m, Sig(m,sk_A,pk_B),sk_B,pk_A)=1
    \end{equation*}
\end{definicia}

Formálne si teraz môžeme definovať experiment na testovanie, či naša
podpisová schéma $\Pi$ je podpisová schému s určeným overovateľom.
Budeme na to používať algoritmus
\ref{proc:ind-dv}, skrátene IND-DV, ktorý bude fungovať v roli
rozlišovateľa.

\begin{procedure}[H]
    \caption{Indistinguishable-DesignatedVerifier($D,\Pi,n$)}
    \label{proc:ind-dv}
    $\langle pk_A,sk_A,pk_B,sk_B \rangle \assign Gen_{\Pi}(1^n)$ \;
    $m \inr M$ \;
    $b \inr \{0,1\}$ \;
    \eIf{$b==0$}{$O \assign Sig_\Pi(m,sk_A,pk_B)$}
        {$O \assign Simulate_\Pi(m,pk_A,sk_B)$}
    spusť algoritmus $D$ s prístupom ku oráklu $O$. Výsledok nech je
    $b'$ \;
    \Return $b==b'$\;
\end{procedure}

Teraz môžeme hovoriť, že schéma má vlastnost určeného overovateľa, ak
pre každého polynomiálneho rozlišovateľa platí
\begin{equation*}
    Pr[IND-DV(D,\Pi,n)=1] \le \frac{1}{2} + negl(n)
\end{equation*}

Čitateľ si iste sám vie zobšeobecniť dané definície aj na silnejšiu
verziu schémy.

\subsection{Schéma od pánov Saeednia, Kremer a Markowitch}
\todo{}

\todo{co sme mali dalej, potrebujem k tomu zosit s poznamkami}



