\section{CPA odolnosť symetrických šifier}

Majme nejaký symetrický šifrovací systém. Formálne nech:
$\Pi = \langle Gen(1^n), E_k(\cdot), D_k(\cdot) \rangle$, 
kde $Gen$ je generátor kľúča s dĺžkou $n$, 
$E_k(\cdot)$ je šifrovacia transformácia (môže byť
pravdepodobnostná a nakoniec sa ukáže, že je dobré aby aj bola)
a $D_k(\cdot)$ je dešifrovacia transformácia. Samozrejme, požadujeme, aby
systém bol korektný teda $\forall x, \forall k\colon D_k(E_k(x))=x$.

Chceme, aby systém bol odolný voči útoku s možnosťou voľby otvoreného textu.
Teda útočník má k dispozícií orákulum, ktoré mu umožňuje zašifrovať ľubovoľný
text, ktorý si vyberie. Na to, aby sme ukázali, že systém je odolný vočí 
útoku s voľbou otvoreného textu, je dobré ukázať, že útočník
nevie rozlišovať šifrové texty, teda:

\begin{definicia}
    Šifrovací systém $\Pi$ má nerozlíšiteľné šifrové texty pri $CPA$, ak platí:
    \begin{equation*}
        Pr[PrivK_{A,\Pi}^{CPA}(n) = 1] \leq \frac{1}{2} + negl(n)
    \end{equation*}

    Kde $PrivK_{A,\Pi}^{CPA}$ je experiment s útočníkom $A$ na
    šifrovacom systéme $\Pi$ definový nasledovne:
    \begin{enumerate}
        \item $K = Gen(1^n)$ -- vygenerujeme nejaký kľúč
        \item Útočníkový dáme šifrovacie orákulum,
            to si označme ako $A^{E_k(\cdot)}$. 
            
        \item Útočník vygeneruje 2 otvorené texty $m_0,m_1$ 
            (požadujeme aby $|m_0| = |m_1|$). 
        \item Zvolíme si $b \inr \{0,1\}$ a položíme $c = E_k(m_b)$.
            Hodnotu $c$ povieme útočníkovi.
        \item Útočník sa pokúsi rozlíšiť šifrový text $c$, teda snaží
            sa uhádnuť $b'$ -- číslo správy, z ktorej daný šifrový
            text vznikol.
        \item Ak $b=b{'}$, tak experiment vráti $1$, ináč $0$.
    \end{enumerate}
\end{definicia}

Všimnime si, že nijako nezakazujeme útočníkovi sa priamo spýtať na zašifrovanie $m_0$, resp. $m_1$. 
Preto $E_k$ nemôže byť deterministický. 

\subsection{Pseudonáhodné funkcie}

Podobne ako sme chceli pseudonáhodným generátorom konštruovať
postupnosť, ktorá sa čo najviac podobala na náhodnú, tak tu
chceme zostrojiť funkciu, ktorá sa čo najviac podobná na náhodnú. 
Tá má následné mnohostranné využitie, napr. pri konštrukciu symetrických šifier.

Pseudonáhodnú funkciu definujeme podobne ako pseudonáhodnú postupnosť 
(nevieme ju v pravdepodobnostnom polynomiálnom čase odlíšiť od náhodnej). 
Navyše to bude funkcia s kľúčom. 
Teda $F: \{0,1\}^* \times \{0,1\}^* \to \{0,1\}^*$. 
Pre jednoduchosť, nech dĺžky
kľúča, správy a výstupu funkcie $F$ sú rovnaké, teda $n = |k| = |x| = |F_k(x)|$.

\begin{definicia}
    Funkcia $F_k(x)$ je pseudonáhodná ak $\forall$ PPT rozlišovateľa 
        $D$ platí
    \begin{equation*}
        \Big| Pr[D^{F_k} = 1] - Pr[D^f = 1] \Big| \leq negl(n)
    \end{equation*}
    kde $k$ je náhodný kľúč $k \inr \{0,1\}^n$ a 
    $f$ je náhodná funkcia $f \inr \{ \{0,1\}^n \mapsto \{0,1\}^n\}$.
\end{definicia}

Pokiaľ máme pseudonáhodný generátor $G$,
tak vieme pomocou neho zostrojiť pseudonáhodnú funkciu nasledovne:
Nech $|G(x)|=2n$. Rozdeľme si $G$ na 2 časti, teda $G(x) = G_0(x)||G_1(x)$.
Potom môžeme definovať
\begin{equation*}
    F_k(x_1 x_2 \dots x_n) = G_{x_n}(\dots(G_{x_2}(G_{x_1}(k)))\dots)
\end{equation*}

Teraz si ukážeme ako pomocou pseudonáhodnej funkciu zostrojiť symetrickú šifru $\Pi$, ktorá má nerozlíšiteľné
šifrové texty pri CPA útoku:

\begin{procedure}[h!]
    \caption{Gen($1^k$)}
    $key \inr \{0,1\}^k$\;
    \Return $key$\;
\end{procedure}

\begin{procedure}[h!]
    \caption{Encrypt($key,m$)}
    $r \inr \{0,1\}^k$\;
    $c = F_{key}(r) \oplus m$\;
    \Return $\langle r,c \rangle$\;
\end{procedure}

\begin{procedure}[h!]
    \caption{Decrypt($key,\langle r,c\rangle$)}
    $m \assign F_{key}(r) \oplus c$.
    \Return $m$\;
\end{procedure}

\begin{veta}
$\Pi$ má nerozlíšiteľné šifrové texty pri CPA útoku, ak $F_{key}(\cdot)$ 
je pseudonáhodná funkcia.
\end{veta}

\begin{dokaz}
    Dôkaz rozdelíme na 2 kroky.

    \begin{enumerate}
    \item Zostrojíme šifrovací systém $\tilde{\Pi}$, 
        kde miesto $F_k(\cdot)$ použijeme náhodnú funkciu $f$.
        Ukážeme, že tento systém má nerozlíšiteľný šifrové texty pri CPA útoku.
        Útočník teda dostane zašifrovaný text 
        $\langle r_c, f(r_c) \oplus m_b \rangle$.

        Môžu nastať 2 situácie. Buď $r_c$ bolo použité v odpovediach 
        na útočníkové otázky alebo nie. Ak bolo, tak túto situáciu
        označme ako $Repeat$. Počet otázok označme ako $q(n)$ 
        (keďže útočník pracuje v polynomiálnom čase, 
        aj tento počet je polynomiálny). 
        Potom šanca na úspech útočníka je nasledovná:
        \begin{equation*}
            Pr[PrivK_{A,\tilde{\Pi}}^{CPA}(n)=1] = 
                Pr[PrivK_{A,\tilde{\Pi}}^{CPA}(n)=1\land Repeat]  + 
                Pr[PrivK_{A,\tilde{\Pi}}^{CPA}(n)=1 \land \neg Repeat]
        \end{equation*}

        Vieme, že platí
        \begin{equation*}
            Pr[PrivK_{A,\tilde{\Pi}}^{CPA}(n)=1\land Repeat] \leq 
            Pr[Repeat] = \frac{q(n)}{2^n} = neql(n)
        \end{equation*}
        Navyše
        \begin{equation*}
            Pr[PrivK_{A,\tilde{\Pi}}^{CPA}(n)=1\land \neg Repeat] = 
                \frac{1}{2}
        \end{equation*}
        lebo v tomto prípade útočník nevie nič a môže len tipovať.
        Celkovo máme
        \begin{equation*}
            Pr[PrivK_{A,\tilde{\Pi}}^{CPA}(n)=1] \leq \frac{1}{2} + negl(n)
        \end{equation*}
        čo sme chceli dokázať.

    \item Ukážeme, že ak $\Pi$ nie je bezpečná, 
        tak vieme rozlišovať medzi $F_k(\cdot)$ a $f$.
        Nech útočník $A$ má nezanedbateľnú pravdepodobnosť úspechu 
        $\frac{1}{2}+\epsilon(n)$. 
        Zostrojme rozlišovač $D$, ktorý bude simulovať $A$ 
        pričom pri šifrovaní použije funkciu, ktorú sa práve snaží odlíšiť. 
        $D$ vráti $1$ ak útočník uspel a $0$ ak neuspel.

        Potom platí: 
        \begin{equation*}
            Pr[D^{F_k} = 1] - Pr[D^f = 1] = 
                Pr[PrivK_{A,\Pi}^{CPA}(n)=1] - 
                Pr[PrivK_{A,\tilde{\Pi}}^{CPA}(n)=1] 
            = \epsilon(n) - negl(n)
        \end{equation*}
        čo je nezanedbateľná šanca na úspech.
    \end{enumerate}
\end{dokaz}

Podobne ako pseudonáhodnú funkciu môžeme definovať pseudonáhodnú permutáciu. Tá nesmie byť rozlíšiteľná od úplne
náhodnej permutácie. Takúto permutáciu vieme využiť priamo na budovanie symetrických šifier.
Navyše vieme zadefinovať aj silne pseudonáhodnú permutáciu, kde rozlišovač okrem permutácie dostane prístup aj
k inverznej permutácii.

\begin{priklad}
    Ako vyrobiť z pseudonáhodnej funkcie pseudonáhodnú permutáciu?
    Využijeme klasický princíp
    Feistelovskej šifry. Rozdelíme vstup permutácie na dva bloky $L_0, H_0$ 
    a kľúč na podkľúče $k_1, k_2, \dots$. Následne položíme
    $L_{i+1} = H_{i}, H_{i+1} = L_{i} \oplus F_{k_i}(H_i)$.
    Pokiaľ by sme ako výstup zobrali $L_1, H_1$, 
    tak nedostaneme náhodnú permutáciu, lebo niektoré bity sa len kopírujú.
    Pokiaľ by sme zobrali ako výstup $L_2, H_2$, 
    máme problém s tým, že pokiaľ znegovaním bit v $L_0$ sa presne ten
    istý bit zneguje aj v $L_2$.
    Dá sa však ukázať, že pokiaľ vezmeme $L_3, H_3$ tak máme 
    pseudonáhodnú permutáciu a ak vezmeme $L_4, H_4$
    tak máme dokonca silne pseudonáhodnú permutáciu.
\end{priklad}

\subsection{Bezpečnosť módov blokových šifier}

V tejto časti budeme predpokladať, že samotná šifrovacia transformácia je pseudonáhodná permutácia.
Ukážeme si, že niektoré módy majú vlastnosť nerozlíšiteľnosti pri CPA a niektoré nie.

\begin{itemize}
    \item ECB mód -- nezvnikne pseudonáhodná permutácia, 
        napr. po zašifrovaní textov $(m_0,m_0), (m_0, m_1)$ vieme rozlišovať.
    \item CBC mód -- pri použití náhodného inicializačného vektora 
        vznikne pseudonáhodná permutácia.
    \item OFB mód -- to isté ako CBC
    \item CTR mód -- funguje nasledovne: 
        Zvolíme náhodne inicializačný vektor $IV$.
        Šifrujeme nasledovne: $C_0 = IV, C_i = P_i \oplus E_k(IV + i)$.
        Ukážeme, že máme nerozlíšiteľné šifrové texty.
\end{itemize}
\begin{veta}
    Ak $E_k$ pri CTR móde je pseudonáhodná permutácia,
    tak CTR mód má nerozlíšiteľné šifrové texty pri CPA.
\end{veta}
\begin{dokaz}
    Opäť najprv ukážeme, že upravená schéma $\tilde{\Pi}$,
    kde miesto $E_k$ použijeme náhodnú permutáciu $f$
    má nerozlíšiteľné šifrové texty. 
    Predpokladajme, že všetky texty majú najviac $q(n)$ blokov 
    a útočník položí najviac $q(n)$ otázok.
    Označme countre použité v odpovediach na útočníkové otázky ako 
    $c^1, c^2, \dots, c^{q(n)}$. 
    Útočník vie rozlišovať správy vtedy, ak 
    $\exists i, \exists j \colon i\ne j \land  |c^i - c^j| \leq q(n)$ 
    (vtedy keď sa nám veci vstupujúce do $E_k$ \clqq prekrývajú\crqq).
    Túto situáciu nazvime Overlap.
    Potom šanca útočníka na úspech je:
    \begin{equation*}
        Pr[X_{\tilde{\Pi}} = 1] = 
            Pr[X_{\tilde{\Pi}} = 1 \land Overlap] + 
            Pr[X_{\tilde{\Pi}} = 1 \land \neg Overlap]
    \end{equation*}
    Pritom platí
    \begin{equation*}
        Pr[X_{\tilde{\Pi}} = 1 \land Overlap] \leq 
            \sum_{i=1}^{q(n)} \frac{2q(n)}{2^n} = \frac{2q(n)^2}{2^n}
    \end{equation*}
    kedže voči každej otázke (ktorých je $q(n)$) máme $2q(n)$ miest, 
    kde môže začínať náš IV tak, aby vznikol prienik.
    Zároveň platí
    \begin{equation*}
        Pr[X_{\tilde{\Pi}} = 1 \land \neg Overlap] = \frac{1}{2}
    \end{equation*}
    lebo v tomto prípade opäť nevieme povedať nič.
    A teda celkovo platí
    \begin{equation*}
        Pr[X_{\tilde{\Pi}} = 1] \leq \frac{1}{2} + \frac{2q(n)^2}{2^n} = 
            \frac12 + negl(n)
    \end{equation*}

    Zbytok dôkazu je presne rovnaký ako pri konštrukcii 
    zo pseudonáhodnej funkcie.
\end{dokaz}

\subsection{CCA odolnosť}
Ukazuje sa, že pokiaľ chceme mať CCA odolnosť voči útočníkom,
tak bežné módy sú nedostačujúce.
Treba ešte pridať kontrolu integrity správy 
(aby si útočník nevedel len tak správu vyskladať po častiach).
