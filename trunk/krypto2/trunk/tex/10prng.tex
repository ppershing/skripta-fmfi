\section{Generátory pseudonáhodných čísel}

Cieľom pseudonáhodných generátorov je vytvárať postupnosti čísel, ktoré sa
navonok javia ako náhodné, hoci v skutočnosti to tak nie je. Dôvodov prečo
by sme chceli niečo takéko je niekoľko, najdôležitejšími sú neprístupnosť
náhodných generátorov a/alebo ich slabá výkonnosť. Ak máme totiž naozaj
náhodný generátor, ktorý ale generuje len 1 bit za sekundu, na
vygenerovanie RSA kľúča by sme potrebovali čakať vyše hodinu.

To, že (pseudo)náhodné čísla potrebujeme v kryptografii je evidentné --
stačí si spomenúť na kryptografické kľúče, príležitostné slová,
inicializačné vektory blokových šifier, náhodné prvky v asymetrických
šifrovacích schémach a schémach na digitálne podpisy, náhodných paddingoch.
Navyše, v mnohých z týchto aplikácii nenáhodnosť ohrozuje bezpečnosť (od
možnosti dešifrovať správu ako pri \todo{} až po úplné prezradenie
súkromného kľúča ako pri \todo{}).

My za preudonáhodný generátor budeme považovať deterministický\footnote{
čo samozrejme priamo znamená, že nenáhodný} algoritmus, ktorý z
počiatočného ``seedu'' vygeneruje dlhú postupnosť. Bez ujmy na strate
všeobecnosti, budeme sa venovať len pseudonáhodným generátorom bitov, čiže
generátorom do postupnosti zloženej z prvkov $\{0,1\}$.

\todo{Fyzikálne generátory}


