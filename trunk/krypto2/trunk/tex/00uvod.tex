\section{0. prednáška - Ako (ne)šifrovať disky}

V decembri 2009 bola nájdená bezpečnostná chyba v niektorých šifrovaných USB diskoch
(Kingston DataTraveler BlackBox, SanDisk Cruzer Enterprise FIPS Edition a
Verbatim Corporate Secure FIPS Edition). Všetky výrobcovia uvádzajú, že disky
spĺňajú bezpečnostný štandart FIPS 140-2 a používajú úplne rovnaký systém zabezpečenia,
ktorý vyzerá nasledovne:
\begin{itemize}
\item Používateľ zadá disku heslo.
\item Heslo za pretransformuje cez MD5 hash a prvá polovica výslednej hashe sa použije ako kľúč K.
\item Následne sa pomocou AES-256 a kľúča K odšifruje daných 32 bajtov z disku (označme ich $X$). Potom zistí, či
$D_K(X)=C$, kde $C$ je pevne známa konštanta (u všetkých výrobcov dokonca rovnaká). Ak áno, tak sa disk odomkne a dáta sa sprístupnia.
Ak nie, tak sa požiadavka zamietne. Dešifrovanie ostatných dát nezávisí od hesla.
\end{itemize}

\begin{figure}[htp]
    \centering
    \includegraphics[scale=1]{img/00/extern-drive-encryption.1.mps}
    \label{fig:extern_drive_encryption}
    \caption{Šifrovanie externého disku}
\end{figure}

Útok na tento systém je vcelku jednoduchý. Stačí v pamäti prepísať výsledok dešifrovacej transformácie. 

%Viac na:
%\url{http://www.h-online.com/security/news/item/NIST-certified-USB-Flash-drives-with-hardware-encryption-cracked-895308.html}

%A ešte na (pekny dokument nie priamo suvisiaci):
%Investigating 'secure'USB stickspsu.edu [PDF]
%PJ Bakker… - Citeseer
%\url{http://citeseerx.ist.psu.edu/viewdoc/download?doi=10.1.1.84.2539&rep=rep1&type=pdf}

