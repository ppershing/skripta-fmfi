\subsection{PSS - Probabilistic signature scheme}

PSS je dokázateľne bezpečná (za istých predpokladov) schéma na
digitálne podpisy. Navrhli ju páni Mihir Bellare a Philip Rogaway
v \cite{pss}. Celá schéma je vlastne akýmsi znáhodnením hashovania - k
správe sa pridá náhodný reťazec dĺžky \todo{} a hash sa počíta až
potom. Samozrejme, pri overovaní hashu treba nejakým spôsobom
vyriešiť, aby sme sa dozvedeli použitý náhodný reťazec. Ako sa ukáže
neskôr, toto nie je až taký veľký problém ak použijeme ďalšiu
hashovaciu funkciu.

PSS má oproti doteraz spomínaným schémam jednu veľkú výhodu - pri
dôkaze jej bezpečnosti sa ukáže "tesná" hranica. T.j., možnosť
prelomiť PSS s pravdepodobnosťou $\eps$ priamo umožňuje lámanie RSA s
rovnakou pravdepodobnosťou.

\todo{dopisat}
