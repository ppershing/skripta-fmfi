
%% Template by Michal Forisek


\documentclass[a4paper]{report}
\usepackage{slovak}
\usepackage[utf8]{inputenc}
\usepackage{a4wide}
\usepackage{tabularx}
\usepackage{amsfonts}
\usepackage{amssymb}
\usepackage{amsmath}
\usepackage{epsfig}
\usepackage{color}
\usepackage{mathrsfs}
\usepackage{verbatim}
\usepackage{hyperref}
\usepackage{algorithm2e}
\usepackage{subfigure}
\usepackage{float}

\def\todo#1{[{\color{red} TODO:} {\bf  #1}]}
\def\fixme#1{[{\color{red} FIXME:} {\bf  #1}]}
\def\verify#1{\todo{verify: #1}}

\renewcommand{\implies}{\rightarrow}
\newcommand{\notmodels}{\nvDash}
\newcommand{\union}{\cup}
\newcommand{\provable}{\vdash}
\newcommand{\unprovable}{\nvdash}

\def\noheader{\relax}

\newtheorem{definicia}{Definícia}[section]
\newtheorem{HLPpriklad}{Príklad}[section]
\newenvironment{priklad}[1][]{
    \ifthenelse{\equal{#1}{}}{
        \begin{HLPpriklad}
    }{
        \begin{HLPpriklad}[#1]
    }
    \rm}{\end{HLPpriklad}
}
\newtheorem{veta}{Veta}[section]
\newtheorem{lema}{Lema}[section]


\newenvironment{dokaz}{\trivlist
  \item[\hskip \labelsep{\bfseries Dôkaz:}]}{\endtrivlist}


\begin{document}

\thispagestyle{empty}
\begin{minipage}{0.25\textwidth}
\includegraphics[width=0.9\textwidth]{img/komlogo-new}
\end{minipage}
\begin{minipage}{0.69\textwidth}
\begin{center}
\sc Katedra Informatiky \\
Fakulta Matematiky, Fyziky a Informatiky \\
Univerzita Komenského, Bratislava
\end{center}
\end{minipage}

\vfill
\begin{center}
\begin{minipage}{0.8\textwidth}
\hrule
\bigskip\bigskip
\centerline{\LARGE\sc Krypto II}
\smallskip
\centerline{(spísané poznámky, draft)}
\bigskip
\bigskip
\centerline{\large\sc Vladimír Boža, Peter Perešíni}
\bigskip
\centerline{\large\sc (prednášal RNDr. Martin Stanek, PhD.)}
\bigskip\bigskip
\hrule
\end{minipage}
\end{center}
\vfill
{~}
\hfill verzia zo dňa {\bf\today} 
\eject % EOP i

\section*{Úvod a disclaimer}

Tieto poznámky obsahujú študijné materiály k predmetu 
\emph{Kryptológia II}
na Fakulte matematiky, fyziky a informatiky UK.
Základná verzia bola spísaná podľa prednášky RNDr. Martina Staneka v
roku 2010. Poznámky však nie sú oficiálny študijný materiál, preto
autori neručia za ich aktuálnosť a vhodnosť na štúdium. Navyše, obsah
prednášky sa môže z roka na rok meniť, a preto je odporúčané dávať
pozor na prípadné rozdiely a dopísať si časti nepokryté týmito
poznámkami.

Aby sme umožnoli jednoduchšie spravovanie a udržali poznámky dlhšie
aktuálne, rozhodli sme sa verejne publikovať zdrojové kódy na stránke
\url{http://code.google.com/p/krypto2}. Ak máte akékoľvek pripomienky,
návrhy, opravy, môžete nám ich prostredníctvom tejto stránky oznámiť.

PPershing a U\$ama.


\tableofcontents

\chapter{Úvod}
\label{chapter:uvod}
\section{Prerekvizity a označenia}

\todo{odkaz na skripta z krypto I}

V zvyšnom texte budeme dodržiavať (až na občasné výnimky) nasledujúce
označenia:
\begin{itemize}
\item $A,B$ - účastníci komunikácie, $E$ - útočník, $E(A)$ - útočník
            tváriaci sa ako účastník $A$.
\item $E(p,k), E_k(p)$ - zašifrovanie otvoreného textu $p$ pomocou kľúča $k$
\item $D(c,k), D_k(c)$ - odšifrovanie šifrového textu $c$ pomocou kľúča $k$
\item $E_A(m)$ - zašifrovanie správy $m$ pomocou verejného kľúča účastníka $A$
\item $D_A(c)$ - odšifrovanie správy $c$ pomocou súkromného kľúča účastníka $A$
\item $H(t)$ - spracovanie textu $t$ pomocou hashovacej funkcie $H$
\item $x \inr M$ - $x$ je \emph{náhodne zvolený} prvok množiny $M$
\item $\exists !$ - existuje práve jeden
\item $p(A)$ - pravdepodobnosť javu $A$
\item $p(A|B)$ - podmienená pravdepodobnosť, t.j. aká je pravdepodobnosť javu $A$, ak platí $B$
\end{itemize}

\section{0. prednáška - Ako (ne)šifrovať disky}

V decembri 2009 bola nájdená bezpečnostná chyba v niektorých šifrovaných USB diskoch
(Kingston DataTraveler BlackBox, SanDisk Cruzer Enterprise FIPS Edition a
Verbatim Corporate Secure FIPS Edition). Všetky výrobcovia uvádzajú, že disky
spĺňajú bezpečnostný štandart FIPS 140-2 a používajú úplne rovnaký systém zabezpečenia,
ktorý vyzerá nasledovne:
\begin{itemize}
\item Používateľ zadá disku heslo.
\item Heslo za pretransformuje cez MD5 hash a prvá polovica výslednej hashe sa použije ako kľúč K.
\item Následne sa pomocou AES-256 a kľúča K odšifruje daných 32 bajtov z disku (označme ich $X$). Potom zistí, či
$D_K(X)=C$, kde $C$ je pevne známa konštanta (u všetkých výrobcov dokonca rovnaká). Ak áno, tak sa disk odomkne a dáta sa sprístupnia.
Ak nie, tak sa požiadavka zamietne. Dešifrovanie ostatných dát nezávisí od hesla.
\end{itemize}

\begin{figure}[htp]
    \centering
    \includegraphics[scale=1]{img/00/extern-drive-encryption.1.mps}
    \label{fig:extern_drive_encryption}
    \caption{Šifrovanie externého disku}
\end{figure}

Útok na tento systém je vcelku jednoduchý. Stačí v pamäti prepísať výsledok dešifrovacej transformácie. 

%Viac na:
%\url{http://www.h-online.com/security/news/item/NIST-certified-USB-Flash-drives-with-hardware-encryption-cracked-895308.html}

%A ešte na (pekny dokument nie priamo suvisiaci):
%Investigating 'secure'USB stickspsu.edu [PDF]
%PJ Bakker… - Citeseer
%\url{http://citeseerx.ist.psu.edu/viewdoc/download?doi=10.1.1.84.2539&rep=rep1&type=pdf}



\chapter{Krypto I}
\label{chapter:krypto}
\section{Interaktívne dokazovacie systémy}

V tejto časti sa budeme venovať dokazovacím systémom. Pôjde o akýsi
typ spoločného výpočtu dvoch účastníkov - jedného výpočtovo
neobmedzeného provera $P$ a výpočtovo obmedzeného overovateľa $V$.
Cieľom provera je akýmsi spôsobom presvedčiť overovateľa o znalosti
nejakého faktu.
Formálne,
interaktívnym dokazovacím systémom (IDS) nazveme dvojicu
$\langle P,V \rangle$, kde $P$ je pravdepodobnostný TS
s neobmedzenou výpočtovou silou,
$V$ je pravdepodobnostný TS pracujúci v polynomiálnom čase.
Oba stroje zdieľajú spoločný vstup $x$, môžu počas svojho výpočtu
komunikovať a o akceptovaní resp. zamietaní vstupu $x$ rozhoduje iba
$V$.
IDS pre jazyk $L$ je dvojica $\langle P,V \rangle$ pre ktorú platí
\begin{itemize}
\item {\bf úplnosť} -- $\forall x \in L:
    Pr[V\textit{ akceptuje } x \textit{ v systéme }
        \langle P,V \rangle ] \ge 2/3$
\item {\bf korektnosť} -- $\forall P^*: \forall x \not \in L:
    Pr[V\textit{ akceptuje } x \textit{ v systéme }
        \langle P^*,V \rangle ] \le 1/3$
\end{itemize}
Prvá podmienka hovorí o tom, že ak $x\in L$, dokazovateľ s veľkou
pravdepodobnosťou presvedčí overovateľa o správnosti.
Naopak, korektnosť tvrdí, že ľubovoľný (podvodný) dokazovateľ
presvedčí overovateľa na zlom vstupe len s nízkou pravdepodobnosťou.

\begin{poznamka}
    Pre $L \in P$ je jednoduché navrhnúť IDS. Overovateľ bude ignorovať
    komunikáciu a môže si vypočítať príslušnosť slova sám.
    Pre $L \in NP$ je jednoduché navrhnúť IDS posielajúci práve jednu
    správu -- konkrétny dôkaz, či výpočet NTS pre problém L.
\end{poznamka}

\begin{priklad}
    Uvažujme problém $GNI \not \in NP$ -- problém grafového neizomorfizmu.
    Vstup pozostáva zo zápisu dvoch grafov $G_0, G_1$, akceptovať chceme, keď
    dané dva grafy nie sú izomorfné. Môžeme použiť nasledovný protokol
    pri dôkaze: Uvažujme $k$ kôl, v každom z nich prebehne takáto
    komunikácia:
    %% FIXME: preco 'P \send V' je vizualne dlhsie ako 'V \send P'?
    \begin{itemize}
        \item $V$ si zvolí $i \inr \{0,1\}$ a permutáciu
         $\pi \inr perm(|G_i|)$
        \item $V \send P: H = \pi(G_i)$.
        \item $P \send V: i'$ reprezentujúce graf $G_{i'}$, s ktorým je $H$
        izomorfný ($P$ je neobedzene výpočtovo silný).
        \item $V$ zamietne vstup ak $i \not = i'$.
    \end{itemize}
    Po $k$ úspešných kolách $V$ akceptuje.

    Ak $G_0 \isomorph G_1$, tak $P$ má v každom kole šancu 50\% na
    uhádnutie indexu $i$, ktorý si vymyslí $V$. 
    Pravdepodobnosť akceptovania po $k$ kolách je teda $2^{-k}$.
    Naopak, ak $G_0 \not \isomorph G_1$, tak čestný dokazovateľ vie
    vždy odlíšiť permutáciu $G_0$ a $G_1$,
    čize akceptujeme s pravdepodobnosťou 1.
\end{priklad}

Pre interaktívne dokazovacie systémy sa dá dokázať mnoho zaujímavých
vlastností. Napríklad, že $IP=PSPACE$,
čiže inak povedané, čokoľvek čo
vieme robiť v polynomiálnom priestore vieme robiť interaktívnym
dokazovacím systémom. O tom, ako na to je písané napr. v
\cite{ip-pspace}. Aby sme neostali staromódni, nedávnym dôkazom
(August 2009) bolo $QIP=PSPACE$ \cite{qip-pspace}, čiže to,
že kvantové počítače nepomôžu sile interaktívnych dôkazov.
Dôkaz je veľmi technický a využíva viacero rôznych redukcií a známych
rovností tried zložitosti. Odporúčame si ho prečítať hlavne ak má
čitateľ pocit, že kvantovým výpočtom aspoň trochu chápe.

Ďalšou možnosťou, kam môžeme rozvíjať interaktívne dokazovacie systémy
sú takzvané MIP -- multiprover IP. Pri týchto akoby overovaťel mohol
krížovo vyslúchať dokazovateľov, ktorí so sebou nemôžu komunikovať.

\input{tex/x2zero_knowledge.tex}
\section{Bit commitment}

Bit commitment schéma je protokol pre dvoch účastníkov, kde sa najprv účastník
zaviaže k nejakému bitu (ktorý zatiaľ zostáva utajený pre ostatných) 
a následne po istom čase ho odhalí. Formálne to môžeme definovať takto:

\begin{definicia}
    Majme dve množiny $X,Y$ a funkciu $f\colon \{0,1\} \times X \mapsto Y$,
    ktorú vieme ``ľahko'' počítať.
    Od $f$ požadujeme navyše ešte tieto vlastnosti:
    \begin{itemize}
    \item \emph{vlastnosť utajenia} --
        Zo znalosti $f(b,x)$ je ťažké určiť $b$.

    \item \emph{vlastnosť záväznosti} --
        Je ťažké nájsť $x, y$ také, že $x \neq y$ a 
            $f(0,x) = f(1,y)$.
    \end{itemize}

    Potom bit commitment protokol vyzerá nasledovne:

    \begin{enumerate}
    \item $A$ si zvolí $b \in \{0,1\}$ ku ktorému sa chce zaviazať a
            taktiež si zvolí náhodný prvok $x \inr X$
    \item $A \to B$: $y = f(b,x)$ -- záväzok
    \item $A \to B$: $x$ -- odhalenie, môže prísť po istom čase
    \item $B$ overí, či $y = f(0,x)$ alebo $y = f(1,x)$
    \end{enumerate}
\end{definicia}

\noindent
Tento protokol môžeme realizovať viacerými spôsobmi:

\subsection{Bit commitment pomocou RSA}

Majme nejakú inštanciu RSA systému, teda trojicu $(n,e,d)$,
kde účastník $B$ nepozná súkromný kľúč.
Bit commitment realizujeme nasledovne:
\begin{enumerate}
    \item Záväzok: $A$ si zvolí $x \inr \mathbb{Z}_n^*$, také, že
        $b$ je najmenej významný bit $x$ a pošle 
        $y = x^e \pmod n$.
    \item Odhalenie: $A \send B:x$. Účastník $B$ overí, či
        $x^e \pmod n = y$.
\end{enumerate}

Vlastnosť utajenia je dodržaná, keďže možnosť zistiť $b$
je ekvivalentná rozbitiu RSA schémy.
Vlastnosť záväznosti je tiež dodržaná,
keďže k jednému $y$ existuje iba jedno $x$.
V tomto prípade ide o nepodmienenú bezpečnosť.

\begin{poznamka}
    V krypto I sme si spomínali, že posledný bit je v RSA bezpečný.
    Tak trochu sa tam ale zavádzalo. To, čo sa v skutočnosti dokázalo
    bolo totiž ``Ak vieme zisťovať posledný bit $\then$ vieme lámať
    RSA''. Otázka ale znie, čo ak vieme určovať posledný bit napr. s
    pravdepodobnosťou 51\% bez toho, aby to implikovalo zlomenie celej
    schémy? Odpoveď je, že nevieme, ako dokázali Chor a Goldreich v 
    \cite{rsa-lsb}.
\end{poznamka}

\subsection{Bit commitment pomocou diskrétneho logaritmu}
Majme konečnú grupu $G$ a jej prvky $g, h \in G$,
pričom v tejto grupe nevieme efektívne počítať diskrétny logaritmus.
Zároveň predpokladáme, že
nevieme diskrétny logaritmus $h$ pri základe $g$.
Realizácia funkcie $f(b,x)$ bude nasledovná: 
\begin{equation*}
    f(b,x) = g^x h^b
\end{equation*}

Utajenie je v tomto prípade nepodmienene bezpečné,
lebo existujú $x, y$ také, že: $g^x = g^y h$ 
a teda nevieme jednoznačne určiť $b$ zo znalosti $f(b,x)$.
Vlastnosť záväznosti výplýva z toho, že nepoznáme
diskrétny logaritmus $h$ pri základe $g$.

\todo{BC pomocou kvadratických rezíduí}

\subsubsection{Nepodmienená bezpečnosť a záväznosť}
Môžeme si všimnúť, že sme vedeli dosiahnúť pri jednej vlastnosti 
(utajenie, záväznosť) nepodmienenú bezpečnosť.
Nasledujúca veta hovorí o tom, že nepodmienenú bezpečnosť 
nevieme dosiahnúť pri oboch vlastnostiach.

\begin{veta}
Neexistuje bit commitment schéma, ktorá by garantovala 
nepodmienenú bezpečnosť pri utajení a záväznosti.
\end{veta}

\begin{dokaz}
Sporom. 
Uvažujeme funkciu bit commitment schémy $f(b,x)$.
Ak je táto schéma garantuje nepodmienenú záväznosť, tak
platí $\forall x, y\colon f(x,b_1) = f(y,b_2) \Rightarrow b_1 = b_2$
(teda neexistuje vhodná dvojica
$x, y$ ktorou by sme vedeli porušiť záväzok).
Z toho vyplýva, že keď dostaneme $z = f(x,b)$, tak vieme 
vyskúšaním všetkých možných hodnôt $(x,b)$ určiť vyhovujúce dvojice,
tieto ale musia mať rovnaký commitnutý bit.
Preto táto schéma nemôže garantovať nepodmienené utajenie.
\end{dokaz}
\todo{IDS pre ham cycle pomocou BC}



\section{Oblivious transfer}

Ďalším základným primitívom, na ktorom vieme budovat kryptografické
prvky je takzvaný oblivious transfer. Ide o akýsi prenos údajov,
pričom sender sa nedozvie, či boli údaje prenesené, prípadne ktoré
údaje boli prenesené.

\begin{definicia}[1-2 OT]
    1-2 oblivious transferom nazveme komunikáciu podľa nasledujúcej
    schémy:
    \begin{itemize}
        \item $A \send B: m_0, m_1$, kde $m_0$ a $m_1$ sú 2 rôzne
        správy, ktoré chce $A$ poslať.
        \item $B$ si vyberie niektorú zo správ, ktorú chce prijať a
        toto prijatie mu bude umožnené.
        \item $A$ sa nedozvie, ktorú správu $B$ prijal.
        \item $B$ nemá možnosť prijať obe správy naraz.        
    \end{itemize}
\end{definicia}

\begin{definicia}[50\% OT]
    50\% oblivious transferom nazveme komunikáciu podľa nasledujúcej
    schémy:
    \begin{itemize}
        \item $A \send B: m$, kde $m$ je správa.
        \item $B$ s spravdepodobnosťou 50\% správu prijme, inak sa o
        nej nedozvie nič.
        \item $A$ sa nedozvie, či $B$ správu prijal
    \end{itemize}
\end{definicia}

\begin{priklad}[Realizácia 50\% OT]
    Nech $A$ chce poslať správu účastníkovi $B$ s 50\%-nou
    pravdepodobnosťou úspechu. Na začiatku si $A$ zvolí inštanciu RSA
    s $n=pq$, kde $p,q$ sú veľké prvočísla. Bude nasledovať
    komunikácia
 \begin{itemize}
    \compactlist
    \item $A \send B: n,e,E(m)$ 
    \item $B$ si zvolí $x \inr Z_n$
    \item $B \send A: x^2$
    \item $A$ s pomocou faktorizácie vyberie nejakú odmocninu
        $z \inr \{x,-x,y,-y\}$.
    \item $A \send B: z$.
    $B$ vie faktorizovať $n$ (a dešifrovať správu) s
    pravdepodobnosťou 50\% (ak $z \not = \pm x$, tak jednoducho
    spočíta pomocou $\nsd(z-x, n)$).
 \end{itemize} 
\end{priklad}

\begin{priklad}[Realizácia 1-2 OT]
 1-2 OT budeme realizovať za pomoci diskrétneho logaritmu.
 Účastník $A$ si zvolí grupu $G$, $n=|G|$ a generátor $g\in G$.
 \begin{itemize}
    \compactlist
    \item $A \send B: c \inr G$.
    \item $B$ si zvolí $b\in\{0,1\}$ - číslo správy, ktorú chce
    prijať.
    \item $B$ si zvolí $x \inr G$ a vypočíta hodnoty $y_b = g^x$,
    $y_{1-b} = c/ y_b$. Platí $y_0 y_1 = c$ a tiež existuje (pretože
    $g$ je generátor) hodnota $x': g^{x'} = y_{1-b}$. Teda, $A$ nevie
    odlíšiť jednotlivé hodnoty.
    \item $B \send A: y_0$.
    \item $A$ si zvolí $k_0,k_1 \inr G$ a vytvorí šifrované správy 
    $E_0 = \langle g^{k_0}, y_0^{k_0} \xor m_0 \rangle$,
    $E_1 = \langle g^{k_1}, y_1^{k_1} \xor m_1 \rangle$.
    \item $A \send B: E_0, E_1$.
    \item $B$ dešifruje $m_b$ ako $(g^{k_b})^x \xor (y_b ^ {k_b} \xor
    m_b)$.
    Pokiaľ $B$ nevie riešiť DH problém, druhú správu nedostane.
    \todo{dokaz}
 \end{itemize}
\end{priklad}

\subsection{Konverzie}

\begin{lema}
    50\% OT sa dá simulovať pomocou 1-2 OT.
\end{lema}
\begin{dokaz}
    Postup je jednoduchý. Nech $A$ chce poslať správu 
    $m \ne 0$.\footnote{Predpokladáme, že vieme posielať správy, ktoré
        sú dlhšie ako 1 bit.\fixme{ako spravit konveziu z 1 bitu?}
    }
    Nech $c \inr \{0,1\}$ a nech $m_c = m, m_{1-c} = 0$. $A$ pošle
    správy $m_0, m_1$ pomocou 1-2 OT. $B$ prijme správu $m$ s
    pravdepodobnosťou 50\%, inak prijme 0 a vie, že sa prenos
    nepodaril.
\end{dokaz}

\todo{1-2 OT pomocou 50}
\todo{BC pomocou 50 OT}
Pre záujemcov je predchádzajúca konštrukcia presnejšie popísaná v
\cite{ot_equiv}.

\section{Bezpečný výpočet viacerých účastníkov}

Uvažujme nasledujúci problém ``starých dám''. Dve dámy sa stretnú a
chceli by zistiť, ktorá z nich je staršia.\footnote{V skutočnosti
sa každá chce ubezpečiť, že tá druhá je staršia :-)}
Neboli by to však dámy v svojom veku, kebyže o sebe nechcú nič prezradiť.
Presnejšie povedané -- nechcú prezradiť nič iné okrem informácie, ktorá z
nich je staršia. Navyše budeme predpokladať, že dámy budú
spolupracovať a jedna druhú nepodrazí.
Teraz navrhneme protokol, ktorý túto ich dilemu vyrieši.

Uvažujme, že vek môže nadobúdať len jednu z konečne veľa diskrétnych
hodnôť (napríklad $v_A,v_B \in V = \{1,\dots,100\}$).
Ďalej uvažujme inštanciu asymetrického šifrovacieho systému s
veľkosťou šifrovacieho kľúča $N$. 
Budeme požadovať, aby účastník $A$ vedel (efektívne) iba šifrovať.

\begin{itemize}
    \compactlist
    \item $A$ si zvolí $x$ -- náhodné $N$-bitové číslo.
        Pomocou tohoto čísla budeme ``maskovať''  $v_A$.
    \item $A \send B: c = E(x) - v_A $ (pošleme maskované $v_A$).
    \item $B$ vygeneruje čísla $y_1, \dots, y_{100}$ tak aby
            $y_i = D(c + i)$ (kde $y_{v_A} =x$, ostatné hodnoty
            sú nepredikovateľné).
    \item $B$ si teraz vygeneruje náhodné $N/2$ bitové prvočíslo $p$ a
    vypočíta $z_1,\dots,z_{100}$ ako $z_i = y_i \pmod{p}$.
    Ak by náhodou existovali indexy $i,j: |z_i - z_j| < 2$, tak si
    zvolí iné prvočíslo.
    \item $B \send A:p$ a 100 čísel $t_1, \dots, t_{100}$ --
        budú to hodnoty $z_i$ zvýšené o hodnotu predikátu
        $v_B < i$, teda hodnoty
        $z_1,\ z_2,\ \dots,\ z_{v_B}, \
         z_{v_B+1}+1,\ z_{v_B+2}+1,\ \dots,\ z_{100}+1$.
    \item $A$ sa teraz pozrie na pravdivosť
        $x \pmod{p} \overset{?}{=} t_{v_A}$. Ak tvrdenie platí,
        tak $v_A \le v_B$.
    \item $A \send B:$ výsledok porovnania
\end{itemize}
Je jasné, že $B$ nemá šancu sa dozvedieť nič o hodnote $v_A$ počas
výpočtu, pretože $A$ si mohol zvoliť $x$ pre ľubovoľnú hodnotu
$v_A$ ako $x=D(c + v_A)$ s rovnakou pravdepodobnosťou.\footnote{V
    skutočnosti musí kryptografický systém spĺňať nejaké požiadavky
    navyše, aby to platilo, čitateľ si môže premyslieť aké.
}
Otázne teda je, či $A$ sa môže dozvedieť niečo o $v_B$.
Na to by ale potreboval vedieť dešifrovať hodnoty $y_i$, čo sme
zamietli v predpokladoch.
\begin{poznamka}
    Samozrejme, náš dôkaz ``správnosti'' by ostražitého čitateľa nemal
    presvedčiť úplne. V takomto prípade je odporúčané nazrieť do
    \cite{yao}.
\end{poznamka}

\todo{bezpecny vypocet funkcie}



\chapter{Krypto II}
\label{chapter:krypto2}
\section{Absolútne bezpečná šifra}

Tento krát sa ideme zaoberať nepodmienenou bezpečnosťou. Uvažujme 
výpočtovo neobmedzene silného útočníka a cypher text only attack (COA).
Požadujeme aby útočník zo znalosti šifrového textu nezískal nič nové.

Označme množinu otvorených textov ako $P$, kľúčov ako $K$, šifrových textov ako $C$
(uvažujeme iba množinu reálne možných šifrových textov). Každému $x \in P$ vieme priradiť
pravdepodobnosť výskytu $pr(x)$, pravdepodobnosť
výskytu konkrétneho kľúča $k \in K$ označíme ako $pr(k)$. Predpokladáme, že
konkrétny kľúc používame iba raz a že jeho voľba nezávisí od otvoreného textu.
Takisto pravdepodobnosť výskytu šifrového textu $c \in C$ označíme $pr(c)$. Táto pravdepodobnosť
bude vždy väčšia ako nula (keďže uvažujeme len šifrové texty, ktoré môžu vzniknúť zašifrovaním).
Všetky tieto množiny považujeme za konečné.

\begin{definicia}
    Šifrovací systém $(E,D)$ je absolútne bezpečný ak:
    $$\forall x \in P, \forall y \in C\colon pr(x | y) = pr(x)$$
\end{definicia}
\begin{komentar}
    Táto definícia vlastne hovorí o tom, že znalosť šifrového textu útočníkovi nepovie nič
    nové, keďže distribúciu pravdepodobnosti otvorených textov pozná.
\end{komentar}

Niekoľko zaujímavých vlastností:
Vzhľadom na to, že dešifrovanie musí byť uskutočniteľné, musí platiť $|C| \geq |P|$.
Ďalej určite platí $|K| \geq |P|$, ináč by sme pri znalosti šifrového textu mohli 
vyskúšať dešifrovanie všetkými kľúčmi a niektoré otvorené texty by sme mohli vylúčiť
($pr(x|c) = 0$). Zaujímavá je situácia keď $|P| = |C| = |K|$, tú popisuje nasledujúca veta:

\begin{veta}{(Shanon)}
    Nech $(E,D)$ je šifrovací systém, kde $|P|=|C|=|K|$. Potom tento šifrovací systém je
    absolútne bezpečný práve vtedy, keď sú splnené nasledujúce vlastnosti:
    \begin{enumerate}
        \item $\forall k \in K\colon pr(k) = \frac{1}{|K|}$
        \item $\forall x \in P, \forall y \in C: \exists! k \in K\colon E_k(x) = y$
    \end{enumerate}
\end{veta}

\begin{dokaz}
    \noindent
    \begin{itemize}
    \item[$\Rightarrow:$] 
        Nech systém $(E,D)$ je absolútne bezpečný.
        Ak by pre $x \in P, y \in C$ neexistoval kľúč $k \in K$ taký, 
        že $E_k(x) = y$, útočník zo znalosti $y$ vie vylúčiť otvorený text $x$,
        čo odporuje tomu, že systém $(E,D)$ je absolútne bezpečný. 
        Keďže $|C|=|K|$, tak tento kľúč môže byť maximálne jeden 
        (keby ich bolo viac, tak by $\exists z \in C$ taký, že
        $x$ nevieme zašifrovať na $z$). 

        Keďže $P$ je konečná, môžeme otvorené texty očíslovať, 
        teda $P = \{ x_1, x_2, \dots, x_n\}$. 
        Teraz fixujme $c \in C$. Pre každý otvorený text musí platiť:
        \begin{equation*}
            pr(x_i) = pr(x_i | c) = \frac{pr(c | x_i) pr(x_i)}{pr(c)}
        \end{equation*}
        a teda
        \begin{equation*}
            pr(c) = pr(c | x_i) = pr(k_i)
        \end{equation*}
        kde $k_i \in K$ je taký kľúč, že platí $E_{k_i}(x_i)=c$. 
        Z toho vyplýva, že všetky kľúče majú rovnakú pravdepodobnosť a keďže
        súčet výskytu ich pravdepodobnosti je $1$, tak 
        $\forall k \in K\colon pr(k) = \frac{1}{|K|}$.

    \item [$\Leftarrow:$]
        Vypočítame $pr(x|c)$ a ukážeme, že definícia platí. Vieme, že:
        \begin{equation*}
            pr(x|c) = \frac{pr(c|x)pr(x)}{pr(c)}
        \end{equation*}
        Ďalej $pr(c|x) = pr(k)$, kde $E_k(x)=c$ a teda 
        $pr(c|x) = \frac{1}{|K|}$. Ešte treba určiť $p(c)$.
        \begin{equation*}
            pr(c) = \sum_{k \in K} pr(k) pr(D_k(c)) = 
                \frac{1}{|K|} \sum_{x \in P} pr(x) = \frac{1}{|K|}
        \end{equation*}
        (Pri dešifrovaní všetkými možnými kľúčmi dostaneme 
        všetky možné otvorené texty (a každý práve raz). 
        A súčet pravdepodobností výskytu všetkých otvorených textov je 1).

        Takže dostávame:
        \begin{equation*}
            pr(x|c) = \frac{pr(c|x)pr(x)}{pr(c)} = 
                \frac{\frac{1}{|K|} pr(x)}{\frac{1}{|K|}} = pr(x)
        \end{equation*}

        A teda systém $(E,D)$ je absolútne bezpečný, čím je náš dôkaz hotový.
    \end{itemize}
\end{dokaz}

\begin{priklad}
    Vernamova šifra je príklad absolútne bezpečného šifrovacieho systému.
\end{priklad}

Iným uhlom pohľadu na absolútne bezpečnú šifru je nerozlíšiteľnosť 2 otvorených textov. 
Teda útočník dostane dva otvorené texty a jeden šifrový text a má určiť, z ktorého otvoreného
textu šifrový text vznikol. Tu nesmie uspieť.

\begin{veta}
    Šifrovací systém $(E,D)$ je absolútne bezpečný práve vtedy, keď:
    \begin{equation*}
        \forall x_1 \in P, \forall x_2 \in P, \forall c \in C
            \colon pr(c|x_1)=pr(c|x_2)
    \end{equation*}
\end{veta}

\begin{dokaz}
    \noindent
    \begin{itemize}
    \item[$\Rightarrow:$] 
        Vieme, že $pr(c|x_1) = \frac{pr(x_1|c) pr(c)}{pr(x_1)}$. 
        Keďže systém $(E,D)$ je absolútne bezpečný, 
        tak $pr(x_1|c) = pr(x_1)$, čiže $pr(c|x_1) = p(c)$. 
        To isté dostaneme aj pre $x_2$.

    \item[$\Leftarrow:$]
        $pr(c|x)$ je rovnaká pre všetky $x \in P$, čiže je konštantná. 
        Nech $pr(c|x) = t$. Potom
        \begin{equation*}
            pr(c) = \sum_{k \in K} pr(k) pr(c|D_k(c)) = 
                t \sum_{k \in K} pr(k) = t
        \end{equation*}
        Čiže $pr(x|c) = pr(x)$, čo sme chceli dokázať.
    \end{itemize}
\end{dokaz}

Takto môžeme podať trochu inú definíciu absolútne bezpečnej šifry.

\begin{definicia}
    Uvažujme kryptoanalytickú hru s nasledovným priebehom:
    \begin{enumerate}
        \item Útočník $A$ pošle druhej strane $B$ dva otvorené texty $p_1, p_2$.
        \item $B$ si zvolí náhodne $b \inr \{0,1\}$
            a náhodný kľúč $k \in K$.
        \item $B\send A: c = E_k(p_b)$.
        \item $A$ sa snaží zistiť z ktorého otvoreného textu $c$ vznikol. 
            Svoj úsudok pošle ako $b'$.
        \item Ak $b' = b$, tak $A$ uspel. V opačnom prípade neuspel.
    \end{enumerate}
    Pokiaľ je systém $(E,D)$ absolútne bezpečný, 
    tak pravdepodobnosť úspechu $A$ je $\frac{1}{2}$.
\end{definicia}

Dá sa ukázať, že tieto dve definície sú ekvivalentné.
\todo{dokaz}

\section{PSS -- Probabilistic signature scheme}

PSS je dokázateľne bezpečná (za istých predpokladov) schéma na
digitálne podpisy. Navrhli ju páni Mihir Bellare a Philip Rogaway
v \cite{pss}. Celá schéma je vlastne akýmsi znáhodnením hašovania -- k
správe sa pridá náhodný reťazec dĺžky $l_0$ a haš sa počíta až
potom. Samozrejme, pri overovaní hašu treba nejakým spôsobom
vyriešiť, aby sme sa dozvedeli použitý náhodný reťazec. Ako sa ukáže
neskôr, toto nie je až taký veľký problém ak použijeme ďalšiu
hašovaciu funkciu.

PSS má oproti doteraz spomínaným schémam jednu veľkú výhodu -- pri
dôkaze jej bezpečnosti sa ukáže ``tesná'' hranica. T.j., možnosť
prelomiť PSS s pravdepodobnosťou $\eps$ priamo umožňuje lámanie RSA s
rovnakou pravdepodobnosťou.


\subsubsection{RSA-PSS}

Podpisová schéma $PSS[l_0,l_1]=\langle GenPSS,SignPSS,VerifyPSS\rangle$ je
schéma s kľúčom dĺžky $k$ parametrizovaná dvoma hodnotami $k_0, k_1$.
Generovanie kľúča dĺžky $k$ je presne rovnaké ako v $GenRSA\mbox{-}FDH$.

Pri podpisovaní bude náš algoritmus používať dve hašovacie funkcie.
Prvú si označíme $h$ a nazveme ju kompresor, pretože bude skracovat
správu, $h:\{0,1\}^* \rightarrow \{0,1\}^{k_1}$.
Druhá funkcia bude $g$ (nazvaná aj generátor), pričom
$g:\{0,1\}^{k_1} \rightarrow \{0,1\}^{k-k_1-1}$.

Ešte si označíme $g_1$, resp. $g_2$ ako funkciu, ktorá vracia
prvých $l_0$ (resp. zvyšných $k-k_0-k_1-1$) bitov funkcie $g$.

\begin{figure}[h]
    \centering
    \includegraphics{img/02/pss.1.mps}
    \caption{Postup podpisovania v schéme PSS}
    \label{fig:pss}

\end{figure}
Na obrázku \ref{fig:pss} je schematicky naznačený postup podpisovania
podľa pseudokódu \ref{proc:signpss}

% {{{ proc SignPSS
\begin{procedure}
    \caption{SignPSS($m$)}
    \label{proc:signpss}
    $r \inr \{0,1\}^{k_0}$\;
    $w \assign h(M \concat r)$\;
    $r^* \assign g_1(w) \xor r$\;
    $y \assign 0 \concat w \concat r^* \concat g_2(w)$\;
    \Return $y^d \bmod N$\;
\end{procedure}
%% }}}

Môžeme si všimnúť, že náhodný reťazec $r$ sme neuviedli v
``otvorenom'' tvare, ale zviazali sme ho (prexorovaním) funkciou $g_1$. 
Intuitívne,
toto nám umožní zaručiť jeho integritu a zároveň máme možnosť ho
zrekonštruovať.

Overovanie podpisu je popísané v pseudokóde \ref{proc:verifypss}

%%% {{{ proc VerifyPSS
\begin{procedure}
    \caption{VerifyPSS($m$,$x$)}
    \label{proc:verifypss}
    $y \assign x^e \bmod N$\;
    rozdeľ $y$ na $b \concat w \concat r^* \concat \gamma$\;
    $r \assign r^* \xor g_1(w)$\;
    \eIf{$(b \ne 0) \vee (g_2(w) \ne \gamma) 
            \vee (h(m \concat r) \ne w)$}
    {% then
        \Return reject\;
    }{% else
        \Return accept\;
    }
\end{procedure}
%%% }}}

\todo{nejake omacky okolo funkcnosti a preco je  to zhruba tak
spravene}

\subsubsection{Dôkaz bezpečnosti}
\todo{dokaz bezpecnosti - tesna redukcia}

\section{Faktorizácia}

\subsection{Náhodné zobrazenia}
Ešte predtým, ako sa vrhneme na algoritmy pre faktorizáciu, zhrnieme
si niekoľko užitočných tvrdení, ktoré budeme neskôr využívať. Jedná sa
hlavne o vlastnosti náhodného zobrazenia.

Majme náhodné zobrazenie $f:X \rightarrow X$, kde $|X| = n$.
Budeme uvažovať postupnosť $x_0 = s,\ x_1=f(s)=f(x_0),\ x_2 =
f(f(s))=f(x_1),\ \dots,\ x_{i+1} = f(x_{i})$ pre nejaké začiatočné $s$.
Keďže obor hodnôt je konečný, najviac po $n$ krokoch sa nám nejaké
číslo zopakuje a dostaneme cyklus. Vo všeobecnosti môžeme postupnosť
rozdeliť na začiatočný ``chvost'' dĺžky $\lambda$ a cyklus dĺžky
$\mu$, ako na obrázku \ref{fig:cyclelen}.

\begin{figure}[h!]
    \centering
    \includegraphics[scale=0.9]{img/03/cyclelen.1.mps}
    \caption{Chvost a telo cyklu pri náhodnom zobrazení}
    \label{fig:cyclelen}
\end{figure}

\noindent
Základom pre ďalšiu analýzu bude nasledujúce tvrdenie:

\begin{lema}
    Nech $f:X\rightarrow X, |X|=n$ je náhodné zobrazenie.
    Potom pre $n\rightarrow \infty$ platí
    $\lambda \sim \sqrt{\pi n/8}$ a 
    $\mu \sim \sqrt{\pi n/8}$.
\end{lema}
\begin{dokaz}
    Dôkaz nebudeme robiť, nakoľko vyžaduje netriviálne znalosti
    z generujúcich funkcií. Čitateľ ho môže nájsť napríklad v článku
    \cite{randommap}.
\end{dokaz}

Za pomoci predchádzajúceho tvrdenia môžeme hľadať ``kolízie'' v
postupnosti v očakávanom čase $\Theta(\sqrt{n})$.
Jednoduchým riešením je napríklad použiť hashovanie na každý prvok
postupnosti $x_i$. Praktický problém, ku ktorému ale narážame je
pamäť, ktorá by musela byť $\Theta(\sqrt{n})$, čo je v dnešnej dobe
limitujúci faktor. Existujú však aj metódy na hľadanie cyklov,
ktoré používajú konštantnú pamäť.

\subsubsection{Metódy na hľadanie cyklov}

Začneme Floydovou metódou, ktorá sa niekedy aj označuje metódou dvoch
bežcov. Pracuje na veľmi jednoduchom princípe -- predstavme si, že máme
2 bežce -- ukazovatele na prvky postupnosti. Ak jedným bežcom budeme
pohybovať rýchlejšie ako druhým a oba tieto prvky ležia v cykle, po
istom čase (najneskôr celý prechod cyklu) jeden z ukazovateľov dobehne
ten prvý. Formálne, metóda porovnáva nasledujúce dvojice prvkov:
$(x_1,x_2),\ (x_2,x_4),\ (x_3,x_6),\ (x_4,x_8),\ \dots$.
Jej funkčnosť dokážeme nasledovne: Predpokladajme, že v aktuálnom
kroku algoritmu máme prvky $x_i, x_{j=2i}$. Ďalej redpokladajme,
že oba tieto prvky ležia v cykle, ktorého dĺžka je $\lambda$.
Ak by platilo $i \equiv j \pmod{\lambda}$, tak nutne platí $x_i = x_j$.
Uvažujme teda, že rovnosť nenastáva. V tom prípade ale platí 
$j-i \equiv i \pmod{\lambda}$. Na začiatku je táto hodnota 0 
a každým krokom algoritmu vzrastie o 1. Čiže,
najneskôr o $\lambda$ krokov bude $j-i \equiv 0 \pmod{\lambda}$.
Preto môžeme časovú zložitosť algoritmu ohraničiť ako $O(\mu+\lambda)$.

\begin{algorithm}
    \caption{Floydov algoritmus na hľadanie cyklov}
    \label{algo:floyd}
    $x1 \assign f(s)$\;
    $x2 \assign f(f(s))$\;
    \While{$x1 \ne x2$}{
        $x1 \assign f(x1)$\;
        $x2 \assign f(f(x2))$\;
    }
    output $\assign$ prvok cyklu je $x1$\;
\end{algorithm}

\medskip
Ďalšou metódou je Brentova metóda detekcie cyklov. Má rovnakú
asymptotickú časovú zložitosť ako Floydova, v praxi však používa menej
výpočtov funkcie $f$ a preto je zväčša preferovaná.
Metóda funguje v akýchsi blokoch mocniny dvojky.
Porovnáva hodnotu $x_{2^i}$ postupne s nasledujúcimi číslami
$x_{2^i+k}$, až dosiahneme ďalšiu mocninu dvojky, vtedy začneme ďalším
blokom. Algoritmus skončí v okamihu, keď je dĺžka aktuálneho bloku
(mocnina dvojky) väčšia ako dĺžka cyklu a už sme do cyklu vošli.
Preto je čas Brentovej metódy $O(\lambda + \mu)$.
Graficky je to naznačené na obrázku \ref{fig:brent} a príslušný
algoritmus je \ref{algo:brent}.

\begin{figure}[h!]
    \centering
    \includegraphics{img/03/brent.1.mps}
    \caption{Brentova metóda na hľadanie cyklov}
    \label{fig:brent}
\end{figure}

\begin{algorithm}
    \caption{Brentov algoritmus}
    \label{algo:brent}
    $x1 \assign f(s)$\;
    $x2 \assign f(f(s))$\;
    $i \assign 2$\;
    \While{$x1 \ne x2$}{
        \If{$i$ je 2. mocnina}{
            $x1 \assign x2$\;
        }
        $x2 \assign f(x2)$\;
        $i \assign i+1$\;
    }
    output $\assign$ prvok cyklu je $x1$\;
\end{algorithm}

\subsection{Pollardova metóda na faktorizáciu}
\todo{Pollard}
\todo{Dixon}
\todo{Quadratic sieve}

\input{tex/04dlog.tex}
\section{Bezpečnosť niektorých hašovacích funkcíí}

S hašovacími funkciami sme sa zoznámili v zimnom semestri.
Pripomeňme, že je to zobrazenie $h: X \to Y$, kde $Y$ je konečná množina.
Pre jednoduchosť predpokladajme, že $Y = \{0,1\}^n$. 

Dobrá hašovacia funkcia by mala spĺňať nasledovné požiadavky 
(vychádzajú z toho, že by sa mala čo najviac priblížiť náhodnému orákulu). 
Okrem požiadaviek zo zimného semestra formulujeme niekoľko ďalších:
\begin{enumerate}
    \itemsep -1.2mm
    \item Odolnosť prvého vzoru: K danému $y$ je ťažké nájsť $x$ také, 
        že $h(x) = y$. Očakávaný čas hľadania $x$ je $\Theta(2^n)$.

    \item Odolnosť druhého vzoru: K danému $x$ je ťažké nájsť 
        $z \neq x$ také, že $h(x) = h(z)$. Očakávaný čas hľadania je opäť
        $\Theta(2^n)$.

    \item Odolnosť proti kolíziám: Je ťažké nájsť $x \neq x$ také, 
        že $h(x) = h(z)$. Očakávaný čas hľadania je $\Theta(2^{n/2})$ 
        (narodeninový útok).

    \vskip 0.5cm

    \item $k$-odolnosť prvého vzoru: K danému $y$ je ťažké nájsť navzájom rôzne 
        $x_1, x_2, \dots, x_k$ také, že $h(x_i) = y$. Očakávaný čas hľadania
        je $\Theta(k 2^n)$.

    \item $k$-odolnosť druhého vzoru: K danému $x$ je ťažké nájsť
        navzájom rôzne 
        $x_1, x_2, \dots, x_k$ také, že $h(x_i) = h(x)$ a $x_i \neq x$.
        Očakávaný čas hľadania je $\Theta(k 2^n)$.

    \item $k$-odolnosť proti kolíziám: Je ťažké nájsť navzájom rôzne 
        $x_1, x_2, \dots, x_k$ také, že $\forall i, j\colon h(x_i) = h(x_j)$. 
        Dá sa ukázať, že pokiaľ je $k$ zanedbateľné oproti $2^n$, tak očakávaný 
        čas hľadania $k$-kolizie je $\Theta(2^{n(k-1)/k})$.
        Pre väčšie $k$ je očakávaný čas väčší.
\end{enumerate}

\subsection {Iteratívne konštrukcie hašovacích funcíí a ich slabiny}

Ide o konštrukciu hashovacej funkcie, ktorá opakovaným aplikovaním
hashovacej funkcie pracujúcej na vstupe pevnej dĺžky vyrobí hashovaciu
funkciu pracujúcu na binárnom reťazci neobmedzenej dĺžky.
Rozdeľme text $M$ na bloky $m_1, m_2, \dots, m_l$ rovnakej dĺžky
(v poslednom bloku môžeme očakávať nejaký padding).
Položme $h_0 = IV$ a $h_i = f(h_{i-1}, m_i)$ pre $i \geq 1$, kde $f$ 
je kompresná funkcia, ktorej obor hodnôt je $\{0,1\}^n$.
Výstupom hašovacej funkcie je hodnota $h_l$.

Takúto konštrukciu 
používa väčšina v súčasnosti používaných hašovacích funkcií
(MD5, SHA-1, SHA-xxx, \dots).

\begin{figure}[h!]
    \centering
    \includegraphics[scale=1.0]{img/05/iter.1.mps}
    \caption{Iteratívna hašovacia funkcia}
    \label{fig:iter}
\end{figure}


\noindent
V roku 2004 Joux našiel nasledujúci útok (\cite{Joux04}):

\begin{enumerate}
    \itemsep -1.2mm
    \item Položme $h_0 = IV$.

    \item Nájdime $m_1 \neq m_1'$ také, že $f(h_0, m_1) = f(h_0, m_1') = h_1$, 
        toto vieme v čase $O(2^{n/2})$.

    \item Podobne nájdime dvojice $m_2 \neq m_2', \dots, m_k \neq m_k'$ také, 
        že $f(h_{i-1}, m_i) = f(h_{i-1}, m_i') = h_i$.

    \item Toto celé vieme spraviť v čase $O(k 2^{n/2})$. A dostaneme $2^k$ 
        kolidujúcich textov (pre každý z $k$ blokov si môžeme vybrať
        2 rôzne texty). Hash každého z nich bude $h_k$.
\end{enumerate}

\begin{figure}[h!]
    \label{fig:joux1}
    \centering
    \includegraphics[scale=1.0]{img/05/joux.1.mps}
    \caption{Útok na multikolíziu}
\end{figure}


Samotný tento útok ešte veľa neznamená. Ale ukazuje sa, že
ho vieme využiť na nájdenie podstatnejších slabín.

Ukážeme si ako ho využiť pri rýchlejšom hľadaní $2^k$-vzoru:
\begin{enumerate}
    \itemsep -1.2mm

    \item Nájdeme $2^k$-kolíziu.

    \item Následne nájdeme posledný blok $m_{k+1}$ tak,
        aby $f(h_k, m_{k+1}) = y$. Toto vieme v čase $O(2^n)$.
\end{enumerate}

\begin{figure}[h!]
    \centering
    \includegraphics[scale=1.0]{img/05/joux.2.mps}
    \caption{Využitie multikolízie pri hľadaní multivzoru}
    \label{fig:joux2}
\end{figure}

Celkový čas hľadania bude $O(2^n)$ namiesto očakávaného $\Theta(k 2^n)$.
Presne rovnako vieme hľadať aj druhý vzor.

\subsubsection{Kaskádové hashovanie}
Niekedy v záujme zvýšenia bezpečnosti zreťazujeme za sebou výstup
dvoch rôznych hašovacích funkcíí, napr.
MD5 a SHA-1. Dostaneme takzvanú kaskádovú hašovaciu funkciu,
formálne $H(m) = H_1(m)||H_2(m)$. 
Ukazuje sa, že ak v jednej z týchto dvoch hašovacích funkcií
vieme efektívne hľadať multikolízie, tak vieme
v kaskádovej hašovacej funkcii hľadať kolízie efektívnejšie
ako len narodeninovým útokom.
Pre jednoduchosť predpokladajme, že obor hodnôt $H_1$
a $H_2$ je $\{0,1\}^n$. Teda dĺžka výstupu $H$ je $2n$ a teda
očakávaný čas hľadania kolízie je $\Theta(2^n)$.
Predpokladajme, že $H_1$ je iteratívna. Potom vieme v čase 
$O(\frac{n}{2} 2^{n/2})$ nájsť $2^{n/2}$ kolízií. Tieto dosadíme
na vstup $H_2$ a narodeninový paradox nám zaručí
vysokú úspešnosť nájdenia kolízie aj pre $H_2$. Takto sme dostali
kolíziu pre $H$ v čase $O(\frac{n}{2} 2^{n/2})$.

\subsection{Hľadanie expandovateľnej správy a jej využitie}

Pre útočníka môže byť výhodné mať kolidujúce správy rôznej dĺžky. 
V nasledujúcom texte si popíšeme nájdenie takej multikolízie,
kde všetky správy majú rôzne dĺžky. Nech $m^*$ je ľubovoľný
text dĺžky 1 blok.
Označme ako $F(h_i, a_1||a_2||\dots||a_x) = 
    f(f(\dots f(h_i, a_1), a_2), \dots, a_x)$,
teda viackrát za sebou použitú kompresnú funkciu $f$.
Postup je podobný ako pri hľadaní multikolízie:
\begin{enumerate}
    \itemsep -1.2mm

    \item Položme $h_0 = IV$.

    \item Nájdime $m_1$ a $m_1'$ také, že: 
        $f(h_0, m_1) = f(f(h_0, m^*), m_1') = F(h_0, m^*||m_1') = h_1$. 

    \item Nájdeme ďalšie dvojice $m_2, m_2', \dots, m_k, m_k'$ také, že platí:
        $f(h_{i-1}, m_i) = F(h_{i-1}, (m^*)^{2^{i-1}}||m_i') = h_i$.

    \item Každú dvojicu vieme nájsť v čase $O(2^{n/2})$.
        Celkový čas bude $O(k 2^{n/2})$.
        
        Dostali sme opäť $2^k$ rôznych kolidujúcich textov, ktoré tentokrát 
        majú dlžky $k, k+1, \dots, k+2^k-1$ blokov. 
        Blok príslušnej dĺžky nájdeme tak, že od dĺžky odpočítame $k$, následne
        sa pozrieme na binárny zápis výsledku postupne odzadu.
        Ak je tam $0$, ideme po hrana, ktorá nemá $m^*$,
        v prípade $1$ ideme po tej druhej.
\end{enumerate}

\begin{figure}[h!]
    \centering
    \includegraphics{img/05/expand.1.mps}
    \caption{Konštrukcia multikolízie rôznej dĺžky}
    \label{fig:expand1}
\end{figure}

\subsubsection{Davis-Meyerova konštrukcia}

Pri niektorých hašovacích funkciách, ktoré využívajú
Davies-Meyerovu konštrukciu vieme expandovateľnú správu hľadať
ešte jednoduchšie. Príkladom môže byť SHA-1, ktorej kompresná funkcia
je definovaná nasledovne:
$f(h, m) = E_m(h)+h$.

Zoberme náhodné $m$. Potom vieme vypočítať tzv. pevný bod, teda vnútorný
stav ktorý sa po \clqq zhašovaní\crqq $m$ nezmení:
\begin{align*}
    f(h,m) = h \\
    E_m(h) + h= h \\
    h = E_m^{-1}(0)
\end{align*}

Útočiť na DM konštrukcie je možné nasledovne -- vypočítajme si $2^{n/2}$ 
pevných bodov $h_1, h_2, \dots, h_{2^{n/2}}$ a k ním príslušné správy
$m_1, m_2, \dots, m_{2^{n/2}}$. 
Následne skúšajme rôzne $m'$ a hľadajme také, kde $f(h_0, m') = h_i$,
$i \in \{1, 2, \dots, 2^{n/2}\}$.
Vďaka narodeninovému paradoxu stačí $2^{n/2}$ pokusov.
Takto sme v čase $O(2^{n/2})$ našli správu, ktorú
vieme ľubovoľne natiahnuť.
\begin{figure}[h!]
    \centering
    \subfigure[Pevný bod]{
        \includegraphics{img/05/expand.3.mps}
        \hspace{1cm}
    }
    \subfigure[Hľadanie kolízií]{
        \includegraphics{img/05/expand.4.mps}
    }

    \caption{Hľadanie kolízií v MD konštrukcii}
\end{figure}

\subsubsection{Hľadanie druhého vzoru}
Následne hľadanie expandovateľnej správy vieme využiť na efektívnejšie
hľadanie druhého vzoru ako hrubou silou.
Majme správu, ktorá je dlhá $l=2^k + k - 1$ blokov.
Následne zostrojme expandovateľnú správu, ktorá môže nadobúdať dĺžky $k$
až $2^k + k -1$, ktorej hash je $h_{expand}$.
Skúšajme nájsť  $\tilde{m}$ také, že $f(h_{expand}, \tilde{m})$ 
sa rovná niektorej z hodnôt $h_{k+1}, \dots, h_{k+2^k-1}$.
Ak sa nám podarí nájsť vhodnú hodnotu
(označme ju ako $h_x$), tak môžeme bloky $m_1, m_2, \dots, m_x$ 
nahradiť príslušnou variantou expandovateľnej správy.
Očakávaný počet potrebných pokusov je $\Theta(2^{n-k})$. 

\begin{figure}{h}
    \centering
    \includegraphics{img/05/expand.5.mps}
    \caption{Hľadanie druhého vzoru}
\end{figure}

Čo to v praxi znamená? Zoberme bežne používanú SHA-1, 
ktorej výstup má 160 bitov. Teda,
očákavaný počet operácií pre nájdenie druhého vzoru je úmerný $2^{160}$.
Maximálna dĺžka správy pre SHA-1 je $2^{64} - 1$ bytov,
čo je asi $2^{55}$ blokov.
Zoberme správu dĺžky $54 + 2^{54} - 1$ blokov.
Najprv v čase $2^{80}$ nájdeme expandovateľnú správu (SHA-1 je DM konštrukcia).
Následne potrebujeme ešte $2^{106}$ pokusov na nájdenie 
``spájacieho'' bloku.
Toto je síce stály veľký počet, ale oproti očakávanému počtu $2^{160}$
je to značné zlepšenie.

\subsection{Obrana pred týmto druhom útokov}
Tieto útoky ukazujú, že iteratívna konštrukcia nie je najvhodnejšia.
To, čo ju môže zachrániť je mať aspoň 2-krát dlhší medzistav 
(t.j. hodnoty $h_1, h_2, \dots, h_{k-1}$). Potom sa skomplikuje hľadanie
kolízií v medzistavoch a konštrukcia je proti týmto útokom bezpečná.

\subsection{Nostradamov útok (alebo tiež herding attack)}

Uvažujme nasledujúcu situáciu. Svetovo známy prorok Nostradamus
vyriekne začiatkom roka 2009 proroctvo, týkajúce sa budúcnosti.
Nostradamus je prorok veľkého kalibru a preto vyriekne proroctvá
možné i všemožné, skrátka, natára veľa vecí.
Môže to byť napríklad stav akciového indexu na konci roka 2010.
Akurát ho nezverejní (nechce predsa ovplyvniť burzu, spôsobiť globálnu
ekonomickú krízu, ...) a zverejní len hash tohoto proroctva 
(v podstate je to niečo ako commitment).

Následne na konci roka odhalí svoje proroctvo na svojej stránke.
Hash je v poriadku, proroctvo
o akciách hovorí pravdu, za ním sú ešte nejaké ďalšie proroctvá.
Otázkou je ako veľmi môžeme veriť tomu, že prorok poznal stav
akcií na konci roka. Nostradamus sa nám síce môže snažiť vsugerovať,
že on je naozaj ten pravý prorok, ale je to tak?

Ukazuje sa, že na iteratívnu konštrukciu
hašovacích funkciu existuje tzv. chosen prefix attack.

\subsubsection{Konštrukcia ``proroctva''}

Pripravme si najprv $2^k$ hodnôt $h_{0,1},\ h_{0,2},\ \dots,\ h_{0,2^k}$.
Nad týmito hodnotami ideme vybudovať strom nasledovne:
hľadajme $m_{0,1},\ m_{0,2},\ \dots,\ m_{0,2^k}$ také, že
\begin{align*}
    f(h_{0,1},\ m_{0,1}) &= f(h_{0,2},\ m_{0,2}) = h_{1,1} \\
  &  \dots \\
    f(h_{0,2^k-1},\ m_{0,2^k-1}) &= f(h_{0,2^k},\ m_{0,2^k}) = h_{1,
        2^{k-1}} 
\end{align*}
Následne pokračujme podobným spôsobom aj na ďalších úrovniach
pokiaľ nedostaneme jedinú hodnotu $h_{k,1}$.
Túto hodnotu zverejníme ako hash nášho proroctva.
Navyše, aby sme za správu vedeli ``prilepiť'' rozumný obsah,
je vhodné aby tieto správy dávali aspoň nejaký zmysel.

Aký čas strávime prípravou tohoto stromu?
Nájdenie jednej kolízie trvá čas $O(2^{n/2})$, na prvej úrovni treba 
$2^{k-1}$ kolízii, na druhej $2^{k-2}$ a na poslednej
potrebujeme jednu kolíziu. Celkový čas je teda
$O\big(2^n (2^{k-1} + 2^{k-2} + \dots + 1)\big) = O(2^{n/2} 2^k)$.
Lepší spôsob je nehľadať kolízie na jednotlivých úrovniach po dvojiciach,
ale rovno pre celú úroveň.
Vtedy na prvej úrovni strávime čas $O(2^{k/2 + n/2})$. 
Celkový čas bude tiež $O(2^{k/2 + n/2})$.

\begin{figure}[h]
    \centering
    \includegraphics[scale=0.9]{img/05/nostradamus.1.mps}
\end{figure}

Nostradamus na konci roka (po zatvorení burzy) zistí stav akciového
trhu a pripraví si podľa neho správu.
Označme ju $m$. Po jej zahashovaní sa dostaneme do odtlačku
$h_m$. Teraz hľadáme správu $\tilde{m}$, pre ktorú platí
$f(h_m, \tilde{m}) = h_{0,x_0}$, kde $x_0 \in \{1, 2, \dots, 2^k\}$.
Následne na základe pripraveného stromu vieme za správu dolepiť patričné 
$m_{0,x_0},\ m_{1,x_1},\ \dots,\ m_{k,x_k}$ také, že
výsledný hash bude predtým zverejnený $h_{k,1}$.
Teda na konci zverejníme správu 
$m || \tilde{m} || m_{0,x_0} || m_{1,x_1} || \dots || m_{k, 1}$.

Čas hľadania $\tilde{m}$ bude $2^{n-k}$.

\subsubsection{Celkový čas útoku}
Bolo by dobré, aby čas trvania oboch častí útoku bol viacmenej vyvážený. 
Uvažujme najprv pomalšiu variantu predspracovania.
Vtedy chceme, aby $2^{n-k} = 2^{k+n/2}$, z čoho dostaneme
$k=n/4$ a celkový čas bude $2^{3n/4}$.
Pri rýchlejšom predspracovaní máme podmienku $2^{n-k} = 2^{k/2 + n/2}$,
z čoho máme $k = n/3$ a celkový čas $2^{2n/3}$.

\fixme{realne priklady}

\section{Jednorázové a fail-stop podpisové schémy}

Jednorázové podpisové schémy, ako už ich názov napovedá, slúžia na
podpísanie práve jednej správy. Ich bezpečnosť je v prípade viacnásobného
podpisovania ohrozená. Načo sú nám teda takéto podpisové schémy? Zatiaľ to
vyzerá tak, že sú ia menej výhodné. Existujú však dôvody, prečo sa zaoberať
aj takýmito zjavne ``okrátenými'' podpisovými schémami.

Ich hlavná výhoda bude spočívať v jednoduchších predpokladoch pri dôkaze
bezpečnosti. Kým pri bežných podpisových schémach sme stavali na
tažkosti istých matematických problémov (RSA, dlog, Diffie-Hellman),
pri jednorázových schémach nám bude stačiť napríklad jednosmernosť
hashovacej funkcie.\footnote{Už aj toto je pomerne náročný predpoklad,
    keďže nevieme povedať veľa o existencii one-way funkcií}
Druhou výhodou môže byť rýchlosť - zrejme je jednoduchšie hashovať hodnoty
ako napríklad umocňovať.
Treťou výhodou (i keď skôr teoretickou) je možnosť odolať kvantovým
výpočtom - pre väčsinu používaných ťažkých problémov sú známe kvantové
algoritmy, ktoré ich efektívne počítajú. Pre invertovanie hashovacích
funkcií ale takéto algoritmy nie sú známe.

Otázka teda môže znieť, že či jednorázovosť je až taká obmedzujúca
vlastnosť. Môžeme napríklad uvažovať komunikáciu s bankou a podpisovanie
prevodných príkazov. Je jednoducho predstaviteľné, že povedzme za mesiac
bežný človek nespraví viac ako povedzme 5 príkazov. Preto môžeme použiť
niečo ako pohľad z opačnej strany - namiesto toho, aby sme navrhovali
podpisové schémy na polynomiálny počet podpísaných správ,
môžeme sa snažiť navrhnúť schémy na jednorázové podpisy a tie potom
rozšíriť nejakým spôsobom pre viac správ.

Jednorázovú podpisovou schému formálne definujeme veľmi podobne ako bežné
podpisové schémy

\begin{definicia}[Jednorázová podpisová schéma]
    je trojica algoritmov 
    $\langle Gen, Sign_{sk}, Verify_{pk} \rangle$ kde
    $Gen(1^k) \implies \langle pk, sk \rangle$ je generátor kľúčov,
    $Sign_{sk}(m) \implies \sigma$ je podpisovací algoritmus a
    $Verify_{pk}(m,\sigma) \implies \{0,1\}$ je overovací algoritmus.
\end{definicia}

Pojem bezpečnosti takejto schémy si upravíme na jednu správu.
\begin{definicia}[Bezpečnosť]
    Uvažujme útočníka ako PPT algoritmus, ktorý má navyše k dispozícii
    orákulum $Sig_{sk}$. Útočník sa môže raz opýtať orákula na podpis
    $\sigma$ správy $m$ a jeho cieľom je zostrojiť
    správu $m' \ne m$ a k nej platný podpis $\sigma'$.
    Budeme hovoriť, že schéma je bezpečná ak pravdepodobnosť,
    že ľubovoľný útočník uspeje (t.j. nájde $(m',\sigma')$), je zanedbateľná.
\end{definicia}

\subsection{Lamportova schéma}

Uvažujme, že máme funkciu $f: X \rightarrow Y$, ktorá je jednosmerná.
Budeme podpisovať správy $m$ fixnej veľkosti $|m|=n$. Toto samo o sebe nie
je žiadny problém, ak uvážime, že budeme podpisovať len hash správy, ktorý je
fixnej veľkosti.

Generovanie kľúča bude vyzerať nasledovne:
%%% {{{ proc GenLamport
\begin{procedure}
    \caption{GenLamport($n$)}
    \For{$i:=1$ \KwTo $n$}{
        $x_{i,0} \inr X$\;
        $x_{i,1} \inr X$\;
        $y_{i,0} \assign f(x_{i,0})$\;
        $y_{i,1} \assign f(x_{i,1})$\;
    }
    \Return $sk=(x_{1,0},x_{1,1},\ldots,x_{n,0},x_{n,1}),\quad
             pk=(y_{1,0},y_{1,1},\ldots,y_{n,0},y_{n,1})$\;
\end{procedure}
%%% }}}

Čitateľ už môže tušiť, ako budeme podpisovať správu - jednoducho postupne
podpíšeme všetyk jej bity tým, že zverejníme príslušnú časť súkromného
kľúča.

%%% {{{ SignLamport
\begin{procedure}
    \caption{SignLamport($m$)}
    \For{$i:=1$ \KwTo $n$}{
        $\sigma_i \assign f(x_{i, m_i})$\;
    }
    \Return $\sigma = (\sigma_1, \ldots, \sigma_n)$
\end{procedure}
%%% }}}

Overovanie spočíva v overení každého podpísaného bitu správy.

\begin{procedure}
    \caption{VerifyLamport($m, \sigma$)}
    \For{$i:=1$ \KwTo $n$}{
        \If{$f(x_{i, m_i}) \ne y_{i, m_i}$}{
            \Return reject\;
        }
    }
    \Return accept\;
\end{procedure}

Bezpečnosť schémy je založená na nasledujúcom pozorovaní:
Aby bol útočník k správe $m$ a jej podpisu $\sigma$ vygenerovať
falošnú správu $m' \ne m$ a jej podpis $\sigma'$, musel by byť schopný
vygenerovať $x_{i,b}$ pre nejakú novú dvojicu $(i,b)$.
To ale znamená invertovať niektorú hodnotu $y_{i,b}$ z verejnej
funkcie, predpokladáme, že je možné iba so zanedbateľnou
pravdepodobnosťou.

Na druhej strane, schéma je evidentne jednorázová.
Až na špeciálny prípad, keď sú podpísané dve správy $m_1,m_2$ líšiace
sa v práve jednom bite (vtedy si útočník nepomôže), vieme kombinovať
jednotlivé bity podpisov a podpísať inú správu. V prípade, že
podpisujeme hash hodnotu, je navyše očakávané, že správy sa budú líšiť
na zhruba polovici bitov.

Uvažujme teraz praktické aspekty používania takejto schémy. Podpisujme
hash, napríklad výstup z funkcie SHA-256. Ďalej predpokladajme, že
$|X|=|Y|=256 \unit{bit}$. Potom dostávame pre súkromný kľúč veľkosť
$|sk|=2*256*256 bit =16 \unit{kB}$. Verejný kľúč je rovnako dlhý, čiže
$|pk|=16 \unit{kB}$ a podpis má polovičnú dĺžku kľúčov - $|\sigma|=8
\unit{kB}$.
V porovnaní napríklad s RSA je to výrazne horšie. Bolo by teda dobré
nejakým spôsobom skrátiť kľúče.

Skrátenie súkromného kľúča: Namiesto celého kľúča $(x_{1,0}, x_{1,1},
\ldots x_{n,1})$ si budeme pamätať iba náhodné $x \inr X$ - seed pre
pseudonáhodý generátor, ktorý postupne vygeneruje dané hodnoty
$x_{i,b}$. Problém v tomto prípade je ďalší predpoklad - bezpečnost
pseudonáhodného generátora (t.j. že z niektorých hodnôt postupnosti
$x_{1,0}, \ldots, x_{n,1}$ nevieme efektívne vypočítať žiadnu ďalšiu -
to by bolo ekvivalentné zlomeniu podpisovej schémy).

Skrátenie verejného kľúča: Namiesto celého kľúča zverejníme iba hash
$y=H(y_{1,0},y_{1,1},\ldots,y_{n,1})$. Tým pádom ale pri podpisovaní
musíme uviesť aj všetky hodnoty $y$, aby si to overovateľ mohol
overiť. Po chvíli zamyslenia sa ale môžeme pozorovať, že na overenie
stačí poslať tie hodnoty $y_{i,b}$, ktoré si overovateľ nemôže
spočítať. Tieto sú presne negácie bitov správy a teda nám stačí poslať
iba polovicu hodnôt $y_{i,b}$. Podpis sa nám tým predĺži na $16
\unit{kB}$.

\subsubsection{Merkleho konštrukcia}
Ďalši nezávislú možnosť ako skrátiť dĺžku postupnosti vymyslel
Merkle. Hlavnou pointou bude pridať akúsi formu checksumu - počtu
nulových bitov správy. Potom budeme pri podpisovaní podpisovať iba
jednotkové bity, čím ušetríme v priemernom prípade takmer polovicu bitov
(čiže v našom prípade $|\sigma| \approx 4 \unit{kB}$).

Presnejšie, majme správu dĺžky $l$. K nej pridáme checksum dĺžky
$\lceil \log l \rceil$ a na výsledok dĺžky $n=l+\lceil \log l \rceil$
použijeme podpis, kde podpíšeme iba jednotkové bity.

V prvom rade by sme mali ukázať, že takáto zmena nepokazí bezpečnosť
schémy. Majme preto známu správu $m$ s podpisom $\sigma$ a
predpokladajme, že sa útočník snaží vyrobiť $m'$.
Ak existuje bit $i$, $m_i=0$ a platilo by $m'_i=1$, tak sa dostávame
do známej situácie, kedy útočník musí vyrobiť platný vzor pre $y_{i,1}$.
Ošemetná situácia ale nastáva, ak máme bit $m_i=1$ a útočník ho zmení
na nulu. V tomto prípade totiž nemusí nič podvrhnúť, lebo pre nulový
bit nemusí nič uviesť. Zachráni nás však checksum - totiž, ak zmenšíme
celkový počet jednotiek v správe (a to jediné nám ostáva, ak nemáme
nablyšťanú guľu na lámanie one-way funkcie), vznikne nám aspoň jedna
jednotka na doteraz neodhalenom mieste - nie je totiž možné, aby sme
po pričítaní čísla k checksume dostali checksumu pozostávajúci iba zo
známych bitov - to by znamenalo, že daná checksuma používa iba
niektoré jednotky z pôvodnej, lenže to je v spore s tým, že je
väčšia). Preto aj v tomto prípade musí útočník úspešne nájsť vzor
jednosmernej fukncie a schéma ostáva naďalej bezpečná.

Ďalšou príjemnou vlastnosťou tejto úpravy je automatické zmenšenie
súkromného a verejného kľúča na polovicu - vôbec nepotrebujeme
generovať $x_{i,0}$ a $y_{i,0}$.

\subsection{Merkleho stromy}
\subsection{Stanekova schéma}
\subsection{Fail/stop podpisové schémy}

Predstavme si, že chceme schému pri ktorej sme (ako podpisovateľ) chránený pred neobmedzene výpočtovo silným
falšovateľom. Inak povedané, žiadny útočník, nech je akokoľvek silný,
by nemal z prístupu k môjmu verejnému kľúču byť schopný generovať
platné podpisy. Toto je samozrejme mierne v rozpore s tým, že máme
vedieť overiť podpis. Preto budeme požadovať miernejšiu vec -
podpisovateľ bude schopný preukázať, že to nebolo podpísané jeho
súkromným kľúčom.
Opäť si predstavíme jednorázovú schému. Útočník v prípade získania
správy, jej podpisu a verejného kľúča nebude schopný identifikovať
jednoznačne súkromný kľúč (napríklad preto, že možných súkromných
kľúčov bude veľmi veľa)
a nebude schopný vyrobiť správny podpis inej správy.
Pokiaľ sa falšovateľ pokúsi podpisať inú správu, tak podpisovateľ zistí, že bol použitý iný
ako jeho súkromný kľúč a je to schopný preukázať.

\subsubsection{Heyst Pedersenova schéma}

Uvažujme grupu $G$, kde $|G| = q$ a $q$ je nejaké (dostatočne veľké) prvočíslo.
Zoberme generátory $g, h \in G$.
Súkromný kľúč bude štvorica  $sk = (x_1, x_2, y_1, y_2) \inr Z_q$.
Poznamenajme, že narozdiel od ElGamalovej schémy, hodnoty $x,y$ sú
náhodné a \emph{nezávislé}.
Verejný kľúč bude dvojica
$pk = (g^{x_1} h^{x_2}, g^{y_1} h^{y_2}) = (z_1, z_2)$, teda akýmsi
spôsobom previažeme obe hodnoty. Algoritmický zápis generovania je vo
funkcii \ref{funct:genhp}

\begin{function}[h!]
    \caption{GenHP($G$)}
    \label{funct:genhp}
    $g,h \assign $ rôzne generátory $G$\;
    $x_1,x_2,y_1,y_2 \inr G$\;
    $z_1 \assign g^{x_1} h^{x_2}$\;
    $z_2 \assign g^{y_1} h^{y_2}$\;
    \Return $sk=(g,h,x_1,x_2,y_1,y_2), pk=(g,h,z_1,z_2)$\;
\end{function}

Správu $m$ podpíšeme svojim súkromným kľúčom ako lineárnu kombináciu
$x_i, y_i, m$ pomocou funkcie \ref{funct:signhp}
\begin{function}[h!]
    \caption{SignHP($m$)}
    \label{funct:signhp}
    $\sigma_1 \assign x_1 + m y_1$\;
    $\sigma_2 \assign x_2 + m y_2$\;
    \Return $\sigma=(\sigma_1,\sigma_2)$\;
\end{function}

Overenie podpisu je jednoduché otestovanie
rovnosti:
\begin{function}[h!]
    \caption{VerifyHP($\sigma,m$)}
    \eIf{$g^{\sigma_1} h^{\sigma_2} == z_1 z_2^m$}{
        \Return accept\;
    }{
        \Return reject\;
    }
\end{function}

Teraz si dokážeme niekoľko vlastností tejto schémy.

\begin{lema}
Pre ľubovoľnú trojicu $pk, m, \sigma$, kde $\sigma = SigHP_{sk}(m)$
a $sk$ je nejaký vyhovujúci súkromný kľúč
existuje $q$ rôznych kľúčov $sk^*$, takých že $\sigma = Sig_{sk^*}(m)$.
\end{lema}

\begin{dokaz}
Nech $h = g^a$, $z_1 = g^{e_1}$ a $z_2 = g^{e_2}$ 
(neobmedzene silný útočník vie $a, e_1, e_2$ vypočítať).
Z toho, že $z_1 = g^{x_1} h^{x_2}$ máme rovnicu $e_1 = x_1 + ax_2$.
Podobne z $z_2 = g^{y_1} h^{y_2}$ dostaneme $e_2 = y_1 + a y_2$.
A ešte vďaka tomu, že $\sigma$ je podpis správy $m$ máme rovnice
$\sigma_1 = x_1 + my_1$, $\sigma_2 = x_2 + my_2$. 

Dostali sme 4 rovnice. Máme 4 neznáme. V maticovom tvare dostávame:
\begin{equation*}
    \left ( \begin{matrix}
                1 & a & 0 & 0 \cr 
                0 & 0 & 1 & a \cr
                1 & 0 & m & 0 \cr
                0 & 1 & 0 & m
            \end{matrix} \right )
    \left ( \begin{matrix}
                x_1 \cr x_2 \cr y_1 \cr y_2
            \end{matrix} \right )
    =
    \left ( \begin{matrix} 
                e_1 \cr e_2 \cr \sigma_1 \cr \sigma_2
            \end{matrix} \right )
\end{equation*}

Matica sústavy má ale hodnosť $3$. Jedno riešenie už máme (pôvodný
súkromný kľúč) a teda sústava má práve $q$ rôznych riešení (sme v
priestore $Z_q^4$).

\end{dokaz}

Toto dokazuje, že potenciálnych súkromných kľúčov je exponenciálne veľa.
Na to aby sme ukázali, že útočník má
šancu na úspech $\frac{1}{q}$ treba ešte ukázať jednu vec.
Môže totiž nastať patologický prípad, kde veľa súkromných kľúčov
pri podpise falošnej správy $m^*$ dá rovnaký podpis.
Teda budeme mať rôzne kľúče $sk^{'}, sk^{''}, \dots$, 
kde $Sig_{sk^{'}}(m^*) = Sig_{sk^{''}}(m^*) = \dots$.
To, že tento prípad nenastane ukazuje nasledujúca lema.

\begin{lema}
Pre ľubovoľné $pk, m, \sigma, m^*, \sigma^*$, kde $\sigma = Sig_{sk}(m)$, $\sigma^* = Sig_{sk}(m^*)$ a $sk$ vyhovuje $pk$
existuje maximálne jeden jediný vyhovujúci $sk$.
\end{lema}

\begin{dokaz}
Prvé 4 rovnice máme rovnaké ako v predchádzajúcej leme. Navyše dostaneme ešte rovnice $x_1 + m^*y_1 = \sigma_1^*$, 
$x_2 + m^* y_2 = \sigma_2^*$.
Pokiaľ zostrojíme maticu tejto sústavy bude mať hodnosť $4$ a teda bude mať maximálne jedno riešenie.
\end{dokaz}


Ešte treba vyriešiť otázku ako môže podpisujúci spochybniť podpis 
a či tomuto spochybneniu môžeme veriť.
Keďže je to jednorázová schéma, tak podpisujúci v prípade,
že chce spochybniť podpis, tak stačí ak zverejní vlastný súkromný kľúč.
V tom prípade vieme overiť, že nesedí s podpisom falošnej správy
a navyše vyhovuje verejnému kľúču.

Ukažeme ešte, že podpisujúci si nevie vymyslieť iný súkromný kľúč
(samozrejme ak je výpočtovo obmedzený). 
Nech jeho pôvodný súkromný kľúč je $sk = (x_1, x_2, y_1, y_2)$ a
chce si pripraviť nový kľúč $sk^{'} = (x_1^{'}, x_2^{'}, y_1^{'}, y_2^{'})$.
Potom platí $z_1 = g^{x_1} h^{y_1} = g^{x_1^{'}} h^{y_1^{'}}$ z čoho máme 
$h = g^{(x_1 - x_1^{'})(y_1^{'} - y_1)^{-1}}$.
A teda vieme zistiť $dlog_g h$.

%tuna redukciu na dlog stanek robil trochu inac (cez rovnost podpisov), toto sa mi zda byt priamejsie

\section{Inkrementálne hašovanie}

Predstavme si, že máme dlhý dokument (súbor alebo disk),
označme ho $m$ a chceme si uchovávať jeho haš napríklad kvôli
integrite alebo autenticite. Pri klasickom riešení
by sme museli prejsť celý dokument a spočítať jeho 
haš $H(m)$. Následne keď urobíme čo i len najmenšiu
zmenu, tak na to, aby sme získali nový haš musíme opäť
prejsť celý súbor. Toto je náročné na systémové prostriedky
a veľmi neefektívne. Predstavíme si preto niekoľko riešení, ktoré sa
snažia riešiť tento problém a ukážeme ich nedostatky.

\subsection{Triviálne riešenia}

Môžeme náš dokument rozdeliť na časti (disk na sektory)
a uchovávať haš každej časti osobitne. V prípade, keď
$m = m_1 m_2 \dots m_k$, tak haš bude 
$H(m) = \langle h(m_1), h(m_2), \dots, h(m_k) \rangle$.
Toto riešenie má ale príliš dlhý výsledný haš.
Takisto, z pohľadu autenticity nie je nič moc, že vieme
preusporiadavať bloky.

Iné riešenie je použiť Merkleho stromy. Ale aby
sme mali rýchly update hašu musíme pamätať zloženie celého stromu,
čo tiež nie je potešujúce.


\subsection{Lepšie riešenia}

Zoberme konečnú komutatívnu grupu $(G, \odot)$ (napríklad $(2^n, \oplus)$).
Následne predpokladajme, že máme hašovaciu funkciu s oborom hodnôť $G$.
Rozdeľme dokument na $k$ blokov $m = m_1 m_2 \dots m_k$, každý
veľkosti $n$. Náš haš bude 
\begin{equation*}
    H(m) = \bigodot_{i=1}^k h(i, m_i)
\end{equation*}
Pokiaľ sa blok $m_i$ zmení na $m_i'$, tak nový haš vypočítame
nasledovne:
\begin{equation*}
    H(m') = H(m) \odot h(i, m_i)^{-1} \odot h(i, x_i')
\end{equation*}

\subsubsection{Súvislosť s problémom vyvažovania}

Dá sa ukázať, že nájsť kolíziu pre inkrementálnu hašovaciu
funkciu nad grupou $G$ je aspoň tak ťažké ako riešiť
problém vyvažovania (ktorého obťiažnosť závisí hlavne od grupy 
$G$). Vo všeobecných grupách sa verí, že tento problém je
NP-ťažký.\footnote{Pozor, nie NP-úplný.}

\begin{definicia}[Problém vyvažovania]
    Máme zadanú grupu $(G, \odot)$ a postupnosť $a_1, a_2, \dots, a_n$.
    Našou úlohou je nájsť čísla $w_1, w_2, \dots, w_n \in \{-1, 0, 1\}$,
    kde aspoň jedno z nich je nenulové a platí 
    $a_1^{w_1} \odot a_2^{w_2} \odot \dots \odot a_n^{w_n} = e$.
    Tomuto problému hovoríme problém vyvažovania.
\end{definicia}

\begin{poznamka}
    Pokiaľ je grupa $G$ komutatívna, tak vlastne hľadáme
    dve neprázdne disjuktné podmnožiny
    $I, J \subseteq \{1, 2, \dots, n\}$, kde platí 
    $\bigodot_{i\in I} a_i = \bigodot_{j \in J} a_j$.
\end{poznamka}

\begin{lema}
    Ak vieme v grupe $(G, \odot)$ hľadať kolízie pre inkrementálne hašovacie
    funkcie, tak vieme rovnako efektívne riešiť problém vyvažovania.
\end{lema}

\begin{dokaz}
    Majme program $A$, ktoré hľadá kolíziu pre inkrementálnu hašovaciu
    funkciu nad grupou $(G, \odot)$. Tento program nám položí $q$ otázok
    typu ``aká je hodnota $h(i, x_i)$?'' Na tieto otázky mu odpovieme postupne
    hodnotami $a_1, a_2, \dots, a_q$.
    Program následne vyprodukuje odpoveď $H(x) = H(y)$, čo je vlastne 
    $\bigodot_{i \in I} a_i = \bigodot_{j \in J} a_j$.
    (Keďže predpokladáme, že funkcia $h$ sa správa ako náhodné orákulum,
    tak sa na všetky zložky $x$ a $y$ musí $A$ opýtať.)
\end{dokaz}
\fixme{Tento dôkaz je pochybný -- problémy: 1. čo ak máme otázok viac,
 2. A čo randomizácia? Ak vieme hľadať kolízie iba pravdepodobnostne,
 tak tento postup nefunguje}

\subsection{XOR-HASH}

Najprv si ukážeme tzv. XOR-HASH.
Predpokladajme, že $h\colon \{0,1\}^n \to \{0,1\}^l$.
Haš bude daný vzorcom
\begin{equation*}
    H(m) = \bigoplus_{i=1}^k h(i, m_i)
\end{equation*}

Ukážeme, že aj v prípade, že aj v prípade ak $h$ je kvalitná hašovacia funkcia
(správa sa ako náhodné orákulum) vieme XOR-HASH invertovať.

Na vstupe majme hodnotu $z \in \{0,1\}^l$, ktorý chceme invertovať.
Najprv si pripravíme dva náhodné dokumenty (postupnosti blokov)
$m^{(0)} = m_1^{(0)} m_2^{(0)} \dots m_k^{(0)}$ a 
$m^{(1)} = m_1^{(1)} m_2^{(1)} \dots m_k^{(1)}$. Bolo by pritom
vhodné, aby zodpovedajúce si bloky boli rôzne, teda aby $\forall i:
m_i^{(0)} \ne m_i^{(1)}$.
Vypočítame si haše jednotlivých častí a označíme ich ako
$a_i^{(b)} = h(i, m_i^{(b)})$. 
Teraz chceme nájsť vektor $y = (y_1,y_2, \dots ,y_k)$, kde $y_i \in \{0,1\}$,
taký, aby platilo 
\begin{equation*}
    z = a_1^{(y_1)} \oplus a_2^{(y_2)} \oplus \dots \oplus a_k^{(y_k)}
\end{equation*}
(čo je inak povedané $H(m_1^{(y_1)} m_2^{(y_2)} \dots m_k^{(y_k)}) = z$).
Táto rovnica sa dá napísať aj ako:
\begin{equation*}
    z = a_1^{(0)} (1 - y_1) \oplus a_1^{(1)} y_1 \oplus \dots 
        \oplus a_k^{(0)} (1 - y_k) \oplus a_k^{(1)} y_k
\end{equation*}

Keďže $z$ má $l$ bitov a všetky tieto bity sa počítajú nezávisle od
ostatných, vieme zostaviť $l$ rovníc nad $Z_2$. 
A máme $k$ neznámych.
V praktických aplikáciách je $k > l$ 
a teda riešením sústavy dostaneme skoro vždy riešenie.

\todo{sanca na najdenie riesenia}

\subsection{Ad-hash}

Tento krát spravíme iteratívnu hašovaciu funkciu v grupe $(Z_u, +)$.
\begin{equation*}
    H(x) = \sum_{i=1}^k h(i, m_i) \pmod{u}
\end{equation*}
Tu sa dá ukázať, že problém vyvažovania pre $(Z_u, +)$ je ťažký. 
V praxi sa používajú napr. tieto 2 konštrukcie:

NASD konštrukcia:
\begin{equation*}
    H(x) = \sum_{i=1}^k h(i, m_i) \pmod{2^{256}}
\end{equation*}

DCIHF konštrukcia:
\begin{equation*}
    H(x) = \sum_{i=1}^{k-1} \textrm{SHA-1}(m_i, m_{i+1}) \pmod{2^{160}+1}
\end{equation*}

\subsection{Zovšeobecnený narodeninový útok}

Predstavme si, že máme dva zoznamy slov $L_1, L_2 \in \{0,1\}^n$ vygenerované
napríklad pomocou hašovacej funkcie. Teraz chceme nájsť
také $x_1 \in L_1, x_2 \in L_2$, že $x_1 = x_2$
resp. inak povedané $x_1 \oplus x_2 = 0$.

Toto je starý známy narodeninový útok. Pokiaľ obidva zoznamy budú mať veľkosť
$2^{n/2}$, tak máme celkom dobrú šancu, že takáto dvojica existuje. Nájdeme ju
už ľahko - jeden zoznam vložíme do hašovacej tabuľký a skúšame potom v tejto
tabuľke hľadať prvky z druhého zoznamu.
Celkový čas útoku by bol $O(2^{n/2})$.

Tento druh útoku môžeme zovšeobecniť pre $k$ zoznamov $L_1, L_2, \dots, L_k$.
Potom naša požiadavka na vybrané slová je: $x_1 \oplus x_2 \oplus \dots \oplus x_k = 0$.

Ukážeme si ako vieme jednoducho hľadať takéto slová pre $k=4$.

Vygenerujeme zoznamy $L_1, L_2, L_3, L_4$ s veľkosťou $2^{n/4}$. 
Následne vyrobíme všetky kombinácie $x_1 \oplus x_2$ a $x_3 \oplus x_4$, kde
$x_1 \in L_1, x_2 \in L_2, x_3 \in L_3, x_4 \in L_4$.
Každej z týchto kombinácií bude $2^{n/2}$. Následne prevedieme útok pre $k=2$.
Takže celkový čas opäť bude $O(2^{n/2})$.

Zatiaľ sme si ale veľmi nepomohli. Ale v roku 2002 Wagner \cite{birthday}
našiel rýchlejší útok.

Najprv si ale zadefinujme operáciu spojenia (join). Nech $low_l(x)$ predstavuje
posledných $l$ bitov slova $x$. Potom:
$L \bowtie_l L^{'} = \{(x, x^{'}) | x \in L \land x^{'} \in L^{'} \land low_{l}(x) = low_l(x^{'})\}$
Teda vyberieme z $L$ a $L^{'}$ tie dvojice slov, ktoré sa zhodujú na posledných
$l$ bitoch.

Teraz si ukážeme útok pre $4$ zoznamy. 
\begin{enumerate}
\item Pripravíme si zoznamy veľkosti $2^{n/3}$.
\item Vypočítame $L_{12} = L_1 \bowtie_{n/3} L_2$.
\item Vypočítame $L_{34} = L_3 \bowtie_{n/3} L_4$.
\item Vypočítame $L_{12} \bowtie L_{34}$.
\end{enumerate}

\begin{figure}
    \centering
    \includegraphics[scale=1]{img/07/birthday.1.mps}
    \caption{Zovšeobecnený narodeninový útok pre $k = 4$}
\end{figure}


Pozrime sa na časovú zložitosť. Očakávaná veľkosť $L_{12}$ bude $2^{n/3} \cdot 2^{n/3} / 2^{n/3} = 2^{n/3}$ (zoberieme všetky možné
dvojice a požadujeme rovnosť posledných $n/3$ bitov). V tomto istom čase vieme aj vygenerovať tento zoznam. To isté
dostaneme aj pre $L_{34}$. Očakávaná veľkosť výstupu z posledného kroku je $2^{n/3} \cdot 2^{n/3} / 2^{2n/3} = 1$.
Časová zložitosť opať bude $O(2^{n/3})$. Takto sme dosiahli oveľa lepšiu časovú zložitosť ako pôvodne.

Tento postup vieme zovšeobecniť pre ľubovoľné $k = 2^l$. 
Najprv si pripravíme zoznamy veľkosti $2^{\frac{n}{l+1}}$. Následne ich postupne spájame v strome, pričom
vo výške $v$ (výška úplne dole je $1$, smerom nahor rastie) spravíme join na posledných $w_v = \frac{vn}{l + 1}$
bitoch. Výnimka je koreň tam robíme join na všetkých $n$ bitoch.

\paragraph{Časová zložitosť:}
Každý nekoreňový join vyrobí $2^{\frac{n}{l+1}}$ dvojíc 
a trvá mu to čas $O(2^{\frac{n}{l+1}})$.
Koreňový join vyrobí jednu dvojicu a trvá mu to tiež čas 
$O(2^{\frac{n}{l+1}})$. Celkovo musíme urobiť $k-1$ joinov, takže
celková časová zložitosť je $O(k\cdot 2^{\frac{n}{1+lg k}})$.

\begin{figure}
    \centering
    \includegraphics[scale=1]{img/07/birthday.2.mps}
    \caption{Zovšeobecnený narodeninový útok pre $k=8$}
\end{figure}

Zatiaľ tento útok generuje iba $k$-tice pre ktoré platí: $x_1 \oplus x_2 \oplus \dots \oplus x_k = 0$. 
Ak chceme dosiahnúť rovnosť $x_1 \oplus x_2 \oplus \dots \oplus x_k = c$, stačí zmeniť $L_k$ na $L_k^{'}$
nasledovne: $L_k^{'} = \{x_k \oplus c | x_k \in L_k\}$. Potom keď nájdeme $x_1 \oplus x_2 \oplus \dots \oplus x_k^{'} = 0$,
tak $x_k^{'}$ zameníme za príslušné $x_k$ a máme splnenú podmienku, ktorú sme chceli.

Zároveň toto ukazuje, že pokiaľ chceme hľadať kolíziu pre nejaké $k^{'} > k$, tak to vieme
určite tak rýchlo ako pre $k$. Vyberme si nejaké $x_{k+1}, x_{k+2}, \dots, x_{k^{'}}$ a položme
$c = x_{k+1} \oplus x_{k+2} \oplus \dots \oplus x_{k^{'}}$. A potom nájdeme kolíziu pre prvých $k$ zoznamov. 

Takto by sme vedeli útočiť napríklad už na XOR-HASH. Vygenerujeme zoznamy a nájdeme kolíziu, ktorá nám po
vyxorovaní dá požadovanú hašovaciu hodnotu. Pre XOR to ale žiadny pokrok nie je, keďže vieme robiť
útok s ďaleko lepšou zložitosťou.

Útok sa dá uplatniť aj na Ad-Hash.
Pri grupe $(Z_{2^n}, +)$ bude join vyzerať nasledovne: $L_1 \bowtie_l L_2 = \{(x_1, x_2) | x_1 \in L_1 \land x_2 \in L_2 \land low_l(x_1 + x_2) = 0\}$.

Takto vieme zaútočiť na NASD schému. Napríklad pokiaľ by schéma mal 128 blokov, tak by čas útoku bol: $128 \cdot 2^{256/8} = 2^{39}$.

Pre grupu $(Z_m, +)$ použijeme nasledovný join: $L_1 \bowtie_{[a,b]} L_2 = \{(x_1, x_2) | x_1 \in L_1 \land x_2 \in L_2 \land x_1 + x_2 \in <a,b>\}$.


Pokiaľ chceme útočiť na PCIHF schému, tak najprv si zvolíme napevno hodnoty $x_2, x_4, x_6, \dots$ a následne dopočítame vhodné hodnoty medzi nimi.


\section{Bezpečnosť trojitého šifrovania}

Uvažujme trojité šifrovanie pomocou dvoch kľúčov v EDE móde.
Čiže $TE_{k_1,k_2} (x) = E_{k_1}(D_{k_2}(E_{k_1}(x)))$.
Predpokladajme, že oba kľúče $k_1, k_2$ majú rovnakú veľkosť a to
menovite $n$.
V krypto I sme hovorili o tom, že takéto šifrovanie je odolné voči
útoku so znalosťou otvoreného textu (KPA).
Ukazuje sa však, že voľba iba 2 kľúčov má nepríjemne vlastnosti pri
CPA -- útokom s možnosťou voľby otvoreného textu.

Jeden taký možný útok si ukážeme:
\begin{itemize}
    \item Predpripravme si hashovaciu tabuľku
        $\langle m \mapsto \{k^{(1)}, k^{(2)}, \dots \} \rangle$,
        kde správe $m$ priradíme také hodnoty kľúčov $k^{(i)}$,
        aby platilo $E_{k^{(i)}}(m)=0$. Táto tabuľka nám bude slúžiť
        na hľadanie kľúčov $k_2$.
        Tabuľku môžeme predpripraviť napríklad tak, že pre každý možný
        kľúč $k$ vypočítame $m=D_k(0)$. Predpríprava bude zaberať čas
        $O(2^n)$.
    \item Pointou celého útoku bude, že sa začneme venovať takým
        dvojiciam otvoreného a šifrového textu, pre ktoré po prvom kroku
        trojitého šifrovania dostaneme číslo 0.
        \begin{figure}[h]
            \centering
            \includegraphics{img/08/triple.1.mps}
            \caption{Trojité šifrovanie}
            \label{fig:triple}
        \end{figure}
        Preberajme teraz všetkých kandidátov $k_1$ na prvý kľúč.
        Vypočítajme plaintex ako $p=D_{k_1}(0)$.
        Následne vypočítame šifrový text ako $c=TE_{k_1,k_2}(p)$.
        Pretože vieme $k_1$, vráťme sa o 1 krok vo výpočte šifrového
        textu, dostávame $z=D_{k_1}(c)$.
        Teraz nám ale nič nebráni pozrieť sa do našej hashovacej
        tabuľky -- chceme nájsť kľúč $k_2$, taký, že
        $D_{k_2}(0) = z$, čiže pozrieme zoznam kľúčov v hashovacej
        tabuľke pri hodnote $z$.
        Zložitosť tejto časti útoku je teda tiež $O(2^n)$ (za
        predpokladu, že lookup hashu je v $O(1)$).
        Výstupom je teda množina dvojíc potenciálnych kľúčov
        $(k_1,k_2)$, ktorými sa môže šifrovať.
        Jej očakávaná veľkosť bude $O(2^n)$.
        \begin{figure}[h]
            \centering
            \includegraphics{img/08/triple.2.mps}
            \caption{Útok na trojité šifrovanie}
            \label{fig:triple-attack}
        \end{figure}
\end{itemize}
Máme teda útok, ktorého časová zložitosť je $O(2^n)$, čo je menej ako
očakávaných $O(2^{2n})$. V praxi sa samozrejme daný útok nedá
realizovať, pokiaľ nám naša obeť nie je ochotná zašifrovať zhruba
$O(2^n)$ otvorených textov. Morálnym poučením teda bude, že trojité
šifrovanie môže byť síce jednoduchá a pomerne úspešná forma zosilnenia
šifrovania pre praktické účely, pre teoretické účely je to ale
nevhodná konštrukcia.

\section{Eliptické krivky}

Tento často počuteľný výraz znie veľmi matematicky. Napriek tomu si
ukážeme, že je to pomerne jednoduchý spôsob, ako generovať istý týp
grúp, v ktorých budeme vedieť následne počítať známe kryptografické
konštrukcie.

\begin{poznamka}
    Osobný názor doc. Staneka je, že eliptické krivky nahradia do
    niekoľko rokov RSA. Napríklad odporúčania na šifrovanie vládnych
    dokumentov v USA sa už ani nezmieňujú o používaní RSA. Nevýhodou
    RSA je totiž veľmi dlhý modulus pri ekvivalentnej bezpečnosti.
    Môžeme teda povedať, že diskrétny logaritmus začne vládnuť svetu
    :-)
\end{poznamka}

Ešte predtým, ako si povieme, čo to vlastne tieto krivky sú, uvedieme
si najväčšiu výhodu -- pre sofistikované algoritmy na
výpočet diskrétneho algoritmu ako napríklad general number field
sieve, index calculus nepoznáme v súčasnosti úpravy, ktoré by umožnili
ich aplikáciu na eliptické krivky. Dajú sa teda použiť iba generické
algoritmy. Tým pádom môžeme používať menšie kľúče, budeme mať menšie
podpisy, ... Hneď ale pripojíme aj varovanie -- existujú špecifické
útoky na isté typy eliptických kriviek, preto je potrebné venovať
generovaniu krivky dostatočnú pozornosť.

\begin{definicia}[Eliptická krivka]
    Pod eliptickou krivkou nad poľom $F$ budeme označovať množinu
    všetkých takých bodov, ktoré vyhovujú rovnici
    \begin{equation*}
        x^3 + A x + B = y^2
    \end{equation*}
    kde $A,B$ sú vopred zvolené konštanty.
    Pre body $(x,y)$ si zavedieme grupovú operáciu sčítania ``+''
    a neutrálny prvok $0$, ktorý bude predstavovať bod v nekonečne.
\end{definicia}

Ak si zoberieme pole $F=R$, eliptická krivka $x^3 -3x+3=y^2$ je
zobrazená na obrázku \ref{fig:elliptic1}

\begin{figure}[h]
    \centering
    \includegraphics{img/09/elliptic.1.mps}
    \caption{Eliptická krivka $x^3 - 3x +3 = y^2$ v reálnych číslach}
    \label{fig:elliptic1}
\end{figure}

Ešte predtým, ako si ukážeme sčítanie bodov, ktoré nie je úplne
triviálne, budeme sa zaoberať podmienkou na to, aby daná krivka bola
regulárna.

Predstavme si, že krivka je singulárna, t.j. existuje dvojitý koreň,
označme ho $a$ a označme tretí koreň $b$ (neplatí nutne $a\ne b$).
Dostávame
\begin{align*}
    (x-a)^2 (x-b) &= x^3 + Ax + B \\
    x^3 + x^2 (-b -2a) + x (2ab + a^2) + (-b a^2) &= x^3 + Ax + B
\end{align*}
Z koeficientu pred $x^2$ dostávame $b=-2a$ a teda
\begin{align*}
    A &= 2ab + a^2 = -3 a^2 \\
    B &= -b a^2 = 2 a^3
\end{align*}
Čo nastáva v prípade, ak
\begin{equation*}
    4 A^3 + 27 B^2 = 0
\end{equation*}
Pri generovaní kriviek budeme teda musieť otestovať túto rovnice a v
prípade rovnosti generovať novú krivku.

Poďme ale späť k sčítavaniu
\begin{definicia}[Sčítanie bodov eliptickej krivky]
    Majme body $P=[x_1,y_1]$ a $Q=[x_2,y_2]$.\footnote{Body
        eliptických kriviek zvykneme označovať veľkými písmenami}
    Nech $P \ne -Q$, kde znamienkom ``-'' budeme označovať
    bod $(x,-y)$ (Prípad $P = -Q$ definujeme ako $P+Q=0$).
    Potom operáciu sčítania $P+Q$ definujeme nasledovne:

    \begin{equation*}
        P+Q = [ \underbrace{\lambda^2 - x_1 - x_2}_{x_3}, \ 
            \lambda (x_1 - x_3) - y_1]
    \end{equation*}
    kde $\lambda$ je definované ako
    \begin{equation*}
        \lambda =
            \begin{cases}
                (y_2 -y_1)(x_2 - x_1)^{-1} \quad & P \ne Q \\
                (3 x_1^2 + A)(2 y_1)^{-1} \quad & P = Q
            \end{cases}
    \end{equation*}
\end{definicia}

Budeme tvrdiť, že množina bodov $\{(x,y)\} \union 0$
eliptickej krivky spolu s operáciu ``+'' bude tvoriť grupu. Toto
netriviálne algebraické tvrdenie prenecháme na neveriaceho
čitateľa. Ak teda zoberieme ako $F$ nejaké konečné
pole, dostávame konečnú grupu.

Predchádzajúca definícia sčítania možno nebola veľmi intuitívna.
Ukážeme si teda iný prístup k definícii sčítania bodov na eliptickej
krivke. Spôsob je to čisto grafický.
Ak máme body $P,Q$, body $-P, 2P, P+Q$ získame postupne ako
zrkadlenie bodu $P$ podľa osi $x$,
zkradlenie priesečníka dotyčnice v bode $P$ s krivkou
a zkradlenie tretieho priesečníka priamky prechádzajúcej bodmi $P,Q$ a
eliptickej krivky. Názorne to možno vidieť na obrázku
$\ref{fig:elliptic-plus}$.

\begin{figure}
    \centering
    \subfigure[$-P$]{
        \includegraphics{img/09/elliptic.2.mps}
    }
    \subfigure[$2P$]{
        \includegraphics{img/09/elliptic.3.mps}
    }
    \subfigure[$P+Q$]{
        \includegraphics{img/09/elliptic.4.mps}
    }

    \caption{Grafické sčítanie na eliptických krivkách}
    \label{fig:elliptic-plus}
\end{figure}

\begin{priklad}
    Uvedieme si ilustratívny príklad eliptickej krivky. Uvažujme
    rovnicu
    \begin{equation*}
        y^2 = x^3 + x + 1 \pmod{11}
    \end{equation*}
    Vypočítajme body vyhovujúce krivke:

    \begin{table}[h!]
        \centering
        \begin{tabular}{c|c|c}
            $x$ & $x^3 + x + 1 \pmod{11}$ & $y$ \\
            \hline 0 & 1 & 1, 10 \\
            \hline 1 & 3 & 5, 6 \\
            \hline 2 & 0 & 0 \\
            \hline 3 & 9 & 3, 8 \\
            \hline 4 & 3 & 5, 6 \\
            \hline 5 & 4 & 2, 9 \\
            \hline 6 & 3 & 5, 6 \\
            \hline 7 & 10 & - \\
            \hline 8 & 4 & 2, 9 \\
            \hline 9 & 2 & - \\
            \hline 19 & 10 & - \\
        \end{tabular}
        \caption{Body ležiace na eliptickej krivke $x^3 + x + 1
        \pmod{11}$}
    \end{table}
    Dostávame teda, že množina bodov našej krivky je
    \begin{equation*}
       E=\Big\{ [0,1], [0,10], [1,5], [1,6],
                [2,0], [3,3], [3,8],
                [4,5], [4,6], [5,2], [5,9],
                [6,5], [6,6], [9,2], [9,9], 0 \Big\}
    \end{equation*}
    Príklad niektorých sčítaní:
    \begin{align*}
        [2,0] + [2,0] &= 0 \\
        [1,5] + [1,6] &= 0 \\
        [0,1] + [3,3] &= [6, 6] 
    \end{align*}
\end{priklad}

Dostali sme sa do stavu, že máme grupu bodov eliptickej krivky. Toto
samo o sebe je síce pekné, ale na naše účely to nestačí. To, čo by sme
momentálne potrebovali vedieť je, že aká veľká daná grupa je (koľko má
prvkov) a či obsahuje malé podgrupy.

Druhá vlastnosť nás nezaujíma len tak zo zvedavosti -- Pohlig Hellmanov
algoritmus, ktorý dobre poznáme by v tomto prípade narobil paseku a
bezpečnosť schémy by bola ohrozená. Preto by sme boli najradšej, ak by
sme mali prvočíselnú grupu. Začneme však postupne:

\begin{veta}[\fixme{nazov}]
    Nech $E$ je grupa bodov eliptickej krivky $\pmod{p}$.
    Potom platí
    \begin{equation*}
        P+1 - 2 \sqrt{P} \le |E| \le P + 1 + 2 \sqrt{P}
    \end{equation*}
\end{veta}

Dá sa teda garantovať, že počet bodov eliptickej krivky je dostatočne
veľký, no o veľkých prvočíselných faktoroch nám to nič nepovie.
Na pomoc príde \fixme{meno} Schufov algoritmus, ktorý v čase
$O((\log{p})^6)$ vie presne určiť počet bodov krivky. Problém je však
v tom, že na praktické účely je aj toto priveľa -- v roku 2001 vraj
výpočet pre 200 bitové čísla trval niekoľho hodín. Teda, v súčasnosti
nie je efektívne počítať veľkosť grupy tvojenej bodmi eliptickej
krivky, minimálne nie na takej úrovni ako sa používa RSA napríklad pri
zabezpečovaní secure http.

Ako teda, že sa eliptické krivky používajú v praxi? Odpoveď je
jednoduchá -- niekto si dal záležať a našiel takú grupu, že $|E|$ je
prvočíslo. Napríklad, v štandarde DSS (Digital Signature standard), v
časti o ECDSA (Elliptic Curve DSA) sa explicitne uvádzajú krivky,
ktoré sa majú používať. Tieto krivky majú napevno zvolené $A=-3$ a $B$
bolo volené tak, aby $|E|$ bolo prvočíslo -- napríklad tak, že sa
postupne preberali možné parametre $B$ a $p$.

\begin{poznamka}[Pre fanúšikov algebry]
    Dá sa ukázať, že grupa $(E,+)$ je izomorfná so
    $Z_{n_1} \times Z_{n_2}$, kde $n_2$ delí $n_1$, $n_2$ delí $p-1$ a
    $n_1,n_2$ sú jednoznačne určené.
    Navyše, vhodnou voľbou parametrov sa dá dosiahnuť, že
    $n_2=1$, t.j. dostávame izomorfizmus so $Z_p^*$. Prirodzenou
    otázkou v takomto prípade je, či izomorfizmus nezmenšuje
    bezpečnosť -- mohli by sa dať aplikovať klasické algoritmy na
    faktorizáciu aj na eliptické krivky. \fixme{ako to vlastne je?}
\end{poznamka}

Teraz si predstavíme algoritmus ECDSA určený na digitálne podpisovanie
správ.

Začneme parametrami, ktoré sa zvyknú uvádzať v tabuľkách eliptických
kriviek:
\begin{itemize}
    \item $A$
    \item $B$
    \item $p$ -- prvočíselný modulus
    \item $FR$ -- field representation -- buď ako polynómy nad
        $GF(2^m)$ (binárne polia)
        alebo ako mocniny generátora (prvočíselné polia)
        alebo Koblitzove krivky.
    \item $seed$ -- keď niekto neverí vygenerovaným krivkám, môže si
        overiť, že boli vygenerované podľa tohoto seedu.
    \item $G$ -- generátor prvočíselnej podgrupy, je to bod patriaci krivke
    \item $n$ -- rád $G$ (generátora podgrupy)
    \item $k$ -- kofaktor, $k=|E| / n$
\end{itemize}
\begin{poznamka}
    V štandarde sa používajú krivky s kofaktorom $k=1,2,4$, teda (ako
    sa ukáže), dĺžka verejného a privátneho kľúča je približne
    rovnaká.
\end{poznamka}

Čo sa týka prvočíselného modula $p$, štandard prechádza plejádou
hodnôt od 192 po 521 bitov. \fixme{nie 512?}
Príklady:
\begin{itemize}
    \item $2^{192} - 2^{64} - 1$ 
    \item $2^{384} ?? 2^{126} ...$ \fixme{}
    \item $2^{521} - 1$.
\end{itemize}
Vo všeobecnosti sa volia takmer mersenovské prvočísla.

Ale vráťme sa k samotnému podpisovaniu. Prvou častou je generovanie
kľúčov:
\begin{procedure}
    \caption{GenECDSA()}
    \label{proc:genecdsa}
    $sk = d \inr \{1, \dots, n-1\}$ \;
    $pk = Q = d*G = \underbrace{G+G+G+\dots+G}_{d \times}$ \;
\end{procedure}

\todo{zvyšok ECDSA}

\todo{ElGamal v eliptickych}

\todo{zvysok co Stanek rozpraval a nemam v poznamkach}

\input{tex/10identity.tex}
\section{Generátory pseudonáhodných čísel}

Cieľom pseudonáhodných generátorov je vytvárať postupnosti čísel, ktoré sa
navonok javia ako náhodné, hoci v skutočnosti to tak nie je. Dôvodov prečo
by sme chceli niečo takéko je niekoľko, najdôležitejšími sú neprístupnosť
náhodných generátorov a/alebo ich slabá výkonnosť. Ak máme totiž naozaj
náhodný generátor, ktorý ale generuje len 1 bit za sekundu, na
vygenerovanie RSA kľúča by sme potrebovali čakať vyše hodinu.

To, že (pseudo)náhodné čísla potrebujeme v kryptografii je evidentné --
stačí si spomenúť na kryptografické kľúče, príležitostné slová,
inicializačné vektory blokových šifier, náhodné prvky v asymetrických
šifrovacích schémach a schémach na digitálne podpisy, náhodných paddingoch.
Navyše, v mnohých z týchto aplikácii nenáhodnosť ohrozuje bezpečnosť (od
možnosti dešifrovať správu ako pri \todo{} až po úplné prezradenie
súkromného kľúča ako pri \todo{}).

My za preudonáhodný generátor budeme považovať deterministický\footnote{
čo samozrejme priamo znamená, že nenáhodný} algoritmus, ktorý z
počiatočného ``seedu'' vygeneruje dlhú postupnosť. Bez ujmy na strate
všeobecnosti, budeme sa venovať len pseudonáhodným generátorom bitov, čiže
generátorom do postupnosti zloženej z prvkov $\{0,1\}$.

\subsection{Fyzikálne generátory náhodných čísel}
Existuje veľa rôznych spôsobov, ako zkonštruovať generátor náhodných čísel
na základe určitého fyzikálneho javu. Či už ide o jednoduché hádzanie
mince, meranie šumu v polovodičových prvkoch, klopné obvody, ...

Tieto generátory ale môžu byť zaťažené istými chybami -- môžu mať odchýlku
alebo rôzne závislosti medzi jednotlivými vygenerovanými bitmi. Na riešenie
tejto situácie existujú rôzne tzv. ``korektory''. Prirodzene ale musí byť
známe, akou chybou daný generátor trpí.

Uveďme si jednoduchý príklad
\begin{priklad}[Korektor na jednoduchú odchýlku v pravdepodobnosti]
  Uvažujme náhodný generátor bitov, v ktorom sú všetky vygenerované bity
  nezávislé, môže sa ale stať, že bit 0 generujeme s inou pravdepodobnosťou
  ako bit 1.

  Existuje jednoduchý Van Neumanov\fixme{spelling} korektor, ktorý rieši
  daný problém nasledovne: Zakaždým zoberme dvojicu bitov $r_{2i},r_{2i+1}$
  a na výstup dajme nasledovné:
  \begin{itemize}
    \item[00] - nevypíšem nič, opakuj s ďalšou dvojicou bitov
    \item[11] - nevypíšem nič, opakuj s ďalšou dvojicou bitov
    \item[01] - 0
    \item[10] - 1
  \end{itemize}
  Pretože bity sú nezávislé, evidentne pravdepodobnosť dvojice 01 a 10 je
  rovnaká. Problémom tejto konštrukcie je ale nejasná priepustnosť --
  môže sa nám stať, že zaradom zahodíme veľa dvojíc. Taktiež, ak je
  vyváženosť narušená dosť výrazne, korekcia je neefektívna pretože
  dominantnou dvojicou bude 00 alebo 11 a budeme teda zahadzovať väčšinu
  vygenerovaných bitov.
\end{priklad}


\section{CPA odolnosť symetrických šifier}

Majme nejaký symetrický šifrovací systém. Formálne nech:
$\Pi = <Gen(1^n); E_k(\cdot); D_k(\cdot)>$, kde $Gen$ je generátor
kľúča s dĺžkou $n$, $E_k(\cdot)$ je šifrovacia transformácia (môže byť
pravdepodobnostná a nakoniec sa ukáže, že je dobré aby aj bola)
a $D_k(\cdot)$ je dešifrovacia transformácia. Samozrejme, požadujeme, aby
systém bol korektný teda $\forall x, k\colon D_k(E_k(x))=x$.

Chceme, aby systém bol odolný voči útoku s možnosťou voľby otvoreného textu.
Teda útočník má k dispozícií orákulum, ktoré mu umožňuje zašifrovať ľubovoľný
text, ktorý si vyberie. Na to, aby sme ukázali, že systém je odolný vočí 
útoku s voľbou otvoreného textu, tak je dobré ukázať, že útočník
nevie rozlišovať šifrové texty, teda:

\begin{definicia}
Šifrovací systém $\Pi$ má nerozlíšiteľné šifrové texty pri $CPA$, ak platí:
$$Pr[PrivK_{A,\Pi}^{CPA}(n) = 1] \leq \frac{1}{2} + negl(n)$$

Kde $PriK_{A,\Pi}^{CPA}$ je experiment s útočníkom $A$ na šifre $\Pi$ definový nasledovne:
\begin{enumerate}
\item $K = Gen(1^n)$ - vygenerujeme nejaký kľúč
\item Útočníkový dáme šifrovacie orákulum, to si označme ako $A^{E_k(\cdot)}$. Následne $A^{E_k(\cdot)} \to (m_0,m_1)$ -
útočník vygeneruje 2 otvorené texty (navyše požadujeme $|m_0| = |m_1|$). 
\item Zvolíme $b \inr \{0,1\}$ a položíme $c = E_k(m_b)$, to pošleme útočníkovi.
\item Útočník sa pokúsi rozlíšiť šifrový text $A^{E_k(\cdot)}(c) \to b^{'} \in \{0,1\}$.
\item Ak $b=b{'}$, tak experiment vráti $1$, ináč $0$.
\end{enumerate}
\end{definicia}

Všimnime si, že nijako nezakazujeme útočníkovi sa priamo spýtať na zašifrovanie $m_0$, resp. $m_1$. 
Preto $E_k$ nemôže byť deterministický. 

\subsection{Pseudonáhodné funkcie}

Podobne ako sme chceli pseudonáhodným generátorom konštruovať
postupnosť, ktorá sa čo najviac podobala na náhodnú, tak tu
chceme zostrojiť funkciu, ktorá sa čo najviac podobná na náhodnú. 
Tá má následné mnohostranné využitie, napr. pri konštrukciu symetrických šifier.

Pseudonáhodnú funkciu definujeme podobne ako pseudonáhodnú postupnosť (nevieme ju
v pravdepodobnostnom polynomiálnom čase odlíšiť od náhodnej). 
Navyše to bude funkcia s kľúčom. 
Teda $F: \{0,1\}^* \times \{0,1\}^* \to \{0,1\}^*$. Pre jednoduchosť, nech dĺžky
kľúča, správý a výstupu funkciu sú rovnaké, teda $n = |k| = |x| = |F_k(x)|$.

\begin{definicia}
Funkcia $F_k(x)$ je pseudonáhodná ak $\forall PPT D \exists negl(\cdot)\colon$
$$|Pr[D^{F_k} = 1] - Pr[D^f = 1]| \leq negl(n)$$
kde: $k \inr \{0,1\}^n; f \inr \{ \{0,1\}^n \to \{0,1\}^n\}$.
\end{definicia}

Pokiaľ máme pseudonáhodný generátor $G$, tak vieme pomocou neho zostrojiť pseudonáhodnú
funkciu nasledovne:
Nech $|G(x)=2n|$ a $G(x) = G_0(x)||G_1(x)$. Potom $F_k(x_1 x_2 \dots x_n) = G_{x_n}(\dots(G_{x_2}(G_{x_1}(k)))\dots)$.

Teraz si ukážeme ako pomocou pseudonáhodnej funkciu zostrojiť symetrickú šifru $\Pi$, ktorá má nerozlíšiteľné
šifrové texty pri CPA útoku:\\
$Gen(1^n)\colon k \inr \{0,1\}^n$\\
$E_k(m)\colon <r, F_k(r) \oplus m> m \in \{0,1\}^n, r \inr \{0,1\}^n$\\
$D_k(<r,s>) = F_k(r) \oplus s$

\begin{veta}
$\Pi$ má nerozlíšiteľné šifrové texty pri CPA útoku, ak $F_k(\cdot)$ je pseudonáhodná.
\end{veta}

\begin{dokaz}
Dôkaz rozdelíme na 2 kroky.

{\bf 1. krok:} Zostrojíme šifrovací systém $\tilde{\Pi}$, kde miesto $F_k(\cdot)$ použijeme
náhodnú funkciu $f$. Ukážeme, že tento systém má nerozlíšiteľný šifrové texty pri CPA útoku.
Útočník teda dostane zašifrovaný text $<r_c, f(r_c) \oplus m_b>$.
Môžu nastať 2 situácie. Buď $r_c$ bolo použité v odpovediach na útočníkové otázky alebo nie. Ak bolo, tak túto situáciu
označme ako $Repeat$. Počet otázok označme ako $q(n)$, keďže útočník pracuje v polynomiálnom čase, aj
tento počet je polynomiálny. Potom šanca na úspech útočníka je nasledovná:
$$Pr[PrivK_{A,\tilde{\Pi}}^{CPA}(n)=1] = Pr[PrivK_{A,\tilde{\Pi}}^{CPA}(n)=1\land Repeat]  + Pr[PrivK_{A,\tilde{\Pi}}^{CPA}(n)=1 \land \neg Repeat]$$

Vieme, že $Pr[PrivK_{A,\tilde{\Pi}}^{CPA}(n)=1\land Repeat] \leq Pr[Repeat] = \frac{q(n)}{2^n} = neql(n)$.
A navyše $Pr[PrivK_{A,\tilde{\Pi}}^{CPA}(n)=1\land \neg Repeat] = \frac{1}{2}$, lebo v tomto prípade útočník
nevie nič a môže len tipovať, teda celkovo: $Pr[PrivK_{A,\tilde{\Pi}}^{CPA}(n)=1] \leq \frac{1}{2} + negl(n)$, čo sme chceli dokázať.

{\bf 2. krok:} Ukážeme, že ak $\Pi$ nie je bezpečná, tak vieme rozlišovať medzi $F_k(\cdot)$ a $f$.
Nech útočník $A$ má nezanedbateľnú pravdepodobnosť úspechu $\frac{1}{2}+\epsilon(n)$. 
Potom náš rozlišovač $D$ bude simulovať $A$ pričom pri šifrovaní použije funkciu, ktorú sa práve 
snaží odlíšiť. A vráti $1$ ak útočník uspel a $0$ ak neuspel.
Potom platí: $Pr[D^{F_k} = 1] - Pr[D^f = 1] = Pr[PrivK_{A,\Pi}^{CPA}(n)=1] - Pr[PrivK_{A,\tilde{\Pi}}^{CPA}(n)=1] 
= \epsilon(n) - negl(n)$, čo je nezanedbateľná šanca na úspech.
\end{dokaz}






\section{Lineárna a diferenciálna kryptoanalýza}

\begin{figure}[h!]
    \centering
    \includegraphics{img/13/cipher.1.mps}
    \caption{Nákres šifry}
\end{figure}

\begin{table}[htp]
\begin{tabular}{r|r|r|r|r|r|r|r|r|r|r|r|r|r|r|r|r|}
&{\bf  0}
&{\bf  1}
&{\bf  2}
&{\bf  3}
&{\bf  4}
&{\bf  5}
&{\bf  6}
&{\bf  7}
&{\bf  8}
&{\bf  9}
&{\bf 10}
&{\bf 11}
&{\bf 12}
&{\bf 13}
&{\bf 14}
&{\bf 15}
\\ \hline
{\bf  0}
    &{\color[rgb]{0.000000,0.000000,0.000000}  16}
    &{\color[rgb]{0.750000,0.750000,0.750000}   0}
    &{\color[rgb]{0.750000,0.750000,0.750000}   0}
    &{\color[rgb]{0.750000,0.750000,0.750000}   0}
    &{\color[rgb]{0.750000,0.750000,0.750000}   0}
    &{\color[rgb]{0.750000,0.750000,0.750000}   0}
    &{\color[rgb]{0.750000,0.750000,0.750000}   0}
    &{\color[rgb]{0.750000,0.750000,0.750000}   0}
    &{\color[rgb]{0.750000,0.750000,0.750000}   0}
    &{\color[rgb]{0.750000,0.750000,0.750000}   0}
    &{\color[rgb]{0.750000,0.750000,0.750000}   0}
    &{\color[rgb]{0.750000,0.750000,0.750000}   0}
    &{\color[rgb]{0.750000,0.750000,0.750000}   0}
    &{\color[rgb]{0.750000,0.750000,0.750000}   0}
    &{\color[rgb]{0.750000,0.750000,0.750000}   0}
    &{\color[rgb]{0.750000,0.750000,0.750000}   0}
\\ \hline
{\bf  1}
    &{\color[rgb]{0.750000,0.750000,0.750000}   0}
    &{\color[rgb]{0.000000,0.000000,0.000000}   8}
    &{\color[rgb]{0.350000,0.350000,0.350000}   4}
    &{\color[rgb]{0.350000,0.350000,0.350000}   4}
    &{\color[rgb]{0.750000,0.750000,0.750000}   0}
    &{\color[rgb]{0.750000,0.750000,0.750000}   0}
    &{\color[rgb]{0.350000,0.350000,0.350000}   4}
    &{\color[rgb]{0.350000,0.350000,0.350000}  -4}
    &{\color[rgb]{0.350000,0.350000,0.350000}  -4}
    &{\color[rgb]{0.350000,0.350000,0.350000}  -4}
    &{\color[rgb]{0.750000,0.750000,0.750000}   0}
    &{\color[rgb]{0.000000,0.000000,0.000000}   8}
    &{\color[rgb]{0.350000,0.350000,0.350000}  -4}
    &{\color[rgb]{0.350000,0.350000,0.350000}   4}
    &{\color[rgb]{0.750000,0.750000,0.750000}   0}
    &{\color[rgb]{0.750000,0.750000,0.750000}   0}
\\ \hline
{\bf  2}
    &{\color[rgb]{0.750000,0.750000,0.750000}   0}
    &{\color[rgb]{0.350000,0.350000,0.350000}  -4}
    &{\color[rgb]{0.350000,0.350000,0.350000}  -4}
    &{\color[rgb]{0.000000,0.000000,0.000000}  -8}
    &{\color[rgb]{0.350000,0.350000,0.350000}   4}
    &{\color[rgb]{0.750000,0.750000,0.750000}   0}
    &{\color[rgb]{0.750000,0.750000,0.750000}   0}
    &{\color[rgb]{0.350000,0.350000,0.350000}  -4}
    &{\color[rgb]{0.750000,0.750000,0.750000}   0}
    &{\color[rgb]{0.350000,0.350000,0.350000}  -4}
    &{\color[rgb]{0.350000,0.350000,0.350000}   4}
    &{\color[rgb]{0.750000,0.750000,0.750000}   0}
    &{\color[rgb]{0.350000,0.350000,0.350000}  -4}
    &{\color[rgb]{0.000000,0.000000,0.000000}   8}
    &{\color[rgb]{0.750000,0.750000,0.750000}   0}
    &{\color[rgb]{0.350000,0.350000,0.350000}  -4}
\\ \hline
{\bf  3}
    &{\color[rgb]{0.750000,0.750000,0.750000}   0}
    &{\color[rgb]{0.350000,0.350000,0.350000}  -4}
    &{\color[rgb]{0.000000,0.000000,0.000000}  -8}
    &{\color[rgb]{0.350000,0.350000,0.350000}  -4}
    &{\color[rgb]{0.350000,0.350000,0.350000}  -4}
    &{\color[rgb]{0.750000,0.750000,0.750000}   0}
    &{\color[rgb]{0.350000,0.350000,0.350000}   4}
    &{\color[rgb]{0.750000,0.750000,0.750000}   0}
    &{\color[rgb]{0.350000,0.350000,0.350000}  -4}
    &{\color[rgb]{0.750000,0.750000,0.750000}   0}
    &{\color[rgb]{0.350000,0.350000,0.350000}  -4}
    &{\color[rgb]{0.000000,0.000000,0.000000}   8}
    &{\color[rgb]{0.750000,0.750000,0.750000}   0}
    &{\color[rgb]{0.350000,0.350000,0.350000}  -4}
    &{\color[rgb]{0.750000,0.750000,0.750000}   0}
    &{\color[rgb]{0.350000,0.350000,0.350000}   4}
\\ \hline
{\bf  4}
    &{\color[rgb]{0.750000,0.750000,0.750000}   0}
    &{\color[rgb]{0.750000,0.750000,0.750000}   0}
    &{\color[rgb]{0.350000,0.350000,0.350000}   4}
    &{\color[rgb]{0.350000,0.350000,0.350000}  -4}
    &{\color[rgb]{0.350000,0.350000,0.350000}   4}
    &{\color[rgb]{0.350000,0.350000,0.350000}   4}
    &{\color[rgb]{0.000000,0.000000,0.000000}  -8}
    &{\color[rgb]{0.750000,0.750000,0.750000}   0}
    &{\color[rgb]{0.750000,0.750000,0.750000}   0}
    &{\color[rgb]{0.750000,0.750000,0.750000}   0}
    &{\color[rgb]{0.350000,0.350000,0.350000}  -4}
    &{\color[rgb]{0.350000,0.350000,0.350000}   4}
    &{\color[rgb]{0.350000,0.350000,0.350000}   4}
    &{\color[rgb]{0.350000,0.350000,0.350000}   4}
    &{\color[rgb]{0.750000,0.750000,0.750000}   0}
    &{\color[rgb]{0.000000,0.000000,0.000000}   8}
\\ \hline
{\bf  5}
    &{\color[rgb]{0.750000,0.750000,0.750000}   0}
    &{\color[rgb]{0.750000,0.750000,0.750000}   0}
    &{\color[rgb]{0.750000,0.750000,0.750000}   0}
    &{\color[rgb]{0.750000,0.750000,0.750000}   0}
    &{\color[rgb]{0.350000,0.350000,0.350000}   4}
    &{\color[rgb]{0.350000,0.350000,0.350000}  -4}
    &{\color[rgb]{0.350000,0.350000,0.350000}   4}
    &{\color[rgb]{0.350000,0.350000,0.350000}  -4}
    &{\color[rgb]{0.350000,0.350000,0.350000}  -4}
    &{\color[rgb]{0.350000,0.350000,0.350000}   4}
    &{\color[rgb]{0.350000,0.350000,0.350000}   4}
    &{\color[rgb]{0.350000,0.350000,0.350000}  -4}
    &{\color[rgb]{0.750000,0.750000,0.750000}   0}
    &{\color[rgb]{0.750000,0.750000,0.750000}   0}
    &{\color[rgb]{0.000000,0.000000,0.000000}   8}
    &{\color[rgb]{0.000000,0.000000,0.000000}   8}
\\ \hline
{\bf  6}
    &{\color[rgb]{0.750000,0.750000,0.750000}   0}
    &{\color[rgb]{0.350000,0.350000,0.350000}   4}
    &{\color[rgb]{0.750000,0.750000,0.750000}   0}
    &{\color[rgb]{0.350000,0.350000,0.350000}  -4}
    &{\color[rgb]{0.750000,0.750000,0.750000}   0}
    &{\color[rgb]{0.350000,0.350000,0.350000}   4}
    &{\color[rgb]{0.750000,0.750000,0.750000}   0}
    &{\color[rgb]{0.350000,0.350000,0.350000}  -4}
    &{\color[rgb]{0.750000,0.750000,0.750000}   0}
    &{\color[rgb]{0.350000,0.350000,0.350000}   4}
    &{\color[rgb]{0.750000,0.750000,0.750000}   0}
    &{\color[rgb]{0.350000,0.350000,0.350000}  -4}
    &{\color[rgb]{0.000000,0.000000,0.000000}  -8}
    &{\color[rgb]{0.350000,0.350000,0.350000}  -4}
    &{\color[rgb]{0.000000,0.000000,0.000000}  -8}
    &{\color[rgb]{0.350000,0.350000,0.350000}   4}
\\ \hline
{\bf  7}
    &{\color[rgb]{0.750000,0.750000,0.750000}   0}
    &{\color[rgb]{0.350000,0.350000,0.350000}  -4}
    &{\color[rgb]{0.350000,0.350000,0.350000}   4}
    &{\color[rgb]{0.750000,0.750000,0.750000}   0}
    &{\color[rgb]{0.000000,0.000000,0.000000}   8}
    &{\color[rgb]{0.350000,0.350000,0.350000}  -4}
    &{\color[rgb]{0.350000,0.350000,0.350000}  -4}
    &{\color[rgb]{0.750000,0.750000,0.750000}   0}
    &{\color[rgb]{0.350000,0.350000,0.350000}  -4}
    &{\color[rgb]{0.750000,0.750000,0.750000}   0}
    &{\color[rgb]{0.750000,0.750000,0.750000}   0}
    &{\color[rgb]{0.350000,0.350000,0.350000}   4}
    &{\color[rgb]{0.350000,0.350000,0.350000}  -4}
    &{\color[rgb]{0.000000,0.000000,0.000000}  -8}
    &{\color[rgb]{0.750000,0.750000,0.750000}   0}
    &{\color[rgb]{0.350000,0.350000,0.350000}  -4}
\\ \hline
{\bf  8}
    &{\color[rgb]{0.750000,0.750000,0.750000}   0}
    &{\color[rgb]{0.350000,0.350000,0.350000}   4}
    &{\color[rgb]{0.750000,0.750000,0.750000}   0}
    &{\color[rgb]{0.350000,0.350000,0.350000}  -4}
    &{\color[rgb]{0.350000,0.350000,0.350000}  -4}
    &{\color[rgb]{0.750000,0.750000,0.750000}   0}
    &{\color[rgb]{0.350000,0.350000,0.350000}  -4}
    &{\color[rgb]{0.000000,0.000000,0.000000}   8}
    &{\color[rgb]{0.750000,0.750000,0.750000}   0}
    &{\color[rgb]{0.350000,0.350000,0.350000}   4}
    &{\color[rgb]{0.000000,0.000000,0.000000}   8}
    &{\color[rgb]{0.350000,0.350000,0.350000}   4}
    &{\color[rgb]{0.350000,0.350000,0.350000}  -4}
    &{\color[rgb]{0.750000,0.750000,0.750000}   0}
    &{\color[rgb]{0.350000,0.350000,0.350000}   4}
    &{\color[rgb]{0.750000,0.750000,0.750000}   0}
\\ \hline
{\bf  9}
    &{\color[rgb]{0.750000,0.750000,0.750000}   0}
    &{\color[rgb]{0.350000,0.350000,0.350000}   4}
    &{\color[rgb]{0.350000,0.350000,0.350000}  -4}
    &{\color[rgb]{0.750000,0.750000,0.750000}   0}
    &{\color[rgb]{0.350000,0.350000,0.350000}   4}
    &{\color[rgb]{0.750000,0.750000,0.750000}   0}
    &{\color[rgb]{0.750000,0.750000,0.750000}   0}
    &{\color[rgb]{0.350000,0.350000,0.350000}  -4}
    &{\color[rgb]{0.350000,0.350000,0.350000}   4}
    &{\color[rgb]{0.750000,0.750000,0.750000}   0}
    &{\color[rgb]{0.000000,0.000000,0.000000}   8}
    &{\color[rgb]{0.350000,0.350000,0.350000}   4}
    &{\color[rgb]{0.000000,0.000000,0.000000}   8}
    &{\color[rgb]{0.350000,0.350000,0.350000}  -4}
    &{\color[rgb]{0.350000,0.350000,0.350000}  -4}
    &{\color[rgb]{0.750000,0.750000,0.750000}   0}
\\ \hline
{\bf 10}
    &{\color[rgb]{0.750000,0.750000,0.750000}   0}
    &{\color[rgb]{0.750000,0.750000,0.750000}   0}
    &{\color[rgb]{0.350000,0.350000,0.350000}  -4}
    &{\color[rgb]{0.350000,0.350000,0.350000}   4}
    &{\color[rgb]{0.750000,0.750000,0.750000}   0}
    &{\color[rgb]{0.750000,0.750000,0.750000}   0}
    &{\color[rgb]{0.350000,0.350000,0.350000}  -4}
    &{\color[rgb]{0.350000,0.350000,0.350000}   4}
    &{\color[rgb]{0.000000,0.000000,0.000000}  -8}
    &{\color[rgb]{0.000000,0.000000,0.000000}  -8}
    &{\color[rgb]{0.350000,0.350000,0.350000}   4}
    &{\color[rgb]{0.350000,0.350000,0.350000}  -4}
    &{\color[rgb]{0.750000,0.750000,0.750000}   0}
    &{\color[rgb]{0.750000,0.750000,0.750000}   0}
    &{\color[rgb]{0.350000,0.350000,0.350000}  -4}
    &{\color[rgb]{0.350000,0.350000,0.350000}   4}
\\ \hline
{\bf 11}
    &{\color[rgb]{0.750000,0.750000,0.750000}   0}
    &{\color[rgb]{0.000000,0.000000,0.000000}   8}
    &{\color[rgb]{0.750000,0.750000,0.750000}   0}
    &{\color[rgb]{0.000000,0.000000,0.000000}  -8}
    &{\color[rgb]{0.750000,0.750000,0.750000}   0}
    &{\color[rgb]{0.750000,0.750000,0.750000}   0}
    &{\color[rgb]{0.750000,0.750000,0.750000}   0}
    &{\color[rgb]{0.750000,0.750000,0.750000}   0}
    &{\color[rgb]{0.350000,0.350000,0.350000}  -4}
    &{\color[rgb]{0.350000,0.350000,0.350000}  -4}
    &{\color[rgb]{0.350000,0.350000,0.350000}  -4}
    &{\color[rgb]{0.350000,0.350000,0.350000}  -4}
    &{\color[rgb]{0.350000,0.350000,0.350000}   4}
    &{\color[rgb]{0.350000,0.350000,0.350000}  -4}
    &{\color[rgb]{0.350000,0.350000,0.350000}   4}
    &{\color[rgb]{0.350000,0.350000,0.350000}  -4}
\\ \hline
{\bf 12}
    &{\color[rgb]{0.750000,0.750000,0.750000}   0}
    &{\color[rgb]{0.350000,0.350000,0.350000}  -4}
    &{\color[rgb]{0.350000,0.350000,0.350000}   4}
    &{\color[rgb]{0.750000,0.750000,0.750000}   0}
    &{\color[rgb]{0.750000,0.750000,0.750000}   0}
    &{\color[rgb]{0.000000,0.000000,0.000000}  12}
    &{\color[rgb]{0.350000,0.350000,0.350000}   4}
    &{\color[rgb]{0.750000,0.750000,0.750000}   0}
    &{\color[rgb]{0.750000,0.750000,0.750000}   0}
    &{\color[rgb]{0.350000,0.350000,0.350000}  -4}
    &{\color[rgb]{0.350000,0.350000,0.350000}   4}
    &{\color[rgb]{0.750000,0.750000,0.750000}   0}
    &{\color[rgb]{0.750000,0.750000,0.750000}   0}
    &{\color[rgb]{0.350000,0.350000,0.350000}  -4}
    &{\color[rgb]{0.350000,0.350000,0.350000}   4}
    &{\color[rgb]{0.750000,0.750000,0.750000}   0}
\\ \hline
{\bf 13}
    &{\color[rgb]{0.750000,0.750000,0.750000}   0}
    &{\color[rgb]{0.350000,0.350000,0.350000}   4}
    &{\color[rgb]{0.000000,0.000000,0.000000}  -8}
    &{\color[rgb]{0.350000,0.350000,0.350000}   4}
    &{\color[rgb]{0.000000,0.000000,0.000000}   8}
    &{\color[rgb]{0.350000,0.350000,0.350000}   4}
    &{\color[rgb]{0.750000,0.750000,0.750000}   0}
    &{\color[rgb]{0.350000,0.350000,0.350000}   4}
    &{\color[rgb]{0.350000,0.350000,0.350000}   4}
    &{\color[rgb]{0.750000,0.750000,0.750000}   0}
    &{\color[rgb]{0.350000,0.350000,0.350000}  -4}
    &{\color[rgb]{0.750000,0.750000,0.750000}   0}
    &{\color[rgb]{0.350000,0.350000,0.350000}  -4}
    &{\color[rgb]{0.750000,0.750000,0.750000}   0}
    &{\color[rgb]{0.350000,0.350000,0.350000}   4}
    &{\color[rgb]{0.750000,0.750000,0.750000}   0}
\\ \hline
{\bf 14}
    &{\color[rgb]{0.750000,0.750000,0.750000}   0}
    &{\color[rgb]{0.750000,0.750000,0.750000}   0}
    &{\color[rgb]{0.750000,0.750000,0.750000}   0}
    &{\color[rgb]{0.750000,0.750000,0.750000}   0}
    &{\color[rgb]{0.350000,0.350000,0.350000}  -4}
    &{\color[rgb]{0.350000,0.350000,0.350000}  -4}
    &{\color[rgb]{0.350000,0.350000,0.350000}  -4}
    &{\color[rgb]{0.350000,0.350000,0.350000}  -4}
    &{\color[rgb]{0.000000,0.000000,0.000000}   8}
    &{\color[rgb]{0.000000,0.000000,0.000000}  -8}
    &{\color[rgb]{0.750000,0.750000,0.750000}   0}
    &{\color[rgb]{0.750000,0.750000,0.750000}   0}
    &{\color[rgb]{0.350000,0.350000,0.350000}  -4}
    &{\color[rgb]{0.350000,0.350000,0.350000}  -4}
    &{\color[rgb]{0.350000,0.350000,0.350000}   4}
    &{\color[rgb]{0.350000,0.350000,0.350000}   4}
\\ \hline
{\bf 15}
    &{\color[rgb]{0.750000,0.750000,0.750000}   0}
    &{\color[rgb]{0.750000,0.750000,0.750000}   0}
    &{\color[rgb]{0.350000,0.350000,0.350000}  -4}
    &{\color[rgb]{0.350000,0.350000,0.350000}   4}
    &{\color[rgb]{0.350000,0.350000,0.350000}  -4}
    &{\color[rgb]{0.350000,0.350000,0.350000}   4}
    &{\color[rgb]{0.000000,0.000000,0.000000}  -8}
    &{\color[rgb]{0.000000,0.000000,0.000000}  -8}
    &{\color[rgb]{0.350000,0.350000,0.350000}  -4}
    &{\color[rgb]{0.350000,0.350000,0.350000}   4}
    &{\color[rgb]{0.750000,0.750000,0.750000}   0}
    &{\color[rgb]{0.750000,0.750000,0.750000}   0}
    &{\color[rgb]{0.750000,0.750000,0.750000}   0}
    &{\color[rgb]{0.750000,0.750000,0.750000}   0}
    &{\color[rgb]{0.350000,0.350000,0.350000}   4}
    &{\color[rgb]{0.350000,0.350000,0.350000}  -4}
\\ \hline
\end{tabular}
\caption{Lineárna aproximačná tabuľka pre S-box}
\label{tab:lat}
\end{table}


\subsubsection{Príklad 1}
\begin{figure}[h!]
    \centering
    \includegraphics{img/13/lin.1.mps}
    \caption{Lineárna aproximácia v príklade 1}
\end{figure}
\begin{itemize}
\item {\bf 1. kolo:}
Použijeme lineárnu aproximáciu $
x_{1,9} \oplus x_{1,10} \oplus x_{1,12} \oplus x_{1,15}  \oplus k_{1} 
 \approx 
y_{1,10} \oplus y_{1,11} \oplus y_{1,14} \oplus y_{1,15} $,
ktorá má balancie po jednotlivých S-boxoch $
0.5,0.5,-0.25,0.25
$čo podľa piling-up lemy dáva pravdepodobnosť 
$1/2 + 2^3*( 0.5)*(0.5)*(-0.25)*(0.25 )= 0.375 $
Balancia je $-0.125$.

\item {\bf 2. kolo:}
Použijeme lineárnu aproximáciu $
x_{2,10} \oplus x_{2,14} \oplus x_{2,11} \oplus x_{2,15}  \oplus k_{2} 
 \approx 
y_{2,8} \oplus y_{2,10} \oplus y_{2,12} \oplus y_{2,14} $,
ktorá má balancie po jednotlivých S-boxoch $
0.5,0.5,0.375,0.375
$čo podľa piling-up lemy dáva pravdepodobnosť 
$1/2 + 2^3*( 0.5)*(0.5)*(0.375)*(0.375 )= 0.78125 $
Balancia je $0.28125$.

\item {\bf 3. kolo:}
Použijeme lineárnu aproximáciu $
x_{3,2} \oplus x_{3,10} \oplus x_{3,3} \oplus x_{3,11}  \oplus k_{3} 
 \approx 
y_{3,0} \oplus y_{3,2} \oplus y_{3,8} \oplus y_{3,10} $,
ktorá má balancie po jednotlivých S-boxoch $
0.375,0.5,0.375,0.5
$čo podľa piling-up lemy dáva pravdepodobnosť 
$1/2 + 2^3*( 0.375)*(0.5)*(0.375)*(0.5 )= 0.78125 $
Balancia je $0.28125$.

\item {\bf Spolu:}  Máme lineárnu kombináciu $ \Big(
in_{9} \oplus in_{10} \oplus in_{12} \oplus in_{15}
\Big) \oplus \Big( k_1 \oplus k_2 \oplus k_3 \oplus 
key_{0,9} \oplus key_{0,10} \oplus key_{0,12} \oplus key_{0,15} \oplus key_{1,10} \oplus key_{1,11} \oplus key_{1,14} \oplus key_{1,15} \oplus key_{2,2} \oplus key_{2,3} \oplus key_{2,10} \oplus key_{2,11} \oplus key_{3,0} \oplus key_{3,2} \oplus key_{3,8} \oplus key_{3,10} \Big) \approx \Big(
out_{0} \oplus out_{2} \oplus out_{8} \oplus out_{10}
\Big) $.
Podľa piling-up lemy máme balanciu $4* -0.125*0.28125*0.28125 ~= -0.0396 $.
\end{itemize}

\begin{table}[h]
 \centering
 \subtable[$key_5=cbd9$]{
     \begin{tabular}{cc}
         0.0640 & \_b\_f \\
         0.0510 & \_b\_9 \\
         0.0500 & \_e\_6 \\
         0.0480 & \_3\_f \\
         0.0440 & \_e\_4 \\
         0.0430 & \_e\_0 \\
         0.0430 & \_a\_e \\
         0.0420 & \_8\_0 \\
         0.0420 & \_e\_2 \\
         0.0410 & \_d\_9 \\
     \end{tabular}
 }
 \subtable[$key_5=f4aa$]{
     \begin{tabular}{cc}
         0.0510 & \_0\_a \\
         0.0430 & \_b\_7 \\
         0.0410 & \_d\_5 \\
         0.0410 & \_7\_5 \\
         0.0410 & \_4\_2 \\
         0.0400 & \_d\_b \\
         0.0400 & \_3\_5 \\
         0.0390 & \_3\_0 \\
         0.0390 & \_a\_0 \\
         0.0380 & \_3\_a \\
     \end{tabular}
 }
\end{table}


\subsubsection{Príklad 2}
\begin{figure}[h!]
    \centering
    \includegraphics{img/13/lin.2.mps}
    \caption{Lineárna aproximácia v príklade 2}
\end{figure}
\begin{itemize}
\item {\bf 1. kolo:}
Použijeme lineárnu aproximáciu $
x_{1,9} \oplus x_{1,10} \oplus x_{1,12} \oplus x_{1,15}  \oplus k_{1} 
 \approx 
y_{1,10} \oplus y_{1,11} \oplus y_{1,14} \oplus y_{1,15} $,
ktorá má balancie po jednotlivých S-boxoch $
0.5,0.5,-0.25,0.25
$čo podľa piling-up lemy dáva pravdepodobnosť 
$1/2 + 2^3*( 0.5)*(0.5)*(-0.25)*(0.25 )= 0.375 $
Balancia je $-0.125$.

\item {\bf 2. kolo:}
Použijeme lineárnu aproximáciu $
x_{2,10} \oplus x_{2,14} \oplus x_{2,11} \oplus x_{2,15}  \oplus k_{2} 
 \approx 
y_{2,8} \oplus y_{2,10} \oplus y_{2,12} \oplus y_{2,14} $,
ktorá má balancie po jednotlivých S-boxoch $
0.5,0.5,0.375,0.375
$čo podľa piling-up lemy dáva pravdepodobnosť 
$1/2 + 2^3*( 0.5)*(0.5)*(0.375)*(0.375 )= 0.78125 $
Balancia je $0.28125$.

\item {\bf 3. kolo:}
Použijeme lineárnu aproximáciu $
x_{3,2} \oplus x_{3,10} \oplus x_{3,3} \oplus x_{3,11}  \oplus k_{3} 
 \approx 
y_{3,0} \oplus y_{3,2} \oplus y_{3,8} \oplus y_{3,10} $,
ktorá má balancie po jednotlivých S-boxoch $
0.375,0.5,0.375,0.5
$čo podľa piling-up lemy dáva pravdepodobnosť 
$1/2 + 2^3*( 0.375)*(0.5)*(0.375)*(0.5 )= 0.78125 $
Balancia je $0.28125$.

\item {\bf Spolu:}  Máme lineárnu kombináciu $ \Big(
in_{9} \oplus in_{10} \oplus in_{12} \oplus in_{15}
\Big) \oplus \Big( k_1 \oplus k_2 \oplus k_3 \oplus 
key_{0,9} \oplus key_{0,10} \oplus key_{0,12} \oplus key_{0,15} \oplus key_{1,10} \oplus key_{1,11} \oplus key_{1,14} \oplus key_{1,15} \oplus key_{2,2} \oplus key_{2,3} \oplus key_{2,10} \oplus key_{2,11} \oplus key_{3,0} \oplus key_{3,2} \oplus key_{3,8} \oplus key_{3,10} \Big) \approx \Big(
out_{0} \oplus out_{2} \oplus out_{8} \oplus out_{10}
\Big) $.
Podľa piling-up lemy máme balanciu $4* -0.125*0.28125*0.28125 ~= -0.0396 $.
\end{itemize}

\begin{table}[h]
 \centering
 \subtable[$key_5=cbd9$]{
     \begin{tabular}{cc}
         0.0640 & \_b\_f \\
         0.0510 & \_b\_9 \\
         0.0500 & \_e\_6 \\
         0.0480 & \_3\_f \\
         0.0440 & \_e\_4 \\
         0.0430 & \_e\_0 \\
         0.0430 & \_a\_e \\
         0.0420 & \_8\_0 \\
         0.0420 & \_e\_2 \\
         0.0410 & \_d\_9 \\
     \end{tabular}
 }
 \subtable[$key_5=f4aa$]{
     \begin{tabular}{cc}
         0.0510 & \_0\_a \\
         0.0430 & \_b\_7 \\
         0.0410 & \_d\_5 \\
         0.0410 & \_7\_5 \\
         0.0410 & \_4\_2 \\
         0.0400 & \_d\_b \\
         0.0400 & \_3\_5 \\
         0.0390 & \_3\_0 \\
         0.0390 & \_a\_0 \\
         0.0380 & \_3\_a \\
     \end{tabular}
 }
\end{table}


\subsection{Diferenciálna kryptoanalýza}
\begin{table}[H]
\begin{tabular}{r|r|r|r|r|r|r|r|r|r|r|r|r|r|r|r|r|}
&{\bf  0}
&{\bf  1}
&{\bf  2}
&{\bf  3}
&{\bf  4}
&{\bf  5}
&{\bf  6}
&{\bf  7}
&{\bf  8}
&{\bf  9}
&{\bf 10}
&{\bf 11}
&{\bf 12}
&{\bf 13}
&{\bf 14}
&{\bf 15}
\\ \hline
{\bf  0}
    &{\color[rgb]{0.000000,0.000000,0.000000}  16}
    &{\color[rgb]{0.750000,0.750000,0.750000}   0}
    &{\color[rgb]{0.750000,0.750000,0.750000}   0}
    &{\color[rgb]{0.750000,0.750000,0.750000}   0}
    &{\color[rgb]{0.750000,0.750000,0.750000}   0}
    &{\color[rgb]{0.750000,0.750000,0.750000}   0}
    &{\color[rgb]{0.750000,0.750000,0.750000}   0}
    &{\color[rgb]{0.750000,0.750000,0.750000}   0}
    &{\color[rgb]{0.750000,0.750000,0.750000}   0}
    &{\color[rgb]{0.750000,0.750000,0.750000}   0}
    &{\color[rgb]{0.750000,0.750000,0.750000}   0}
    &{\color[rgb]{0.750000,0.750000,0.750000}   0}
    &{\color[rgb]{0.750000,0.750000,0.750000}   0}
    &{\color[rgb]{0.750000,0.750000,0.750000}   0}
    &{\color[rgb]{0.750000,0.750000,0.750000}   0}
    &{\color[rgb]{0.750000,0.750000,0.750000}   0}
\\ \hline
{\bf  1}
    &{\color[rgb]{0.750000,0.750000,0.750000}   0}
    &{\color[rgb]{0.750000,0.750000,0.750000}   0}
    &{\color[rgb]{0.416667,0.416667,0.416667}   2}
    &{\color[rgb]{0.750000,0.750000,0.750000}   0}
    &{\color[rgb]{0.750000,0.750000,0.750000}   0}
    &{\color[rgb]{0.416667,0.416667,0.416667}   2}
    &{\color[rgb]{0.416667,0.416667,0.416667}   2}
    &{\color[rgb]{0.416667,0.416667,0.416667}   2}
    &{\color[rgb]{0.750000,0.750000,0.750000}   0}
    &{\color[rgb]{0.750000,0.750000,0.750000}   0}
    &{\color[rgb]{0.750000,0.750000,0.750000}   0}
    &{\color[rgb]{0.416667,0.416667,0.416667}   2}
    &{\color[rgb]{0.750000,0.750000,0.750000}   0}
    &{\color[rgb]{0.416667,0.416667,0.416667}   2}
    &{\color[rgb]{0.750000,0.750000,0.750000}   0}
    &{\color[rgb]{0.083333,0.083333,0.083333}   4}
\\ \hline
{\bf  2}
    &{\color[rgb]{0.750000,0.750000,0.750000}   0}
    &{\color[rgb]{0.416667,0.416667,0.416667}   2}
    &{\color[rgb]{0.750000,0.750000,0.750000}   0}
    &{\color[rgb]{0.750000,0.750000,0.750000}   0}
    &{\color[rgb]{0.750000,0.750000,0.750000}   0}
    &{\color[rgb]{0.416667,0.416667,0.416667}   2}
    &{\color[rgb]{0.750000,0.750000,0.750000}   0}
    &{\color[rgb]{0.750000,0.750000,0.750000}   0}
    &{\color[rgb]{0.416667,0.416667,0.416667}   2}
    &{\color[rgb]{0.750000,0.750000,0.750000}   0}
    &{\color[rgb]{0.083333,0.083333,0.083333}   4}
    &{\color[rgb]{0.750000,0.750000,0.750000}   0}
    &{\color[rgb]{0.750000,0.750000,0.750000}   0}
    &{\color[rgb]{0.416667,0.416667,0.416667}   2}
    &{\color[rgb]{0.416667,0.416667,0.416667}   2}
    &{\color[rgb]{0.416667,0.416667,0.416667}   2}
\\ \hline
{\bf  3}
    &{\color[rgb]{0.750000,0.750000,0.750000}   0}
    &{\color[rgb]{0.750000,0.750000,0.750000}   0}
    &{\color[rgb]{0.750000,0.750000,0.750000}   0}
    &{\color[rgb]{0.750000,0.750000,0.750000}   0}
    &{\color[rgb]{0.750000,0.750000,0.750000}   0}
    &{\color[rgb]{0.416667,0.416667,0.416667}   2}
    &{\color[rgb]{0.750000,0.750000,0.750000}   0}
    &{\color[rgb]{0.416667,0.416667,0.416667}   2}
    &{\color[rgb]{0.083333,0.083333,0.083333}   4}
    &{\color[rgb]{0.416667,0.416667,0.416667}   2}
    &{\color[rgb]{0.416667,0.416667,0.416667}   2}
    &{\color[rgb]{0.750000,0.750000,0.750000}   0}
    &{\color[rgb]{0.750000,0.750000,0.750000}   0}
    &{\color[rgb]{0.750000,0.750000,0.750000}   0}
    &{\color[rgb]{0.416667,0.416667,0.416667}   2}
    &{\color[rgb]{0.416667,0.416667,0.416667}   2}
\\ \hline
{\bf  4}
    &{\color[rgb]{0.750000,0.750000,0.750000}   0}
    &{\color[rgb]{0.750000,0.750000,0.750000}   0}
    &{\color[rgb]{0.750000,0.750000,0.750000}   0}
    &{\color[rgb]{0.416667,0.416667,0.416667}   2}
    &{\color[rgb]{0.000000,0.000000,0.000000}   6}
    &{\color[rgb]{0.750000,0.750000,0.750000}   0}
    &{\color[rgb]{0.750000,0.750000,0.750000}   0}
    &{\color[rgb]{0.750000,0.750000,0.750000}   0}
    &{\color[rgb]{0.750000,0.750000,0.750000}   0}
    &{\color[rgb]{0.750000,0.750000,0.750000}   0}
    &{\color[rgb]{0.750000,0.750000,0.750000}   0}
    &{\color[rgb]{0.416667,0.416667,0.416667}   2}
    &{\color[rgb]{0.416667,0.416667,0.416667}   2}
    &{\color[rgb]{0.750000,0.750000,0.750000}   0}
    &{\color[rgb]{0.083333,0.083333,0.083333}   4}
    &{\color[rgb]{0.750000,0.750000,0.750000}   0}
\\ \hline
{\bf  5}
    &{\color[rgb]{0.750000,0.750000,0.750000}   0}
    &{\color[rgb]{0.416667,0.416667,0.416667}   2}
    &{\color[rgb]{0.750000,0.750000,0.750000}   0}
    &{\color[rgb]{0.416667,0.416667,0.416667}   2}
    &{\color[rgb]{0.750000,0.750000,0.750000}   0}
    &{\color[rgb]{0.416667,0.416667,0.416667}   2}
    &{\color[rgb]{0.416667,0.416667,0.416667}   2}
    &{\color[rgb]{0.750000,0.750000,0.750000}   0}
    &{\color[rgb]{0.416667,0.416667,0.416667}   2}
    &{\color[rgb]{0.416667,0.416667,0.416667}   2}
    &{\color[rgb]{0.750000,0.750000,0.750000}   0}
    &{\color[rgb]{0.083333,0.083333,0.083333}   4}
    &{\color[rgb]{0.750000,0.750000,0.750000}   0}
    &{\color[rgb]{0.750000,0.750000,0.750000}   0}
    &{\color[rgb]{0.750000,0.750000,0.750000}   0}
    &{\color[rgb]{0.750000,0.750000,0.750000}   0}
\\ \hline
{\bf  6}
    &{\color[rgb]{0.750000,0.750000,0.750000}   0}
    &{\color[rgb]{0.416667,0.416667,0.416667}   2}
    &{\color[rgb]{0.416667,0.416667,0.416667}   2}
    &{\color[rgb]{0.416667,0.416667,0.416667}   2}
    &{\color[rgb]{0.750000,0.750000,0.750000}   0}
    &{\color[rgb]{0.750000,0.750000,0.750000}   0}
    &{\color[rgb]{0.416667,0.416667,0.416667}   2}
    &{\color[rgb]{0.750000,0.750000,0.750000}   0}
    &{\color[rgb]{0.750000,0.750000,0.750000}   0}
    &{\color[rgb]{0.750000,0.750000,0.750000}   0}
    &{\color[rgb]{0.750000,0.750000,0.750000}   0}
    &{\color[rgb]{0.416667,0.416667,0.416667}   2}
    &{\color[rgb]{0.750000,0.750000,0.750000}   0}
    &{\color[rgb]{0.416667,0.416667,0.416667}   2}
    &{\color[rgb]{0.083333,0.083333,0.083333}   4}
    &{\color[rgb]{0.750000,0.750000,0.750000}   0}
\\ \hline
{\bf  7}
    &{\color[rgb]{0.750000,0.750000,0.750000}   0}
    &{\color[rgb]{0.416667,0.416667,0.416667}   2}
    &{\color[rgb]{0.750000,0.750000,0.750000}   0}
    &{\color[rgb]{0.416667,0.416667,0.416667}   2}
    &{\color[rgb]{0.416667,0.416667,0.416667}   2}
    &{\color[rgb]{0.750000,0.750000,0.750000}   0}
    &{\color[rgb]{0.416667,0.416667,0.416667}   2}
    &{\color[rgb]{0.750000,0.750000,0.750000}   0}
    &{\color[rgb]{0.750000,0.750000,0.750000}   0}
    &{\color[rgb]{0.750000,0.750000,0.750000}   0}
    &{\color[rgb]{0.416667,0.416667,0.416667}   2}
    &{\color[rgb]{0.416667,0.416667,0.416667}   2}
    &{\color[rgb]{0.416667,0.416667,0.416667}   2}
    &{\color[rgb]{0.416667,0.416667,0.416667}   2}
    &{\color[rgb]{0.750000,0.750000,0.750000}   0}
    &{\color[rgb]{0.750000,0.750000,0.750000}   0}
\\ \hline
{\bf  8}
    &{\color[rgb]{0.750000,0.750000,0.750000}   0}
    &{\color[rgb]{0.750000,0.750000,0.750000}   0}
    &{\color[rgb]{0.750000,0.750000,0.750000}   0}
    &{\color[rgb]{0.416667,0.416667,0.416667}   2}
    &{\color[rgb]{0.750000,0.750000,0.750000}   0}
    &{\color[rgb]{0.750000,0.750000,0.750000}   0}
    &{\color[rgb]{0.750000,0.750000,0.750000}   0}
    &{\color[rgb]{0.416667,0.416667,0.416667}   2}
    &{\color[rgb]{0.750000,0.750000,0.750000}   0}
    &{\color[rgb]{0.083333,0.083333,0.083333}   4}
    &{\color[rgb]{0.416667,0.416667,0.416667}   2}
    &{\color[rgb]{0.750000,0.750000,0.750000}   0}
    &{\color[rgb]{0.083333,0.083333,0.083333}   4}
    &{\color[rgb]{0.750000,0.750000,0.750000}   0}
    &{\color[rgb]{0.416667,0.416667,0.416667}   2}
    &{\color[rgb]{0.750000,0.750000,0.750000}   0}
\\ \hline
{\bf  9}
    &{\color[rgb]{0.750000,0.750000,0.750000}   0}
    &{\color[rgb]{0.416667,0.416667,0.416667}   2}
    &{\color[rgb]{0.750000,0.750000,0.750000}   0}
    &{\color[rgb]{0.416667,0.416667,0.416667}   2}
    &{\color[rgb]{0.750000,0.750000,0.750000}   0}
    &{\color[rgb]{0.750000,0.750000,0.750000}   0}
    &{\color[rgb]{0.750000,0.750000,0.750000}   0}
    &{\color[rgb]{0.750000,0.750000,0.750000}   0}
    &{\color[rgb]{0.416667,0.416667,0.416667}   2}
    &{\color[rgb]{0.750000,0.750000,0.750000}   0}
    &{\color[rgb]{0.750000,0.750000,0.750000}   0}
    &{\color[rgb]{0.416667,0.416667,0.416667}   2}
    &{\color[rgb]{0.083333,0.083333,0.083333}   4}
    &{\color[rgb]{0.750000,0.750000,0.750000}   0}
    &{\color[rgb]{0.416667,0.416667,0.416667}   2}
    &{\color[rgb]{0.416667,0.416667,0.416667}   2}
\\ \hline
{\bf 10}
    &{\color[rgb]{0.750000,0.750000,0.750000}   0}
    &{\color[rgb]{0.750000,0.750000,0.750000}   0}
    &{\color[rgb]{0.750000,0.750000,0.750000}   0}
    &{\color[rgb]{0.416667,0.416667,0.416667}   2}
    &{\color[rgb]{0.416667,0.416667,0.416667}   2}
    &{\color[rgb]{0.750000,0.750000,0.750000}   0}
    &{\color[rgb]{0.000000,0.000000,0.000000}   6}
    &{\color[rgb]{0.416667,0.416667,0.416667}   2}
    &{\color[rgb]{0.750000,0.750000,0.750000}   0}
    &{\color[rgb]{0.416667,0.416667,0.416667}   2}
    &{\color[rgb]{0.750000,0.750000,0.750000}   0}
    &{\color[rgb]{0.750000,0.750000,0.750000}   0}
    &{\color[rgb]{0.750000,0.750000,0.750000}   0}
    &{\color[rgb]{0.750000,0.750000,0.750000}   0}
    &{\color[rgb]{0.750000,0.750000,0.750000}   0}
    &{\color[rgb]{0.416667,0.416667,0.416667}   2}
\\ \hline
{\bf 11}
    &{\color[rgb]{0.750000,0.750000,0.750000}   0}
    &{\color[rgb]{0.083333,0.083333,0.083333}   4}
    &{\color[rgb]{0.416667,0.416667,0.416667}   2}
    &{\color[rgb]{0.416667,0.416667,0.416667}   2}
    &{\color[rgb]{0.416667,0.416667,0.416667}   2}
    &{\color[rgb]{0.416667,0.416667,0.416667}   2}
    &{\color[rgb]{0.750000,0.750000,0.750000}   0}
    &{\color[rgb]{0.750000,0.750000,0.750000}   0}
    &{\color[rgb]{0.750000,0.750000,0.750000}   0}
    &{\color[rgb]{0.083333,0.083333,0.083333}   4}
    &{\color[rgb]{0.750000,0.750000,0.750000}   0}
    &{\color[rgb]{0.750000,0.750000,0.750000}   0}
    &{\color[rgb]{0.750000,0.750000,0.750000}   0}
    &{\color[rgb]{0.750000,0.750000,0.750000}   0}
    &{\color[rgb]{0.750000,0.750000,0.750000}   0}
    &{\color[rgb]{0.750000,0.750000,0.750000}   0}
\\ \hline
{\bf 12}
    &{\color[rgb]{0.750000,0.750000,0.750000}   0}
    &{\color[rgb]{0.750000,0.750000,0.750000}   0}
    &{\color[rgb]{0.416667,0.416667,0.416667}   2}
    &{\color[rgb]{0.750000,0.750000,0.750000}   0}
    &{\color[rgb]{0.416667,0.416667,0.416667}   2}
    &{\color[rgb]{0.416667,0.416667,0.416667}   2}
    &{\color[rgb]{0.750000,0.750000,0.750000}   0}
    &{\color[rgb]{0.416667,0.416667,0.416667}   2}
    &{\color[rgb]{0.083333,0.083333,0.083333}   4}
    &{\color[rgb]{0.416667,0.416667,0.416667}   2}
    &{\color[rgb]{0.750000,0.750000,0.750000}   0}
    &{\color[rgb]{0.750000,0.750000,0.750000}   0}
    &{\color[rgb]{0.750000,0.750000,0.750000}   0}
    &{\color[rgb]{0.416667,0.416667,0.416667}   2}
    &{\color[rgb]{0.750000,0.750000,0.750000}   0}
    &{\color[rgb]{0.750000,0.750000,0.750000}   0}
\\ \hline
{\bf 13}
    &{\color[rgb]{0.750000,0.750000,0.750000}   0}
    &{\color[rgb]{0.750000,0.750000,0.750000}   0}
    &{\color[rgb]{0.083333,0.083333,0.083333}   4}
    &{\color[rgb]{0.750000,0.750000,0.750000}   0}
    &{\color[rgb]{0.750000,0.750000,0.750000}   0}
    &{\color[rgb]{0.416667,0.416667,0.416667}   2}
    &{\color[rgb]{0.750000,0.750000,0.750000}   0}
    &{\color[rgb]{0.416667,0.416667,0.416667}   2}
    &{\color[rgb]{0.750000,0.750000,0.750000}   0}
    &{\color[rgb]{0.750000,0.750000,0.750000}   0}
    &{\color[rgb]{0.416667,0.416667,0.416667}   2}
    &{\color[rgb]{0.416667,0.416667,0.416667}   2}
    &{\color[rgb]{0.416667,0.416667,0.416667}   2}
    &{\color[rgb]{0.750000,0.750000,0.750000}   0}
    &{\color[rgb]{0.750000,0.750000,0.750000}   0}
    &{\color[rgb]{0.416667,0.416667,0.416667}   2}
\\ \hline
{\bf 14}
    &{\color[rgb]{0.750000,0.750000,0.750000}   0}
    &{\color[rgb]{0.750000,0.750000,0.750000}   0}
    &{\color[rgb]{0.083333,0.083333,0.083333}   4}
    &{\color[rgb]{0.750000,0.750000,0.750000}   0}
    &{\color[rgb]{0.416667,0.416667,0.416667}   2}
    &{\color[rgb]{0.750000,0.750000,0.750000}   0}
    &{\color[rgb]{0.750000,0.750000,0.750000}   0}
    &{\color[rgb]{0.416667,0.416667,0.416667}   2}
    &{\color[rgb]{0.416667,0.416667,0.416667}   2}
    &{\color[rgb]{0.750000,0.750000,0.750000}   0}
    &{\color[rgb]{0.416667,0.416667,0.416667}   2}
    &{\color[rgb]{0.750000,0.750000,0.750000}   0}
    &{\color[rgb]{0.416667,0.416667,0.416667}   2}
    &{\color[rgb]{0.416667,0.416667,0.416667}   2}
    &{\color[rgb]{0.750000,0.750000,0.750000}   0}
    &{\color[rgb]{0.750000,0.750000,0.750000}   0}
\\ \hline
{\bf 15}
    &{\color[rgb]{0.750000,0.750000,0.750000}   0}
    &{\color[rgb]{0.416667,0.416667,0.416667}   2}
    &{\color[rgb]{0.750000,0.750000,0.750000}   0}
    &{\color[rgb]{0.750000,0.750000,0.750000}   0}
    &{\color[rgb]{0.750000,0.750000,0.750000}   0}
    &{\color[rgb]{0.416667,0.416667,0.416667}   2}
    &{\color[rgb]{0.416667,0.416667,0.416667}   2}
    &{\color[rgb]{0.416667,0.416667,0.416667}   2}
    &{\color[rgb]{0.750000,0.750000,0.750000}   0}
    &{\color[rgb]{0.750000,0.750000,0.750000}   0}
    &{\color[rgb]{0.416667,0.416667,0.416667}   2}
    &{\color[rgb]{0.750000,0.750000,0.750000}   0}
    &{\color[rgb]{0.750000,0.750000,0.750000}   0}
    &{\color[rgb]{0.083333,0.083333,0.083333}   4}
    &{\color[rgb]{0.750000,0.750000,0.750000}   0}
    &{\color[rgb]{0.416667,0.416667,0.416667}   2}
\\ \hline
\end{tabular}
\caption{Diferenčná tabuľka pre S-box}
\label{tab:dif}
\end{table}


\subsubsection{Príklad 1}
\begin{figure}[h!]
    \centering
    \includegraphics{img/13/dif.1.mps}
    \caption{Diferenciálna analýza v príklade 1}
\end{figure}
\begin{itemize}
\item {\bf 1. kolo:}
Použijeme differenciu $\langle in= 00d0 , out= 0020 \rangle $,
ktorá má pravdepodobnosti po jednotlivých S-boxoch $
1.0,0.25,1.0,1.0
$ čo spolu dáva pravdepodobnosť 
$ 1.0*0.25*1.0*1.0 = 0.25 $

\item {\bf 2. kolo:}
Použijeme differenciu $\langle in= 0020 , out= 00a0 \rangle $,
ktorá má pravdepodobnosti po jednotlivých S-boxoch $
1.0,0.25,1.0,1.0
$ čo spolu dáva pravdepodobnosť 
$ 1.0*0.25*1.0*1.0 = 0.25 $

\item {\bf 3. kolo:}
Použijeme differenciu $\langle in= 2020 , out= a0a0 \rangle $,
ktorá má pravdepodobnosti po jednotlivých S-boxoch $
1.0,0.25,1.0,0.25
$ čo spolu dáva pravdepodobnosť 
$ 1.0*0.25*1.0*0.25 = 0.0625 $

\item {\bf Spolu:}  Máme diferenciu $\langle = plaintext\_diff 00d0 , ciphertext\_diff= a0a0 \rangle$.
Celková pravdepodobnosť je $ 0.25*0.25*0.0625 ~= 0.00391 $.
\end{itemize}

\begin{table}[h]
 \centering
 \subtable[$key_5=1d87$]{
     \begin{tabular}{cc}
         0.0060 & 1\_8\_ \\
         0.0040 & 8\_8\_ \\
         0.0040 & 8\_1\_ \\
         0.0040 & 1\_1\_ \\
         0.0010 & f\_6\_ \\
         0.0010 & f\_1\_ \\
         0.0010 & e\_f\_ \\
         0.0010 & e\_8\_ \\
         0.0010 & e\_1\_ \\
         0.0010 & e\_0\_ \\
     \end{tabular}
 }
 \subtable[$key_5=b02b$]{
     \begin{tabular}{cc}
         0.0030 & b\_2\_ \\
         0.0020 & b\_b\_ \\
         0.0020 & 5\_b\_ \\
         0.0020 & 4\_2\_ \\
         0.0020 & 2\_b\_ \\
         0.0010 & d\_d\_ \\
         0.0010 & d\_8\_ \\
         0.0010 & d\_7\_ \\
         0.0010 & d\_4\_ \\
         0.0010 & c\_b\_ \\
     \end{tabular}
 }
\end{table}


\subsubsection{Príklad 2}
\begin{figure}[h!]
    \centering
    \includegraphics{img/13/dif.2.mps}
    \caption{Diferenciálna analýza v príklade 2}
\end{figure}
\paragraph{1. kolo:}
Použijeme differenciu $in= 0400 , out= 0400 $,
ktorá má pravdepodobnosti po jednotlivých S-boxoch $
1.0,1.0,0.375,1.0
$čo spolu dáva pravdepodobnosť 
$ 1.0*1.0*0.375*1.0 = 0.375 $

\paragraph{2. kolo:}
Použijeme differenciu $in= 0400 , out= 0400 $,
ktorá má pravdepodobnosti po jednotlivých S-boxoch $
1.0,1.0,0.375,1.0
$čo spolu dáva pravdepodobnosť 
$ 1.0*1.0*0.375*1.0 = 0.375 $

\paragraph{3. kolo:}
Použijeme differenciu $in= 0400 , out= 0400 $,
ktorá má pravdepodobnosti po jednotlivých S-boxoch $
1.0,1.0,0.375,1.0
$čo spolu dáva pravdepodobnosť 
$ 1.0*1.0*0.375*1.0 = 0.375 $

\paragraph{Spolu:}  Máme diferenciu $in= 0400 , out= 0400 $.
Celková pravdepodobnosť je $ 0.375*0.375*0.375 ~= 0.0527 $.

\begin{table}[h]
 \centering
 \subtable[$key_5=5f46$]{
     \begin{tabular}{cc}
         0.0780 & \_f\_\_ \\
         0.0420 & \_3\_\_ \\
         0.0420 & \_1\_\_ \\
         0.0390 & \_d\_\_ \\
         0.0280 & \_b\_\_ \\
         0.0230 & \_5\_\_ \\
         0.0200 & \_9\_\_ \\
         0.0130 & \_c\_\_ \\
         0.0130 & \_7\_\_ \\
         0.0120 & \_4\_\_ \\
     \end{tabular}
 }
 \subtable[$key_5=4fdc$]{
     \begin{tabular}{cc}
         0.0630 & \_f\_\_ \\
         0.0320 & \_3\_\_ \\
         0.0270 & \_d\_\_ \\
         0.0250 & \_1\_\_ \\
         0.0140 & \_b\_\_ \\
         0.0140 & \_4\_\_ \\
         0.0130 & \_c\_\_ \\
         0.0120 & \_9\_\_ \\
         0.0100 & \_5\_\_ \\
         0.0090 & \_7\_\_ \\
     \end{tabular}
 }
\end{table}


\subsubsection{Príklad 3}
\begin{figure}[h!]
    \centering
    \includegraphics{img/13/dif.3.mps}
    \caption{Diferenciálna analýza v príklade 3}
\end{figure}
\paragraph{1. kolo:}
Použijeme differenciu $in= 0400 , out= 0400 $,
ktorá má pravdepodobnosti po jednotlivých S-boxoch $
1.0,1.0,0.375,1.0
$čo spolu dáva pravdepodobnosť 
$ 1.0*1.0*0.375*1.0 = 0.375 $

\paragraph{2. kolo:}
Použijeme differenciu $in= 0400 , out= 0400 $,
ktorá má pravdepodobnosti po jednotlivých S-boxoch $
1.0,1.0,0.375,1.0
$čo spolu dáva pravdepodobnosť 
$ 1.0*1.0*0.375*1.0 = 0.375 $

\paragraph{3. kolo:}
Použijeme differenciu $in= 0400 , out= 0c00 $,
ktorá má pravdepodobnosti po jednotlivých S-boxoch $
1.0,1.0,0.125,1.0
$čo spolu dáva pravdepodobnosť 
$ 1.0*1.0*0.125*1.0 = 0.125 $

\paragraph{Spolu:}  Máme diferenciu $in= 0400 , out= 0c00 $.
Celková pravdepodobnosť je $ 0.375*0.375*0.125 ~= 0.0176 $.

\begin{table}[h]
 \centering
 \subtable[$key_5=f4db$]{
     \begin{tabular}{cc}
         0.0140 & f4\_\_ \\
         0.0090 & f8\_\_ \\
         0.0090 & b4\_\_ \\
         0.0090 & 34\_\_ \\
         0.0070 & d4\_\_ \\
         0.0070 & 74\_\_ \\
         0.0060 & 94\_\_ \\
         0.0060 & 14\_\_ \\
         0.0050 & fa\_\_ \\
         0.0050 & f6\_\_ \\
     \end{tabular}
 }
 \subtable[$key_5=194c$]{
     \begin{tabular}{cc}
         0.0150 & 19\_\_ \\
         0.0110 & d9\_\_ \\
         0.0100 & 15\_\_ \\
         0.0080 & f9\_\_ \\
         0.0080 & d5\_\_ \\
         0.0080 & 39\_\_ \\
         0.0080 & 1b\_\_ \\
         0.0080 & 17\_\_ \\
         0.0070 & db\_\_ \\
         0.0060 & f5\_\_ \\
     \end{tabular}
 }
\end{table}


\section{Autentizačné protokoly s heslami}

Predstavme si veľmi klasickú situáciu. Použivateľ $A$ sa chce pripojiť
na server $S$ , čo obnáša samozrejme potvrdenie vlastnej identity.
V dnešnom svete je asi najbežnejšie overenie identity cez kombináciu
uživateľského mena a hesla $P$. A následne počas tohoto overenia
by sme chceli aj vygenerovať kľúč pre šifrovanie dát počas nasledovnej
komunikácie (session key $K$).

Pokiaľ ale navrhneme protokol zle narážame na jeden problém. Priestor
hesiel, ktoré sa bežne používajú je dosť malý. Na ten sa dá dosť úspešne
útočiť, či už úplným preberaním alebo slovníkovým útokom (keďže veľa použivateľov
má heslá, ktoré sú zmysluplné slová). Demonštrujme si to na jednoduchom
príklade protokolu:
\begin{enumerate}
\item $A \to S\colon A$ - uživateľ povie svoje meno
\item $S \to A\colon E_p(K)$ - server pomocou hesla zašifruje session key
\item $A \to S\colon E_k(msg)$ - uživateľ zašifruje vopred dohodnutú správu, aby ukázal, že dostal správny kľúč
\end{enumerate}

Tento protokol má niekoľko problémov. Okrem toho, že je náchýlný
na útok opakovaním, tak po zachytení druhej a tretej správy vieme úspešne útočiť
preberaním všetkých hesiel, keďže pre každé heslo, vieme povedať, či je správne alebo nie.

\subsection{EKE protokol}

Trochu lepším prístupom je EKE (Encrypted key exchange) protokol vymyslený 
v roku 1992. Má viacero variantov. Jednou z nich je napríklad DHEKE:

\begin{enumerate}
\item $A\to S\colon A, E_p(g^x)$
\item $S\to A\colon E_p(g^y), E_K(N_S)$, kde $K = g^{xy}$
\item $A\to S\colon E_K(N_A, N_S)$
\item $S\to A\colon E_K(N_A)$
\end{enumerate}

Čiže v podstate prebehne DH algoritmus, ktorý je ale šifrovaný heslom a navyše
si uživateľ a server vymenia príležitostné slová.
Iné varianty EKE môžu fungovať bez výmeny príležitostných slov prípadne využijú asymetrické
šifrovanie ($A\to S\colon A, E_p(pk_a)$; $S\to A\colon E_p(E_{pk_a}(K))$).

Tento protokol sa zdá byť bezpečný, keďže útočník po odposluchu sekvencie nie je schopný
robiť slovníkový útok. A šifrovanie heslom navyše zabraňuje man-in-the-middle útoku.

Ale je tu niekoľko problémov. Jednak $pk_a$, resp. $g^x$ musia byť úplne náhodne.
Napríklad verejný kľúč RSA $(n,e)$ je úplne nevhodný, keďže v ňom platí, že $e$ je nepárne
a nesúdeliteľný s $n$. Z tejto znalosti vie útočník po vyskúšaní všetkých hesiel približne polovicu vylúčiť.
Pokiaľ odpočuje viac konverzácií, tak sa časom dostane k správnemu heslu. Tomuto sa hovorí partitioning attack.
Podobná situácia je pri $g^x \pmod m$, kde niektoré hodnoty nemôžeme vôbec dostať a toto vie útočník
rozoznať.

A navyše je tu ďalší problém. Server musí udržiavať heslá v otvorenej podobe, čo má dosť
veľké bezpečnostné problémy. Môžeme miesto hesla si ukladať a používať na šifrovanie napríklad hash hesla (a tou aj šifrovať), ale
to stále nebráni útočníkovi napodobniť použivateľa, keď získa túto hash.

Tento problém sa snaží riešiť Augmented EKE protokol (A-EKE).
Zoberme si funkciu, ktorá
nám z hesla vygeneruje kľúče pre asymetrické podpisovanie: 
$\langle sk_A, pk_A \rangle = f(P)$. 
Server si uloží iba hodnotu $pk_A$.
Protokol bude podobný ako EKE, teda budeme šifrovať \emph{symetricky}
pomocou kľúča odvodeného z hesla, v našom prípade to bude $pk_A$.
Nakoniec pomocou súkromného kľúča (ktorý vie zostrojiť iba vlastník hesla)
podpíšeme komunikáciu a tým zaručíme, že komunikujeme s pravým človekom.

\begin{enumerate}
    \item $A\to S\colon A, E_{pk_A}(g^x)$.
        Pozor, v celej schéme šifrujeme iba \emph{symetricky}!
    \item $S\to A\colon E_{pk_A}(g^y), E_K(N_S)$
    \item $A\to S\colon E_K(N_S, N_A)$
    \item $S\to A\colon E_K(N_A)$
    \item $A\to S\colon E_K(Sig_{sk_A}(K))$ --
        tento krok je tu navyše oproti EKE, slúži nato,
        aby útočník pokiaľ získa $pk_A$ sa nevedel vložiť do
        komunikácie.
\end{enumerate}

\subsection{SRP (Secure remote password) protokol}

Ďalším protokolom použiteľným na autentifikáciu heslom je SRP protokol.
Pracuje v grupe $Z_n^{*}$, kde $n$ je prvočíslo. Nech $g$ je generátor
tejto grupy.
Pre každého uživateľa $A$ je potrebné vygenerovať náhodnú soľ $s$.
Privátnym kľúčom účastníka v ďalšom výpočte bude hash hesla, čiže
hodnota $x = H(s,P)$ a jeho verejný kľúč označíme ako $v = g^x$.
Server $S$ si následne
pre každého uživateľa uloží záznam $\langle A, s, v \rangle$.
Samotné prihlasovanie prebieha nasledovne:
\begin{enumerate}
\item $A\to S\colon A$
\item $S\to A\colon s$
\item $A\to S\colon \alpha = g^a$ pričom $a \inr Z_n^{*}$
\item $S\to A\colon \beta = g^b + v,\ u$ pričom $b,u \inr Z_n^{*}$
\item Server vypočíta kľúč $K = H((\alpha v^u)^b)$
\item Uživateľ vypočíta kľúč $K = H((\beta - g^x)^{a+ux})$
\item $A\to S\colon M_1 = H(A,S,K)$
\item $S\to A\colon H(A,M_1,K)$
\end{enumerate}

Ukážme si najprv, že aj server aj užívateľ vypočítajú tú istú hodnutu $K$.
Ak si odmyslíme hašovanie na konci, dostávame
\begin{align*}
(\beta - g^x)^{a+ux} = (g^b + v - g^x)^{a+ux} &= g^{ba+bux}\\
(\alpha v^u)^b = (g^{a+xu})^b &= g^{ba+bux}
\end{align*}
Zaoberajme sa teraz rôznymi vlastnosťami tohoto protokolu.

Prvou bude to, že útočník, ktorý odpočuje protokol,
nie je schopný vypočítať kľúč $K$.
Útočník má k dispozícií hodnoty $\alpha, \beta, u, g, n$.
Dajme mu navyše k dispozícii aj hodnotu $x$.
Ak útočník dokáže zistiť z týchto hodnôt hodnotu $K$,
vieme ho využiť na riešenie DH problému nasledovne:
Majme na vstupe hodnoty $g, g^a, g^b$. Zvolíme si $u,x$ tak, 
aby $n-1 | ux$, napríklad $u = 2, x = \frac{n-1}{2}$.
Útočníkovi dáme na vstup $\alpha = g^a, \beta = g^b + g^x$ a obvyklý zbytok.
On určí $K = H(g^{ba+bux})$. 
Vzhľadom na to, že uvažujeme hash ako náhodné orákulum,
tak je jasné, že niekedy musel útočník vedieť hodnotu $g^{ba+bux}$,
teda spýtať sa hašovacieho orákula na $H(g^{ba+bux})$.
Toto orákulum môžeme prevádzkovať útočníkovi sami a
tým pádom budeme vedieť hodnotu
$g^{ba+bux} = g^{ba+b(n-1)m} = g^{ba}$,
čím sme vyriešili úspešne DH problém.

\begin{poznamka}
Na tomto mieste by sme upozornili, že daná redukcia funguje len za
predpokladu, ak máme útočníka riešiaceho všetky inštancie problému.
V prípade, že máme iba pravdepodobnostného útočníka, ktorý na vstupe
dostane hodnoty z rovnomernej distribúcie, máme problém.
Väzba $n-1 | ux$ nám totiž výrazne redukuje možné dvojice hodnôt
$u,x$, čo je v rozpore s predpokladmi pre útočníka. Tým pádom útočník
nemusí byť vôbec schopný riešiť takéto inštancie problému.
\end{poznamka}

\todo{SRP zbytok}

\section{Podpisové schémy s určeným overovateľom}

V tejto kapitole sa budeme baviť o podpisových schémach, ktoré fungujú
tak trochu podobne ako bezznalostné dôkazy -- chceli by sme vedieť
podpísať správu špeciálne pre jedného overovateľa. Navyše, úplne
ideálne by bolo, ak by nikto iný okrem neho nebol schopný overiť náš
podpis a dokonca aby overovateľ nebol schopný dokázať niekomu inému,
že sme to podpísali my.

Pretože myšlienka určeného overovateľa (designated verifier) je veľmi
blízka bezznalostným dôkazom, budú aj definície veľmi podobné.

\begin{definicia}[Schéma s určeným overovateľom]
    Uvažujme dve komunikujúce strany -- Alicu $A$ a Boba $B$. Nech
    $P(A,B)$ je protokol pre Alicu na presvedčenie
    Boba o pravdivosti nejakého výroku $\Omega$. Môžeme hovoriť, že
    Bob je určený overovateľ, ak vie produkovať komunikácie, ktorých
    distribúcia je identická s distrubúciou komunikácii protokolu
    $P(A,B)$.
\end{definicia}

\begin{poznamka}
    Pre schémy s určeným overovateľom nemusí nutne platiť to, čo sme si
    povedali na začiatku, teda že nikto iný okrem Boba nevie overiť podpis.
    Túto vlastnosť budú mať až silné schémy s určeným overovateľom.
    V štandardnej schéme bude stačiť, že Alica dokáže Bobovi výrok
    \quoteme{Platí $\Omega$ (podpísala som správu) alebo som Bob}.
    Inak povedané, ktokoľvek iný nebude presvedčený, či správu naozaj
    podpísala Alica alebo ju nafingoval Bob.
\end{poznamka}

Niekedy ale ani táto situácia nestačí. Predstavme si Evu, ktorá sa
snaží zistiť niečo viac o podpise. Ak je presvedčená, že
\begin{enumerate}
    \item Bob sa zdá byť čestný, je nepravdepodobné, že by on
    vygeneroval falošný podpis.

    \item Ak je Eva predvedčená, že Bob ešte nevidel daný podpis, je
    tiež evidentné, že podpisovať musela Alica.
\end{enumerate}

V tomto prípade môžeme zaviesť striktnejšie predpoklady.

\begin{definicia}[Silná schéma s určeným overovateľom]
    Nech $P(A,B)$ je protokol pre Alicu a Boba na ukázanie pravdivosti
    nejakého výroku $\Omega$. Hovoríme, že $P(A,B)$ je protokol so
    silne určeným overovateľom, ak ktokoľvek vie produkovať
    komunikácie, ktoré sú nerozlíšiteľné od komunikácii protokolu
    $P(A,B)$ pre všetkých s výnikou Boba (pretože Bob musí byť schopný
    overiť pravosť podpisu). \fixme{a bob vioe generovat falosne podpisy}
\end{definicia}

\begin{poznamka}
    Z predchádzajúcej definície je evidentné, že nikto okrem Boba nesmie
    vedieť overiť pravdivosť podpisu,
    inak by vedel odlíšiť nafingovanú správu od správy od Alice.
\end{poznamka}

Samozrejme, celú túto definíciu môžeme zaviesť aj omnoho formálnejšie,
ako sme robili pri ostatných podpisových schémach.

\begin{definicia}[Designated verifier signature scheme]
    Nech $M$ je priestor všetkých možných správ (štandardne
    $M=\{0,1\}^*$). Podpisová schéma s určeným overovateľom je
    štvorica algoritmov
    $\langle Gen, Sig, Verify, Simulate \rangle$ polynomiálnych
    algoritmov kde
    \begin{itemize}
        \item $Gen(1^n)$ je pravdepodobnostný algoritmus ktorý zo
            vstupu $1^n$ vyrobí štvoricu kľúčov
            $\langle pk_A,sk_A,pk_B,sk_B \rangle$.
            Štandardne sa kľúče pre $A$ a $B$ generujú nezávisle a
            preto je možné tento algoritmus rozdeliť na dva rôzne
            algoritmy (resp. dvakrát ten istý algoritmus).

        \item $Sig(m,sk_A,pk_B)$ je pravdepodobnostný algoritmus,
            ktorý na vstupe dostane správu, privátny kľúč Alice a
            verejný kľúč Boba a na výstupe vegeneruje podpis $\sigma$
            správy $m$.

        \item $Verify(m,\sigma,sk_B,pk_A)$ je deterministický
            algoritmus, ktorý zo správy, jej podpisu, verejného kľúča
            Alice a súkromného kľúča Boba vie rozhodnúť, či bol podpis
            $\sigma$ správy $m$ podpísaný Alicou a určený pre Boba.
            Algoritmus vypíše na vstup jeden bit $b$, ktorý je 1 práve
            vtedy, ak bol podpis korektný.

        \item $Simulate(m,pk_A,sk_B)$ je pravdepodobnostný algoritmus
            produkujúci podpisy nerozlíšiteľné od podpisov algoritmu
            $Sig(m,sk_A,pk_B)$.
            V prípade, že uvažujeme silnú verziu schémy, $Simulate$ na
            vstupe nedostáva $sk_B$ ale $pk_B$.
    \end{itemize}
    Navyše, pre každú štvoricu $\langle pk_A,sk_A,pk_B,sk_B \rangle$
    vygenerovaný algoritmom $Gen(1^n)$ a pre každú správu $m\in M$
    musí platiť
    \begin{equation*}
        Verify(m, Sig(m,sk_A,pk_B),sk_B,pk_A)=1
    \end{equation*}
\end{definicia}

Formálne si teraz môžeme definovať experiment na testovanie, či naša
podpisová schéma $\Pi$ je podpisová schému s určeným overovateľom.
Budeme na to používať algoritmus
\ref{proc:ind-dv}, skrátene IND-DV, ktorý bude fungovať v roli
rozlišovateľa.

\begin{procedure}[H]
    \caption{Indistinguishable-DesignatedVerifier($D,\Pi,n$)}
    \label{proc:ind-dv}
    $\langle pk_A,sk_A,pk_B,sk_B \rangle \assign Gen_{\Pi}(1^n)$ \;
    $m \inr M$ \;
    $b \inr \{0,1\}$ \;
    \eIf{$b==0$}{$O \assign Sig_\Pi(m,sk_A,pk_B)$}
        {$O \assign Simulate_\Pi(m,pk_A,sk_B)$}
    spusť algoritmus $D$ s prístupom ku oráklu $O$. Výsledok nech je
    $b'$ \;
    \Return $b==b'$\;
\end{procedure}

Teraz môžeme hovoriť, že schéma má vlastnost určeného overovateľa, ak
pre každého polynomiálneho rozlišovateľa $D$ platí
\begin{equation*}
    Pr[IND-DV(D,\Pi,n)=1] \le \frac{1}{2} + negl(n)
\end{equation*}

Čitateľ si iste sám vie zobšeobecniť dané definície aj na silnejšiu
verziu schémy.

\subsection{Schéma od Saeednia, Kremera a Markowitcha}

Schéma je prebratá z článku \cite{designated_verifier}.
Na začiatok by sme si mohli popísať, aké spoločné parametre budú
zdieľať účastníci schémy. Bude to niečo podobné ako pri DSA podpisoch.

Uvažujme dve veľké prvočísla $p,q$ také, že $q | p-1$. 
Prvočíslo $p$ nám určuje grupu $Z_p^*$.
K tejto grupe vygenerujeme generátor $g$ jej podgrupy, ktorá má $q$
prvkov. Uvažujme ďalej hashovaciu funkciu $H$, o ktorej predpokladáme,
že je odolná voči kolíziam a s výstupom do $Z_q$.

Každý užívateľ si sám zvolí svoj súkromný kľúč $sk=x \inr Z_q$ a
zverejní svoj verejný kľúč $pk=y=g^{x} \pmod{p}$.

Potom, ako máme všetky náležitosti pripravené, Alica môže podpísať
správu určenú pre Boba nasledovne:
\begin{procedure}
    \caption{SignSKM($m$)}
    
    $k \inr Z_q$\;
    $t \inr Z_q^*$\;
    $c \assign y_B^k \mod{p}$\;
    $r \assign H(m, c)$\;
    $s \assign k \cdot t^{-1} - r \cdot x_A \mod{q}$\;
    \Return $\sigma=\langle r,s,t\rangle$\;
\end{procedure}

Ak bude teraz Bob overovať podpis, stačí mu overiť podmienku
\begin{equation*}
    H(m, (g^s y_A^r)^{t \cdot x_B} \mod{p}) == r
\end{equation*}
% (g^s y_A^r)^{t x_B} =
% g^{k x_B - r t x_A x_B + r t x_A x_B} = y_B^k

\begin{poznamka}
    Vidíme, že vďaka previazanosti Bobovho súkromného kľúča v overovacej
    rovnici by mal byť schopný overiť správu len on.
    Formálny dôkaz však nepotrebujeme (táto vlastnosť nás u obyčajnej
    schémy s určeným overovateľom netrápi) a ani autori dôkaz
    neuvádzajú.
\end{poznamka}

Aby sme ukázali, že schéma je s určeným overovateľom, musíme ukázať,
že Bob vie simulovať podpis. Simulačný algoritmus môže byť napríklad
tento:
\begin{procedure}
    \caption{SimulateSKM()}
    $r' \inr Z_q^*$\;
    $s' \inr Z_q^*$\;
    $c \assign g^{s'} y_A^{r'}$\;
    $r \assign H(m,c)$\;
    $s \assign s' r r'^{-1} \pmod{q}$\;
    $t \assign x_A r' r^{-1} \pmod{q}$\;
    \Return $\langle r,s,t \rangle$\;
\end{procedure}

\todo{co sme mali dalej, potrebujem k tomu zosit s poznamkami}

%%%%%%%%%%%%%%%%%%%%%%%%%%%%%%%
\subsection{Yang, Liao}
Táto podpisová schéma patrí medzi horúce novinky, pôvodný článok
\cite{designated_verifier2}
je z Mája 2010 (ale vyskytla sa aj jeho podoba zo Septembra 2009).
Vo svojej podstate je podobná predchádzajúcej.
Máme ale pridanú hodnotu -- schéma je navrhnutá tak, 
aby bola možná extrakcia správy.
Uvažujme opäť, že Alica chce poslať Bobovi podpísanú správu.

O správe $m$ predpokladajme, že jej veľkosť je $\log_2 p - \log_2 q$
bitov.\footnote{V originálnom článku sa uvádza $m\in Z_q$,
    náš prístup je však vhodnejší pre svoju názornosť a všeobecnosť}
Toto obmedzenie na veľkosť zavádzame, pretože budeme chcieť, aby
$m||t$ sa zmestilo do $Z_p^*$ kde $t \in Z_q$.

\begin{poznamka}
    Je evidentné, že podpísaná správa nemôže byť veľmi veľká.
    Štandardne však môžeme uvažovať, že
    pokiaľ nechceme možnosť extrakcie správy, 
    môžeme podpisovať iba hash správy a nie celú správu.
    V tomto prípade ale treba dať pozor, pretože daná zmena môže
    zmeniť správnosť nasledujúcich dôkazov.
\end{poznamka}

Generácia kľčov je štandardná:
\begin{procedure}
    \caption{GenYangLiao($p,q,g$)}
    $x \inr Z_q$\;
    $y \assign g^x \pmod{p}$\;
    \Return $sk=x,\ pk=y$\;
\end{procedure}


Podpis budeme generovať nasledovne:
\begin{procedure}
    \caption{SignYangLiao()}
    $t \inr Z_q^*$\;
    $s \assign H(m||t)$\;    
    $r \assign (m||t) \cdot y_B^{x_A \cdot s} \pmod{p}$\; 
    \Return $\sigma=\langle r, s \rangle$\;
\end{procedure}

Na overovanie použijeme súkromné a verejné kľúče presne obrátene,
keďže platí $y_B^{x_A} = y_A^{x_B}$.

\begin{procedure}
    \caption{VerifyYangLiao()}
    $m'||t \assign r \cdot y_A^{-x_B\cdot s} \pmod{p}$\;
    \If{$(s \ne H(m||t))$}{
        Reject\;
    }
    Accept\;
\end{procedure}

Nato, aby sme tvrdili, že schéma je designated verifier nám treba
dokázať tieto 3 vlastnosti
\begin{enumerate}
    \item Schéma je bezpečná, teda korektný podpis pre Boba vie
        vyrobiť iba Alica.

    \item Bob vie simulovať podpisy a teda nemôže byť schopný nikoho
        presvedčiť o tom, že správu podpísala Alica.
\end{enumerate}


\begin{lema}
    Prezentovaná schéma je bezpečná.
\end{lema}
\begin{dokaz}
    \todo{}
\end{dokaz}

\begin{lema}
    Prezentovaná schéma je designed verifier schéma.
\end{lema}
\begin{dokaz}
    Ako dôkaz ukážeme, že Bob vie generovať platné podpisy na
    nerozoznanie od skutočných. Prvou časťou je simulačný algoritmus
    -- procedúra \ref{proc:simyang}
    \begin{procedure}
        \caption{SimulateYangLiao()}
        \label{proc:simyang}
        $t' \inr Z_q^*$\;
        $s' \assign H(m||t')$\;
        $r' \assign (m||t') \cdot y_A^{x_B \cdot s} \pmod{p}$\; 
        \Return $\sigma'=\langle r',s' \rangle$\;
    \end{procedure}

    Malo by byť evidentné, že simulačný algoritmus generuje iba platné
    podpisy, nakoľko $y_A^{x_B} = y_B^{x_A}$.
    Otázkou teda ostáva, či sedia aj pravdepodobnostné
    distribúcie vygenerovaných podpisov. Ak hej, tak môžeme prehlásiť,
    že schéma je designated verifier.
    Uvažujme, že
    $\overline{\sigma}=\langle \overline{r}, \overline{s} \rangle$ 
    je správa, ktorá je
    náhodne vybraná spomedzi všetkých platných podpisov.
    Uvažujme najskôr podpis $\sigma = \langle r,s \rangle$ od Alice.
    Platí
    \begin{equation*}
        Pr[\langle r,s \rangle = 
            \langle \overline{r}, \overline{s} \rangle] =
        Pr\left[ \left.
            \begin{array}{l}
            H(m||t) = \overline{s} \quad \land \\ 
            (m||t) \cdot y_B^{x_A \cdot s} = \overline{r}
            \end{array} \right| t \inr Z_q^*
        \right] = Pr[t = \overline{t} | t \inr Z_q^*] = \frac{1}{q-1}
    \end{equation*}
    Bolo by dobré vysvetliť predposlednú rovnosť, keďže nie je úplne
    transparentná. K podpisu $\overline{\sigma}$ existuje
    $\overline{t}$, podľa ktorého bol podpis zostrojený.
    Uvažujme, že $t \ne \overline{t}$. 
    Ak $H(m||t) \ne H(m||\overline{t})$, podpisy sú určite rôzne,
    pretože $s \ne \overline{s}$. Naopak, v nepravdepodobnom prípade,
    že nastane kolízia a $s = \overline{s}$, muselo by platiť
    $(m||t) \cdot y_B^{x_A \cdot s} = (m||t') \cdot y_B^{x_A \cdot s}
    \pmod{p}$, čo ale implikuje $(m||t) = (m||t') \pmod{p}$.
    Posledná rovnosť ale určite nemôže nastať, pokiaľ platí, 
    že zreťazenie $m||t$ nemá viac bitov ako číslo $p$.
    To sme ale na začiatku zakázali. Preto jediná možnosť,
    kedy sa podpisy $\sigma,\overline{\sigma}$ rovnajú je ak 
    $t=\overline{t}$.
     
    Podobne ako sme vypočítali pravdepodobnosť zhody podpisu Alice s
    náhodnou korektnou správou,
    vieme vypočítať aj pravdepodobnosť pre zhodu simulovanej
    správy s náhodnou korektnou správou a vyjde presne rovnako, $1/(q-1)$.
    Tým pádom je ale schéma perfect\footnote{Inak povedané,
        s rovnakými pravdepodobnosťami. Tak isto sme to značili aj pri
        bezznalostných dôkazoch} designated verifier.
\end{dokaz}

Autori článku tvrdia, že daná schéma je silná schéma s určeným
overovateľom. Žiaľ, dôkaz tohoto tvrdenia sa im zdal byť natoľko evidentný, 
že ho z článku vynechali a pre istotu ani nenapísali simulovanie
podpisovania iba z verejných kľúčov.

\begin{poznamka}
    Ak sa čitateľovi nepozdáva GDH problém, pretože sa mu zdá byť
    príliš silný, môže si prečítať o modifikácii Yang-Liaovej schémy,
    ktorej pri dôkaze bezpečnosti stačí obyčajný DH. Článok tiež
    obsahuje pekné zhrnutie pojmov z oblasti určeného overovateľa.
    Takže, enjoy \cite{designated_verifier_stanek}.
\end{poznamka}

\input{tex/16rainbow.tex}

%\backmatter fixme: preco to tu nefunguje? asi chyba nejaky package
\listoffigures
\listoftables

\bibliographystyle{alpha}
\bibliography{literatura}



\end{document}
