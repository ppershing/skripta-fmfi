\section{Oblivious transfer}

Ďalším základným primitívom, na ktorom vieme budovat kryptografické
prvky je takzvaný oblivious transfer. Ide o akýsi prenos údajov,
pričom sender sa nedozvie, či boli údaje prenesené, prípadne ktoré
údaje boli prenesené.

\begin{definicia}[1-2 OT]
    1-2 oblivious transferom nazveme komunikáciu podľa nasledujúcej
    schémy:
    \begin{itemize}
        \item $A \send B: m_0, m_1$, kde $m_0$ a $m_1$ sú 2 rôzne
        správy, ktoré chce $A$ poslať.
        \item $B$ si vyberie niektorú zo správ, ktorú chce prijať a
        toto prijatie mu bude umožnené.
        \item $A$ sa nedozvie, ktorú správu $B$ prijal.
        \item $B$ nemá možnosť prijať obe správy naraz.        
    \end{itemize}
\end{definicia}

\begin{definicia}[50\% OT]
    50\% oblivious transferom nazveme komunikáciu podľa nasledujúcej
    schémy:
    \begin{itemize}
        \item $A \send B: m$, kde $m$ je správa.
        \item $B$ s spravdepodobnosťou 50\% správu prijme, inak sa o
        nej nedozvie nič.
        \item $A$ sa nedozvie, či $B$ správu prijal
    \end{itemize}
\end{definicia}

\begin{priklad}[Realizácia 50\% OT]
    Nech $A$ chce poslať správu účastníkovi $B$ s 50\%-nou
    pravdepodobnosťou úspechu. Na začiatku si $A$ zvolí inštanciu RSA
    s $n=pq$, kde $p,q$ sú veľké prvočísla. Bude nasledovať
    komunikácia
 \begin{itemize}
    \compactlist
    \item $A \send B: n,e,E(m)$ 
    \item $B$ si zvolí $x \inr Z_n$
    \item $B \send A: x^2$
    \item $A$ s pomocou faktorizácie vyberie nejakú odmocninu
        $z \inr \{x,-x,y,-y\}$.
    \item $A \send B: z$.
    $B$ vie faktorizovať $n$ (a dešifrovať správu) s
    pravdepodobnosťou 50\% (ak $z \not = \pm x$, tak jednoducho
    spočíta pomocou $\nsd(z-x, n)$).
 \end{itemize} 
\end{priklad}

\begin{priklad}[Realizácia 1-2 OT]
 1-2 OT budeme realizovať za pomoci diskrétneho logaritmu.
 Účastník $A$ si zvolí grupu $G$, $n=|G|$ a generátor $g\in G$.
 \begin{itemize}
    \compactlist
    \item $A \send B: c \inr G$.
    \item $B$ si zvolí $b\in\{0,1\}$ - číslo správy, ktorú chce
    prijať.
    \item $B$ si zvolí $x \inr G$ a vypočíta hodnoty $y_b = g^x$,
    $y_{1-b} = c/ y_b$. Platí $y_0 y_1 = c$ a tiež existuje (pretože
    $g$ je generátor) hodnota $x': g^{x'} = y_{1-b}$. Teda, $A$ nevie
    odlíšiť jednotlivé hodnoty.
    \item $B \send A: y_0$.
    \item $A$ si zvolí $k_0,k_1 \inr G$ a vytvorí šifrované správy 
    $E_0 = \langle g^{k_0}, y_0^{k_0} \xor m_0 \rangle$,
    $E_1 = \langle g^{k_1}, y_1^{k_1} \xor m_1 \rangle$.
    \item $A \send B: E_0, E_1$.
    \item $B$ dešifruje $m_b$ ako $(g^{k_b})^x \xor (y_b ^ {k_b} \xor
    m_b)$.
    Pokiaľ $B$ nevie riešiť DH problém, druhú správu nedostane.
    \todo{dokaz}
 \end{itemize}
\end{priklad}

\subsection{Konverzie}

\begin{lema}
    50\% OT sa dá simulovať pomocou 1-2 OT.
\end{lema}
\begin{dokaz}
    Postup je jednoduchý. Nech $A$ chce poslať správu 
    $m \ne 0$.\footnote{Predpokladáme, že vieme posielať správy, ktoré
        sú dlhšie ako 1 bit.\fixme{ako spravit konveziu z 1 bitu?}
    }
    Nech $c \inr \{0,1\}$ a nech $m_c = m, m_{1-c} = 0$. $A$ pošle
    správy $m_0, m_1$ pomocou 1-2 OT. $B$ prijme správu $m$ s
    pravdepodobnosťou 50\%, inak prijme 0 a vie, že sa prenos
    nepodaril.
\end{dokaz}

\todo{1-2 OT pomocou 50}
\todo{BC pomocou 50 OT}
Pre záujemcov je predchádzajúca konštrukcia presnejšie popísaná v
\cite{ot_equiv}.
