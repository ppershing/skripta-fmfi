\section{Bezpečný výpočet viacerých účastníkov}

Uvažujme nasledujúci problém ``starých dám''. Dve dámy sa stretnú a
chceli by zistiť, ktorá z nich je staršia.\footnote{V skutočnosti
sa každá chce ubezpečiť, že tá druhá je staršia :-)}
Neboli by to však dámy v svojom veku, kebyže o sebe nechcú nič prezradiť.
Presnejšie povedané -- nechcú prezradiť nič iné okrem informácie, ktorá z
nich je staršia. Navyše budeme predpokladať, že dámy budú
spolupracovať a jedna druhú nepodrazí.
Teraz navrhneme protokol, ktorý túto ich dilemu vyrieši.

Uvažujme, že vek môže nadobúdať len jednu z konečne veľa diskrétnych
hodnôť (napríklad $v_A,v_B \in V = \{1,\dots,100\}$).
Ďalej uvažujme inštanciu asymetrického šifrovacieho systému s
veľkosťou šifrovacieho kľúča $N$. 
Budeme požadovať, aby účastník $A$ vedel (efektívne) iba šifrovať.

\begin{itemize}
    \compactlist
    \item $A$ si zvolí $x$ -- náhodné $N$-bitové číslo.
        Pomocou tohoto čísla budeme ``maskovať''  $v_A$.
    \item $A \send B: c = E(x) - v_A $ (pošleme maskované $v_A$).
    \item $B$ vygeneruje čísla $y_1, \dots, y_{100}$ tak aby
            $y_i = D(c + i)$ (kde $y_{v_A} =x$, ostatné hodnoty
            sú nepredikovateľné).
    \item $B$ si teraz vygeneruje náhodné $N/2$ bitové prvočíslo $p$ a
    vypočíta $z_1,\dots,z_{100}$ ako $z_i = y_i \pmod{p}$.
    Ak by náhodou existovali indexy $i,j: |z_i - z_j| < 2$, tak si
    zvolí iné prvočíslo.
    \item $B \send A:p$ a 100 čísel $t_1, \dots, t_{100}$ --
        budú to hodnoty $z_i$ zvýšené o hodnotu predikátu
        $v_B < i$, teda hodnoty
        $z_1,\ z_2,\ \dots,\ z_{v_B}, \
         z_{v_B+1}+1,\ z_{v_B+2}+1,\ \dots,\ z_{100}+1$.
    \item $A$ sa teraz pozrie na pravdivosť
        $x \pmod{p} \overset{?}{=} t_{v_A}$. Ak tvrdenie platí,
        tak $v_A \le v_B$.
    \item $A \send B:$ výsledok porovnania
\end{itemize}
Je jasné, že $B$ nemá šancu sa dozvedieť nič o hodnote $v_A$ počas
výpočtu, pretože $A$ si mohol zvoliť $x$ pre ľubovoľnú hodnotu
$v_A$ ako $x=D(c + v_A)$ s rovnakou pravdepodobnosťou.\footnote{V
    skutočnosti musí kryptografický systém spĺňať nejaké požiadavky
    navyše, aby to platilo, čitateľ si môže premyslieť aké.
}
Otázne teda je, či $A$ sa môže dozvedieť niečo o $v_B$.
Na to by ale potreboval vedieť dešifrovať hodnoty $y_i$, čo sme
zamietli v predpokladoch.
\begin{poznamka}
    Samozrejme, náš dôkaz ``správnosti'' by ostražitého čitateľa nemal
    presvedčiť úplne. V takomto prípade je odporúčané nazrieť do
    \cite{yao}.
\end{poznamka}

\todo{bezpecny vypocet funkcie}

