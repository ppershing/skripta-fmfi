\section{Bit commitment}

Bit commitment schéma je protokol pre dvoch účastníkov, kde sa najprv účastník
zaviaže k nejakému bitu a následne po istom čase ho odhalí.
Formálne to môžeme definovať takto:

\begin{definicia}
Majme dve množiny $X,Y$ a funkciu $f\colon \{0,1\} \times X \to Y$, ktorú vieme
\clqq ľahko\crqq počítať. Od $f$ požadujeme navyše ešte tieto vlastnosti:
\begin{itemize}
\item Zo znalosti $f(b,x)$ je ťažké určiť $b$ - vlastnosť utajenia.
\item Je ťažké nájsť $x, y$ také, že $x \neq y$ a $f(0,x) = f(1,y)$ - vlastnosť záväznosti.
\end{itemize}
Potom bit commitment protokol vyzerá nasledovne:
\begin{enumerate}
\item $A$ zvolí $b \in \{0,1\}$ ku ktorému sa chce zaviazať a $x \inr X$
\item $A \to B$: $y = f(b,x)$ - záväzok
\item $A \to B$: $x$ - odhalenie, môže prísť po istom čase
\item $B$ overí, či $y = f(0,x)$ alebo $y = f(1,x)$
\end{enumerate}
\end{definicia}

Tento protokol môžeme realizovať viacerými spôsobmi. 

\noindent{\bf{Bit commitment pomocou RSA}}

Majme nejakú inštanciu RSA systému, teda trojicu $(n,e,d)$, kde účastník $b$ nepozná súkromný kľúč.
Bit commitment realizujeme nasledovne:
\begin{enumerate}
\item Záväzok: $A$ si zvolí $x \inr \mathbb{Z}_n^*$ a pošle $B$: $y = x^e \pmod n$, v tomto prípade je $b$ najmenej signifikatný bit z $x$
\item Odhalenie: $A$ pošle $x$. $B$ overí, či $x^e \pmod n = y$
\end{enumerate}

Vlastnosť utajenia je dodržaná, keďže možnosť zistiť $b$ je ekvivalentná rozbitiu RSA schémy.
Vlastnosť záväznosti je tiež dodržaná, keďže k jednému $y$ existuje iba jedno $x$. V tomto prípade ide o nepodmienenú bezpečnosť.

\todo{realizacia cez diskretny logaritmus}
\todo{IDS pre ham cycle pomocou BC}


