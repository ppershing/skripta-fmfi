\section{Bezpečnosť niektorých hašovacích funkcíí}

S hašovacími funkciami sme sa zoznámili v zimnom semestri.
Pripomeňme, že je to zobrazenie $h: X \to Y$, kde $Y$ je konečná množina.
Pre jednoduchosť predpokladajme, že $Y = \{0,1\}^n$. 

Dobrá hašovacia funkcia by mala spĺňať nasledovné požiadavky 
(vychádzajú z toho, že by sa mala čo najviac priblížiť náhodnému orákulu). 
Okrem požiadaviek zo zimného semestra formulujeme niekoľko ďalších:
\begin{enumerate}
\itemsep -1.2mm
\item Odolnosť prvého vzoru: K danému $y$ je ťažké nájsť $x$ také, 
že $h(x) = y$. Očakávaný čas hľadania $x$ je $\Theta(2^n)$.
\item Odolnosť druhého vzoru: K danému $x$ je ťažké nájsť 
$z \neq x$ také, že $h(x) = h(z)$. Očakávaný čas hľadania je opäť
$\Theta(2^n)$.
\item Odolnosť proti kolíziám: Je ťažké nájsť $x \neq x$ také, 
že $h(x) = h(z)$. Očakávaný čas hľadania je $\Theta(2^{n/2})$ (narodeninový útok).
\vskip 0.5cm
\item $k$-odolnosť prvého vzoru: K danému $y$ je ťažké nájsť navzájom rôzne 
$x_1, x_2, \dots, x_k$ také, že $h(x_i) = y$. Očakávaný čas hľadania
je $\Theta(k 2^n)$.
\item $k$-odolnosť druhého vzoru: K danému $x$ je ťažké nájsť navzájom rôzne 
$x_1, x_2, \dots, x_k$ také, že $h(x_i) = h(x)$ a $x_i \neq x$.
Očakávaný čas hľadania je $\Theta(k 2^n)$.
\item $k$-odolnosť proti kolíziám: Je ťažké nájsť navzájom rôzne 
$x_1, x_2, \dots, x_k$ také, že $\forall i, j\colon h(x_i) = h(x_j)$. 
Dá sa ukázať, že pokiaľ je $k$ zanedbateľné oproti $2^n$, tak očakávaný 
čas hľadania $k$-kolizie je $\Theta(2^{\frac{n(k-1)}{k}})$.
Pre väčšie $k$ je očakávaný čas väčší.
\end{enumerate}

\subsection {Iteratívne konštrukcie hašovacích funcíí a ich slabiny}

Rozdeľme text $M$ na rovnakodlhé bloky $m_1, m_2, \dots, m_l$
(v poslednom bloku môžeme očakávať nejaký padding).
Položme $h_0 = IV$ a $h_i = f(h_{i-1}, m_i)$ pre $i \geq 1$, kde $f$ 
je kompresná funkcia, ktorej obor hodnôt je $\{0,1\}^n$.
Výstupom hašovacej funkcie je hodnota $h_l$. Takúto konštrukciu 
používa väčšina v súčasnosti používaných hašovacích funkcií
(MD5, SHA-1, SHA-xxx, \dots).

\begin{figure}[h!]
    \label{fig:iter}
    \centering
    \includegraphics[scale=1.0]{img/5/iter.1.mps}
    \caption{Iteratívna hašovacia funkcia}
\end{figure}


V roku 2004 \cite{Joux04} bol nájdený nasledovný útok:
\begin{enumerate}
\itemsep -1.2mm
\item Položme $h_0 = IV$.
\item Nájdime $m_1 \neq m_1'$ také, že $f(h_0, m_1) = f(h_0, m_1') = h_1$, 
toto vieme v čase $O(2^{n/2})$.
\item Podobne nájdime dvojice $m_2 \neq m_2', \dots, m_k \neq m_k'$ také, 
že $f(h_{i-1}, m_i) = f(h_{i-1}, m_i') = h_i$.
\item Toto celé vieme spraviť v čase $O(k 2^{n/2})$. A dostaneme $2^k$ 
kolidujúcich textov (pre každý z $k$ blokov si môžeme vybrať
2 rôzne texty). Hash každého z nich bude $h_k$.
\end{enumerate}

\begin{figure}[h!]
    \label{fig:joux1}
    \centering
    \includegraphics[scale=1.0]{img/5/joux.1.mps}
    \caption{Útok na multikolíziu}
\end{figure}


Samotný tento útok ešte veľa neznamená. Ale ukazuje sa, že
ho vieme využiť na nájdenie podstatnejších slabín.

Ukážeme si rýchlejšie hľadanie $2^k$-vzoru:
\begin{enumerate}
\itemsep -1.2mm
\item Nájdeme $2^k$-kolíziu.
\item Následne nájdeme posledný blok $m_{k+1}$ tak, aby $f(h_k, m_{k+1}) = y$. Toto vieme v čase $O(2^n)$.
\end{enumerate}

\begin{figure}[h!]
    \label{fig:joux2}
    \centering
    \includegraphics[scale=1.0]{img/5/joux.2.mps}
    \caption{Využitie multikolízie pri hľadaní multivzoru}
\end{figure}

Celkový čas hľadania bude $O(2^n)$ namiesto očakávaného $\Theta(k 2^n)$. Presne rovnako vieme hľadať aj druhý vzor.

Niekedy v záujme zvýšenia bezpečnosti niekedy zreťazujeme za sebou výstup dvoch rôznych hašovacích funkcíí, napr.
MD5 a SHA-1. Dostaneme takzvanú kaskádovú hašovaciu funkciu, formálne $H(m) = H_1(m)||H_2(m)$. 
Ukazuje sa, že ak v jednej z týchto dvoch hašovacích funkcíí vieme efektívne hľadať multikolízie, tak vieme
v kaskádovej hašovacej funkcii hľadať kolízie efektívnejšie ako len narodeninovým útokom.
Pre jednoduchosť predpokladajme, že obor hodnôt $H_1$
a $H_2$ je $\{0,1\}^n$. Teda dĺžka výstupu $H$ je $2n$ a teda očakávaný čas hľadania kolízie je $\Theta(2^n)$.
Predpokladajme, že $H_1$ je iteratívna. Potom vieme v čase $O(\frac{n}{2} 2^{n/2})$ nájsť $2^{n/2}$ kolízií. Tieto dosadíme
na vstup $H_2$ a narodeninový paradox nám zaručí vysokú úspešnosť nájdenia kolízie aj pre $H_2$. Takto sme dostali
kolíziu pre $H$ v čase $O(n 2^{n/2})$.

\subsection{Hľadanie expandovateľnej správy a jej využitie}

Niekedy je výhodné mať kolidujúce správy rôznej dĺžky. 
V nasledujúcom texte si popíšeme nájdenie takej multikolízie,
kde všetky správy majú rôzne dĺžky. Nech $m^*$ je ľubovoľný
text dĺžky 1 blok.
Označme ako $F(h_i, a_1||a_2||\dots||a_x) = f(f(\dots f(h_i, a_1), a_2), \dots, a_x)$,
teda viackrát za sebou použitú kompresnú funkciu $f$.
Postup je podobný ako pri hľadaní multikolízie:
\begin{enumerate}
\itemsep -1.2mm
\item Položme $h_0 = IV$
\item Nájdime $m_1$ a $m_1'$ také, že: $f(h_0, m_1) = f(f(h_0, m^*), m_1') = F(h_0, m^*||m_1') = h_1$. 
\item Nájdeme ďalšie dvojice $m_2, m_2', \dots, m_k, m_k'$ také, že platí:
$f(h_{i-1}, m_i) = F(h_{i-1}, (m^*)^{2^{i-1}}||m_i') = h_i$
\item Každú dvojicu vieme nájsť v čase $O(2^{n/2})$. Celkový čas bude $O(k 2^{n/2})$.
Dostali sme opäť $2^k$ rôznych kolidujúcich textov, ktoré tentokrát 
majú dlžky $k, k+1, \dots, k+2^k-1$ blokov. 
Blok príslušnej dĺžky nájdeme tak, že od dĺžky odpočítame $k$, následne
sa pozrieme na binárny zápis výsledku postupne odzadu. Ak je tam $0$, ideme
po hrana, ktorá nemá $m^*$, v prípade $1$ ideme po tej druhej.
\end{enumerate}

\begin{figure}[h!]
    \centering
    \includegraphics{img/5/expand.1.mps}
    \caption{Konštrukcia multikolízie rôznej dĺžky}
    \label{fig:expand1}
\end{figure}


Pri niektorých hašovacích funkciách, ktoré využívajú
Davies-Meyerovu konštrukciu vieme expandovateľnú správu hľadať
ešte jednoduchšie. Príkladom môže byť SHA-1, ktorej kompresná funkcia
je definovaná nasledovne:
$f(h, m) = E_m(h)+h$

Zoberme náhodné $m$. Potom vieme vypočítať tzv. pevný bod, teda vnútorný
stav ktorý sa po \clqq zhašovaní\crqq $m$ nezmení:
$$f(h,m) = h$$
$$E_m(h) + h= h$$
$$h = E_m^{-1}(0)$$

Vypočítajme si $2^{n/2}$ pevných bodov $h_1, h_2, \dots, h_{2^{n/2}}$ a k ním príslušné správy
$m_1, m_2, \dots, m_{2^{n/2}}$. 
Následne skúšajme rôzne $m'$ a hľadajme také, kde $f(h_0, m') = h_i$, kde $i \in \{1, 2, \dots 2^{n/2}\}$.
Vďaka narodeninovému paradoxu stačí $2^{n/2}$ pokusov. Takto sme v čase $O(2^{n/2})$ našli správu, ktorú
vieme ľubovoľne natiahnuť.
\todo{OBRAZOK}

Následne hľadanie expandovateľnej správy vieme využiť na efektívnejšie hľadanie druhého vzoru ako hrubou silou.
Majme správu, ktorá je dlhá $k + 2^k - 1$ blokov. Následne zostrojme expandovateľnú správu dĺžky $k$, ktorej hash je $h_k'$.
Následne skúšajme nájsť  $\tilde{m}$ také, že $f(h_k', \tilde{m})$ sa rovná niektorej z hodnôt $h_{k+1}, \dots h_{k+2^k-1}$. Ak áno 
(označme takú hodnotu ako $h_x$), tak môžeme bloky $m_1, m_2, \dots, m_x$ nahradiť príslušnou expandovateľnou správou.
Očakávaný počet pokusov je $\Theta(2^{n-k})$. 

Čo to v praxi znamená? Zoberme bežne používanú SHA-1, ktorej výstup má 160 bitov. Teda,
očákavaný počet operácií pre nájdenie druhého vzoru je úmerný $2^{160}$. Maximálna dĺžka správy
pre SHA-1 je $2^{64} - 1$ bytov, čo je asi $2^{55}$ blokov. Zoberme správu dĺžky
$54 + 2^{54} - 1$ blokov. Najprv v čase $2^{80}$ nájdeme expandovateľnú správu (SHA-1 je DM konštrukcia).
Následne potrebujeme ešte $2^{106}$ pokusov na nájdenie \clqq spájacieho\crqq bloku.
Toto je síce stály veľký počet, ale oproti očakávanému počtu je to značné zlepšenie.

\subsection{Obrana pred týmto druhom útokov}
Tieto útoky ukazujú, že iteratívna konštrukcia nie je najvhodnejšia.
To, čo ju môže zachrániť je mať aspoň 2-krát dlhší medzistav 
(t.j. hodnoty $h_1, h_2, \dots, h_{k-1}$). Potom sa skomplikuje hľadanie
kolízií v medzistavoch a konštrukcia je proti týmto útokom bezpečná.

\subsection{Nostradamov útok}

Uvažujme nasledujúcu situáciu. Prorok vyriekne začiatkom roka 2009
vyriekne proroctvo, napríklad stav akciového indexu na konci roka 2010.
Akurát ho nezverejní, ale zverejní len hash tohoto proroctva (v podstate je to
niečo ako commitment).
Následne na konci roka odhalí svoje proroctvo. Hash je v poriadku, proroctvo
o akciách hovorí pravdu, za ním sú ešte nejaké balastové proroctvá.
Otázkou je ako veľmi môžeme veriť tomu, že prorok poznal stav
akcií na konci roka. Ukazuje sa, že na iteratívnu konštrukciu
hašovacích funkciu existuje tzv. chosen prefix attack.

\subsubsection{Konštrukcia}

Pripravme si najprv $2^k$ hodnôt $h_{0,1}, h_{0,2}, \dots, h_{0,2^k}$.
Následne hľadajme $m_0,1, m_0,2, \dots, m_{0,2^k}$ také, že
$f(h_{0,1}, m_{0,1}) = f(h_{0,2}, m_{0,2}) = h_{1,1}, \dots f(h_{0,2^k-1}, m_{0,2^k-1}) = f(h_{0,2^k}, m_{0,2^k}) = h_{1, 2^{k-1}}$
Následne pokračujme podobným spôsobom aj na ďalších úrovniach. Až dostaneme jednu hodnotu $h_{k,1}$. Túto
hodnotu zverejníme ako hash nášho proroctva.
Navyše, aby sme za správu vedeli \clqq prilepiť\crqq rozumný obsah, je vhodné aby tieto správy dávali aspoň nejaký zmysel.

Aký čas strávime prípravou tohoto stromu? Nájdenie jednej kolízie trvá čas $O(2^n)$, na prvej úrovni treba $2^{k-1}$ kolízii,
na druhej $2^{k-2}$ a na poslednej potrebujeme jednu kolíziu. Celkový čas teda je: $O(2^n (2^{k-1} + 2^{k-2} + \dots + 1)) = O(2^n 2^k)$.
Lepší spôsob je nehľadať kolízie na jednotlivých úrovniach po dvojiciach, ale rovno pre celú úroveň
potom na jednej úrovni strávime čas $O(2^{k/2 + n/2})$. Celkový čas bude tiež $O(2^{k/2 + n/2})$.

\begin{figure}[h]
    \centering
    \includegraphics[scale=0.9]{img/5/nostradamus.1.mps}
\end{figure}

Teraz na konci roka dostaneme pôvodnú správu. Označme ju $m$. Po jej zahashovaní sa dostaneme do odtlačku
$h_m$. Teraz hľadáme správu $\tilde{m}$, pre ktorú platí $f(h_m, \tilde{m}) = h_{0,x_0}$, kde $x_0 \in \{1, 2, \dots, 2^k\}$.
Následne na základe pripraveného stromu, vieme za správu dolepiť patričné $m_{0,x_0}, m_{1,x_1}, \dots, m_{k,x_k}$ také, že
výsledná hash bude predtým zverejnená $h_{k,1}$. Teda na konci zverejníme správu $m\tilde{m}m_{0,x_0}, m_{1,x_1}, \dots, m_{k, x_k}$.

Čas hľadania $\tilde{m}$ bude $2^{n-k}$. 
Celkový čas útoku. Bolo by dobré, aby čas trvania oboch časti útoku bol viacmenej vyvážený. 
Uvažujme najprv pomalšiu variantu predspracovania. Vtedy chceme, aby $2^{n-k} = 2^{k+n/2}$, z čoho dostaneme
$k=\frac{n}{4}$ a celkový čas bude $2^{\frac{3}{4} n}$. Pri rýchlejšom predspracovaní
máme podmienku $2^{n-k} = 2^{k/2 + n/2}$, z čoho máme $k = \frac{n}{3}$ a celkový čas $2^{\frac{2}{3}n}$.
