\section{Identity based crypto}

Asymetrické kryptovacie systémy, ktoré veľmi dobre poznáme, vyžadujú,
aby si každý vygeneroval vlastnú inštanciu danej schémy. Problém pri
takomto riešení je distribúcia verejného kľúča. Ak nemám dostatočne
dôveryhodný kanál na distribúciu môjho verejného kľúča, útočník si
môže pripraviť vlastný kľúč a presvedčiť obeť o tom, že je to môj
kľúč. Navyše, existujú isté ťažkosti s udržiavaním verejných kľúčov a
limity sú aj v chápanlivosti ľudí o potrebe bezpečného kanálu pri
distribúcii.
Isté riešenie tohoto problému,
ktoré sa ujalo najmä pri webových aplikáciach sú certifikáty --
kľúč sa podpíše dôveryhodnou autoritou a každý si môže
overiť, že po kanáli dostal správny kľúč. V tejto kapitole sa budeme
venovať alternatívnemu riešeniu -- verejným kľúčom bude priamo moja
identita.

Ako si to teda môžeme predstaviť? Nebudeme generovať inštanciu schémy,
ale uvažujme rovno, že mojím kľúčom bude priamo moja identifikácia -
či už email, OpenID konto a pod.

Na pomoc si ale budeme musieť prizvať dôveryhodnú tretiu stranu $T$.
Táto strana bude mať 2 kľúče master public key $M_{pk}$ a master
secret key $M_{sk}$. Túto tretiu stranu budeme potrebovať, aby nám
vygenerovala náš súkromný kľúč k môjmu verejnému kľúču (kontu).
Formálne, toto generovanie si môžeme popísať ako algoritmus, ktorý
užívateľovi $U$ priradí na záklede jeho identity $ID_U$ jeho súkromný
kľúč $U_{sk} = Extract(M_{sk}, ID_U)$. Samozrejme, kľúč ešte treba
dôveryhodne doručiť, toto ale nebudeme riešiť.

Potom môžeme šifrovať a dešifrovať nasledovne:
$c=E(m,ID_U,M_{pk})$ a opačne $m = D(c, U_{sk}, M_{pk})$, pričom
samozrejme požadujeme
$\forall ID_U \forall m: D(E(m, ID_U, M_{pk}), Extract(M_{sk},ID_U),
M_{pk})=m$.

Takáto schéma má ale výraznú bezpečnostnú slabinu - kompromitácia
$M_{sk}$ je fatálna. Je teda otázne, do akej miery môžeme dôverovať
zabezpečeniu dôveryhodnej strany. Napriek tomuto problému si ukážeme,
ako môže taká schéma fungovať.

\subsection{Coksova schéma (2001)}
\todo{cela schema}
