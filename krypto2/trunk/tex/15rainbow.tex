\section{Dúhové tabuľky}

Hacker Ivan Ivanovič získal prístup k zašifrovanému heslu šéfa FBI. Bolo uskladnené v tvare
$H(password)$, kde $H$ je nejaká hašovacia funkcia. Rozhodol sa ho prelomiť.
Postupne vygeneroval všetky heslá dĺžky 1, potom dĺžky 2, \dots. Keďže heslo malo $n$ bitov, tak
mu to zabralo nakoniec čas $O(2^n)$, čo bol asi rok, takže mu to nakoniec bolo nanič, lebo heslo bolo dávno iné.
 Ale pamäť neuvyužil skoro vôbec. Sklamaný svojím neúspechom sa rozhodol, že si 
predráta zahašovanú hodnotu hesla, pre všetky rôzne heslá do dĺžky $n$. Dúfal, že
keď následne uloví nejaké heslo, tak ho rozšifruje skoro hneď.  Tak sa pustil do rátania.
Čoskoro zaplnil prvý disk svojou tabuľkou. Tak si kúpil ďalších 10. A potom ďalších 100.
A takto čoskoro vykúpil disky z celého Ruska, ale stále nemal uložené všetko čo treba.
Pozorný čitateľ si isto všimol, že keď dostaneme na nejaké heslo na rozbitie, tak čas
bude $O(1)$, ale pamäť máme $O(2^n)$. Prirodzená otázka znie, že keďže nechcem dopadnúť ako Ivan, tak
či neexistuje niečo medzi týmito hodnotami (obetujeme trochu času, aby sme mohli zabrať menej pamäte).
Ukazuje sa, že áno. Ako na to sa dozviete v ďalšom texte.
