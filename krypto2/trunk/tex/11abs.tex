\section{1. prednáška Absolútne bezpečná šifra}

Tento krát sa ideme zaoberať nepodmienenou bezpečnosťou. Uvažujme 
výpočtovo neobmedzene silného útočníka a cypher text only attack (COA).
Požadujeme aby útočník zo znalosti šifrového textu nezískal nič nové.

Označme množinu otvorených textov ako $P$, kľúčov ako $K$, šifrových textov ako $C$
(uvažujeme iba množinu reálne možných šifrových textov). Každému $x \in P$ vieme priradiť
pravdepodobnosť výskytu $p(x)$, pravdepodobnosť
výskytu konkrétneho kľúča $k \in K$ označíme ako $p(k)$. Predpokladáme, že
konkrétny kľúc používame iba raz a že jeho voľba nezávisí od otvoreného textu.
Takisto pravdepodobnosť výskytu šifrového textu $c \in C$ označíme $p(c)$. Táto pravdepodobnosť
bude vždy väčšia ako nula (keďže uvažujeme len šifrové texty, ktoré môžu vzniknúť zašifrovaním).
Všetky tieto množiny považujeme za konečné.

\begin{definicia}
Šifrovací systém $(E,D)$ je absolútne bezpečný ak:
$$\forall x \in P \forall y \in C\colon p(x | y) = p(x)$$
\end{definicia}
\begin{komentar}
Táto definícia vlastne hovorí o tom, že znalosť šifrového textu útočníkovi nepovie nič
nové, keďže distribúciu pravdepodobnosti otvorených textov pozná.
\end{komentar}

Niekoľko zaujímavých vlastností:
Vzhľadom na to, že dešifrovanie musí byť uskutočniteľné, musí platiť $|C| \geq |P|$.
Ďalej určite platí $|K| \geq |P|$, ináč by sme pri znalosti šifrového textu mohli 
vyskúšať dešifrovanie všetkými kľúčmi a niektoré otvorené texty by sme mohli vylúčiť
($p(x|c) = 0$). Zaujímavá je situácia keď $|P| = |C| = |K|$, tú popisuje nasledujúca veta:

\begin{veta}{(Shanon)}
Nech $(E,D)$ je šifrovací systém, kde $|P|=|C|=|K|$. Potom tento šifrovací systém je
absolútne bezpečný práve vtedy, keď sú splnené nasledujúce vlastnosti:
\begin{enumerate}
\item $\forall k \in K\colon p(k) = \frac{1}{|K|}$
\item $\forall x \in P \forall y \in C \exists! k \in K\colon E_k(x) = y$
\end{enumerate}
\end{veta}

\begin{dokaz}
($\Rightarrow$) Nech systém $(E,D)$ je absolútne bezpečný. Ak by pre $x \in P, y \in C$ neexistoval
kľúč $k \in K$ taký, že $E_k(x) = y$, útočník zo znalosti $y$ vie vylúčiť otvorený text $x$,
čo odporuje tomu, že systém $(E,D)$ je absolútne bezpečný. 
Keďže $|C|=|K|$, tak tento kľúč môže byť maximálne jeden (keby ich bolo viac, tak by $\exists z \in C$ taký, že
$x$ nevieme zašifrovať na $z$). 

Keďže $P$ je konečná, môžeme otvorené texty očíslovať, teda $P = \{ x_1, x_2, \dots, x_n\}$. 
Teraz fixujme $c \in C$. Pre každý otvorený text musí platiť:
$$p(x_i) = p(x_i | c) = \frac{p(c | x_i) p(x_i)}{p(c)}$$
$$p(c) = p(c | x_i) = p(k_i)$$

Kde $k_i \in K\colon E_{k_i}(x_i)=c$. Z toho vyplýva, že všetky kľúče majú rovnakú pravdepodobnosť a keďže
súčet výskytu ich pravdepodobnosti je $1$, tak $\forall k \in K\colon p(k) = \frac{1}{|K|}$.

($\Leftarrow$) Spočítame $p(x|c)$ a ukážeme, že definícia platí. Vieme, že:
$$p(x|c) = \frac{p(c|x)p(x)}{p(c)}$$

$p(c|x) = p(k)$, kde $E_k(x)=c$, teda $p(c|x) = \frac{1}{|K|}$. Ešte treba určiť $p(c)$.
$$p(c) = \displaystyle\sum_{k \in K} p(k)p(D_k(c)) = \frac{1}{|K|} \displaystyle\sum_{x \in P} p(x) = \frac{1}{|K|}$$

(Pri dešifrovaní všetkými možnými kľúčmi dostaneme všetky možné otvorené texty (a každý práve raz). A súčet pravdepodobností
výskytu všetkých otvorených textov je 1).

Takže dostávame:
$$p(x|c) = \frac{p(c|x)p(x)}{p(c)} = \frac{\frac{1}{|K|} p(x)}{\frac{1}{|K|}} = p(x)$$

A teda $(E,D)$ je absolútne bezpečný, čím je náš dôkaz hotový.

\end{dokaz}

\begin{priklad}
Vernamova šifra je príklad absolútne bezpečného šifrovacieho systému.
\end{priklad}

Iným uhlom pohľadu na absolútne bezpečnú šifru je nerozlíšiteľnosť 2 otvorených textov. 
Teda útočník dostane dva otvorené texty a jeden šifrový text a má určiť, z ktorého otvoreného
textu šifrový text vznikol. Tu nesmie uspieť.

\begin{veta}
Šifrovací systém $(E,D)$ je absolútne bezpečný práve vtedy, keď:
$$\forall x_1 \in P \forall x_2 \in P \forall c \in C\colon p(c|x_1)=p(c|x_2)$$
\end{veta}

\begin{dokaz}
($\Rightarrow$) Vieme, že $p(c|p_1) = \frac{p(p_1|c) p(c)}{p(p_1)}$ Keďže $(E,D)$ je absolútne
bezpečný, tak $p(p_1|c) = p(p_1)$, čiže $p(c|p_1) = p(c)$. To isté dostaneme aj pre $p_2$.

($\Leftarrow$) $p(c|x)$ je rovnaká pre všetky $x \in P$, čiže je konštantná. Nech $p(c|x) = t$.
Ešte dopočítame:
$$p(c) = \displaystyle\sum_{k \in K} p(k) p(c|D_k(c)) = t \displaystyle\sum_{k \in K} p(k) = t$$
Potom $p(x|c) = p(x)$, čo sme chceli dokázať.
\end{dokaz}

Takto môžeme podať trochu inú definíciu absolútne bezpečnej šifry.

\begin{definicia}
Uvažujme kryptoanalytickú hru s nasledovným priebehom:
\begin{enumerate}
\item Útočník (A) pošle druhej strane B dva otvorené texty $p_1, p_2$.
\item B si zvolí náhodne $b \inr \{0,1\}$, náhodný $k \in K$ a pošle A: $c = E_k(p_b)$
\item A sa snaží zistiť z ktorého otvoreného textu $c$ vznikol. Svoj úsudok pošle ako $b'$.
\item Ak $b' = b$, tak A uspel. V opačnom prípade neuspel.
\end{enumerate}
Pokiaľ je $(E,D)$ absolútne bezpečný, tak pravdepodobnosť úspechu A je $\frac{1}{2}$.
\end{definicia}

Dá sa ukázať, že tieto dve definície sú ekvivalentné.
\todo{dokaz}
