\section{Eliptické krivky}

Tento často počuteľný výraz znie veľmi matematicky. Napriek tomu si
ukážeme, že je to pomerne jednoduchý spôsob, ako generovať istý týp
grúp, v ktorých budeme vedieť následne počítať známe kryptografické
konštrukcie.

\begin{poznamka}
    Osobný názor doc. Staneka je, že eliptické krivky nahradia do
    niekoľko rokov RSA. Napríklad odporúčania na šifrovanie vládnych
    dokumentov v USA sa už ani nezmieňujú o používaní RSA. Nevýhodou
    RSA je totiž veľmi dlhý modulus pri ekvivalentnej bezpečnosti.
    Môžeme teda povedať, že diskrétny logaritmus začne vládnuť svetu
    :-)
\end{poznamka}

Ešte predtým, ako si povieme, čo to vlastne tieto krivky sú, uvedieme
si najväčšiu výhodu -- pre sofistikované algoritmy na
výpočet diskrétneho algoritmu ako napríklad general number field
sieve, index calculus nepoznáme v súčasnosti úpravy, ktoré by umožnili
ich aplikáciu na eliptické krivky. Dajú sa teda použiť iba generické
algoritmy. Tým pádom môžeme používať menšie kľúče, budeme mať menšie
podpisy, ... Hneď ale pripojíme aj varovanie -- existujú špecifické
útoky na isté typy eliptických kriviek, preto je potrebné venovať
generovaniu krivky dostatočnú pozornosť.

\begin{definicia}[Eliptická krivka]
    Pod eliptickou krivkou nad poľom $F$ budeme označovať množinu
    všetkých takých bodov, ktoré vyhovujú rovnici
    \begin{equation*}
        x^3 + A x + B = y^2
    \end{equation*}
    kde $A,B$ sú vopred zvolené konštanty.
    Pre body $(x,y)$ si zavedieme grupovú operáciu sčítania ``+''
    a neutrálny prvok $0$, ktorý bude predstavovať bod v nekonečne.
\end{definicia}

Ak si zoberieme pole $F=R$, eliptická krivka $x^3 -3x+3=y^2$ je
zobrazená na obrázku \ref{fig:elliptic1}

\begin{figure}[h]
    \centering
    \includegraphics{img/09/elliptic.1.mps}
    \caption{Eliptická krivka $x^3 - 3x +3 = y^2$ v reálnych číslach}
    \label{fig:elliptic1}
\end{figure}

Ešte predtým, ako si ukážeme sčítanie bodov, ktoré nie je úplne
triviálne, budeme sa zaoberať podmienkou na to, aby daná krivka bola
regulárna.

Predstavme si, že krivka je singulárna, t.j. existuje dvojitý koreň,
označme ho $a$ a označme tretí koreň $b$ (neplatí nutne $a\ne b$).
Dostávame
\begin{align*}
    (x-a)^2 (x-b) &= x^3 + Ax + B \\
    x^3 + x^2 (-b -2a) + x (2ab + a^2) + (-b a^2) &= x^3 + Ax + B
\end{align*}
Z koeficientu pred $x^2$ dostávame $b=-2a$ a teda
\begin{align*}
    A &= 2ab + a^2 = -3 a^2 \\
    B &= -b a^2 = 2 a^3
\end{align*}
Čo nastáva v prípade, ak
\begin{equation*}
    4 A^3 + 27 B^2 = 0
\end{equation*}
Pri generovaní kriviek budeme teda musieť otestovať túto rovnice a v
prípade rovnosti generovať novú krivku.

Poďme ale späť k sčítavaniu
\begin{definicia}[Sčítanie bodov eliptickej krivky]
    Majme body $P=[x_1,y_1]$ a $Q=[x_2,y_2]$.\footnote{Body
        eliptických kriviek zvykneme označovať veľkými písmenami}
    Nech $P \ne -Q$, kde znamienkom ``-'' budeme označovať
    bod $(x,-y)$ (Prípad $P = -Q$ definujeme ako $P+Q=0$).
    Potom operáciu sčítania $P+Q$ definujeme nasledovne:

    \begin{equation*}
        P+Q = [ \underbrace{\lambda^2 - x_1 - x_2}_{x_3}, \ 
            \lambda (x_1 - x_3) - y_1]
    \end{equation*}
    kde $\lambda$ je definované ako
    \begin{equation*}
        \lambda =
            \begin{cases}
                (y_2 -y_1)(x_2 - x_1)^{-1} \quad & P \ne Q \\
                (3 x_1^2 + A)(2 y_1)^{-1} \quad & P = Q
            \end{cases}
    \end{equation*}
\end{definicia}

Budeme tvrdiť, že množina bodov $\{(x,y)\} \union 0$
eliptickej krivky spolu s operáciu ``+'' bude tvoriť grupu. Toto
netriviálne algebraické tvrdenie prenecháme na neveriaceho
čitateľa. Ak teda zoberieme ako $F$ nejaké konečné
pole, dostávame konečnú grupu.

Predchádzajúca definícia sčítania možno nebola veľmi intuitívna.
Ukážeme si teda iný prístup k definícii sčítania bodov na eliptickej
krivke. Spôsob je to čisto grafický.
Ak máme body $P,Q$, body $-P, 2P, P+Q$ získame postupne ako
zrkadlenie bodu $P$ podľa osi $x$,
zkradlenie priesečníka dotyčnice v bode $P$ s krivkou
a zkradlenie tretieho priesečníka priamky prechádzajúcej bodmi $P,Q$ a
eliptickej krivky. Názorne to možno vidieť na obrázku
$\ref{fig:elliptic-plus}$.

\begin{figure}
    \centering
    \subfigure[$-P$]{
        \includegraphics{img/09/elliptic.2.mps}
    }
    \subfigure[$2P$]{
        \includegraphics{img/09/elliptic.3.mps}
    }
    \subfigure[$P+Q$]{
        \includegraphics{img/09/elliptic.4.mps}
    }

    \caption{Grafické sčítanie na eliptických krivkách}
    \label{fig:elliptic-plus}
\end{figure}

\begin{priklad}
    Uvedieme si ilustratívny príklad eliptickej krivky. Uvažujme
    rovnicu
    \begin{equation*}
        y^2 = x^3 + x + 1 \pmod{11}
    \end{equation*}
    Vypočítajme body vyhovujúce krivke:

    \begin{table}[h!]
        \centering
        \begin{tabular}{c|c|c}
            $x$ & $x^3 + x + 1 \pmod{11}$ & $y$ \\
            \hline 0 & 1 & 1, 10 \\
            \hline 1 & 3 & 5, 6 \\
            \hline 2 & 0 & 0 \\
            \hline 3 & 9 & 3, 8 \\
            \hline 4 & 3 & 5, 6 \\
            \hline 5 & 4 & 2, 9 \\
            \hline 6 & 3 & 5, 6 \\
            \hline 7 & 10 & - \\
            \hline 8 & 4 & 2, 9 \\
            \hline 9 & 2 & - \\
            \hline 19 & 10 & - \\
        \end{tabular}
        \caption{Body ležiace na eliptickej krivke $x^3 + x + 1
        \pmod{11}$}
    \end{table}
    Dostávame teda, že množina bodov našej krivky je
    \begin{equation*}
       E=\Big\{ [0,1], [0,10], [1,5], [1,6],
                [2,0], [3,3], [3,8],
                [4,5], [4,6], [5,2], [5,9],
                [6,5], [6,6], [9,2], [9,9], 0 \Big\}
    \end{equation*}
    Príklad niektorých sčítaní:
    \begin{align*}
        [2,0] + [2,0] &= 0 \\
        [1,5] + [1,6] &= 0 \\
        [0,1] + [3,3] &= [6, 6] 
    \end{align*}
\end{priklad}
