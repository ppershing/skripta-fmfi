\section{Algoritmy na výpočet diskrétneho logaritmu}
Počas celej tejto kapitoly sa budeme pohybovať v cyglickej grupe 
$G, |G|=n$ generovanej generátorom $g \in G$.
Našim cieľom bude vypočítať diskrétny logaritmus prvku $y$
pri základe $g$, t.j. nájsť hodnotu $x$, takú, že $g^x=y$ (počítané
v grupe $G$).


Triviálnym algoritmom je postupné skúšanie možných hodnôt $x$, teda
$g^0,g^1,\dots,g^{n-1}$.
Časová zložitosť je v tomto prípade $O(n)$, pamäťová $O(1)$.

\subsection{Shanksov Baby-step/giant-step algoritmus}
Baby-step/giant step je spôsob, ako preliať čas do pamäte.
Algoritmus sa skadá z 2 častí - ``veĺkých'' a ``malých'' krokov.
Veľké kroky sa dajú predpočítať raz, malé kroky treba robiť pre každé
číslo, z ktorého chceme zistiť diskrétny logaritmus.

Najskôr budeme robiť "veľké kroky".
Zoberme si $t=\lfloor \sqrt{n} \rfloor$.
Vypočítajme si postupne hodnoty $g^0,g^t,\dots,g^{\lfloor n/t \rfloor t}$.
Pre každú hodnotu, ktorú takto vypočítame si zapamätáme (napríklad do
hashovacej tabuľky) aj jej diskrétny logaritmus, t.j. zapamätáme si hodnoty
$(k,g^{kt})$ pre $k=0,1,\dots,\lfloor n/t \rfloor t$.
Táto časť algoritmu trvá $O(t)$, teda $O(\sqrt{n})$, potrebná pamäť je
tiež $O(t)=O(\sqrt{n})$.

Po tomtp predpočítaní môžeme pokračovať malými krokmi.
Počítajme postupne hodnoty $y g^0,yg,yg^2,\dots, yg^{t-1}$ a pozeráme sa,
či niektorá z týchto hodnôt sa nenachádza v zapamätanej tabuľke.
Grafické znázornenie činnosti celého algoritmu je na obrázku
\ref{fig:babygiant}.
\begin{figure}[h]
    \centering
    \label{fig:babygiant}
    \caption{Kroky v baby-step/giant-step algoritme}
    \includegraphics{img/4/babygiant.1.mps}
\end{figure}

Zložitosť druhého kroku je maximálne $t$ krát tabuľkový lookup, pretože
najneskôr o $t$ hodnôt narazíme na nejakú hodnotu, ktorá už v tabuľke bude.
Dokopy teda po oboch krokoch nájdeme diskrétny logaritmus
v čase $O(\sqrt{n})$ a pamäti $O(\sqrt{n})$.

Problém pamäťovej zložitosti sa snaží riešiť Pollardov algoritmus.

\subsection{Pollardov $\rho$ algoritmus}
Podobne ako pri faktorizácii sa budeme snažiť nájsť nejakú funkciu, ktorá
by sa správala ako čo najlepšie náhodné zobrazenie. Pri takejto funkcii
potom môžeme očakávať nájdenie cyklu/kolízie v čase $O(\sqrt{n})$.

Zvoľme si $x_0 = g^{a_0} y^{b_0}$ kde $a_0,b_0 \inr Z_{n-1}$.
$x_0$ bude začiatočný prvok postupnosti. Na ``znáhodnenie'' našej funkcie
si grupu $G$ rozdelíme na 3 disjunktné množiny:
\def\union{\fixme{union}}
$G=S_1 \union S_2 \union  S_3$.

Potom môžeme uvažovať nasledujúcu funkciu $f$:
\begin{equation*}
    x_{i+1} = f(x_i) =
        \begin{cases}
         y x_i & x_i \in S_1 \\
         x_i^2 & x_i \in S_2 \\
         g x_i & x_i \in S_3
        \end{cases}
\end{equation*}
Je dôležité, že vďaka operáciam, ktoré sme zvolili, vieme vypočítať aj
"rozklad" $x_{i+1}$ nasledovne:

\begin{equation*}
    (a_{i+1},b_{i+1}) =
        \begin{cases}
         (a_i,b_i+1) & x_i \in S_1 \\
         (2a_i,2b_i) & x_i \in S_2 \\
         (a_i+1,b_i) & x_i \in S_3 \\
        \end{cases}
\end{equation*}

Našim cieľom je teda nejak rozdeliť $G$ na $S_1,S_2,S_3$ tak, aby
funkcia $f$ čo najviac pripomínala náhodné zobrazenie.
Pre prvočíselné grupy to môžeme napríklad rozdeliť modulo 3.
Pre všeobecné grupy môže byť dobrý spôsob zobrať hash vnútornej binárnej
reprezentácie prvku a tento hash zobrať modulo 3.

Keď už máme funkciu $f$ danú, môžeme použiť niektorý z algoritmov na
hľadanie cyklov spomínaných v kapitole o faktorizácii.

Ak nájdeme kolíziu, dostávame
\begin{align*}
    g^{a_i} y^{b_i} &= g^{a_j} y^{b_j} \\
    y^{b_i-b_j} &= g^{a_j-a_i} \\
    x &=(a_j - a_i)(b_i-b_j)^{-1} \pmod n
\end{align*}
a teda vieme $x$ vypočítať.

Časová zložitosť algoritmu (ak predpokladáme,
že $f$ má charakteristiku náhodného zobrazenia) je
$O(\sqrt{n})$, pamäťová zložitosť $O(1)$.

\subsection{Pohligov-Hellmanov algoritmus}
Tento algoritmus funguje vo všeobecnosti, ale veľmi efektívny je hlavne v
prípade, že $n$ má prvočíselný rozklad s malými prvočíslami.
$n=p_1^{m_1} \dots p_k^{n_k}$
\todo{dokoncit algoritmus}



\subsection{Index calculus}
Narozdiel od predchádzajúcich algoritmov, tento algoritmus nie je
všeobecný. Dá sa použiť napríklad $(Z_p^*,\cdot), GF(2^m)$.
V prípade eliptických kriviek sa nedá použiť. Tento jeho hendikep
ale kompenzuje jeho časová zložitosť, ktorá je subexponenciálna.

Uvažujme pre jednoduchosť grupu $G=(Z_p^*,\cdot)$.
Algoritmus pozostáva z 2 krokov - predvýpočtu a samotného výpočtu.
Predvýpočet sa môže spraviť raz pre jednu grupu.

Predvýpočet:

Majme zvolenú prvočíselnú bázu $B=\{p_1,\dots,p_b\}$.
Položme $a \inr Z_{p-1}$. 
Našou snahou bude rozložiť $g^a$ nad bázou $B$, teda nájsť hodnoty
$e_1,\dots,e_b$ také, že $g^a \mod p = p_1^{e_1} p_2^{e_2} \dots
p_b^{e_b}$.
Našou ambíciou je nazbierať čo najviac takýchto závislostí, optimálne
$b+1$. Ak totiž zlogaritmujeme dané rovnosti, dostaneme sústavu
lineárnych rovníc.
\fixme{makro na dlog}
$a=e_1 dlog_g p_1 + e_2 dlog_g p_2 + \dots + e_b dlog_g p_b \pmod{p-1}$.
V tejto sústave sú neznáme $dlog_g p_1, dlog_g p_2, \dots, dlog_g p_b$.

Ak sme úspešne zvládly tento predvýpočet, môžeme skúšať počítať dlog y
nasledovne.
Zvolíme si $b\inr Z_{p-1}$ a skúšame, či vieme hodnotu
$y g^a = p_1 ^ {e_1} \dots p_b ^ {e_b}$. Tým pádom
$x+a = e_1 d_1 + \dots + e_b d_b$.

Časová zložitosť (veľmi podobná Dixonovmu algoritmu) je
$O(e^{(1+O(1))\sqrt{\log p \log \log p}})$.

\begin{poznamka}
    Výpočet v prípade, že $G$ je $GF(p^m)$ je veľmi podobný, bázu budú ale
    tvoriť ireducibilné polynómy.
\end{poznamka}
\begin{poznamka}
    Nech G je podgrupa $Z_p^*$, ktorej veľkost $|G|=q$ je prvočíslo.
    Ak mame generator $g\in G$, potom výpočet dlog g v G buď pollard rho
    metodou.
    \todo{doplnit}
\end{poznamka}
