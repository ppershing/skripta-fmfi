\section{Authenticated encryption}

Doteraz sme sa pri správach zaberali buď autentickosťou (Rôzne podoby
MAC-u) alebo dôvernosťou a utajením (šifrovanie). V praxi však niekedy
potrebujeme dosiahnuť oboje naraz.
\begin{priklad}
    Ako naivné riešenie tohoto problému by sme mohli zobrať nasledujúcu
    konštrukciu: $AE = E_k(m, H(m))$. Táto konštrukcia bola použitá v
    praxi, a to dokonca viackrát. Prvým kandidátom je $E=\mbox{RC4}$ a
    $H=\mbox{CRC-32}$, v naších končinách je táto kombinácia známa aj pod
    názvom WEP (Wired Equivalent Privacy). Podobne, ak
    $E=\mbox{DES}$ a $H=\mbox{CRC-32}$, ide o SSH verzia 1.
    Obidve spomínané konštrukcie sú v súčasnosti rozbité.
\end{priklad}

Ešte predtým, ako začneme rozmýšľať nad úpravami, je dobré sa
zamyslieť nad tým, že MAC aj šifrovanie potrebujú kľúče. Použijeme
jeden kľúč? Dva kľúče? Aký bude medzi nimi vzťah a ako kvalitné musia
byť?

Taktiež, bezpečnosť celej konštrukcie bude závisieť od módu blokovej
šifry. Ako príklad si ukážeme (ne)bezpečnosť CBC-MAC.

\begin{priklad}
    Konštrukcia CBC-MAC funguje nasledovne. Na vstupe máme otvorený
    text $m_1,m_2,\dots,m_n$. Vstup najskôr zašifrujeme v CBC-móde
    (vstupom šifrovania je inicializačný vektor $IV$ a kľúč $k$),
    následne vypočítame autentifikačný token (tag)
    $\sigma=E_k(D_{k'}(c_n))$ kde $c_n$ je posledný blok výstupu
    šifry a $k'$ je autentifikačný kľúč.
    Celá správa je teda $\langle IV, c_1, c_2,\dots,c_n,\sigma
    \rangle$.

    Uvažujme teraz útočníka, ktorý si môže voliť správy a nechať si
    ich autentifikovane šifrovať. Cieľom útočníka je vytvoriť novú
    (podvrhnutú) správu a k nej korektný MAC.

    Uvažujme 2 podpísané správy
    \begin{align*}
        M &= \langle IV, c_1, c_2, \dots, c_n, \sigma \rangle \\
        M' &= \langle IV', c_1', c_2', \dots c_n', \sigma' \rangle
    \end{align*}
    Ak nastane situácia, že $\exists i,j: c_i = c_j'$, vieme zobrať
    polovicu šifrového textu z jednej správy a polovicu šifrového
    textu z druhej. V našom prípade, korektne autentifikované a
    šifrované správy budú
    \begin{align*}
        X_1 &= \langle IV, c_1, c_2, \dots, c_i, c_{j+1}', c_{j+2}' \dots
                c_n', \sigma' \rangle \\
        X_2 &= \langle IV', c_1', c_2', \dots c_j,
           c_{i+1}, c_{i+2}, \dots c_n, \sigma \rangle
    \end{align*}
    Aká je pravdepodobnosť vzniku takejto kolízie? Spomeňme si na
    narodeninový paradox. Ak označíme veľkosť kľúča pri šifrovaní ako 
    $k$ (poznamenajme, že celková dĺžka kľúčov potrebných pri tejto
    konštrukcii je $2k$),
    stačí ak si útočník podpíše správy o celkovom počte blokov
    $2^{k/2}$ a má vysokú šancu na úspech. Tento útok je síce skôr
    teoretický ako praktický, ale poukazuje na možné probémy.
\end{priklad}

Budeme pokračovať ďalším ilustratívnym príkladom, ktorý je historický
a poučný.
\begin{priklad}[Dual Counter Mode]
    Tento algoritmus bol navrhovaný v čase, keď už existovali aspoň 2
    jednoprechodové algoritmy (menovite IAPM a IACBC). Problémom však
    bolo, že tieto algoritmy boli patentované. Navyše, IAPM má problém
    s latenciou, čo pri šifrovaní sieťovej prevádzky (krátke IP
    pakety, požiadavky nízkej latencie) nie je vyhovujúce.
    Zaujímavosťou je, že tento algoritmus (ktorého autormi sú Boyle a
    Salter) navrhla v roku 2001 priamo NSA. No a ešte väčšou
    zaujímavosťou je, že bol do dňa zlomený.

    Pozrime sa naň teda bližšie. Klasicky predpokladajme, že správa
    pozostáva z blokov $m_1, \dots, m_n$. Ďalej, uvažujme že na
    začiatku session  sa dohodnú účastníci na hodnote $x_0$.
    Algoritmus potom vyzerá nasledovne:

    \begin{procedure}[H]
        \caption{DualCounterMode($m$)}
        $checksum \assign 0$\;
        \For{$i:=1$ \KwTo $n$}{
            $x_i \assign f(x_{i-1})$\;
            $c_i \assign E_k(m_i \xor x_i) \xor x_i$\;
            $checksum \assign checksum \xor m_i$\;
        }
        $x_{n+1} \assign f(x_n)$\;
        $\sigma \assign E_k(checksum \xor x_{n+1}) \xor x_0$\;
    \end{procedure}
    a jeho grafické znároznenie je na obrázku \ref{fig:dcm}.

    Funkcia $f$ je pre nás nezaujímavá, pre referenciu ale povieme, že
    v reálnej implementácii je to lineárnu posuvný register.
    Zaujímavá koštrukcia $E(m \xor x) \xor x$ sa volá whitening,
    údajne pomáha pri dôkazoch bezpečnosti.
    \begin{figure}
        \centering
        \includegraphics{img/18/dual-counter-mode.1.mps}
        \caption{Priebeh algoritmu DualCounterMode}
        \label{fig:dcm}
    \end{figure}
\end{priklad}

\todo{zvysok dual counter mode}
\todo{iapm}
\todo{gcm}
