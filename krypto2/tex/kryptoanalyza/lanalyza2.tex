\begin{itemize}
\item {\bf 1. kolo:}
Použijeme lineárnu aproximáciu
\begin{equation*}
( x_{1,0} \oplus x_{1,2} \oplus x_{1,3}  ) 
 \oplus ( key_{1,0} \oplus key_{1,2} \oplus key_{1,3} ) 
 \approx ( y_{1,1} )
\end{equation*}
ktorá má balancie po jednotlivých S-boxoch $
-0.25,\ 0.5,\ 0.5,\ 0.5
$ čo podľa piling-up lemy dáva pravdepodobnosť 
$1/2 + 2^3*( -0.25)*(0.5)*(0.5)*(0.5 )= 0.25 $. 
Výsledná balancia je $-0.25$.

\item {\bf 2. kolo:}
Použijeme lineárnu aproximáciu
\begin{equation*}
( x_{2,4}  ) 
 \oplus ( key_{2,4} ) 
 \approx ( y_{2,4} )
\end{equation*}
ktorá má balancie po jednotlivých S-boxoch $
0.5,\ 0.25,\ 0.5,\ 0.5
$ čo podľa piling-up lemy dáva pravdepodobnosť 
$1/2 + 2^3*( 0.5)*(0.25)*(0.5)*(0.5 )= 0.75 $. 
Výsledná balancia je $0.25$.

\item {\bf 3. kolo:}
Použijeme lineárnu aproximáciu
\begin{equation*}
( x_{3,1}  ) 
 \oplus ( key_{3,1} ) 
 \approx ( y_{3,0} \oplus y_{3,1} )
\end{equation*}
ktorá má balancie po jednotlivých S-boxoch $
-0.25,\ 0.5,\ 0.5,\ 0.5
$ čo podľa piling-up lemy dáva pravdepodobnosť 
$1/2 + 2^3*( -0.25)*(0.5)*(0.5)*(0.5 )= 0.25 $. 
Výsledná balancia je $-0.25$.

\item {\bf Spolu:}  Máme lineárnu kombináciu 
\begin{equation*}
\begin{split}
( in_{0} \oplus in_{2} \oplus in_{3} )
 & \\
 \phantom{x} \oplus ( key_{1,0} \oplus key_{1,2} \oplus key_{1,3} )
 & \\
 \phantom{x} \oplus ( key_{2,4} )
 & \\
 \phantom{x} \oplus ( key_{3,1} )
 & \approx  (
key_{4,0} \oplus key_{4,4}
 ) \\
 & \phantom{\approx} \oplus (
out_{0} \oplus out_{1}
) \end{split}
\end{equation*}
Podľa piling-up lemy máme balanciu $4* -0.25*0.25*-0.25 ~= 0.0625 $.
\end{itemize}
