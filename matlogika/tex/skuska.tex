\section{Písomná časť na konci semestra}

Na konci semestra existuje k predmetu písomka.
Táto písomka by sa mala skladať z 5 príkladov.
\begin{enumerate}
    \item Preniesť formulu do prenexného tvaru. Ako podotázka môže byť
        dôkaz niektorej z prenexných operácii.

    \item Napíšte axiómy nejakej teórie s rovnosťou, jej jazyk,
        špeciálne symboly a model (napríklad Teória telies alebo grúp)

    \item Dokážte Lindenbaumovy/Henkinovu/inú z ľahších viet počas
        semestra. Prípadne niečo na štýl 
        ``definujte konzervatívne rozšírenie + príklad''.
        Táto otázka je teoretická otázka.

    \item Daná je množina $S$. Nájdite unifikátor, Skolemov tvar.
        Proste, nejaká úloha z automatického dokazovania.

    \item Metódou rezolvent ukážte nesplniteľnosť zadanej množiny
        klauzúl $S$.
\end{enumerate}

\section{Samotná skúška}

\startFIXME
\begin{itemize}
	\item Dokázať $4$ vety
	\item je daná formula, nájsť prenexný tvar, skolemov tvar, metódou
	rezolvent ukázať, či platí alebo neplatí, napíšte teóriu s rovnosťou
	\item 3 alebo 4 definície
\end{itemize}
\stopFIXME
