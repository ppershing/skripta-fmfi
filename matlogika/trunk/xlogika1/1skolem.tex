\begin{definicia}[Skolemov normálny tvar]
  Uvažujme uzevretú formulu $A$.
  Ak $A$ má prefix $(\exists x_1)(\exists x_2) \dots (\exists x_k)
    (\forall x_{k+1}) (\forall x_{k+2}) \dots (\forall x_n)$, potom 
    hovoríme, že formula $A$ je vyjadrená v Skolemovom normálnom
    tvare, pričom $0 \le k \le n$.
\end{definicia}
\begin{veta}
    Nech $A$ je formula predikátovej logiky. Potom k nej môžeme
    zostrojiť formulu $A'$ v Skolemovom normálnom tvare, pričom platí
    $\provable A \Leftrightarrow \provable A'$.
\end{veta}
\begin{poznamka}
    Všimnime si, že predchádzajúca veta nehovorí nič o existencii
    formuly $A'$ takej, že $\provable A \leftrightarrow A'$ ako to
    bolo u prenexného tvaru. Taká formula totiž v prípade Skolemovho
    normálneho tvaru nemusí existovať.
\end{poznamka}
\todo{dopisat}

\section{Skolemov tvar}
\todo{}
