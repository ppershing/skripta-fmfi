\section{Skolemov tvar formuly}

\begin{definicia}[Skolemov normálny tvar]
  Uvažujme uzevretú formulu $A$.
  Ak $A$ má prefix $(\exists x_1)(\exists x_2) \dots (\exists x_k)
    (\forall x_{k+1}) (\forall x_{k+2}) \dots (\forall x_n)$, potom 
    hovoríme, že formula $A$ je vyjadrená v Skolemovom normálnom
    tvare, pričom $0 \le k \le n$.
\end{definicia}

\begin{veta}
    Nech $A$ je formula predikátovej logiky. Potom k nej môžeme
    zostrojiť formulu $A'$ v Skolemovom normálnom tvare, pričom platí
    $\provable A \Leftrightarrow \provable A'$.
    \label{veta:skolem}
\end{veta}

\begin{poznamka}
    Všimnime si, že predchádzajúca veta nehovorí nič o existencii
    formuly $A'$ takej, že $\provable A \leftrightarrow A'$ ako to
    bolo u prenexného tvaru. Taká formula totiž v prípade Skolemovho
    normálneho tvaru nemusí existovať.
\end{poznamka}

\begin{definicia}{Hodnosť formuly}
    Uvažujme formulu $A$ vyjadrenú v prenexnom tvare. Potom pod
    hodnosťou formuly $A$ označíme počet veľkých kvantifikátorov,
    ktoré v prefixe predchádzajú \emph{posledný} existenčný
    kvantifikátor (počítame zľava doprava).
\end{definicia}
\begin{priklad}
    \todo{ukazka hodnosti formuly na niecom}
\end{priklad}

\begin{dokaz}[vety \ref{veta:skolem}]
    Budeme predpokladať, že $A$ je uzavretá (Podľa vety o uzávere
    platí $\provable A$ práve vtedy, keď je dokázateľný jej uzáver.
    Dôkaz bude prebiehať matematickou indukciou vzhľadom na hodnosť
    $m$ formuly $A$.
    \begin{itemize}
        \item $m=0$ - $A$ je v Skolemovom normálnom tvare
        \item IP: tvrdenie vety platí pre každú
        formulu\footnote{používame tým pádom komplexnejší variant
        indukcie}, ktorej hodnosť je $\le m-1$.
        \begin{equation}
            A: (\exists x_1) (\exists x_2) ... (\exists x_n) (\forall y)
            B(x_1,x_2,\dots,x_n,y)
        \end{equation}
        kde $b$ \emph{nie je} nutne bezkvantifikátorová.
        Predpokladajme, že v $B$ sú voľné iba $x_1,x_2,\dots,x_n,y$.
        Keďže hodnosť $A$ je $m$, tak vo formule $B$ poslednému
        existenčnému kvantifikátoru predchádza práve $m-1$ všeobecných
        kvantifikátorov.
        Nech $P^{n+1}$ je $n+1$-árny predikát, ktorý sa nevyskytuje v
        $A$ (a teda ani $B$). Uvažujme nasledujúcu formulu
        \begin{equation}
            A^*: (\exists x_1) \dots (\exists x_n) \left[
                (\forall y) (B(x_1,\dots,x_n,y) \implies 
                    P^{(n+1}(x_1,\dots,x_n,y)) \implies 
                    (\forall y) P^{(n+1)} (x_1,\dots,x_n,y)
            \right]
        \end{equation}
        Postupne ukážeme, že $\provable A \Leftrightarrow \provable
        A^*$.
        \fixme{vo zvysku Toman robi nejaku drsnu magiu, treba to cele
        pocheckovat na validitu}
        \begin{itemize}
        \item[$\Rightarrow$]
            \begin{itemize}
            \item[0] $\provable A$
            \item[1] $\provable (B \implies P^{(n+1)}) \implies
                            (B \implies P^{(n+1)})$ -- inštancia
                            $X\implies X$.
            \item[2] $\provable (X\implies (Y\implies Z)) \implies
                            (Y\implies (X\implies Z))$ -- veta o zámene
                            predpokladov
            \item[3] $\provable
                (\underbrace{(B \implies P^{(n+1)})}_{X} \implies 
                        (\underbrace{B}_{Y} \implies
                        \underbrace{P^{(n+1)}}_{Z})) \implies
                ( \underbrace{B}_{Y} \implies
                 (\underbrace{(B \implies P^{(n+1)})}_{X} \implies
                        \underbrace{P^{(n+1)}}_{Z}))$ -- inštancia 2
            \item[4] $\provable
                 B \implies
                 ((B \implies P^{(n+1)}) \implies
                        P^{(n+1)}))$ - MP 1,3.
                        Pozn.: Toman: Ideme to obliecť, zatiaľ je
                        obnažená.
            \item[5] $\provable (\forall y) B \implies (\forall y)
                    (B \implies P^{(n+1)}) \implies
                    (\forall y) P^{(n+1)}$ -- veta o distribúcii
                    kvantifikátorov + pravidlo jednoduchého sylogizmu
                    \fixme{isto?}
            \item[6] $\provable (\exists x_1) (\exists x_2) \dots 
                    (\exists x_n) (\forall y) B \implies
                    (\exists x_1) (\exists x_2) \dots (\exists x_n)
                    (\forall y) (B \implies P^{(n+1)})$ --
                    $\provable (\forall x)(A \implies B) \implies
                     ((\exists x) A \implies (\exists x)B)$
                   \begin{itemize}
                    \item $\provable (\forall x)
                        (A \implies B) \implies
                        ((\forall x) A \implies (\forall x) B)$
                    \item $\provable (A \implies B) \leftrightarrow
                        (\neg B \implies \neg A)$
                    \item $\provable (\forall x) (\neg B \implies \neg
                    A) \implies ((\forall x) \neg B \implies
                      (\forall x) \neg A)$.
                    \item $\provable (\forall x) (\neg B \implies \neg
                    A) \implies (\neg (\forall x) \neg A \implies
                       \neg (\forall x) \neg B)$
                    \item $\provable (\forall x) (A \implies B)
                     \implies ( (\exists x) A \implies (\exists x) B)$
                   \end{itemize}
                \item[7] $\provable
                 (\exists x_1) (\exists x_2) \dots (\exists x_n)
                 (\forall y) (B \implies P^{(n+1)}) \implies 
                 (\forall y) P^{(n+1)}$.
            \end{itemize}
        \item[$\Leftarrow$]
            \todo{}
        \end{itemize}
    \end{itemize}
\end{dokaz}
