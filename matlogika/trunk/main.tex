
%% Template by Michal Forisek


\documentclass[a4paper]{report}
\usepackage{ulem}
\usepackage{slovak}
\usepackage[utf8]{inputenc}
\usepackage{a4wide}
\usepackage{tabularx}
\usepackage{amsfonts}
\usepackage{amssymb}
\usepackage{amsmath}
\usepackage{epsfig}
\usepackage[usenames,dvipsnames]{color}
\usepackage{mathrsfs}
\usepackage{verbatim}
\usepackage{hyperref}
\usepackage{ifthen}
\usepackage{subfigure}


\def\todo#1{[{\color{red} TODO:} {\bf  #1}]}
\def\fixme#1{[{\color{red} FIXME:} {\bf  #1}]}
\def\verify#1{\todo{verify: #1}}

\renewcommand{\implies}{\rightarrow}
\newcommand{\notmodels}{\nvDash}
\newcommand{\union}{\cup}
\newcommand{\provable}{\vdash}
\newcommand{\unprovable}{\nvdash}

\def\noheader{\relax}

\newtheorem{definicia}{Definícia}[section]
\newtheorem{HLPpriklad}{Príklad}[section]
\newenvironment{priklad}[1][]{
    \ifthenelse{\equal{#1}{}}{
        \begin{HLPpriklad}
    }{
        \begin{HLPpriklad}[#1]
    }
    \rm}{\end{HLPpriklad}
}
\newtheorem{veta}{Veta}[section]
\newtheorem{lema}{Lema}[section]


\newenvironment{dokaz}{\trivlist
  \item[\hskip \labelsep{\bfseries Dôkaz:}]}{\endtrivlist}


\begin{document}

\thispagestyle{empty}
\begin{minipage}{0.25\textwidth}
\includegraphics[width=0.9\textwidth]{img/komlogo-new}
\end{minipage}
\begin{minipage}{0.69\textwidth}
\begin{center}
\sc Katedra Informatiky \\
Fakulta Matematiky, Fyziky a Informatiky \\
Univerzita Komenského, Bratislava
\end{center}
\end{minipage}

\vfill
\begin{center}
\begin{minipage}{0.8\textwidth}
\hrule
\bigskip\bigskip
\centerline{\LARGE\sc Matematická Logika}
\smallskip
\centerline{(spísané poznámky, draft)}
\bigskip
\centerline{\url{http://code.google.com/p/matlogika}}
\bigskip
\centerline{\large\sc Peter Perešíni, Milan Plžík, Pavol Struhár, Ivan Kováč}
\bigskip\bigskip
\hrule
\end{minipage}
\end{center}
\vfill
{~}
\hfill verzia zo dňa {\bf\today} 
\eject % EOP i

\section*{Úvod}

Tieto poznámky obsahujú študijné materiály
k predmetu \emph{Matematická logika}
na Fakulte matematiky, fyziky a informatiky UK.

Základ poznámok bol spísaný podľa prednášky doc. Eduarda Tomana v roku 2009.
Poznámky ale nie sú oficiálny študijný materiál, preto autori neručia
žiadnym spôsobom za ich aktuálnosť či vhodnosť. Navyše, obsah prednášky sa
počas rôznych rokov môže meniť a preto je silne odporúčané dopísať si
prípadné rozdiely medzi poznámkami a prednáškou.

Aby sme umožnoli jednoduchšie spravovanie a udržali poznámky dlhšie
aktuálne, rozhodli sme sa verejne publikovať zdrojové kódy na stránke
\url{http://code.google.com/p/matlogika}. Ak máte akékoľvek pripomienky,
návrhy, opravy, môžete nám ich prostredníctvom tejto stránky oznámiť.

Za autorov, PPershing.

\tableofcontents

\chapter{Dokončenie úvodu do matematickej logiky}
\section{Prerekvizity a označenia}

V tejto časti si zhrnieme najdôležitejšie označenia, pojmy a vety z
prednášky ``Úvod do matematickej logiky''.

\noindent Označenia:
\begin{itemize}
    \item $\provable A$ -- formula $A$ je dokázateľná
    \item $\mathcal{M} \models A$ -- $\mathcal{M}$ je modelom $A$
    \item $\implies, \lequiv$ -- implikácia, ekvivalencia vo
    výrokovej/predikátovej logike
    \item $\Rightarrow, \Leftrightarrow$ -- implikácia a ekvivalencia v
    našom jazyku
\end{itemize}

\noindent Axiómy:
\begin{itemize}
    \item[A1:] $A \implies (B \implies A)$
    \item[A2:] $(A \implies (B \implies C)) \implies 
                [(A \implies B) \implies (A \implies C)]$
    \item[A3:] $(\neg B \implies \neg A) \implies (A \implies B)$
    \smallskip
    \item[A4:] $(\forall x) A \implies A_x[t]$
    \item[A5:] $(\forall x) (A \implies B) \implies (A \implies
    (\forall x) B)\quad$ (ak $x$ nie je voľná v $A$)
\end{itemize}

\noindent Pravidlá:
\begin{itemize}
    \item Modus ponens: $\displaystyle \frac{A,A\implies B}{B}$
    \item Pravidlo zovšeobecnenia:
            $\displaystyle \frac{A,x}{(\forall x)A}$
    \item Jednoduchý sylogizmus:
            $\displaystyle \frac{A, A\implies B, B\implies C}{C}$, resp.
            $\provable (A \implies B) \implies ((B \implies C) 
                \implies (A \implies C))$.
\end{itemize}

\begin{veta}[O dedukcii]
    \begin{equation*}
        T \provable A \implies B \iff T,A \provable B
    \end{equation*}
    Pozn.: V predikátovej logike ale musí na implikáciu $\Leftarrow$ 
    navyše platiť, že na žiadnu voľnú premennú z formly $A$
    nepoužijeme v dôkaze $T,A \provable B$ pravidlo zovšeobecnenia.
    Špeciálne teda ekvivalencia platí v prípade, že $A$ je uzavretá.
\end{veta}

\begin{veta}[Postove vety a iné užitočné tvrdenia]
    \noindent
    \begin{itemize}
        \item $\provable A \implies A$
        \item $\provable A \implies \neg \neg A$
        \item $\provable \neg \neg A \implies A$
        \item $\provable (A \implies B) \implies (\neg B \implies \neg A)$
    \end{itemize}
\end{veta}

\begin{lema}[O neutrálnej formule]
   Nech $T,A \provable B$ a tiež $T, \neg A \provable B$. Potom
   $T \provable B$.
\end{lema}

\begin{lema}[O distribúcii kvantifikátorov]
    Ak je $\provable A \implies B$, potom
    $\provable (\forall x)A \implies (\forall x) B$ a
    $\provable (\exists x)A \implies (\exists x)B$.
\end{lema}

\begin{veta}[O ekvivalencii]
    Nech formula $A'$ vznikne z formuly A nahradením všetkých výskytov 
    podformúl $B_1,\dots ,B_n$ v uvedenom poradí formulami 
    $B'_1, \dots, B'_n$. Ak platí $\provable B_i \lequiv B'_i$ pre
    $i \in \{1,\dots,n\}$, potom $\provable A \lequiv A'$.
\end{veta}

\todo{vety o variantoch, zavedeni kvantifikatorov, ...}
\begin{lema}[Duálny tvar axiómy špecifikácie]
    $\provable A_x[t] \implies (\exists x) A$
\end{lema}

\begin{lema}[Pravidlo zavedenia existenčného kvantifikátora]
    Ak $\provable A \implies B$ a $x$ nie je voľná v $B$, potom
    $\provable (\exists x) A \implies B$.
\end{lema}

% vim: set fdm=marker: sw=2: sts=2: ts=2:

\section{Prenexné tvary formúl}

Ako sme mali vo výrokovej logike isté normálne tvary - konjunktívnu a
disjunktívnu normálnu formu, budeme mať aj v predikátovej logike isté
špeciálne tvary. Zaujímavé sú najmä prexenxná forma a ešte Skolemov
normálny tvar, čo je špeciálny prípad prexenej formy.
V prípade prenexného tvaru ide o preskupenie kvantifikátorov na
začiatok formuly - tvar vzniká aplikovaním kvantifikátorov na otvorenú
formulu.


\begin{definicia}
    Formula $A$ je v prenexnom tvare, ak $A$ je v nasledujúcom tvare:
    \begin{equation*}
     (Q_1 x_1) (Q_2 x_2) (Q_3 x_3) \dots (Q_n x_n) B
    \end{equation*}
    kde $x_1, \dots , x_n$ sú navzájom rôzne premenné,
    $Q_i \in\{\forall, \exists\}$ sú
    kvantifikátory a  formula $B$ je bez kvantifikátorov.
    Formulu $B$ nazveme otvoreným jadrom formuly $A$, 
    formulu $(Q_1 x_1) (Q_2 x_2) (Q_3 x_3) \dots (Q_n x_n)$ nazveme
    prefixom formuly $A$.
\end{definicia}

\begin{poznamka}
    \noindent
    \begin{itemize}
     \item $x_1, \dots, x_n$ sú navzájom rôzne pre vylúčenie viacnásobných
        kvantifikácií.
     \item ak $n=0$, tak $A$ je otvorená a nemá prefix
     \item $B$ je najväčšia otvorená podformula formuly $A$.
    \end{itemize}
\end{poznamka}

\begin{priklad}
    Formula elementárnej aritmetiky:
    \begin{equation*}
        (\forall x) (\forall y) (\exists z) (x+y=z)
    \end{equation*}
\end{priklad}

\begin{veta}
 Nech $A$ je ľubovoľná formula predikátovej logiky. Potom existuje
 formula $A'$ v prenexnom tvare taká, že
 $\provable A \lequiv A'$.
 \label{veta:prenex}
\end{veta}

Pri prevádzaní formuly na prenexný tvar budeme využívať nasledujúce
\emph{prenexné} operácie, každá z nich nahrádza podformulu
jej ekvivalentom.
\begin{itemize}
    \item[a)] $B$ nahraď variantom formuly $B$ (premenovanie viazaných
    premenných)
    \item[b)] $\neg(Q x) B$ nahraď $(\overline{Q} x) \neg B$
    \item[c)] ak $x$ nie je voľná v $B$, tak podformulu $B\implies (Qx)C$
            nahraď podformulou $(Qx) (B\implies C)$
    \item[d)] ak $x$ nie je voľná v $C$, tak $((Qx) B) \implies C$
        nahraď $(\overline{Q} x) (B \implies C)$
    \item[e)] ak $x$ nie je voľná v $B$, $\square \in \{\land,\lor\}$.
     Potom $B \squareop ((Qx)C)$ resp. $((Qx)B)\squareop C$ nahraď
     $(Qx)(B \squareop C)$
\end{itemize}
\begin{poznamka}
    Asi stojí za zmienku upozorniť, že v časti c) $x$ nie je voľná vo
    formule $C$ narozdiel od častí b), d), kde $x$ nie je voľná v $B$.
\end{poznamka}

\begin{lema}
    Prenexnými operáciami dostaneme ekvivalentné formuly
\end{lema}
\begin{dokaz}
  %%% {{{ 
  \noindent
  \begin{itemize}
    \item[a)] Veta o variantoch
    \item[b)] Platí
        \begin{itemize}
        \item[1]
            $\provable \neg(\forall x) B \lequiv
                \highlighto{
                \neg (\forall x) \highlightb{\neg\neg} B}$ -- 
                pretože platí $B \lequiv \neg \neg B$

        \item[2]
            $\provable \highlighto{
                \highlighta{\neg (\forall x) \neg} \neg B}
             \lequiv \highlightb{(\exists x)} \neg B$ --
                pretože $(\exists x) A$ je z definície
                $\neg (\forall x) \neg A$.
        \end{itemize}
        Následným použítím vety o ekvivalencii môžeme prvú
        ekvivalenciu dodadiť do druhej a dostávame požadovaný
        výsledok.
        Podobne
        \begin{itemize}
        \item[1]
            $\provable \neg(\exists x) B \lequiv
                \neg (\exists x) \highlightb{\neg\neg} B$

        \item[2]
            $\provable \highlighto{
                \highlighta{\neg (\exists x) \neg} \neg B}
                \lequiv \highlightb{(\forall x)} \neg B$
        \end{itemize}

    \item[c)] Nech $Q=\forall$. Chceme ukázať
        $\provable (\forall x) (B \implies C) \lequiv
            (B \implies (\forall x) C)$ kde $x$ nie je voľná v $B$.
        \begin{itemize}
        \item[$\Rightarrow$]
            5. Axióma predikátovej logiky


        \item[$\Leftarrow$]
            \begin{itemize}
            \item[1] $\provable (\forall x) C \implies C$ --
                axióma špecifikácie

            \item[2] $\provable \highlighta{
                 \underbrace{(B \implies (\forall x) C)}_X
                \implies 
                 (\underbrace{((\forall x) C \implies C)}_Y
                    \implies \underbrace{(B \implies C)}_Z)}$ --
                    Jednoduchý sylogizmus

            \item[*] $\provable X \implies (Y \implies Z)$.

            \item[*] $\provable (X \implies (Y \implies Z) \implies
                (Y \implies (X \implies Z))$ -- pravidlo zámeny
                predpokladov

            \item[3] $\provable \highlighta{
                [(B \implies (\forall x) C)
                \implies 
                 (((\forall x) C \implies C)
                    \implies (B \implies C))]
                }
                \implies \highlightb{
                   [((\forall x) C \implies C) \implies
                    ((B \implies (\forall x) C) \implies (B \implies
                    C))]}$.

            \item[4] $\provable
                  \highlightb{
                   ((\forall x) C \implies C) \implies
                    \highlighto{
                    [(B \implies (\forall x) C) \implies (B \implies
                    C)]}}$ - MP 2,3

            \item[5] $\provable \highlighto{
                (B \implies (\forall x)) \implies (B
                    \implies C)}$ - MP 1,4

            \item[6] $\provable (B \implies (\forall x) C) \implies
            \highlighta{(\forall x)} (B \implies C)$ -- 
                pravidlo zavedenia veľkého kvantifikátora
            \end{itemize}
        \end{itemize}
     Druhou možnosťou je $Q=\exists$. Našim cieľom je ukázať
     $\provable (\exists x) (B \implies C) \lequiv 
        (B \implies (\exists x) C)$ za predpokladu že $x$ nie je voľná v $B$.
        \begin{itemize}
        \item[$\Rightarrow$]
            \begin{itemize}
            \item[1] $\provable C \implies (\exists x) C$ --
                duálna verzia axiómy špecifikácie

            \item[2] $\provable (B \implies C) \implies
                 ((C \implies (\exists c) C) \implies (B \implies
                 (\exists x) C ))$ -- jednoduchý sylogizmus (JS)

            \item[3] $\provable
                \highlighta{
                [(B \implies C) \implies (( C \implies (\exists x) C)
                \implies (B \implies (\exists x) C))]}
                \implies
                \highlightb{
                  [(C \implies (\exists x) C) \implies (( B \implies
                  C) \implies (B \implies (\exists x) C))]}$ -- 
                  pravidlo zámeny predpokladov

            \item[4] $\provable \highlightb{
                  ((C \implies (\exists x) C) \implies 
                  \highlighto{[( B \implies
                  C) \implies (B \implies (\exists x) C))]}}$ -- MP 2,3

            \item[5] $\provable
                  \highlighto{(( B \implies
                  C) \implies (B \implies (\exists x) C)))}$ -- MP 1,4

            \item[6] $\provable \highlighta{(\exists x)}
                ( B \implies C) \implies (B \implies (\exists x) C)$
                -- pravidlo zavedenie existenčného kvantifikátora
            \end{itemize}


        \vskip 5mm
        \item[$\Leftarrow$]
            \begin{itemize}
                \item[1] $\provable C \implies (B \implies C)$ -- A1

                \item[2] $\provable (\exists x) C \implies 
                    (\exists x)(B \implies C)$ -- pravidlo
                    distribúcie kvantifikátorov

                \item[3] $\provable \neg B \implies (B \implies C)$ -- postova
                teoréma

                \item[4] $\provable (B \implies C) \implies 
                    (\exists x)(B \implies C)$ -- 
                    duálny tvar axiómy špecifikácie

                \item[5] $\provable \highlighta{\neg B \implies 
                    (\exists x) (B \implies C)}$ -- JS 3,4

                \item[*] $\provable
                    \highlighta{
                    [\neg \underbrace{B}_X \implies 
                        \underbrace{(\exists x) ( B \implies C)}_Z]
                    }
                    \implies
                    \highlightb{
                    [(\underbrace{(\exists x) C}_Y 
                        \implies 
                      \underbrace{(\exists x) (B \implies C)}_Z)
                    \implies
                    ((\underbrace{B}_X \implies 
                        \underbrace{(\exists x) C}_Y)
                      \implies 
                        \underbrace{(\exists x) (B
                    \implies C)}_Z)]}$ -- dokážeme neskôr

                \item[6] $\provable
                    \highlightb{
                    [(\exists x) C \implies (\exists x) (B \implies C)]
                    \implies
                    \highlighto{
                    [(B \implies (\exists x) C) \implies (\exists x) (B
                    \implies C ))]}}$ -- MP 5,*

                \item[7] $\provable
                    \highlighto{
                    (B \implies (\exists x) C) \implies (\exists x) (B
                    \implies C ))}$ -- MP 2,6
            \end{itemize}
            Ešte treba dokázať formulu (*)
            \begin{itemize}
            \item[a] $\neg X \implies Z, Y \implies Z, X \implies Y,
                \highlighta{X}
                \provable Z$
            \item[b] $\neg X \implies Z, Y \implies Z, X \implies Y,
                \highlightb{\neg X}
                \provable Z$
            \item[c] $\neg X \implies Z, Y \implies Z, X \implies Y
                \provable Z$ -- veta o neutrálnej formule ($X,\neg
                X$).
            \item[d] $\provable (\neg X \implies Z) \implies (
                    (Y \implies Z) \implies ((X \implies Y) \implies
                    Z))$ -- veta o dedukcii
            \end{itemize}
        \end{itemize}
    \item[d)]
        \begin{itemize}
        \item $Q=\forall$: Ukazujeme
            $\provable (\exists x) (B \implies C) \lequiv
             ((\forall x) B \implies C)$ ak $x$ nie je voľná v $C$.
            \begin{itemize}
            \item[1]
                $\provable \highlightc{((\forall x) B \implies C)}
                    \lequiv
                    (\neg C \implies \neg (\forall x) B)$
            \item[2]
                $\provable \highlightc{((\forall x) B \implies C)}
                    \lequiv
                    (\neg C \implies \neg (\forall x) \highlighta{\neg
                    \neg} B)$
            \item[3]
                $\provable \highlightc{((\forall x) B \implies C)}
                    \lequiv
                    \highlightp{
                    (\neg C \implies \highlightb{(\exists x)} \neg B)}$
            \item[4]
                $\provable 
                    \highlighto{(\exists x) (\neg C \implies \neg B)}
                \lequiv
                    \highlightp{
                    (\neg C \implies (\exists x) \neg B)}$
                 -- časť c) tohoto dôkazu

            \item[5]
                $\provable \highlightc{((\forall x) B \implies C)}
                    \lequiv
                    \highlighto{(\exists x) 
                        (\highlightb{\neg C \implies \neg B)}}$ --
                    vetou ekvivalentných zámenách sme dosadili 4 do 3
            \item[6] 
                $\provable (\highlighta{B\implies C}) 
                    \lequiv 
                    (\highlightb{\neg C \implies \neg B})$ -- 
                    vieme z výrokovej logiky
            \item[7]
                $\provable \highlightc{((\forall x) B \implies C)}
                    \lequiv
                    (\exists x) (\highlighta{B \implies C})$ --
                    použili sme vetu o ekvivalentných zámenách na
                    5,6.
            \end{itemize}

        \item $Q=\exists$: Chceme ukázať
            $\provable (\exists x) (B \implies C) \lequiv
             ((\forall x) B \implies C)$ ak $x$ nie je voľná v $B$.
             Postupujeme analogicky ako v predchádzajúcom prípade
            \begin{itemize}
            \item[1]
                $\provable \highlightc{((\exists x) B \implies C)}
                    \lequiv
                    (\neg C \implies \neg (\exists x) B)$
            \item[2]
                $\provable \highlightc{((\exists x) B \implies C)}
                    \lequiv
                    (\neg C \implies \neg (\exists x) \highlighta{\neg
                    \neg} B)$
            \item[3]
                $\provable \highlightc{((\exists x) B \implies C)}
                    \lequiv
                    \highlightp{
                    (\neg C \implies \highlightb{(\forall x)} \neg B)}$
            \item[4]
                $\provable 
                    \highlighto{(\forall x) (\neg C \implies \neg B)}
                \lequiv
                    \highlightp{
                    (\neg C \implies (\forall x) \neg B)}$
                 -- časť c) tohoto dôkazu

            \item[5]
                $\provable \highlightc{((\exists x) B \implies C)}
                    \lequiv
                    \highlighto{(\forall x) 
                        (\highlightb{\neg C \implies \neg B)}}$ --
                    vetou ekvivalentných zámenách sme dosadili 4 do 3
            \item[6] 
                $\provable (\highlighta{B\implies C}) 
                    \lequiv 
                    (\highlightb{\neg C \implies \neg B})$ -- 
                    vieme z výrokovej logiky
            \item[7]
                $\provable \highlightc{((\exists x) B \implies C)}
                    \lequiv
                    (\forall x) (\highlighta{B \implies C})$ --
                    použili sme vetu o ekvivalentných zámenách na
                    5,6.
            \end{itemize}
        \end{itemize}
        \item[e)] Ukazujeme
            $\provable (Qx) (B\squareop C) \lequiv (B\squareop (Qx)C)$,
            kde $x$ nie je voľná v $B$.
            Na základe operácii c), d) toto vieme dokázať, pretože platí
            \begin{align*}
                &\provable (A \lor B) \lequiv (\neg A \implies B) \\
                &\provable (A \land B) \lequiv \neg(A \implies \neg B)
            \end{align*}
  \end{itemize}
  %%% }}}
\end{dokaz}

\begin{dokaz}[Dôkaz vety \ref{veta:prenex} o prexexných tvaroch]
%%% {{{
Budeme postupovať matematickou indukciou vzhľadom na zložitosť formuly $A$.
\begin{itemize}
    \item $A$ je atomická formula. $A$ je potom v prenexnom tvare.

    \item $A=\neg B$. Na $B$ sa vzťahuje IP, teda vieme zostrojiť
        $B'$ takú, že platí
        $\provable B \lequiv B'$.
        Položíme $A'=\neg B'$ a niekoľkonásobným aplikovaním 
        prenexnej operácie b) dostaneme 
        $A''$ v správnom tvare.

    \item $A=B \implies C$. Na $B,C$ platí IP a teda existujú formuly
        $B',C'$ v prenexnom tvare, pre ktoré platí
        $\provable B \lequiv B'$, $\provable C \lequiv C'$.
        Nech $A' = B' \implies C'$. Na základe vety o ekvivalencii platí
        $\provable A \lequiv A'$. Teraz potrebujeme dostať
        $A'$ do prenexného tvaru.
        Vezmime variant $C''$ formuly $C'$ taký, že $B',C''$ nemajú
        žiadnu spoločnú premennú.
        \begin{equation*}
            \provable A \lequiv (B' \implies C'')
        \end{equation*}
        Teraz použijeme prenexné operácie c), d) a formulu
        $B' \implies C''$ prevedieme do prenexného tvaru.

    \item $A=(\forall x)B$. Z indukčného predpokladu vyplýva
        existencia $B'$, $\provable B \lequiv B'$.
        Môžu nastať 2 prípady
        \begin{itemize}
        \item $x$ nie je viazaná v $B'$. Položme $A' = (\forall x) B'$
        \item $x$ je viazaná v $B'$. Potom máme $A' = B'$.
        \end{itemize}
\end{itemize}
%%% }}}
\end{dokaz}
\begin{poznamka}
    Ak $A$ obsahuje spojky 
    $\land,\lor$, môžeme použiť prenexnú operáciu e) alebo formulu nahradiť
    ekvivaletnou formulou obsahujúcou $\neg,\implies$.
    Ak sa vo vormule vyskytuje $\lequiv$, nemôžeme priamo
    použiť operácie e), d) ale $A\lequiv B$ prepíšeme na
    $(A\implies B) \land (B \implies A)$.
\end{poznamka}

\begin{priklad}
    \noindent
    Formula $A: B \lequiv (\forall x) C$ kde $x$ nie je voľná
    v $B$ a $y$ sa nevyskytuje v $B,C$.
    \begin{align*}
       (B \implies (\forall x) C) \land ((\forall x) C \implies B) &\\
       (B \implies (\forall x) C) \land ((\forall y) C_x[y] \implies B) 
       & \mbox{ -- podľa a)} \\
       (\forall x)(B \implies C) \land (\exists y) (C_x[y] \implies B) 
       & \mbox{ -- podľa c), d)} \\
       (\forall x)(\exists y)((B \implies C) \land (C_x[y] \implies B)
    \end{align*}
\end{priklad}

\begin{priklad}
    Formula elementárnej aritmetiky:
    \begin{align*}
        (\exists x) (x=y) \implies (\exists x)((x=0) \lor
                    \neg (\exists y)(y<0))& \\
        (\exists x) (x=y) \implies (\exists u)((u=0) \lor
                    \neg (\exists v)(v<0)) &\mbox{ -- podľa a)}\\
        (\exists x) (x=y) \implies (\exists u)((u=0) \lor
                    (\forall  v) \neg(v<0)) &\mbox{ -- podľa b)}\\
        (\exists x) (x=y) \implies (\exists u)(\forall v)
                ((u=0) \lor \neg(v<0)) &\mbox{ -- podľa e)}\\
        (\forall x)(\exists u)(\forall v) (x=y) \implies 
                ((u=0) \lor \neg(v<0)) &\mbox{ -- podľa c), d)}
    \end{align*}
\end{priklad}


\section{Skolemov tvar formuly}

\begin{definicia}[Skolemov normálny tvar]$\!\!\!$\footnote{Pozn.:
    Táto definícia sa mierne líši od štandardnej, v tej sa nemôžu
    vyskytovať žiadne existenčné kvantifikátory
    }
  Uvažujme uzevretú formulu $A$.
  Ak $A$ má prefix $(\exists x_1)(\exists x_2) \dots (\exists x_k)
    (\forall x_{k+1}) (\forall x_{k+2}) \dots (\forall x_n)$, potom 
    hovoríme, že formula $A$ je vyjadrená v Skolemovom normálnom
    tvare, pričom $0 \le k \le n$.
\end{definicia}

\begin{veta}
    Nech $A$ je formula predikátovej logiky. Potom k nej môžeme
    zostrojiť formulu $A'$ v Skolemovom normálnom tvare, pričom platí
    $\provable A \iff \provable A'$.
    \label{veta:skolem}
\end{veta}

\begin{poznamka}
    Všimnime si, že predchádzajúca veta nehovorí nič o existencii
    formuly $A'$ takej, že $\provable A \lequiv A'$ ako to
    bolo u prenexného tvaru. Taká formula totiž v prípade Skolemovho
    normálneho tvaru nemusí existovať.
\end{poznamka}

\begin{definicia}[Hodnosť formuly]
    Uvažujme formulu $A$ vyjadrenú v prenexnom tvare. Potom pod
    hodnosťou formuly $A$ označíme počet veľkých kvantifikátorov,
    ktoré v prefixe predchádzajú \emph{posledný} existenčný
    kvantifikátor (počítame zľava doprava).
\end{definicia}

\begin{priklad}
    Uvažujme formulu
    \begin{equation*}
        \highlighta{(\forall x) (\forall y)}
        (\exists z) 
        \highlighta{(\forall u)}
        (\exists w)
        \highlightc{(\exists v)}
        (\forall s) (\forall t) A
    \end{equation*}
    Jej hodnosť je 3.
\end{priklad}

\def\existsxton{(\exists x_1) (\exists x_2) \dots (\exists x_n)}
\def\forallxton{(\forall x_1) (\forall x_2) \dots (\forall x_n)}

\begin{dokaz}[Vety \ref{veta:skolem} o Skolemovom tvare]
%%% {{{
    Budeme predpokladať, že $A$ je uzavretá 
    (Podľa vety o uzávere
    platí, že je dokázateľné $\provable A$ práve vtedy,
    keď je dokázateľný uzáver formuly $A$)
    a taktiež že je v prenexnom tvare (to vieme zabezpečiť).
    Dôkaz bude prebiehať matematickou indukciou vzhľadom na hodnosť
    $m$ formuly $A$.
    \begin{itemize}
        \item $m=0$ - $A$ je v Skolemovom normálnom tvare
        \item IP: tvrdenie vety platí pre každú
        formulu ktorej hodnosť je $m-1$.
        Nech je teda $A$ tvaru
        \begin{equation}
            A: \existsxton (\forall y) B(x_1,x_2,\dots,x_n,y)
        \end{equation}
        kde $B$ \emph{nie je} nutne bezkvantifikátorová (a je v
        prenexnom tvare).
        Pretože $A$ je uzavretá, v $B$ sú voľné iba $x_1,x_2,\dots,x_n,y$.
        Keďže hodnosť $A$ je $m$, tak vo formule $B$ poslednému
        existenčnému kvantifikátoru predchádza práve $m-1$ všeobecných
        kvantifikátorov.
        Nech $P^{n+1}$ je $(n+1)$-árny predikát, ktorý sa nevyskytuje v
        $A$ (a teda ani $B$). Uvažujme nasledujúcu formulu
        \begin{equation}
            A^*: \existsxton \left[
                (\forall y) [B(x_1,\dots,x_n,y) \implies 
                    P^{(n+1)}(x_1,\dots,x_n,y)] \implies 
                    (\forall y) P^{(n+1)} (x_1,\dots,x_n,y)
            \right]
        \end{equation}
        Postupne ukážeme, že $\provable A \iff \; \provable
        A^*$.
        %%% {{{ \provable A <==> \provable A*
        \begin{itemize}
        \item[$\Rightarrow$]
        %%% {{{ ==>
            \begin{itemize}
            \item[0] $\provable A$

            \item[1] $\provable \highlightb{(B \implies P^{(n+1)}) \implies
                            (B \implies P^{(n+1)})}$ -- inštancia
                            $X\implies X$.

            \item[2] $\provable (X\implies (Y\implies Z)) \implies
                            (Y\implies (X\implies Z))$ -- veta o zámene
                            predpokladov

            \item[3] $\provable 
                \highlightb{[\underbrace{(B \implies P^{(n+1)})}_{X} \implies 
                        (\underbrace{B}_{Y} \implies
                        \underbrace{P^{(n+1)}}_{Z}))]} \implies
                \highlighta{[ \underbrace{B}_{Y} \implies
                 (\underbrace{(B \implies P^{(n+1)})}_{X} \implies
                        \underbrace{P^{(n+1)}}_{Z})]}$ -- inštancia
                        kroku 2

            \item[4] $\provable
                 \highlighta{B \implies
                 ((B \implies P^{(n+1)}) \implies
                        P^{(n+1)})}$ -- MP 1,3.\footnote{
                        Toman: Ideme to obliecť, zatiaľ je
                        obnažená.}

            \item[x] $\provable \highlightc{\highlighto{(\forall y)}
                B} \implies \highlightp{\highlighto{(\forall y)}(
                (B \implies P^{(n+1)}) \implies P^{(n+1)})}$ --
                veta o distribúcii kvantifikátorov aplikovaná na
                5

            \item[y] $\highlightp{(\forall y)(
                (B \implies P^{(n+1)}) \implies P^{(n+1)})}
                \implies
                \highlightb{(\highlighto{(\forall y)}(B \implies P^{(n+1)}) 
                    \implies
                     \highlighto{(\forall y)}P^{(n+1)})}$ --
                Ukážeme nasledovne:
                %%% {{{
                 \begin{itemize}
                    \item[a] $ \highlightdim{(\forall x) (X \implies
                        Y)} \provable 
                            (\forall x) (X \implies Y)$ -- predpoklad
                            je vždy dokázateľný
                    \item[b] $ \provable \highlightdim{
                        (\forall x) (X \implies Y)}
                                \implies (X \implies Y)$ -- axióma
                                špecifikácie
                    \item[c] $ \highlightdim{(\forall x) (X \implies
                        Y)} \provable
                            X \implies Y$ -- MP a,b
                    \item[d] $ \highlightdim{(\forall x) (X \implies
                        Y)} \provable
                            (\forall x) X \implies (\forall x) Y$ --
                            pravidlo zavedenia všeobecného
                            kvantifikátora aplikované na c
                    \item[e] $ \provable \highlightdim{(\forall x) (X
                                \implies Y)}
                        \implies
                            ((\forall x) X \implies (\forall x) Y)$ --
                            veta o dedukcii v predikátovej logike.%
                            \footnote{Pozor, treba si riadne
                                premyslieť, že ju môžeme použiť. Totiž, v
                                dôkaze d sme nikne nepoužili pravidlo
                                zovšeobecnenia na premennú, ktorá by bola
                                voľná v $(\forall x) (X \implies Y)$ --
                                použili sme to iba na premennú $x$ a tá je
                                viazaná.
                            }
                 \end{itemize}
                 %%% }}}

            \item[5] $\provable \highlightc{(\forall y) B} \implies 
                \highlightb{((\forall y)
                    (B \implies P^{(n+1)}) \implies
                    (\forall y) P^{(n+1)})}$ -- pravidlo jednoduchého sylogizmu
                    aplikované na x,y

            \item[z] $\provable 
                \forallxton \left[
                \highlightc{(\forall y) B} \implies 
                \highlightb{((\forall y)
                    (B \implies P^{(n+1)}) \implies
                    (\forall y) P^{(n+1)})}
                \right] $ -- $n$-krát aplikované pravidlo zovšeobecnenia
                 na krok 5

            \item[6] $\provable \highlighta{\existsxton} 
                    \highlightc{(\forall y) B} \implies
                    \highlighta{\existsxton} \left[
                        \highlightb{
                        (\forall y) (B \implies P^{(n+1)})
                        \implies (\forall y) P^{(n+1)}) } \right]$ --
                $n$-krát použijeme
                    $\provable (\forall x)(A \implies B) \implies
                     ((\exists x) A \implies (\exists x)B)$, čo sa dá
                     dokázať nasledovne:
                     %%% {{{
                   \begin{itemize}
                    \item[c] $ \highlightdim{
                        (\forall x) (X \implies Y)}
                                \provable X \implies Y$ -- už sme
                                mali

                    \item[j] $ \provable Y \implies (\exists x) Y$ --
                        duálny tvar axiómy špecifikácie

                    \item[k] $ \highlightdim{
                        (\forall x) (X \implies Y)} \provable
                            X \implies (\exists x) Y$ -- JS c,j

                    \item[l] $ \highlightdim{
                        (\forall x) (X \implies Y)} \provable
                            (\exists x) X \implies (\exists x) Y$ --
                            pravidlo zavedenia existenčného
                            kvantifikátora, $x$ nie je voľné v
                            $(\exists x) Y$.

                    \item[m] $ \provable \highlightdim{
                        (\forall x) (X \implies Y)} \implies
                            (\exists x) X \implies (\exists x) Y$ --
                            Veta o dedukcii, čitateľ si môže
                            premyslieť, že ju môžeme použiť
                   
                   \end{itemize}
                   %%% }}}

            \item[7] $\provable
                 \highlighta{\existsxton} \left[
                 \highlightb{(\forall y) (B \implies P^{(n+1)})
                 \implies (\forall y) P^{(n+1)}} \right]$ -- MP
                 indukčného predpokladu a 6
            \end{itemize}
        %%% }}}

        \medskip
        \item[$\Leftarrow$]
        Predpokladáme, že je dokázateľné $\provable A^* : 
                \existsxton [(\forall y) (B \implies P^{(n+1)}) 
                \implies (\forall y) P^{(n+1)}]$.
        Treba si uvedomiť,
        že na predikát $P^{(n+1)}$ nekladieme žiadne nároky a teda je
        to dokázateľné pre ľubovoľný taký predikát. No ale
        formula $B$ sa dá chápať ako špeciálny prípad $(n+1)$-árneho
        predikátu. Preto bude dokázateľná aj inštancia
        formuly $A^*$, ktorá vyzerá nasledovne:
        $\provable  \existsxton [(\forall y) (B \implies B) 
            \implies (\forall y) B]$

        %%% {{{ <==
            \begin{itemize}
            \item[1] $\provable \highlightc{
                \existsxton [(\forall y) (B \implies B) 
                \implies (\forall y) B]}$

            \item[2] $\provable (\exists x) (X \implies Y) \implies
                ((\forall x) X \implies (\exists x) Y)$ -- dá sa
                dokázať napríklad takto:
                %%% {{{
                \begin{itemize}
                \item $\provable X \implies (\neg Y \implies \neg (X
                        \implies Y))$

                \item $\provable (\forall x) X \implies (\forall x)
                (\neg Y
                   \implies \neg (X \implies Y))$ -- veta o zavedení
                   všeobecných kvantifikátorov

                \item $(\forall x) X \provable
                    (\forall x) (\neg Y \implies \neg (X \implies Y))$
                    -- veta o dedukcii

                \item $(\forall x) X \provable
                    \highlighta{(\forall x) \neg Y} \implies
                    \highlightb{(\forall x) \neg (X \implies Y)}$ --
                    veta o distribúcii kvantifikátorov\footnote{
                        formálne by sme mali ešte spraviť medzikrok
                        $(\forall x) X \provable \neg Y \implies
                                \neg (X \implies Y)$
                    }

                \item $(\forall x) X \provable
                    \highlightb{\highlightc{\neg} (\forall x) 
                                 \neg (X \implies Y)} \implies
                    \highlighta{\highlightc{\neg} (\forall x) 
                                 \neg Y}$ -- obmena implikácie

                \item $\highlighta{(\forall x)} X \provable
                    \highlightb{\highlightp{(\exists x)} (X \implies
                    Y)} \implies
                    \highlightp{(\exists x)} Y$ -- nahradenie
                    kvantifikátorov

                \item $\provable \highlightb{(\exists x) (A \implies B)} 
                    \implies
                     \highlighta{(\forall x) X } \implies
                    (\exists x) Y$ -- 2 krát veta o dedukcii
                \end{itemize}
                %%% }}}
            \item[3] $\provable \highlightc{
                \existsxton [(\forall y) (B \implies B) 
                \implies  (\forall y) B]} \implies $ \\
                \phantom{x}$\quad\quad
                \highlighta{
                [ \forallxton (\forall y_n) (B \implies B)
                \implies \existsxton (\forall y) B]}$ -- $n$-krát
                využijeme 2 nasledujúcim štýlom:
                Použijeme na dané pravidlo vetu o zavedení
                existenčného kvantifikátora. Následne, pravú stranu
                hlavnej implikácie vidíme ako kandidáta na krok 2.
                Preto ju na novom riadku rozpíšeme podľa tohoto pravidla
                a následne použítím jednoduchého sylogizmu tieto 2
                riadky zložíme. Tím dostaneme variantu 2 s pridanou
                ďalšou premennou a toto opakujeme príslušný počet
                krát.

            \item[4] $\provable \highlighta{
                \highlighto{\forallxton (\forall y_n) (B \implies B)}
                \implies \highlightp{\existsxton (\forall y) B}}$.
                -- MP 1,3
            \item[5]
                $\provable \highlighto{\forallxton (\forall y) (B
                \implies B)}$
                -- $(n+1)$-krát pravidlo zovšeobecnenia použité na 
                dokázateľnú formulu $\provable B \implies B$.
            \item[6]
                $\provable \highlightp{\existsxton (\forall y) B}$
                -- MP 4,5
            \end{itemize}
        %%% }}}
        \end{itemize}
        %%% }}}

        Ešte potrebujeme transformovať formulu $A^*$ na formulu s
        hodnosťou $m-1$. Vieme, že $B$ je v prenexnom tvare
        (predpokladami sme to hneď na začiatku) a teda ju môžeme
        zapísať ako 
        $B:(Q_1 z_1)(Q_2 z_2) \dots (Q_l z_l) C$, kde $C$ je formula
        bez kvantifikátorov. Vieme, že hodnosť $B$, resp.
        hodnosť jej prefixu $(Q_1 z_1)(Q_2 z_2) \dots (Q_l z_l)$ je $m-1$
        nakoľko hodnosť $A$ je $m$ a v $A$ sa pred $B$ nachádza
        jeden všeobecný kvantifikátor.
        %%% {{{
        \begin{itemize}
        %%% pomocne makra
        \def\lefta{\scalebox{3}[1]{\highlighta{[}} }
        \def\leftb{\scalebox{3}[1]{\highlightb{[}} }
        \def\righta{\scalebox{3}[1]{\highlighta{]}} }
        \def\rightb{\scalebox{3}[1]{\highlightb{]}} }

        \item[$A^*$] 
        $=\highlightdim{\existsxton} 
            \lefta (\forall y)
            \leftb
              (Q_1 z_1) (Q_2 z_2) \dots (Q_l z_l) C \implies
              P^{(n+1)} \rightb \implies (\forall y)
              P^{(n+1)} \righta$

        \item[] $\Leftrightarrow$
        $ \highlightdim{\existsxton} \lefta
                    (\forall y) (\overline{Q_1} z_1)
                    (\overline{Q_2} z_2) \dots
                    (\overline{Q_l} z_l) \leftb
                        C \implies P^{(n+1)} \rightb
                            \implies (\forall u)
                                P^{(n+1)} \righta$ -- veta o variantoch
                                + $l$ prenexných operácií d)
        \item[] $\Leftrightarrow$
        $ \highlightdim{\existsxton} \highlightc{(\exists y)}
                \lefta
                    (\overline{Q_1} z_1)
                    (\overline{Q_2} z_2) \dots
                    (\overline{Q_l} z_l) \leftb
                        C \implies P^{(n+1)} \rightb
                            \implies (\forall u)
                                P^{(n+1)} \righta$ -- prenexná
                                operácia d) aplikovaná na $y$
        \item[] $\Leftrightarrow$
        $ \highlightdim{\existsxton} (\exists y)
                    (Q_1 z_1)
                    (Q_2 z_2) \dots
                    (Q_l z_l) \lefta \leftb
                        C \implies P^{(n+1)} \rightb
                            \implies (\forall u)
                                P^{(n+1)} \righta$ -- $l$-krát prenexná
                                operácia d)

        \item[] $\Leftrightarrow$
        $ \highlightdim{\existsxton} (\exists y)
                    (Q_1 z_1)
                    (Q_2 z_2) \dots
                    (Q_l z_l) (\forall u) \lefta \leftb
                        C \implies P^{(n+1)} \rightb
                            \implies
                                P^{(n+1)} \righta$ -- prenexná
                                operácia c) aplikovaná na $u$

            %%% zmaz pomocne makra
            \let\lefta\undefined
            \let\leftb\undefined
            \let\righta\undefined
            \let\rightb\undefined
        \end{itemize}
        %%% }}}
        Posledná formula má hodnosť $m-1$ a je v prenexnom tvate.
        Tým pádom ju vieme podľa
        indukčného predpokladu previesť na Skolemov normálny tvar.
    \end{itemize}
%%% }}}    
\end{dokaz}

\section{Predikátová logika s rovnosťou}

Predikátovú logiku môžeme rozšíriť o nové axiómy, ktoré budú
hovoriť o predikáte ``=''.

\par{Axiómy rovnosti:}
\noindent
\begin{itemize}
%%% {{{
    \item[R1:] ak $x$ je premenná, potom formula $x=x$ je axióma
    \item[R2:] ak $x_1,\dots,x_k, y_1, \dots, y_k$ sú premenné a 
        $f$ je $k$-árny funkčný symbol, potom je axiómou formula
        \begin{equation*}
            (x_1 = y_1) \implies ( (x_2 = y_2) \implies ( \dots
                \implies ((x_k = y_k) \implies
                    [f(x_1,\dots,x_k) = f(y_1,\dots,y_k)] \dots )))
        \end{equation*}
    \item[R3:] ak $x_1,\dots,x_k, y_1, \dots y_k$ sú premenné a 
        $P$ je $k$-árny predikátový symbol, potom je axiómou formula
        \begin{equation*}
            (x_1 = y_1) \implies ( (x_2 = y_2) \implies ( \dots
                \implies ((x_k = y_k) \implies
                    [P(x_1,\dots,x_k) \implies P(y_1,\dots,y_k)] \dots )))
        \end{equation*}
%%% }}}        
\end{itemize}

\begin{priklad}[Teória neostrého čiastočného usporiadania $\le$]
\noindent
%%% {{{
    \begin{itemize}
        \item[1.] $(\forall x) \langle x,x\rangle \in \varphi$ -- Identita

        \item[2.] $(\forall x) (\forall y) [
            (\langle x,y \rangle \in \varphi \land
             \langle y,x \rangle \in \varphi) \implies (x=y)]$ -- 
             Antisymetrickosť

        \item[3.] $(\forall x) (\forall y) (\forall z)
            [( \langle x,y \rangle \in \varphi \land 
               \langle y,z \rangle \in \varphi) \implies
               \langle x,z \rangle \in \varphi]$.
               Tranzitívnosť\footnote{Pozn.: Axiómu môžeme uvádzať aj
               v ekvivalentnom tvare
                $(\forall x) (\forall y) (\forall z)
                [ \langle x,y \rangle \in \varphi \implies (
                    \langle y,z \rangle \in \varphi \implies
                    \langle x,z \rangle \in \varphi)]$. Inak povedané,
                    $(A \land B) \implies C$ sme nahradili
                    $A \implies (B \implies C)$.
               }
    \end{itemize}
    Ak pridáme trichotomickosť, dostaneme teóriu neostrého
    usporiadania:
    \begin{itemize}
        \item[4.] $(\forall x) (\forall y) [x \not=y \implies
            (\langle x,y \rangle \in \varphi \lor 
             \langle y,x \rangle \in \varphi)]$

        \item[4'] $(\forall x) (\forall y) [x = y \lor
             \langle x,y \rangle \in \varphi \lor 
             \langle y,x \rangle \in \varphi]$
    \end{itemize}
%%% }}}    
\end{priklad}

\begin{lema}
Rovnosť je symetrická a tranzitívna.
    \begin{itemize}
        \item[1] $\provable (x=y) \implies (y=x)$ -- symetria
        \item[2] $\provable (x=y) \implies ((y=z) \implies (x=z))$ --
            tranzitívnosť
    \end{itemize}
\end{lema}
\begin{dokaz}
%%% {{{
\noindent
\begin{itemize}
    \item Symetria:
        \begin{itemize}
        \item[1] $\provable (x=y) \implies ((x=x) \implies ((x=x)
                \implies (y=x)))$
            pretože \\
                $\provable (x_1=y_1) \implies ((x_2=y_2) \implies
                    ((x_1=x_2) \implies (y_1=y_2)))$ je inštancia R3
        \item[2] $\provable A \implies (B \implies (B \implies C))$.
        \item[3] $\provable B \implies (B \implies (A \implies C))$ --
            2x použité pravidlo zámeny predpokladov + veta o dedukcii
        \item[4] $\provable (x=x) \implies (x=x) \implies (x=y)
                    \implies (y=x)$
        \item[5] $\provable x=x$ -- R1
        \item[6] $\provable (x=y) \implies (y=x)$ -- 2x MP na 5,4
        \end{itemize}

    \item Tranzitívnosť:
        \begin{itemize}
            \item[1] $\provable (y=x) \implies ((z=z) \implies
                ((y=z) \implies (x=z)))$, pretože \\
                $\provable (x_1 = y_1) \implies ((x_2=y_2) \implies
                 ((x_1=x_2) \implies (y_1=y_2)))$ je inštancia R3.
            \item[2] $\provable A \implies (B \implies (C \implies D))$
            \item[3] $\provable B \implies (A \implies (C \implies D))$
                -- pravidlo zámeny predpokladov
            \item[4] $\provable \highlightc{(z=z)} \implies [
                \highlightp{(y=x) \implies ((y=z) \implies
                (x=z))}]$
            \item[5] $\provable \highlightc{z=z}$ -- R1
            \item[6] $\provable \highlightp{
                    (y=x) \implies ((y=z) \implies (x=z))}$
                -- MP 4,5
            \item[7] $\provable (A \implies B) \implies ((B \implies
                C)) \implies (A \implies C))$ -- JS
            \item[8] $\provable \highlighta{[(x=y) \implies (y=x)]}
                \implies \Big[\highlightb{
                [\highlightp{(y=x) \implies ((y=z) \implies (x=z))}] \implies
                [\highlighto{(x=y) \implies ((y=z) \implies (x=z)))}]} \Big]$ 
                -- inštancia 7
            \item[9] $\provable \highlighta{(x=y) \implies (y=x)}$ -- symetria
            \item[10] $\provable \highlightb{ \highlightp{
                [(y=x) \implies ((y=z) \implies (x=z))]} \implies
                [\highlighto{
                (x=y) \implies ((y=z) \implies (x=z))}]}$ -- MP 9,8
            \item[11] $\provable \highlighto{
                (x=y) \implies ((y=z) \implies (x=z))}$ -- MP 6,10
        \end{itemize}
\end{itemize}
%%% }}}
\end{dokaz}

\begin{veta}
    Nech $t_1,\ldots,t_n,s_1,\ldots,s_n$ sú termy, pričom platí
        $\forall i \in \{1,\dots,n\}:\; \provable t_i = s_i$.
    Potom
    \begin{itemize}
    \item[i)] Ak $t$ je term, ktorý vznikne z termu $s$ nahradením
        niektorých výskytov termov $s_i$ za $t_i$, potom 
        $\provable t=s$.
    \item[ii)] Ak $A'$ je formula, ktorá vznikne z formuly $A$
    dosadením $t_i$ za niektoré termy $s_i$, okrem prípadov, keď
    $t_i$ je premenná $x$ v kvantifikácii $(\forall x)$ 
    resp. $(\exists x)$. Potom
    $\provable A \lequiv A'$.
    \end{itemize}
\end{veta}

\begin{dokaz}
%%% {{{
\noindent
\begin{itemize}
    \item[i)] Dôkaz matematickou indukciou vzhľadom na zložitosť termu $t$.
        %%% {{{
        \begin{itemize}
        \item Nech $t$ je premenná alebo $t$ je $s_i$ pre nejaké $i$.
            Potom zjavne $\provable t=s_i$.

        \item Nech term $t$ je $f(r_1,\dots,r_k)$, 
            term $s$ je $f(r_1',\dots,r_k')$.
            Pre $r_1,\dots,r_k$ platí IP, čize $\provable r_i = r_i'$.
            Potom $\provable (r_1=r_1') \implies  \dots \implies
                (r_k = r_k') \implies (f(r_1,\dots,r_k) = f(r_1',
                    \dots, r_k'))$, čo $k$-násobným použitím MP na 
                    indukčný predpoklad vedie k
                    $\provable f(r_1,\dots,r_k) = f(r_1', \dots,
                    r_k')$.
                
        \end{itemize}
        %%% }}}
    \item[ii)] Zámena termov prebieha len v atomických podformulách
        formuly $A$ (každý term je časťou nejakej atomickej podformuly).
        Máme 2 možnosti, ako vyzerá atomická podformula:
        %%% {{{
        \begin{itemize}
        \item Nech $P$ je atomická podformula tvaru
            $P(r_1,\dots,r_l)$. Potom po zámene dostaneme z 
            $P(r_1, \dots, r_k)$ formulu $P':P(r_1',\dots,r_k')$.
            Chceme ukázať $\provable P(r_1,\dots,r_k) \lequiv
                                     P(r_1',\dots,r_k')$, ak vieme, že
            podľa IP platí $\forall i: \provable r_i = r_i'$
            %%% {{{
            \begin{itemize}
            \item [$\Rightarrow:$] 
                \begin{itemize}
                \item $\provable (r_1=r_1') \implies
                                 (r_2=r_2') \implies \dots \implies
                                 (r_k=r_k') \implies
                                 (P(r_1,\dots,r_k) \implies
                                 P(r_1',\dots,r_k'))$ -- inštancia R3
                \item $\provable r_i = r_i'$ -- IP
                \item $\provable P(r_1,\dots,r_k) \implies
                    P(r_1',\dots, r_k')$ -- $k$-krát MP
                \end{itemize}
            \item [$\Leftarrow:$]
                \begin{itemize}
                \item $\provable (r_1'=r_1) \implies
                                 (r_2'=r_2) \implies \dots \implies
                                 (r_k'=r_k) \implies
                                 (P(r_1',\dots,r_k') \implies
                                 P(r_1,\dots,r_k))$ -- inštancia R3
                \item $\provable r_i' = r_i$ -- IP + symetria
                \item $\provable P(r_1',\dots,r_k') \implies
                    P(r_1,\dots, r_k)$ -- $k$-krát MP
                
                \end{itemize}
            \end{itemize}
            %%% }}}
        \item Nech atomická podformula je tvaru $r_1 = r_2$. 
        Tento prípad sa dá ukázať podobne ako predchádzajúci,
        na prednáške bol ale iný dokaz:
        Chceme ukázať $\provable r_1=r_2 \lequiv r_1'=r_2'$.
            %%% {{{
            \begin{itemize}
            \item[$\Rightarrow:$]
                \begin{itemize}
                \item $\provable r_1=r_1'$ -- IP
                \item $\provable r_2=r_2'$ -- IP
                \item $\provable r_1 = r_2 \implies (
                            r_2 = r_2' \implies r_1 = r_2')$ --
                            tranzitívnosť
                \item $r_1=r_2 \provable r_1=r_2'$ -- VD + MP
                \item $r_1=r_2 \provable r_2'=r_1$ -- symetria + MP
                \item $\provable r_2' = r_1 \implies (
                            r_1 = r_1' \implies r_2' = r_1')$ --
                            tranzitívnosť
                \item $r_1=r_2 \provable r_2' = r_1'$ -- 2-krát MP
                \item $r_1=r_2 \provable r_1' = r_2'$ -- symetria + MP
                \item $\provable r_1=r_2 \implies r_1'=r_2'$ -- VD
                \end{itemize}
            \item[$\Leftarrow:$]
                \begin{itemize}
                \item $\provable r_1=r_1'$ -- IP
                \item $\provable r_2=r_2'$ -- IP
                \item $\provable r_1' = r_2' \implies (
                            r_2' = r_2 \implies r_1' = r_2)$ --
                            tranzitívnosť
                \item $r_1'=r_2' \provable r_1'=r_2$ -- VD + MP
                \item $\provable r_1 = r_1' \implies (
                            r_1' = r_2 \implies r_1 = r_2)$ --
                            tranzitívnosť
                \item $r_1'=r_2' \provable r_1 = r_2$ -- 2xMP
                \item $\provable r_1'=r_2' \implies r_1=r_2$ -- VD
                \end{itemize}
            \end{itemize}
            %%% }}}
        \end{itemize}
        %%% }}}
        Dokázali sme, že atomické podformuly sú ekvivalentné. Spolu s
        vetou o ekvivalencii to ale znamená, že aj pôvodné formuly sú
        ekvivalentné.
\end{itemize}
%%% }}}
\end{dokaz}

\begin{veta}
    Majme term $t$, termy $t_1,\dots,t_n, s_1,\dots, s_n$ a formulu
    $A$.
    Potom platí
    \begin{itemize}
        \item[i)] $\provable t_1=s_2 \implies t_2=s_2 \implies
            t_n=s_n \implies (t[t_1,\dots,t_n] = t[s_1,\dots,s_n])$.
        \item[ii)] $\provable t_1=s_2 \implies t_2=s_2 \implies
            t_n=s_n \implies (A[t_1,\dots,t_n] \lequiv 
                              A[s_1,\dots,s_n])$.
    \end{itemize}
    Ak navyše $x$ je premenná, ktorá nie je obsiahnutá v terme $t$,
    potom platí
    \begin{itemize}
        \item[iii)] $\provable A_x[t] \lequiv 
            (\forall x)((x=t) \implies A)$
        \item[iv)] $\provable A_x[t] \lequiv 
            (\exists x)((x=t) \land A)$
    \end{itemize}
\end{veta}
\begin{dokaz}
    \noindent
    \begin{itemize}
        \item[i),ii)] Ak $t_1,\dots,t_n,s_1,\dots,s_n$ neobsahujú
            premenné, tak to vyplýva priamo z predchádzajúcej vety a vety
            o dedukcii. V prípade, že tieto termy obsahujú premenné,
            tieto premenné nahradíme rôznymi konštantami, použijeme
            vetu o konštantách,\footnote{Veta o konštantách hovorí, že
            $T \provable A \iff T \provable A[c_1,\dots,c_m]$ kde
            $c_1,\dots,c_m$ sú nové konštanty. Inak povedané,
            to, že niečo vieme dokázať s premennými je ekvivalentné
            tomu, že to vieme dokázať ak premenné nahradíme novými
            konštantami} predchádzajúcu vetu a vetu o dedukcii.
        \item[iii)]
            %%% {{{
            \begin{itemize}
            \item[$\Rightarrow:$]
                \begin{itemize}
                \item $\provable x=t \implies (A \lequiv
                    A_x[t])$ -- podľa ii).
                \item $\provable \underbrace{x=t}_X \implies (
                    \underbrace{A_x[t]}_Y \implies
                    \underbrace{A}_Z)$ -- platí totiž
                    $\provable (B \implies (C \lequiv D))
                    \implies (B \implies (D \implies C))$
                \item $(X \implies (Y \implies Z)) \implies
                       (Y \implies (X \implies Z))$ -- pravidlo zámeny
                       predpokladov
                \item $\provable A_x[t] \implies (x=t \implies A)$ --
                    MP na pravidlo zámeny predpokladov
                \item $\provable A_x[y] \implies (\forall x)(x=t
                    \implies A)$ -- pravidlo zavedenia všeobecného
                    kvantifikátora
                \end{itemize}

            \item[$\Leftarrow:$]
                \begin{itemize}
                \item $\provable \underbrace{(\forall x)(x=t \implies
                    A)}_X \implies
                        (\underbrace{(t=t)}_Y \implies
                        \underbrace{A_x[t]}_Z)$ -- Axióma špecifikácie

                \item $(X \implies (Y \implies Z)) \implies
                       (Y \implies (X \implies Z))$ -- pravidlo zámeny
                       predpokladov

                \item $\provable (t=t) \implies ((\forall x) ((x=t)
                    \implies A) \implies A_x[t])$ -- MP

                \item $\provable t=t$ - axióma R1

                \item $\provable (\forall x) ((x=t)
                    \implies A) \implies A_x[t]$ -- MP
                \end{itemize}
            \end{itemize}
            %%% }}}
        \item[iv)]
            \begin{itemize}
            \item[1] $\provable (\forall x)((x=t) \implies A)
                \lequiv A_x[t]$ -- iii)
            \item[2] $\provable (\forall x)((x=t) \implies \neg A)
                \lequiv \neg A_x[t]$ -- inštancia 1
            \item[3] $\provable \neg \neg A_x[t] \lequiv
                \neg (\forall x) ((x=t) \implies \neg A)$ -- obmena
                ekvivalencie
            \item[4] $\provable A_x[t] \lequiv (\exists x) \neg
                ((x=t) \implies \neg A)$ -- odstránenie $\neg \neg$ a
                zámena kvantifikátora
            \item[5] $\provable \neg(X \implies \neg Y)
                \lequiv  (X \land Y)$ -- rozpísanie
                $\land$
            \item[6] $\provable A_x[t] \lequiv (\exists x)
                ((x=t) \land A)$ -- veta o ekvivalencii aplikovaná na
                4,5

            \end{itemize}
    \end{itemize}
\end{dokaz}



\chapter{Matematická logika}
% vim: set fdm=marker:
\section{Pravdivosť a dokázateľnosť}

\begin{definicia}[Logická platnosť formuly]
%%% {{{
    Nech $L$ je jazyk prvého rádu a $A$ je formula jazyka
    $L$. Hovoríme, že formula $A$ je \emph{logicky platná},
    označujeme $\models A$,
    ak je splnená v ľubovoľnej realizácii $m$ jazyka $L$.

    \begin{align*}
        \mathcal{M} & \models A[e] \quad \forall \mathcal{M} \\
        \mathcal{M} & \models A \\
        & \models A
    \end{align*}
%%% }}}    
\end{definicia}

\begin{poznamka}
    Formula $A$ je logicky platná, práve vtedy, keď je pravdivá
    bez ohľadu na realizáciu symbolov jazyka $L$.
\end{poznamka}

\begin{definicia}[Teória]
%%% {{{
    Nech $L$ je jazyk prvého rádu a $T$ je množina formúl
    jazyka $L$. Hovoríme, že $T$ je teória 1. rádu predikátovej logiky
    s jazykom $L$ (t.j. množina formúl $T$ je množina axióm teórie).
%%% }}}
\end{definicia}

\begin{definicia}[Model teórie]
%%% {{{
    Nech $T$ je teória v jazyku $L$, $\mathcal{M}$ je realizácia jazyka $L$.
    Hovoríme, že $\mathcal{M}$ je modelom teórie $T$ (označujeme
    $\mathcal{M} \models T$), ak pre každú formulu $A$
    patriacu $T$ platí $\mathcal{M} \models A$.
%%% }}}
\end{definicia}

\begin{definicia}[Tautologický dôsledok]
%%% {{{
    Hovoríme, že formula $A$ je
    sémantickým/tautologickým dôsledkom (vetou teórie) množiny formúl $T$,
    resp. $A$ je $T$-platná, ak $A$ je splnená v každom modeli teórie $T$.
    Túto skutočnosť označujeme $T \models A$.
%%% }}}
\end{definicia}

\begin{priklad}[Teória ostrého usporiadania]
%%% {{{
    Majme predikát $<$ na množine $N$, tak, že platí
    \begin{itemize}
    \item[1.] $(\forall x)(\forall y) ((x < y) \implies \neg (y < x))$
    - asymetrickosť
    \item[2.] $(\forall x)(\forall y)(\forall z) (((x<y) \land (y<z)) \implies
        (x<z))$ - tranzitívnosť
    \item[3.] $(\forall x)(\forall y)( (x\not=y) \implies ((x<y) \lor (y<x)))$ -
     trichotomickosť
    \end{itemize}
    Ak sú splnené axiómy 1,2, tak daná množina je modelom čiastočného
    usporiadania. Ak je
    navyše splnená aj axióma 3, množina tvorí teóriu ostrého
    usporiadania.
%%% }}}    
\end{priklad}

\begin{priklad}[Elementárna aritmetika]
%%% {{{
    Jazyk prvého rádu rozšírme o nasledujúce symboly:
    \begin{itemize}
        \item $0$ -- konštanta (nulárny funkčný symbol),
        \item $S$ -- $S(x)=x+1$, čiže nasledovník (unárny funkčný symbol),
        \item $+,*$ -- binárne funkčné symboly
    \end{itemize}
    Axiómy elementárnej aritmetiky:
    \begin{itemize}
        \item[1.] $\neg (S(x) = 0)$
        \item[2.] $(S(x) = S(y)) \implies (x=y)$
        \item[3.] $(x+0) = x$
        \item[4.] $(x+s(y)) = (s(x) + y)$
        \item[5.] $(x * 0) = 0$
        \item[6.] $(x * S(y)) = ((x*y)+x)$
    \end{itemize}
    Zoberme si realizáciu 
        $\mathcal{N}=\langle N^+,0,S,+,* \rangle, N^+ =
        \{1,2,\dots\}$.
        Potom $\mathcal{N}$ je model pre elementárnu aritmetiku,
        zvykne sa označovať aj ako \emph{štandardný model}.
        Takejto aritmetike sa hovorí Robinsonova aritmetika.
        Ak pridáme axiómu indukcie, dostaneme Peanovu aritmetiku.
%%% }}}
\end{priklad}

\begin{poznamka}
    Nevšímajme si nachvíľu axiómy, ale iba relačnú štruktúru
    $\mathcal{N}=\langle N^+,0,S,+,* \rangle$.
    Zoberme namiesto $S$ konštantu $1$.
    Ciže $\mathcal{N}'=\langle N^+,0,1,+,* \rangle$.
    $\mathcal{N}'$ realizuje jazyk teórie telies.
    Lenže $\mathcal{N}'$ nie je modelom tohoto jazyka --
    na to, aby daná realizácia bola modelom, muselo by platiť, že
    každá formula $T$ je splnená v danej realizácii. V našom prípade,
    v každom telese platí $\provable \neg S(S(0))=S(0)$ (inak
    povedané, $2 \ne 1$, ak má teleso charakteristiku viac ako 2,
    v prípade charakteristiky 2 je to $0 \ne 1$). Nuž ale v 
    $\mathcal{N}'$ je splnená formula $S(S(0))=S(0)$, lebo sa
    realizuje ako $1=1$.
\end{poznamka}

\begin{priklad}[Teória grúp]
%%% {{{
    Špeciálne symboly sú $+,-$.
    Axiómy sú:
    \begin{enumerate}
            \item $((x+y)+z) = (x+(y+z))$ -- asociativita
            \item $(x+0) = (0+x) = x$ -- existuje neutrálny prvok
            označený ako 0
            \item $x+(-x) = 0 = (-x)+x$ -- existujú (ľavé a pravé) inverzné
            prvky
    \end{enumerate}
%%% }}}
\end{priklad}

Ďalším cieľom je stotožniť dokázateľné formuly s tautológiami.
\stopFIXME

\begin{veta}[O korektnosti]
    Ak $T$ je teória v jazyku $L$ a ak formula $A$ je taká,
    že $T \provable A$, potom $T \models A$.
\end{veta}

\begin{dokaz}
%%% {{{
    Nech $A_1, A_2, \ldots, A_n\equiv A$ je odvodenie (dôkaz) formuly $A$
    z predpokladov $T$ (v teórii $T$).
    Nech $\mathcal{M}$ nech je ľubovoľný model teórie $T$ 
    (čiže platí $\mathcal{M} \models A$).
    Ukážeme (indukciou podľa dĺžky dôkazu), že platí $\mathcal{M} \models A_i$
    pri predpoklade, že pre $\forall j:j < i$ platí $m \models A_j$.

    Do dôkazu sa $A_i$ môže dostať niekoľkými spôsobmi:
    \begin{enumerate}
    \item $A_i \in T, \mathcal{M}$ je model $T$. Potom 
        $\mathcal{M} \models A_i$ a teda $T \models A_i$

    \item $A_i$ sa dostane do dôkazu ako axióma predikátovej logiky:
    \begin{enumerate}
        \item $A_i$ je axióma výrokovej logiky -- je poskladaná z
            atomických formúl a logických spojok $\neg$ a $\implies$. Potom
            $A_i$ je tautológia výrokovej logiky (ak formula je tautológia,
            jej pravdivostná hodnota nezávisi od ohodnotenia
            premenných:
            $\mathcal{M} \models A_i$, teda $\mathcal{M} \models A_i[e]$).

        \item $A_i$ je axioma špecifikácie, teda je tvaru 
            $A_i: (\forall x) B \implies B_x[t]$, 
            $t$ je substitúcia za $x$ do $B$. 
            Chceme ukázať, že formula bude pravdivá
            pri každom ohodnotení $e$.
            Zaujíma nás prípad, kedy $(\forall x) B$ je pravdivý.\footnote{
                v opačnom prípade implikácia triviálne platí}
            To znamená, že pre ľubovoľné indivíduum $m$ platí (z Tarského
            definície) $B[e(x/m)]$, teda $e(x)=m$.
        
            Tvrdenie zo zimného semestra: Ak platí
              $\forall i: t_i[e] = m_i$, potom

            \begin{equation*}
                \mathcal{M} \models A_{x_1, \ldots x_n}[t_1, \ldots t_n][e] 
                    \iff
                \mathcal{M} \models A[e(x_1/m_1, \ldots x_n/m_n)]
            \end{equation*}

            Teda namiesto $B_x[t][e]$, vieme použiť $B[e(x/m)]$.
            Táto formula je ale pravdivá v $\mathcal{M}$ (viď vyššie).

        \item $A_i: (\forall x) (B \implies C) \implies (B \implies
            (\forall x) C)$ a $x$ nie je voľná v $B$.
            Mali by sme dokázať, že platí $\mathcal{M} \models A_i$.
            Zaujímavý prípad je, keď
            $\mathcal{M} \models (B \implies C)[e(x/m)]$ platí, vtedy sa
            pozeráme na platnosť $(B \implies (\forall x)C)$.
            Posledná formula je ale ekvivalentná s $\neg B \lor (\forall x) C$.
            Dôležitý je tiež predpoklad, že $x$ nie je voľná v $B$,
            a teda nezávisí od ohodnotenia viazanej premennej.
            Ak $B$ nie je pravdivá, tak disjunkcia je pravdivá a
            problém je vyriešený.
            Ak by $B$ bola pravdivá, tak by malo byť $(\forall x) C$
            pravdivé. Lenže to musí byť, inak by neplatilo
            $(\forall x) (B \implies C)$.
    \end{enumerate}

    \item $A_i$ je niektorá axióma rovnosti:
    \begin{enumerate}
        \item $A_i: x=x$. Potom $A_i[e]$ je $m=m$ a teda pri každom
            ohodnotení $\mathcal{M} \models A_i$.

        \item $A_i: (x_1 = y_1) \implies (x_2 = y_2) \implies \ldots 
            \implies [f(x_1, \ldots, x_n) = f(y_1, \ldots, y_n)]$.
            Zaujíma nás prípad, keď  $e(x_i) = e(y_i)$, teda
            $e(x_i)=e(y_i)=m_i$. Vtedy dostávame
            \begin{equation*}
              m_1=m_1 \implies m_2=m_2 \implies \ldots
                \implies [f(m_1, \dots, m_n) =f(m_1, \dots, m_n)]
            \end{equation*}

        \item $A_i: (x_1 = y_1) \implies (x_2 = y_2) \implies \ldots 
            \implies [P(x_1, \ldots, x_n) \implies P(y_1, \ldots, y_n)]$.
            Zaujíma nás opäť prípad, keď $x_i$ aj $y_i$ reprezentujeme 
            rovnakým indivíduom: $e(x_i/m_i)$ a $e(y_i/m_i)$.
            Potom $(m_1,\dots,m_n)$ buď je alebo nie je v relácii $P$
            a implikácia $P \implies P$ bude pravdivá.
    \end{enumerate}

    \item $A_i$ dostávame ako výsledok odvodzovacieho pravidla.
        \begin{enumerate}
        \item Modus Ponens:
            Vieme
            $T \provable A_j, T \provable A_j \implies A_i$.
            Podľa IP platí
            $T \models A_j, T \models A_j \implies A_i$ a vďaka
            korektnosti MP teda aj $T \models A_i$.

        \item Pravidlo zovšebecnenia: $A_i: (\forall x) A_j$.
            Z IP platí $\models A_j \then
                \mathcal{M} \models A_i[e(x/m)]$
            pre ľubovoľné indivíduum $m$. Tým pádom ale
            $\models (\forall x) A_j$.
            
        \end{enumerate}
    \end{enumerate}
%%% }}}
\end{dokaz}


\begin{priklad} % existencia nedokazatelnej formuly a jej negacie v el. ar.
%%% {{{
    Uvažujme elementárnu aritmetiku, ktorá má svoj štandardný
    model. Uvažujme formulu $x=0$ v $N$.
    \begin{itemize}
        \item Nech $e(x) = m$ kde $m \neq 0$. Potom formula $A:x=0$ nie je
        splnená pre ohodnotenie $e$ a teda $m \notmodels A[e]$.
        \item Nech $e(x) = 0$. Potom formula $A': \neg x=0$ nie je
        splnená v ohodnotení $e$, t.j. $m \notmodels A'[e]$.
    \end{itemize}
    To ale znemaná, že formula $A$ ani jej negácia $A'$
    nie sú splniteľné (a teda dokázateľné) v elementárnej aritmetike.
%%% }}}
\end{priklad}


\begin{poznamka} % neexistencia vety o dedukcii v predikatovej logike
%%% {{{
    Vetu o dedukcii v predikátovej logike nemožno vysloviť pre
    ľubovoľnú formulu. Ukážeme to na nasledujúcom príklade: Majme
    formuly $A,A'$
    \begin{align*}
            A :\ & \neg x=0 \\
            A':\ & \neg y=0 \\
    \end{align*}
    $A'$ je inštancia formuly $A$. Potom ale platí:
    \begin{align*}
            A &\provable A'  \quad \mbox{čiže}\\
            \neg (x=0) &\provable \neg (y=0)
    \end{align*}
    Môžem podľa vety o dedukcii napísať, že 
    $\provable \neg (x=0) \implies \neg (y=0)$? Nie, pretože ak
    vezmeme ohodnotenie $e$ nasledovne
    \begin{align*}
        e(x) &= m, \quad m \neq 0 \\
        e(y) &= 0
    \end{align*}
    dostávame
    \begin{equation*}
        m \notmodels (A \implies A')[e]
    \end{equation*}
%%% }}}
\end{poznamka}

\begin{dosledok}[Vety o korektnosti]
    Ak teória $T$ v jazyku $L$ má model $m$, potom $T$ je
    bezposporná.
\end{dosledok}
\begin{dokaz}
%%% {{{
    Nech $\mathcal{M}$ je model $T$. Teória $T$ je tým pádom
    bezosporná. Uvažujme ďalej, že $A$ je ľubovoľná uzavretá formula jazyka $L$.
    Potom práve jedna z formúl $A$, $\neg A$ je pravdivá v modeli
    $\mathcal{M}$
    (podľa Tarského definície splniteľnosti).
    Lenže tá, ktorá nie je pravdivá, nie je ani dokázateľná 
    (podľa vety o korektnosti).
%%% }}}
\end{dokaz}

\medskip
Tento výsledok hovorí, ze ak máme vyšetriť bezospornosť nejakej teórie,
treba nájsť jej model. Dôkazy bezospornosti môžeme rozdeliť na 2 typy
\begin{itemize}
    \item syntaktické -- sú to konečné posutpnosti symbolov, alebo
        formúl. Ak chceme dokázať bezospornosť $T$, tak ju syntakticky
        prevedieme na teóriu $S$, o ktorej vieme, že je bezosporná.
        Napríklad ak $T$ je predikátová logika, môžeme ju previesť na
        výrokovú logiku $S$.

    \item sémantické -- niekedy nie je možné dokazovať syntakticky.
        Sémanticky dokazujeme tak, že nájdeme (nekonečný) model
        $\mathcal{M}$ teórie $T$
\end{itemize}


\begin{priklad}[Bezospornosť predikátovej logiky]
    Ideme ukázať, že predikátová logika je bezosporná na základe toho,
    že výroková logika je bezosporná.

    Máme jazyk $L$ -- jazyk prvého rádu,
    ktorý rozšírime o konštantu $c \notin L$.
    Dostaneme tak jazyk $L' = L \union \{c\}$ zohrávajúci dôležitú
    rolu v dôkaze.

    Proces dokazovania bude prebiehať nasledovne:
    Každý term formuly $A$ nahradíme konštantou $c$,
    ďalej z danej formuly vynechávame všetky kvantifikátory a
    premenné bezprostredne spojené s kvantifikátorom.

    Každej formule $A$ na jazyku $L$ teda priradíme
    formulu $A^*$ na jazyku $L'$ nasledovne:
    \begin{enumerate}
        \item ak $A$ je tvaru $A: P(t_1, \dots ,t_n)$,
                tak $A^*: P(c, c, \dots,c)$.
                \fixme{A co funkcne symboly?}
        \item ak $A$ je tvaru $A: t_1=t_2$,
                tak $A^*: c=c$.
        \item ak $A$ je tvaru $A: B \implies C$,
                potom $A^*: B^* \implies C^*$.
        \item ak $A$ je tvaru $A: B \squareop C$,
                potom $A^*: B^* \squareop C^*$.
        \item ak $A$ je tvaru $A: \neg B$,
                potom $A^*: \neg B^*$
        \item ak $A$ je tvaru $A: (Qx) B$, potom $A^*: B^*$.
    \end{enumerate}

    Ak jazyk $L$ je jazyk bez rovnosti a $\provable_L A$, potom
    o formule $A^*$ tvrdíme, že je to tautológia. \fixme{WTF? bud
    tomu nechapem, alebo ak $\provable \exists x_1 \exists x_2 x_1 \ne
    x_2$, potom $c \ne c$ je tautologia?}
    Naopak, ak $L$ je jazyk s rovnosťou a $\provable_L A$,
    potom $A^*$ je tautologický dôsledok $c=c$.

    Teraz ukážeme, že predikátová logika nie je sporná.
    Tvrdíme, že pre žiadnu formulu $A$ nie je $\provable A$ aj 
    $\provable \neg A$.
    Ak by to platilo, dostali by sme sa do sporu,
    že vo výrokovej logike je $\provable A^*$ aj $\provable \neg A^*$.
\end{priklad}

\section{Veta o úplnosti}

G\"odelova veta, ktorú si teraz vyslovíme a dokážeme, má 2 varianty.

\begin{veta}[G\"odel, 1. variant]
    Nech $T$ je teória v jazyku $L$ a nech $A$ je
    ľubovoľná formula jazyka $L$. Potom $T \provable A \iff T \models A$,
    čiže $A$ je dokázateľná práve vtedy keď
    je splnená v každom modeli teórie $T$.
\end{veta}

\begin{veta}[G\"odel, 2. variant]
    Teória $T$ je bezosporná práve vtedy, ked $T$ má model.
\end{veta}

\begin{poznamka}
    Varianta 1 G\"odelovej vety vyplýva z variantu 2.
\end{poznamka}

\begin{dokaz}[Poznámky]
    Veta o dedukcii mala nasledovný dôsledok:

    Majme teóriu $T$ a jej formulu $A$. Nech $A'$ označuje uzáver
    formuly $A$.
    Potom platí: $T \provable A \iff T \union \{ \neg A' \}$ je sporná
    teória.

    V našom prípade z 2. varianty G\"odelovej vety dostávame
    $T \provable A \iff T \union \{ \neg A' \}$ nemá model.
    Toto znamená, že v každom modeli teórie $T$ je pravdivý uzáver $A'$.
    Z toho dostávame $T \models A' \Rightarrow T \models A$
    (ak zoberieme ľubovoľný model $\mathcal{M}$ teórie $T$, formula
    $A'$ je v ňom splnená ale to nutne znamená, že aj formula $A$ v
    ňom musí byť splnená)
    a teda z platnosti variantu 2 vyplýva variant 1.
    \\
\end{dokaz}

\begin{dokaz}[2. variantu G\"odelovej vety]
    Budeme sa snažiť zostrojiť model pre teóriu, ktorá je bezosporná.
    Majme bezospornú teóriu s jazykom $L$.
    Potrebujeme v prvom rade univerzum -- $M$.
    K dispozícii máme len syntaktické prostriedky teórie.
    Preto ako kandidát na $M$ prichádza do úvahy
    množina termov bez premenných.
    Tieto termy majú jednoznačnú realizáciu (sami sebe budú
    realizáciou).\footnote{
        Pre ujasnenie tejto myšlienky odporúčame čitateľovi
        nalistovať si kapitolu o Herbrandovských interpretáciach
        a nahliadnuť tak pointu tohoto činu.
    }
    V našom modeli budú teda všetky objekty teórie charakterizované termami.

    Ďalšou otázkou je, ako definovať splniteľnosť. Malo by platiť, že
    formula je splniteľná práve vtedy keď je dokázateľná, čize
    \begin{equation*}
    A[e] \iff T \provable A
    \end{equation*}

    Pri konštrukcii modelu sa nám pritrafia isté nepríjemnosti, ktoré bude 
    treba riešiť:
    \begin{enumerate}
    \item \label{en:model_problem_konst}
        Jazyk $L$ nemusú obsahovať žiadne konštanty (a teda žiadne termy bez
        premenných).

    \item \label{en:model_problem_rovnost}
        Ak jazyk $L$ bude jazyk s rovnosťou, môže sa stať, že v teórii $T$
        bude $T \provable t=s$, ale $t$ a $s$ sú rôzne termy bez
        premenných (rôzne konštanty).

    \item \label{en:model_problem_neg}
        Nech $\mathcal{M}$ je ľubovoľná realizácia jazyka $L$ a $A$ je uzavretá
        formula jazyka $L$. Potom práve jedna z formúl $A$, $\neg A$ je
        pravdivá, ale žiadna z nich nemusí byť dokázateľná v $T$.

    \item \label{en:model_problem_korektnost}
        Môže sa stať, že uzavretá formula $(\exists x)B$ je dokázateľná v teórii
        $T$, ale pre žiaden term $t$ bez premenných formula $B_x[t]$ nie
        je dokázateľná v $T$. To znamená, že
        podľa Tarského definície pravdivosti je $(\exists x)B$
        nepravdivá, čo je spor s vetou o korektnosti.
    \end{enumerate}

    Ako odstránime tieto nedostatky?
    Odstránenie bodu \ref{en:model_problem_rovnost} je jednoduché --
    riešime vhodnou faktorizáciou, čiže zavedieme si množinu $\tau$,
    čo je množina všetkých termov bez premenných
    a na nej zavedieme reláciu ekvivalencie.

    Body č. \ref{en:model_problem_konst}, \ref{en:model_problem_neg} a 
    \ref{en:model_problem_korektnost} sa riešia tzv.
    úplným konzervatívnym rošírením teórie (Henkinovským).
    Budú to tzv.  konzervatívne teórie (na pôvodnom jazyku
    nezískame žiadne nové teorémy a ani nestratíme žiadne).
    Nachvíľu teda opustíme dôkaz G\"odelovej vety, aby sme si mohli niečo
    porozprávať o Henkinovej teórii. K dôkazu sa vrátime, keď na to budeme
    mať pripravenú pôdu.
    \\
\end{dokaz}

\begin{definicia}[Úplná teória]
%%% {{{
    Hovoríme, že teória $T$ s jazykom $L$ je úplná, ak $T$ je
    bezosporná teória a pre ľubovoľnú uzavretú formulu $A$ na jazyku
    $L$ buď $A$ alebo $\neg A$ je dokázateľná v $T$.
%%% }}}
\end{definicia}

\begin{poznamka}
    Pojem úplnosti teda zodpovedá nasledujúcej konštrukcii:
    Majme teóriu $T$ nad jazykom $L$ a jej model $\mathcal{M}$.
    Model $\mathcal{M}$ rozhoduje o pravdivosti každej uzavretej
    formuly.
    Označme ako $T_h(\mathcal{M})$ množinu všetkých
    pravdivých uzavretých formúl $T$.
    Potom platí, že $T_h(\mathcal{M})$ je úplná.

    Je dôležité si uvedomiť, že neberieme otvorené formuly --
    napr. formula $x=0$ v elementárnej aritmetike nemusí byť pradivá, 
    pretože závisí od ohodnotenia $x$.
\end{poznamka}

\begin{poznamka}
    V úplnej teórii (a teda špeciálne v $T_h$) nemôže nastať problém 
    \ref{en:model_problem_neg}, ktorý sme spomínali.
\end{poznamka}

\begin{definicia}[Henkinova teória]
%%% {{{ Henkinova teoria
    Hovoríme, že teória $T$ s jazykom $L$ je Henkinova, ak pre
    ľubovoľnú uzavretú formulu $(\exists x)B$ jazyka $L$ platí
    \begin{equation*}
        T \provable (\exists x)B \implies B_x[c]
    \end{equation*}
    pre nejakú konštantu $c$.
%%% }}}
\end{definicia}

\begin{poznamka}
    Ak je teória Henkinova, tak sme vyriešili problémy
    \ref{en:model_problem_konst}, \ref{en:model_problem_korektnost}.
\end{poznamka}

\begin{lema}
    Ak $T$ je úplná a Henkinova teória, tak potom $T$ má model.
    \label{lema:uplna_henkinova}
\end{lema}
\begin{dokaz}
    Nech $L$ je jazyk teórie $T$ a nech $\tau$ je množina všetkých
    termov jazyka $L$ bez premenných. Na množine $\tau$ definujeme reláciu
    ekvivalencie nasledovne:
    \begin{equation*}
        \forall t_1, t_2 \in \tau :
            t_1 \equiv t_2 \iff T \provable t_1 = t_2
    \end{equation*}
    Rovnosť je reflexívna, symetrická, tranzitívna, teda týmto
    spôsobom definovaná relácia je relácia ekvivalencie a rozdeľuje
    nám množinu $\tau$ na triedy ekvivalencie:
    \begin{equation*}
        [t] = \{ s \in \tau | t \equiv s\}
    \end{equation*}
    Týmto sme vyriešili problém \ref{en:model_problem_rovnost}.

    Zadefinujme si univerzum $M$ tak, že ho budú tvoriť vyššie popísané
    triedy ekvivalencie.
    Nech $f$ je ľubovoľný $n$-árny funkčný symbol a nech
    $[t_1], [t_2], \ldots, [t_n] \in M$. Definujeme funkciu $f$ v relačnej
    štruktúre $\mathcal{M}$ nasledovne:
    \begin{equation*}
        f_\mathcal{M}([t_1], \dots, [t_n]) = [f(t_1, \dots t_n)]
    \end{equation*}
    Bolo by potrebné ukázať, že táto definícia je konzistentná. Teda,
    že platí
    $f_\mathcal{M}([t_1], \dots, [t_n])=f_\mathcal{M}([s_1], \dots, [s_n])$
    za predpokladu $\forall i \in {1,\dots,n}:s_i \equiv t_i$.
    Inak povedané, nesmie záležať na výbere reprezentantov.
    Taktiež je dobré si uvedomiť, že
    $[t_{x_1,\dots,x_n}[t_1,\dots,t_n]]=t[e]$, ak platí $e(x_i/t_i)$ resp.
    $e(x_i)=t_i$.

    Podobne, nech $P$ je $n$-árny predikátový symbol rôzny od `='
    (predikát `=' sme si už zaviedli). V tom prípade definujeme
    \begin{equation*}
     ([t_1], \dots, [t_n]) \in P_\mathcal{M} \iff
        T \provable P(t_1, \dots, t_n)
    \end{equation*}
    Zostáva nám ukázať, že nami definované $\mathcal{M}$ je
    naozaj modelom teórie $T$, čo bude predmetom nasledujúcej vety.
    \\
\end{dokaz}

\begin{veta}[O kanonickej štruktúre]
    Nech $T$ je úplná Henkinovská teória a nech $\mathcal{M}$ je
    definované podľa dôkazu lemy \ref{lema:uplna_henkinova}.
    Potom $T \provable A \iff \mathcal{M} \models A$.
\end{veta}

\begin{dokaz}
    Budeme postupovať matematickou indukciou vzhľadom na zložitosť
    formuly. Pri dokazovaní sa ale obmedzíme iba na uzavreté formuly.
    \begin{itemize}
    \item[1:] Báza indukcie:
        \begin{itemize}
        \item Formula $A$ je tvaru $P(t_1,\dots,t_n)$. Pritom $t_1,
            \dots t_n$ sú termy bez premenných (inak by formula $A$
            nebola uzavretá). Podľa definície splniteľnosti platí, že
            $A$ je pravdivá v $T$ práve vtedy keď je dokázateľná v
            $T$. Zeberme si teraz ohodnotenie $e$ také, že
            $t_i[e] = t_i$ (Čiže každý term realizujeme samým sebou).
            Podľa definície je uzavretá formula $A$ pravdivá práve vtedy, 
            keď je splnená aspoň pre jedno ohodnotenie $e$ (vtedy je
            automaticky splnená pre všetky ohodnotenia $e$).
            Teda
            \begin{equation*}
            \begin{split}
                \mathcal{M} \models A 
                &\iff \mathcal{M} \models A[e] \\
                &\iff (t_1[e],t_2[e],\dots,t_n[e]) \in P_\mathcal{M}
                \\
                &\iff ([t_1],[t_2],\dots,[t_n]) \in P_\mathcal{M} \\
                &\iff T \provable P(t_1,t_2, \dots, t_n) \equiv A
            \end{split}
            \end{equation*}

        \item Formula môže byť tvaru $A:t_1=t_2$. V tom prípade
            $\mathcal{M} \models t_1 = t_2 \iff [t_1]=[t_2] \iff
                T \provable t_1=t_2$
        \end{itemize}

    \item[2:] Indukčný krok:\\
        Máme niekoľko možností, ako mohla formula $A$ vzniknúť:
        \begin{itemize}
        \item $A:\neg B$. Na formulu $B$ sa vzťahuje IP, keďže $B$ je
            uzavretá formula. Teda
            $\mathcal{M} \models A \iff \mathcal{M} \not \models B$.
            Vieme, že teória $T$ je úplná, že buď $T \provable A$
            alebo $T \provable B$. Keďže je bezosporná, platí práve
            jedna možnosť. Tým sme ale ukázali, že
            $\mathcal{M} \models A \iff T \provable A$.

        \item $A: B \implies C$. Ma formuly $B,C$ sa vzťahuje IP.
            Rozoberme si obe implikácie tvrdenia
            $\mathcal{M} \models B \implies C \iff
             T \provable B \implies C$.
            \begin{itemize}
            \item[$\Rightarrow:$] Vieme, že $(B \implies C)
                    \lequiv \neg B \lor C$. Čiže môžeme
                    predpokladať, že aspoň 1 z tvrdení 
                    $\mathcal{M} \models \neg B$,
                    $\mathcal{M} \models C$ platí. Rozoberme obe
                    možnosti -- pri prvej predpokladáme, že $\neg B$ je
                    pravdivá (a z IP dokázateľná\footnote{
                        Presnejšie povedané, IP nestačí, potrebujeme
                        spraviť ešte ďalší krok kvôli negácii.
                        Ten sme už ale rozobrali.}):
                    \begin{itemize}
                    \item $T \provable \neg B$ -- IP
                    \item $\provable \neg B \implies (B \implies C)$ --
                        z vety o neutrálnej formule
                    \item $T \provable B \implies C$ -- MP na prvé 2
                        formuly
                    \end{itemize}
                    V druhom prípade predpokladáme pravdivosť $C$,
                    t.j.
                    \begin{itemize}
                    \item $T \provable C$ -- IP
                    \item $\provable C \implies (B \implies C)$ -- A1
                    \item $T \provable B \implies C$ -- MP
                    \end{itemize}
                    Záver: ak platí
                    $\mathcal{M} \models B \implies C$, tak je formula
                    $B \implies C$ dokázateľná v $T$.
            \smallskip
            \item[$\Leftarrow:$] Uvažujeme, že
                $T \provable B \implies C$.
                Teória $T$ je úplná teória a $B \implies C$ je uzavretá.
                Pre každú uzavretú formulu platí jedna z možností
                $T \provable \neg B$ (potom $\mathcal{M} \models \neg B$ a sme
                hotoví) alebo $T \provable B$.
                Vtedy ale musí platiť $T \provable C$
                a teda $\mathcal{M} \models C$. Tým je tvrdenie dokázané.

                Ešte sa môžeme vrátiť k tomu, prečo platí $T \provable C$.
                Uvažujme totiž $T \provable \neg C$.
                \begin{itemize}
                \item $T \provable B \implies C$
                \item $T \provable B$ -- predpoklad
                \item $T \provable C$ -- MP
                \item $T \provable \neg C$ -- predpoklad a zároveň
                    spor s bezospornosťou teórie $T$.
                \end{itemize}
            \end{itemize}

        \item $A: (\forall x) B$. Pre každú
            inštanciu formuly $B$ tvrdenie platí.
            Pretože je teória úplná, môžu nastať 2 prípady:
            \begin{itemize}
            \item Platí $T\provable A$.
                Teda, $T \provable (\forall x) B$.
                Ďalej platí $\provable (\forall x) B \implies B_x[c]$
                -- axióma špecifikácie.
                Potom aj $T \provable B_x[c]$ (MP na axiómu
                špecifikácie). Lenže z IP dostávame
                $\mathcal{M} \models B_x[c] \then 
                 \mathcal{M} \models (\forall x) B$.
            \smallskip
            \item Platí $T \provable \neg A$. Vtedy
                $T \provable (\exists x) \neg B$. Ďalej, keďže teória je
                Henkinovská, platí tiež 
                $T \provable (\exists x) \neg B \implies \neg B_x[c]$.
                Pomocou MP odvodíme
                $T \provable \neg B_x[c]$. Potobne ako v
                predchádzajúcom prípade máme
                $\mathcal{M} \models \neg B_x[c] \then
                 \mathcal{M} \not \models A$ (formula $B_x[c]$ je
                 nepravdivá a preto aj formula $(\forall x) B$ je
                 nepravdivá). Nuž ale potom
                 $\mathcal{M} \models \neg A$.
            \end{itemize}

        \end{itemize}
    \end{itemize}

    Ukázali sme, že pre uzavreté formuly vieme pomocou indukcie
    dokázať tvrdenie vety. Teraz, nech $A$ je ľubovoľná formula z teórie $T$.
    Pre ňu platí, že $T \provable A$ (predpoklad).
    Potrebujem dokázať, že $T \provable A \iff \mathcal{M} \models A$.
    Zoberiem si $A'$ -- uzáver formuly $A$.
    Na $A'$ sa vzťahuje naša indukcia a teda 
    $\mathcal{M} \models A' \iff T \provable A'$.
    Podľa vety o uzávere ale platí
    $\mathcal{M} \models A \iff  \mathcal{M} \models A'$ čo je 
    $\iff T \provable A'$ a z vety o uzávere $\iff T \provable A$.
    \\
\end{dokaz}

\noindent
Záver: Pre ľubovoľnú teóriu, ktorá je idealizovaná (Henkinova a
úplná), vieme zostrojiť model.


\section{Rozšírenia teórie}

V nasledujúcej časti budeme definovať rozšírenia,
ktoré v istom zmysle ani nepomôžu ani
neuberú sile teórii.

\begin{definicia}[Rozšírenie jazyka]
%%% {{{
    Jazyk $L'$ je rozšírením jazyka $L$, ak každý
    špeciálny symbol jazyka $L$ (prípadne aj symbol ``='')
    je v jazyku $L'$ obsiahnutý s rovnakým významom a s rovnakou árnosťou.
%%% }}}
\end{definicia}

\begin{priklad}
    Nech jazyk $L'$ je jazyk s rovnosťou a špeciálnymi symbolmi
    ``$<$'' a ``$0$''.
    Jazyk $L$ má symboly ``$<$'' a ``$=$''.
    Jazyk $L'$ je potom rozšírením jazyka $L$.
\end{priklad}    

\begin{definicia}[Rozšírenie teórie]
%%% {{{
    Majme teóriu $T'$ s jazykom $L'$. Hovoríme, že $T'$ je
    rozšírením teórie $T$ s jazykom $L$, ak platí,
    že $L'$ je rozšírením jazyka $L$ a pre každý formulu $A$ jazyku $L$
    platí $T \provable A \then T' \provable A$. 
%%% }}}    
\end{definicia}

\begin{priklad}
    Uvažujme teóriu grúp -- je to teória s rovnosťou, ktorá
    používa špeciálny symbol ``$+$'' a existuje v nej neutrálny prvok
    ``0''.
    
    Axiómy tejto teórie sú:
    \begin{itemize}
    \item $\forall x,y,z: (x+y)+z = x+(y+z)$
    \item $(x+0) = x = (0+x)$
    \item $(x+(-x)) = 0 = ((-x)+x)$
    \end{itemize}

    Ak si vezmeme relačnú štruktúru 
    $\mathcal{M}=\langle \mathbb{N}_0,0,+,- \rangle$,
    táto realizuje jazyk teórie grúp.
    Nie je však jej modelom (lebo nie všetky axiómy sú splnené, menovite
    napríklad inverzné prvky).
    Pokiaľ ale vezmeme $\mathcal{M}=\langle \mathbb{Z},0,+,- \rangle$,
    táto množina je realizáciou a zároveň aj modelom teórie grúp.  

    Teraz si zoberme jazyk $L = \{0,1,+,-,*, ^{-1}\}$
    a uvažujme postupne nasledujúce teórie:
    \begin{enumerate}
    \item teória grúp
    \item teória okruhov
    \item teória oborov integrity
    \item teória telies
    \item teória polí
    \item teória zväzov
    \end{enumerate}
    Každá z týchto teórii je rozšírením tej predchádzajúcej.
\end{priklad}

\begin{poznamka}
    Je dôležité si uvedomiť, že nie všetky axiómy pôvodnej teórie $T$
    sa musia nachádzať aj v rozšírenej teórii $T'$ -- 
    stačí, ak sa dajú odvodiť v $T'$.
\end{poznamka}

\begin{poznamka}
 Ak teória $T'$ nad jazykom $L'$ je rozšírením teórie $T$ nad jazykom
 $L$ a vieme že $T'$ je bezosporná, potom aj $T$ je bezosporná.
\end{poznamka}
\begin{dokaz}
    Predpokladajme, že $T$ je sporná teória. Teda je v nej dokázateľná
    každá formula.
    Špeciálne, je dokázateľné $T \provable A$ a $T \provable \neg A$.
    Keďže $T'$ je rozšírenie, musí potom platiť
    $T' \provable A$ a $T' \provable \neg A$. To je ale spor.
    \\
\end{dokaz}

\begin{definicia}[Konzervatívne rozšírenie]
    Hovoríme, že teória $T'$ nad jazykom $L'$ je konzervatívnym
    rozšírením teórie $T$ nad jazykom $L$, ak pre každú formulu
    $A$ nad jazykom $L$ platí $T' \provable A \then T \provable A$.
    Inak povedané, konzervatívnym rozšírením nezískame žiadne nové 
    teorémy.
\end{definicia}

\begin{poznamka}
    Spojením posledných dvoch definícii dostávame, že pre
    konzervatívne rozšírenie $T',L'$ teórie $T,L$ platí
    \begin{equation*}
        \forall A \in L:\quad\quad T \provable A \iff T' \provable A
    \end{equation*}
\end{poznamka}

\begin{poznamka}
    Nech $T',L'$ je konzervatívne rozšírenie $T,L$. Potom platí (z
    predchádzajúcej poznámky)
    $T$ je bezosporná $\iff$ $T'$ je bezosporná.
\end{poznamka}

\begin{veta}[Henkinova] 
    K ľubovoľnej teórii $T$ môžeme zostrojiť Henkinovu
    teóriu $T_H$, ktorá je konzervatívnym rozšírením teórie $T$.
\end{veta}    

\begin{dokaz}
%%% {{{
    Teóriu $T$ rozšírime a priradíme k nej teóriu $T_H$ tak, že
    jednak rozšírime jazyk teórie a jednak pridáme axiómy.
    Nech teória $T$ má jazyk $L$.
    Budeme tvoriť rozšírený jazyk $L_H$ tak, že budeme pridávať konštanty:

    Pre každú formulu $A$ napísanú v jazyku $L$,
    ktorá má jedinú {\tt voľnú} premennú $x$ 
    vytvoríme konštanty $c_A, c_{\neg A}$
    a nasledovné axiomy: 
    \begin{itemize}
        \item[H1:]  $(\exists x) A \implies A_x[c_A]$
        \item[H2:]  $A_x{c_{\neg A}} \implies (\forall x) A$
    \end{itemize}

    Stačil by nám aj jeden typ axiómy, ako sa môžeme ľahko presvedčiť:
    \begin{itemize}
        \item $(\exists x) \neg A \implies \neg A_x[c_{\neg A}]$ -- H1
        \item $\neg \neg A_x[c_{\neg A}] \implies
                (\forall x) \neg \neg A$ -- obmena implikácie
        \item $A_x[c_{\neg A}] \implies (\forall x) A$ -- odstránenie
            dvojitej negácie
    \end{itemize}

    Konštanty $c_A, c_{\neg A}$ nazývame henkinovské konštanty prvého
    rádu. Tieto konštanty nám vytvoria množinu konštánt, označme ju
    $C_1$. Týmito konštantami rozšírime jazyk $L$.
    
    Teraz zoberme formulu $B$ na jazyku $L(C_1)$, ktorá obsahuje práve
    jednu voľnú premennú a aspoň jednu konštantu z $C_1$. K tejto
    formule priradíme podobne ako minule konštanty $c_B, c_{\neg B}$.
    Takto dosiahnuté konštanty budeme nazývať Henkinovské konštanty
    druhého rádu a ich množinu označíme ako $C_2$.
    
    Induktívne pokračujeme ďalej a podobne vytvárame množinu $C_3,
    C_4, \dots$
    Množinu $C_{n+1}$ vo všeobecnosti vytvoríme z množiny $C_n$, tak,
    že zoberieme formuly nad jazykom $L(C_n)$, ktoré obsahujú jednu
    voľnú premennú a aspoň jednu konštantu $n$-tého rádu.

    Označme si teraz
    \begin{equation*}
        L(C) = L \union \bigcup_{i=1}^{\infty} C_i
    \end{equation*}
    Teóriu $T_H$ bude tvoriť rozšírenie $L(C)$ jazyka $L$,
    axiómy budú axiómy $T$ ku ktorým pridáme všetky axiómy $H1$.
    Je ihneď zrejmé, že $T_H$ je rozšírenie teórie $T$.
    Potrebujeme ukázať, že $T_H$ je konzervatívne rozšírenie teórie $T$, t.j.
    že pre každú formulu $A$ jazyka $L$ platí 
    $T_H \provable A \Rightarrow T \provable A$.

    \medskip
    Poďme to dokázať:
    \begin{itemize}
    \item
        Nech platí, že v $T_H$ je dokázateľná formula $A$
        a nech $B_1, B_2, \dots, B_n$ sú všetky Henkinovské
        axiómy vyskytujúce sa v dôkaze $A$. Keďže dôkaz je konečný, týchto
        axióm je konečný počet. Ďalej môžeme uvažovať (ako sme už ukázali),
        že všetky axiómy $B_1, \ldots, B_n$ sú typu $H1$.
        Máme teda
        \begin{equation*}
            T, B_1, \dots ,B_n \provable A
        \end{equation*}

    \item
    Keďže axiómy $B_1,\ldots,B_n$ sú uzavreté formuly,
    môžeme použiť vetu o dedukcii a dostávame
    \begin{equation*}
        T \provable B_1 \implies B_2 \implies \ldots \implies B_n \implies A
    \end{equation*}

    \item
    Axiómy navyše môžeme poprehadzovať tak,
    aby $B_1$ obsahovala konštantu maximálneho rádu.
    Tvrdíme, že táto konštanta sa určite nenachádza v $A$ (pretože je
    nad pôvodným jazykom $L$)
    a ani v ostatných $B_i$
    (pretože všetky axiómy $B_i$ rovnakého rádu majú svoju
    ``hlavnú'' konštantu inú).
    Teda $B_1$ je tvaru $(\exists x) D \implies D_x[c_D]$,
    pričom $c_D$ nie je je použitá
    v $A,B_2,B_3, \ldots, B_n$. 
    \begin{equation*}
        T \provable \highlighta{((\exists x) D \implies D_x[c_D])} \implies 
            \highlightc{[B_2 \implies \ldots \implies B_n \implies A]}
    \end{equation*}

    \item
    Použijeme vetu o konštantách a dostávame
    \begin{equation*}
        T \provable \highlighta{((\exists x) D \implies
            D_x[\highlightb{w}])} \implies 
            \highlightc{[B_2 \implies \ldots \implies B_n \implies A]}
    \end{equation*}
    kde $w$ je nová premenná.
    \item
    Teraz môžeme použiť pravidlo zavedenia existenčného kvantifikátora --
    ak je dokázateľné $T \provable A \implies B$ 
    a $w$ nie je voľná v $B$,
    potom je dokázateľné aj $T\provable (\exists w) A \implies B$.
    Čiže
    \begin{equation*}
        T \provable \highlightb{(\exists w)} 
        \highlighta{((\exists x)D \implies D_x[w])}
            \implies 
            \highlightc{[B_2 \implies \ldots \implies B_n \implies A]}
    \end{equation*}

    \item
    Ďalej použijeme prenexnú operáciu $(Qx) (A\implies B) \iff (A\implies
    (Qx)B)$ (za predpokladu, že $x$ nie je voľná v $A$), aby sme $w$
    preniesli dovnútra. Výsledkom je
    \begin{equation*}
        T \provable \highlighta{((\exists x) D \implies 
                \highlightb{(\exists w)}D_x[w])}
            \implies \highlightc{
            [B_2 \implies \ldots \implies B_n \implies A]}
    \end{equation*}

    \item
    Z vety o variantoch ale vieme, že platí (pretože $(\exists w) D_x[w]$ je
    variant $(\exists x) D$) formula
    \begin{equation*}
        T \provable \highlighta{(\exists x) D \implies (\exists w)
        D_x[w]}
    \end{equation*}

    \item
    A teda pomocou pravidla modus ponens získame
    \begin{equation*}
        T \provable \highlightc{
            B_2 \implies B_3 \implies \ldots \implies B_n \implies A}
    \end{equation*}
    \end{itemize}

    Opakovaním postupu ďalších $n-1$ krát dostaneme $T \provable A$.
    Ukázali sme teda, že $T_H$ s jazykom $L(C)$
    je konzervatívne rozšírenie jazyka $L$.
    \\
%%% }}}
\end{dokaz}

\begin{veta}[Lindenbaum]
    Ak $T$ je bezosporná teória s jazykom $L$, potom existuje úplné
    rozšírenie $T'$ teórie $T$ s rovnakým jazykom $L$.
\end{veta}

\begin{dokaz}
%%% {{{
    Nech $\mathscr{S}$ je množina všetkých uzavretých formúl jazyka $L$.
    Ďalej definujeme množinu (podmnožín $\mathscr{S}$)
    $\mathcal{B} = \{ S \mid S \subseteq \mathscr{S}, T \union S
        \mbox{ je bezosporná}\}$.
    Množina $\mathcal{B}$ je čiastočne usporiadaná vzhľadom na
    inklúziu a (ako si neskôr ukážeme) má 
    takzvanú konečnú vlastnosť -- keď zoberieme ľubovoľnú podmnožinu
    $S \subseteq \mathscr{S}$, bude platiť
    \begin{equation*}
        S \in \mathcal{B} \iff \forall \mbox{ konečné } S' \in
        \mathcal{B}: S' \subseteq S
    \end{equation*}
    %
     Potrebujeme ukázať, že operácia inklúzie $\Psi, \Psi \subseteq
     \mathcal{B}\times\mathcal{B}$ je čiastočné usporiadanie, čiže je
    %
    \begin{itemize}
        \item reflexívna
            \begin{align*}
                &\forall S \in \mathcal{B}: (S,S) \in \Psi,
                \quad \mbox{t.j.} \\
                &S \subseteq S
            \end{align*}
        \item antisymetrická
            \begin{align*}
                &(S_1,S_2) \in \Psi \land (S_2,S_1) \in \Psi
                    \then S_1 = S_2, \quad \mbox{t.j.} \\
                &S_1 \subseteq S_2 \land S_2 \subseteq S_1 \then S_1 = S_2
            \end{align*}
        \item tranzitívna (dokážeme analogicky)
    \end{itemize}
    %
    $\mathcal{B}$ je teda čiastočne usporiadaná inklúziou $\Psi$.
    Navyše
    \begin{equation*}
        \emptyset \in \mathcal{B}, \mbox{ lebo } T \union \emptyset = T
    \end{equation*}
    Ak by totiž $\emptyset$ nebola v $\mathcal{B}$, potom by $T$ bola sporná.

    \noindent
    Teraz ukážeme, že množina $\mathcal{B}$ má konečnú vlastnosť, čiže
    platí: Nech $S \in \mathscr{S}$, potom
    \begin{equation*}
            S \in \mathcal{B} \iff \forall \textrm{
        konečnú podmnožinu }S' \subseteq S \textrm{ platí } T \union S' 
            \textrm{ je bezosporná (teda }S' \in \mathcal{B})
    \end{equation*}
   
   \begin{itemize}
       \item[$\Rightarrow:$]
            Nech $S \in \mathcal{B}$. Potom $T \union S$ je bezosporná.
            Ak teda $S' \subseteq S$ a  $S'$ je konečná, teória 
            $T \union S'$ bude tiež bezosporná $\then S' \in \mathcal{B}$

        \item[$\Leftarrow:$]
            Predpokladajme, že pre každú konečnú podmnožinu $S' \subseteq S$
            je $T \union S'$ je bezosporná.
            Chceme ukázať $S' \in \mathcal{B} \then S \in \mathcal{B}$.

            Tvrdenie dokážeme sporom.
            Predpokladajme, že $S \not \in B$.
            Potom $T \union S$ je sporná,
            čiže pre ľubovoľnú dokázateľnú formulu A je dokázateľné
            $\neg(A \implies A)$.

            Nech $A_1, A_2, \dots, A_n$ je dôkaz platnosti 
                $\provable \neg(A \implies A)$.
            Nech $B_1, B_2, \dots, B_m$ sú tie formuly,
            ktoré sa v tom dôkaze vyskytujú a patria do množiny $S$.
            Tvrdíme, že ich počet $m \ge 1$.
            Prečo? Inak by bola priamo $T$ sporná.
            Zoberme teraz ale konečnú množinu $S'=\{B_1, \dots, B_m\}$.
            Zjavne $T \union S'$ je sporná teória.
            To je ale spor s predpokladmi.
    \end{itemize}

    Teraz nachvíľu opustíme dôkaz Lindenbaumovej vety, aby sme
    sformulovali princíp maximality, ktorý sa nám bude hodiť.
    \\
\end{dokaz}

\begin{lema}[Princíp maximality]
    Každá neprázdna podmnožina potenčnej množiny
    $\mathcal{P}(\mathscr{S})$ s konečnou vlastnosťou 
    má maximálny prvok vzhľadom na inklúziu.
\end{lema}

\begin{dokaz}[Pokračovanie dôkazu Lindenbaumovej vety]
    Dostali sme sa do štádia, keď sme ukázali, že $\mathcal{B}$ má
    konečnú vlastnosť. Budeme teda pokračovať tým, že vytvoríme
    rozšírenie pôvodnej teórie nad rovnakým jazykom.

    Nech $S_0$ je maximálny prvok množiny $\mathcal{B}$ vzhľadom na
    inklúziu, podľa princípu maximality existuje.
    Položme rozšírenie $T' = T \union S_0$. Ukážeme, že $T'$ je úplná
    teória, t.j. že pre ľubovoľnú uzavretú formulu $A$ je dokázateľná
    v teórii $T'$ buď $A$ alebo $\neg A$.

    Uvažujme sporom: nech $T'$ nie je úplná teória. 
    Potom existuje uzavretá formula $A$ taká,
    že $T'\not \provable A$ a $T' \not \provable \neg A$.
    Lenže z tohoto je evidentné (keďže $T'$ je bezosporná), že aj
    $T'' = T' \union \{\neg A\}$ je bezosporná teória.\footnote{
        Odvoláme sa na dôsledok vety o dedukcii dokázaný niekedy v
        Úvode do matematickej logiky: ak $A$ je uzavretá,
        potom $T \provable A \iff T \union \{\neg A'\}$ je sporná teória.
    }
    To je ale v spore s tým, že $S_0$ je maximálny prvok $\mathcal{B}$.

    Dosiahli sme zúplnenie teórie. Pre teóriu $T$ sme získali $T'$,
    ktorá je bezosporná a je úplným rozšírením $T$ na rovnakom jazyku.
    \\
%%% }}}    
\end{dokaz}

Zhrňme si, čo sme doteraz dosiahli:
Keď máme teóriu $T$ nad jazykom $L$,
vieme ju rozšíriť na henkinovskú teóriu $T_H$ s jazykom $L(C)$.
V prípade, že $T$ je bezosporná, bude aj $T_H$ bezosporná.
Podľa Lindenbaumovej vety dokážem vytvoriť teóriu $T_H'$ s jazykom
$L(C)$, ktorá je úplná a má model $\mathcal{M}'$.
Našim cieľom je teraz získať model $\mathcal{M}$ pre pôvodnú teóriu.


\begin{definicia}[Redukcia]
    Majme teóriu $T$ nad jazykom $L$ a jej rozšírenie $T',L'$.
    Nech $\mathcal{M}'$ je realizácia jazyka $L'$.
    Redukovaním štruktúry $\mathcal{M}'$ na jazyk $L$
    získame realizáziu teórie $T$ v jazyku $L$.
    Formálne redukciu definujeme nasledovne:
    \begin{itemize}
    \item Univerzum $\mathcal{M}$ bude to isté univerzum ako univerzum
        $\mathcal{M}'$.

    \item $\mathcal{M}$ obsahuje iba také relácie a zobrazenia,
        ktoré realizujú špecálne symboly jazyka $L$ v realizácii
        $\mathcal{M}'$. Teda, ak $f$ je ľubovoľný $n$-ány funkčný
        symbol jazyka $L$ a $f_{\mathcal{M}'}$ je zobrazenie,
        ktoré realizuje $f$ v $\mathcal{M}'$,
        potom zostáva realizáciou $f$ v $\mathcal{M}$.
        Podobne to bude aj s $n$-árnym predikátom $P$.
    \end{itemize}
    
    Budeme hovoriť, že $\mathcal{M}$ vzniklo redukciou $\mathcal{M}'$ 
    na jazyk $L$ a zapisovať 
    $\mathcal{M} = \mathcal{M}' \triangle L$.\footnote{
        Na prednáške bolo miesto $\triangle$ použíté niečo ako $\land$,
        len ľavá nožička bola dlhšia ako pravá.}
    Alebo opačne, $\mathcal{M}'$ je expanzia realizácie $\mathcal{M}$.
\end{definicia}

\begin{poznamka}
    Nech $\mathcal{M}$ je redukcia $\mathcal{M}'$ na jazyk $L$
    a nech $A$ je formula jazyka $L$.
    Potom platí $\mathcal{M} \models A \iff \mathcal{M}' \models A$.
\end{poznamka}
\begin{dokaz}
    Máme teda tvrdenie $\mathcal{M} \models A$, t.j. pre ľubovoľné
    ohodnotenie v relačnej štruktúre $\mathcal{M}$ je formula $A$
    pravdivá. Podobne je to aj s $\mathcal{M}' \models A$.
    Všimnime si, že univerzum je to isté. Nuž ale potom aj ohodnotenia
    premenných musia byť rovnaké. Čo sa týka realizácie
    $\mathcal{M}, \mathcal{M}'$, tak hodnota formuly $A$ závisí
    od realizácie špeciálnych symbolov jazyka $L$. Tie sa ale podľa
    definície realizujú rovnako. Záverom teda je, že pre formulu
    $A$ su realizácie $\mathcal{M}, \mathcal{M'}$ rovnaké a teda platí
    tvrdenie poznámky.
    \\
\end{dokaz}

\begin{veta}
    Nech $T'$ je rozšírenie teórie $T$ s jazykom $L$.
    Ak $\mathcal{M}'$ je model $T'$ a 
    $\mathcal{M} =\mathcal{M}'\triangle L$, tak $\mathcal{M}$ je model $T$.
\end{veta}
\begin{dokaz}
    Nech $A \in T$. Teda, $T \provable A$. Potom (pretože $T'$ je
    rozšírenie $T$) platí $T' \provable A$.
    Čiže z vety o korektnosti dostávame $\mathcal{M}' \models A$. Ale
    $A$ je nad jazykom $L$ a platí $\mathcal{M}= \mathcal{M'}
    \triangle L$. Teda, z predchádzajúcej poznámky vyplýva
    $\mathcal{M} \models A$.
    \\
\end{dokaz}

\noindent
Rozpamätajme sa na náš pôvodný zámer, načo sme sa vlastne zaoberali
rozšíreniami -- bola to práve G\"odelova veta.

\begin{dokaz}[Dokončenie 2. variantu G\"odelovej vety]
    Naším zámerom bolo zostrojiť model pre teóriu, ktorá je bezosporná.
    Postupujme teda nasledovne: K teórii $T$ existuje henkinovské
    rozšírenie $T_H$, ktoré sa dá (konzervatívne) rozšíriť podľa
    Lindenbaumovej vety na úplnú teóriu $T_H'$. O tejto teórii vieme, že
    je bezosporná a teda vieme podľa vety o kanonickej štruktúre
    zostrojiť jej model $\mathcal{M}'$. Nuž a teraz nám stačí redukovať
    $\mathcal{M}'$ na pôvodný jazyk $L$ a dostávame
    $\mathcal{M}= \mathcal{M}' \triangle L$ -- model pôvodnej teórie $T$.
    Naopak, ak k teórii $T$ existoval model, zrejme sa nemôže stať, že by
    bola sporná. Tým sme dokázali 2. variant G\"odelovej vety.
    \\
\end{dokaz}

\input{tex/04kompaktnost.tex}

\chapter{Dokazovanie formúl -- Metóda rezolvent}
\section{Metóda rezolvent}

Celú túto kapitolu sa budeme venovať sémantike formúl logiky prvého
rádu.

Keď uvažujeme výrokovú logiku, interpretujeme formulu $A$ funkciou
$\bar{v}$ -- valuáciou.
Formulu sme nazvali tautológiou, ak $\bar{v}(A)$ je pre každú
interpretáciu pravda -- treba preskúmať $2^n$ interpretácii.
V predikátovej logike je to ešte horšie.

Zopakujme si, čo už vieme. V predikátovej logike máme funkčné a
predikátové symboly. Univerzum budem označovať netradične ako $D$.
Znakom $M$ budeme totiž označovať jadro(maticu) formuly v prenexnom
tvare.

Funkcia $f$ bude realizovaná ako $n$-árna funkcia $f(x_1,\dots,x_n): D^n \rightarrow D$.
Predikát $P$ bude realizovaný ako $n$-árna funckia
$P(x_1,\dots,x_n): D^n \rightarrow \{0,1\}$.

Vo formule rozlišujeme voľné a viazané premenné.
Ak má formula voľné premenné, nevieme určiť jej pravdivostnú hodnotu,
iba ak za všetky voľné premenné dosadíme konštanty.

Ďalej, ku každej formule $A$ viem zostrojiť takú,
ktorá je v prefixovom (prenexnom) tvare -- t.j. je 
prefix + jadro: $(Q_1 x_+)\cdots(Q_n x_n)M$.

Na upravenie formuly do tohoto tvaru používame prenexné operácie.
Skutočnosť, že premenná $x$ má voľný výskyt vo formule $A$ budeme
značiť ako $A(x)$, v opačnom prípade budeme písať jednoducho $A$.

\medskip
Prenexné operácie:
\begin{align*}
    (Qx)A(x) \lor B &\equiv (Qx)(A(x) \lor B) \tag{1a}\\
    (Qx)A(x) \land B &\equiv (Qx)(A(x) \land B) \tag{1b}\\
    \neg (\forall x) A(x) &\equiv (\exists x) \neg A(x) \tag{2a}\\
    \neg (\exists x) A(x) &\equiv (\forall x) \neg A(x) \tag{2b}\\
    (\forall x) A(x) \land (\forall x) B(x) &\equiv 
        (\forall x) (A(x) \land B(x)) \tag{3a}\\
    (\forall x) A(x) \lor (\forall x) B(x) &\equiv
        (\forall x) (A(x) \lor B(x)) \tag{3b}
\end{align*}
\begin{poznamka}
    S predchádzajúcimi operáciami 3a,3b treba pracovať pozorne. Im veľmi
    podobné úpravy totiž neplatia:

    \begin{align*}
      (\forall x) A(x) \lor (\forall x) B(x)  & \not \equiv
        (\forall x) (A(x) \lor B(x)) \tag{x1}\\
      (\exists x) A(x) \land (\exists x) B(x) & \not \equiv
        (\exists x) (A(x) \land B(x)) \tag{x2}
    \end{align*}
    Zoberme si napríklad $D=\{1,2\}$. Ak položíme
    $A(1)=1, A(2)=0, B(1)=0, B(1)=1$, dostaneme, že pravá strana x1 bude
    platiť, zatiaľ čo ľavá strana nie. Pre x2 naopak.
\end{poznamka}

Pretože môžeme substituovať za premenné --
    $(\forall x) B(x) \equiv (\forall z) B(z)$, môžeme predchádzajúce
prenexné operácie zhrnúť do nasledujúcich všeobecných transformácií.
\begin{align*}
    (Q_1 x)A(x) \lor (Q_2 x)B(x) \equiv 
        (Q_1 x)(Q_2 z)(A(x) \lor B(z)) \tag{4a} \\
    (Q_3 x)A(x) \land (Q_4 x)B(x) \equiv
        (Q_3 x)(Q_4 z)(A(x) \land B(z)) \tag{4b}
\end{align*} 
kde v oboch prípadoch $z$ je premenná, ktorá sa nevyskytuje voľne v $A$
(a ani v pôvodnom $B(x)$).
Tento najvšeobecnejší tvar ale zbytočne pridáva premenné a preto je
vhodný iba v prípadoch, keď nefungujú operácie 1 až 3.
\subsection{Algoritmus na zostrojenie prenexného tvaru}

\begin{enumerate}
\item Odstránenie implikácií a ekvivalencií:
    \begin{align*}
        A \leftrightarrow B &\equiv (A \implies B) \land (B \implies A) \\
        A \implies B &\equiv \neg A \lor B
    \end{align*}

\item Odstránenie dvojitej negácie a presun negácie k formule.
    \begin{align*}
        \neg (\neg A) &\equiv A \\
        \neg (A \lor B) &\equiv (\neg A \land \neg B) \\
        \neg (A \land B) &\equiv (\neg A \lor \neg B) \\
        \neg (\forall x) A(x) &\equiv (\exists x) \neg A(x) \\
        \neg (\exists x) A(x) &\equiv (\forall x) \neg A(x)
    \end{align*}

\item  Premenovanie viazaných premenných, ak je to potrebné.

\item Použijeme zákony:
    \begin{align*}
        (Q x) A(x) \lor B &\equiv (Q x)(A(x) \lor B) \\
        (Q x)A(x) \land B &\equiv (Q x)(A(x) \land B) \\
        (\forall x) A(x) \land (\forall x)B(x) &\equiv 
            (\forall x)(A(x) \land B(x)) \\
        (\exists x)A(x) \lor (\exists x)B(x) &\equiv 
            (\exists x) (A(x)\lor B(x)) \\
        (Q_1 x) A(x) \lor (Q_2 x)B(x) &\equiv
            (Q_1 x)(Q_2 z) (A(x)\lor B(z)) \mbox{ kde $z$ je nová
            premenná} \\
        (Q_3 x) A(x) \land (Q_4 x)(B(X) &\equiv 
            (Q_3 x)(Q_4 z)(A(x) \land (B(z)) \mbox{ kde $z$ je nová
            premenná}
    \end{align*}
\end{enumerate}

\begin{priklad}
    \begin{align*}
        (\forall x)(\forall y) \left[ (\exists z) \left(P(x,z) \land
            P(y,z)\right) \implies (\exists u) Q(x,y,u) \right] &\equiv \\
        (\forall x)(\forall y)\left[\neg (\exists z)(P(x,z)\land 
            P(y,z))\lor(\exists u)Q(x,y,u)\right] &\equiv \\
        (\forall x)(\forall y)\left[(\forall z)(\neg P(x,z)\lor 
            \neg P(y,z))\lor(\exists u)Q(x,y,u)\right] &\equiv \\
        (\forall x)(\forall y)(\forall z)(\exists u) \left[ \neg P(x,z) \lor 
            \neg P(y,z) \lor Q(x,y,u) \right] &
    \end{align*}
\end{priklad}

\section{Herbrandova veta - história}
\startFIXME
\par Leibniz (1646 -- 1716), Peáno (1900), Hilbertova škola (1920), Herbrand
(1930). Fundamentálny výsledok, ktorý dal odpoveď na otázku, či existuje
všeobecná procedúra, ktorá vie zistiť, či je daná formula tautológia, dali
Church a Turing (1936, nezávisle od seba) -- ukázali, že v predikátovej logike
nie je problém rozhodnuteľnosti riešiteľný.
\par Herbrandova metóda je založená na procedúre vyvrátenia. Ide o nájdenie
interpretácie, pri ktorej daná formula nie je splnená. Ak formula nie je
tautológia, v konečnom počte krokov sa procedúra zastaví, pokiaľ ale nie je,
procedúra pokračuje vo výpočte.

\par Gilmore (1960), Davis a Putnem (1960) -- navrhovali procedúry, ktoré by
boli schopné overovať tautologickosť formúl; fungovali ale iba pre jednoduchšie
formuly.

\par V rokoch 1965 -- 1968 Robinson zaviedol pojem \emph{resolventa} (v podstate
iným spôsobom zapísane pravidlo \emph{modus ponens}.
\stopFIXME

\input{tex/07standard.tex}
\section{Herbrandovské univerzum}

\begin{poznamka}
    V nasledujúcom texte budeme používať a medzi sebou zamieňať
    nasledujúce výrazy s rovnakým významom:
    ``nie je splniteľná'', 
    ``je sporná'' a
    ``je protirečivá''. Taktiež, občas budeme zamieňať medzi sebou
    aj ich negácie.
\end{poznamka}

\begin{definicia}[Herbrandovské univerzum množiny klauzúl]
    Nech $H_0$ je množina konštánt, ktoré sa vyskytujú v množine
    klauzúl $S$.
    Ak $S$ neobsahuje žiadnu konštantu, tak položíme $H_0=\{ a \}$,
    kde $a$ je nejaká konštanta.
    Ďalej definujme $H_{i+1}$ ako zjednotenie $H_{i}$ a množiny všetkých termov
    tvaru $f^{(n)}(t_1,\dots, t_n)$, kde $f^{(n)} \in S$ a
    $t_1, \dots, t_n \in H_i$.
    Množinu $H_i$ nazývame
    {\rm herbrandovské univerzum $i$-tej úrovne}.
    Herbrandovské univerzum množiny klauzúl definujeme ako zjednotenie
    cez všetky úrovne:
    \begin{equation*}
        H=\bigcup_{i=0}^{\infty} H_i
    \end{equation*}
\end{definicia}

\begin{priklad}
    Majme množinu klauzúl
    $S= \{ P(x),\ \neg P(x) \lor \neg P(f(x))\}$.
    Potom herbrandovské univerzá jednotlivých úrovní sú
    \begin{align*}
        H_0& = \{ a \} \\
        H_1& = \{ a,\ f(a) \} \\
        H_2& = \{ a,\ f(a),\ f(f(a)) \} \\
         \vdots\ \ &\\
        H_{\phantom{0}} &= \{ a,\ f(a),\ f(f(a),\ f(f(f(a))),\ \ldots \}
    \end{align*}
\end{priklad}

\begin{priklad}
    Nech $S=\{P(x) \lor R(x),\ R(z),\ T(y) \lor \neg W(y) \}$, teda
    množina $S$ neobsahuje žiadnu konštantu.
    Preto kladieme $H_0 = \{ a \}$. Dostávame
    $H_0=H_1=\dots=H=\{a\}$.
\end{priklad}

\begin{priklad}
    Uvažujme množinu klauzúl $S=\{P(f(x),a,g(y), b)\}$. Potom
    %% FIXME: do nasledujuceho alignu som nevedel narvat split
    %% ak to niekto vie, nech to fixne
    \begin{align*}
        H_0 =& \{ a,\ b\} \\
        H_1 =& \{ a,\ b,\ f(a),\ g(a),\ f(b),\ g(b) \} \\
        H_2 =& \{ a,\ b,\ f(a),\ g(a),\ f(b),\ g(b),\
                f(f(a)),\ f(f(b)),\\
              &\quad f(g(b)),\ f(g(b)),\
                g(f(a)),\ g(f(b)),\ g(g(a)),\ g(g(b)) \}
    \end{align*}    
\end{priklad}

\begin{definicia}[výraz]
    Pod pojmom výraz budeme chápať 
    term, množinu termov,
    klauzulu, množinu klauzúl,
    atóm, množinu atómov,
    literál, množinu literálov.
\end{definicia}

\begin{definicia}[podvýraz]
    Podvýraz výrazu $F$ je ľubovoľný výraz, ktorý sa vyskytuje v $F$.
\end{definicia}

\begin{definicia}[základný výraz]
    Ľubovoľný výraz, ktorý neobsahuje premenné, sa nazýva základný výraz
    (základný term, základný atóm, základná klauzula, základný literál, ...).
\end{definicia}

\begin{definicia}[základná inštancia]
    Základnou inštanciou klauzuly $C$ z množiny klauzúl $S$ je klauzula,
    ktorú dostaneme zámenou všetkých premenných
    za prvky Herbrandovského univerza.
\end{definicia}

\begin{priklad}
    Majme množinu $S = \{P(x),\ Q(f(x)) \lor R(y) \}$ a majme klauzulu
    $C: P(x)$. 
    Herbrandovské univerzum $H$ je
    $H = \{ a,\ f(a),\ f(f(a)),\ f(f(f(a))),\ \dots \}$.
    Základné inštancie pre $C$ sú $P(a), P(f(a)), P(f(f(a))), \dots$.
\end{priklad}

\begin{definicia}[Herbrandovská báza]
    Nech $S$ je množina formúl. Potom množina atómov
    tvaru $P^{(n)}(t_1, \ldots, t_n)$ pre všetky $n$-árne predikáty,
    ktoré sa vyskytujú v $S$ a všetky termy $t_i$ z Herbrandovského univerza
    nazývame Herbrandovskou bázou $S$.
    Sú to atómy takého tvaru, že sa v nich nevyskytuje žiadna
    premenná.
\end{definicia}

Položme si otázku, čo znamená interpretovať množinu klauzúl $S$ na
Herbrandovskom univerze $H$. Musíme poznať hodnoty konštánt,
interpretáciu funkčných a predikátových symbolov. Budeme uvažovať
špeciálnu interpretáciu, takzvanú $H$-interpretáciu, pri ktorej
nebudeme interpretovať predikátové symboly (necháme si to ako keby na
neskôr).

\begin{definicia}[H-interpretácia]
    Nech $S$ je množina klauzúl. Ďalej nech $H$ je herbrandovské
    univerzum pre množinu klauzúl $S$ a $I$ je interpretácia v množine
    klauzúl $S$ nad $H$.
    Hovoríme, že interpretácia $I$ je herbrandovská interpretácia 
    (alebo tiež H-interpretácia), ak platí: 

    \begin{enumerate}
	\item Interpretácia $I$ zobrazuje všetky konštanty na samé
            seba, t.j. konštante $a \in S$ priradí tú istú konštantu $a \in H$.

	\item Nech $f^{(n)}$ je $n$-árny funkčný symbol a
            $h_1, \dots, h_n$ sú prvky herbrandovského univerza $H$.
            Potom funkciu $f$ budeme v $I$ realizovať ako funkciu,
            ktorá zobrazuje $(h_1, \dots, h_n) \in H^n$ na element
            $f^{(n)}(h_1,\dots,h_n) \in H$.
    \end{enumerate}
\end{definicia}

\begin{poznamka}
    Ako sme už spomínali, nekladieme žiadne obmedzenia 
    na interpretáciu predikátov.

    Uvažujme ako príklad množinu $A = \{ A_1, A_2, \dots \}$.
    Nech je to herbrandovská báza pre množinu klauzúl $S$.
    Herbrandovskú interpretáciu určíme tak, že $I$ zadáme ako
    $I=\{ m_1, m_2, \ldots \}$, kde
    $m_i$ bude buď $A_j$ (ak $A_j$ je pravdivé) alebo
    $\neg A_j$ (ak $A_j$ je nepravdivé).
\end{poznamka}

\begin{priklad}
    Majme množinu klauzúl $S=\{ P(x) \lor Q(x),\ R(f(y)) \}$.
    Herbrandovské univerzum je
    $H=\{a,\ f(a),\ f(f(a)),\ \ldots \}$.
    V $S$ sa vyskytujú unárne predikáty $P,Q,R$.
    Herbrandovská báza je potom
    \begin{equation*}
        A: \{ P(a),\ Q(a),\ R(a),\ 
            P(f(a)),\ Q(f(a)),\ R(f(a)),\ \dots\}.\footnote{
            Všimnime si, že v danej množine je aj $R(a)$, hoci
            by sme možno očakávali, že to musí začínať $R(f(a))$
            }
    \end{equation*}
    Môžeme mať nasledovné interpretácie:
    \begin{equation*}
        I_1: \{ P(a),\ Q(a),\ R(a),\ 
        P(f(a)),\ Q(f(a)),\ R(f(a)),\ \dots \}
    \end{equation*}
    teda, predikáty sú vždy pravdivé. Alebo
    \begin{equation*}
        I_2: \{ \neg P(a),\ \neg Q(a),\ \neg R(a),\ 
            \neg P(f(a)),\ \neg Q(f(a)),\ \neg R(f(a)),\ \dots \}
    \end{equation*}
    čiže predikáty nie sú nikdy pravdivé.
    Ďalšia možná interpretácia je
    \begin{equation*}
        I_3: \{ P(a),\ Q(a),\ \neg R(a),\ 
            P(f(a)),\ Q(f(a)),\ \neg R(f(a)),\ \dots \}
    \end{equation*}
    V zásade pre každú možnú kombináciu si vieme vytvoriť
    interpretáciu.
\end{priklad}


\medskip
Ďalšou úlohou bude k ľubovoľnej interpretácii nad ľubovoľným univerzom
priradiť Herbrandovskú interpretáciu a zaviesť príslušné tvrdenia a definície.

\begin{priklad}
    \label{prikl:interpretacia}
    Majme množinu klauzúl $S = \{ P(x),\ Q(y, f(y,a)) \}$.
    Nech je oblasť interpretácie $D = \{1,2\}$ a interpretujme
    konštanty, funkčné symboly a predikátové symboly podľa tabuľky
    \ref{tab:priklad-interpretacia}

    \begin{table}[h]
        %%% {{{
        \centering
        \begin{tabular}{|r||c|c|c|c|c|}
            \hline
            symbol & a & f(1,1) & f(1,2) & f(2,1) & f(2,2) \\
            \hline
            hodnota & 2 & 1 & 2 & 2 & 1 \\
            \hline
        \end{tabular}
        
        \medskip
        \begin{tabular}{|r||c|c|c|c|c|c|}
            \hline
            predikát & P(1) & P(2) & Q(1,1) & Q(1,2) & Q(2,1) & Q(2,2) \\
            \hline
            hodnota & true & false & false & true & false & true \\
            \hline
        \end{tabular}
        \caption{Interpretácia $I$ z príkladu
          \ref{prikl:interpretacia}}
        \label{tab:priklad-interpretacia}
        %%% }}}
    \end{table}

    Ideme určiť H-interpretáciu $I^*$, ktorá akýmsi spôsobom bude
    prislúchať našej interpretácii $I$. Najsôr si určíme bázu:
    \begin{equation*}
        A=\{ P(a),\ Q(a,a),\ P(f(a,a)),\ Q(a,f(a,a)),
            Q(f(a,a), a),\ \ldots \}
    \end{equation*}

    Hodnoty pre príslušné predikáty $I^*$ určíme pomocou zadaných tabuliek
    pre $I$:

    \begin{align*}
        P(a) &= P(2) = false \\
        Q(a,a) &= Q(2,2) = true\\
        P(f(a,a)) &= P(f(2,2)) = P(1) =true \\
        Q(f(a,a),a) &= Q(f(2,2),2) = Q(1,2) =true \\
        Q(a,f(a,a)) &= Q(1,f(2,2)) = Q(2,1) =false \\
        Q(f(a,a),f(a,a)) &= Q(f(2,2), f(2,2)) = Q(1,1) =false \\
    \end{align*}

    Interpretácia $I^*$ je teda
    \begin{equation*}
        I^* = \{ \neg P(a),\ Q(a,a),\ P(f(a,a)),\ 
            \neg Q(a,f(a,a)),\ Q(f(a,a),a),\ \ldots \}
    \end{equation*}
\end{priklad}

\begin{priklad}
    \label{prikl:interpretacia2}
    Môže nastať aj iný prípad -- majme množinu klauzúl $S$, ktorá neobsahuje
    konštantu: $S=\{P(x),\ Q(y,f(y,z)) \}$.
    Máme danú interpretáciu $I$ s oblasťou $D=\{1,2\}$ popísanú
    tabuľkou \ref{tab:prikl-interpretacia2}

    \begin{table}[h]
        %%% {{{
        \centering
        \begin{tabular}{|r||c|c|c|c|}
            \hline
            symbol & f(1,1) & f(1,2) & f(2,1) & f(2,2) \\
            \hline
            hodnota  & 1 & 2 & 2 & 1 \\
            \hline
        \end{tabular}
        
        \medskip
        \begin{tabular}{|r||c|c|c|c|c|c|}
            \hline
            predikát & P(1) & P(2) & Q(1,1) & Q(1,2) & Q(2,1) & Q(2,2) \\
            \hline
            hodnota & true & false & false & true & false & false \\
            \hline
        \end{tabular}
        \caption{Interpretácia $I$ z príkladu
          \ref{prikl:interpretacia2}}
        \label{tab:prikl-interpretacia2}
        %%% }}}
    \end{table}
   
    Tejto interpretácii budú zodpovedať dve H-interpretácie
    $I_1^*$ a $I_2^*$, pričom v prvej bude $a$ interpretované ako $1$ 
    a v druhej ako $2$.

    \begin{align*}
        I_1^* &= \{ P(a),\ \neg Q(a,a),\ P(f(a,a)),\ \neg Q(a,f(a,a)),\
            \neg Q(f(a,a),a),\ \dots \} \\
        I_2^* &= \{ \neg P(a),\ \neg Q(a,a),\ P(f(a,a)),\ \neg Q(a,f(a,a)),\
            Q(f(a,a),a),\ \dots \}
    \end{align*}
\end{priklad}

Teraz si formálne zavedieme, čo to znamená priradiť nejakej interpretácii $I$
zodpovedajúcu H-interpretáciu $I^*$:

\begin{definicia}[zodpovedajúca H-interpretácia]
    Majme interpretáciu $I$ pre množinu klauzúl $S$ na oblasti
    interpretácií $D$. Interpretácii $I$ priradíme zodpovedajúcu
    H-interpretáciu $I^*$ nasledovne:
    Nech $h_1, h_2, \ldots, h_n$ sú prvky Herbrandovského univerza
    $H$.
    Nech každé $h_i$ sa zobrazí v interpretácii $I$ na prvok
    $d_i \in D$.
    Zoberme predikát $P^{(n)}$ z množiny klauzúl $S$.
    Definujeme splniteľnosť (a pravdivosť) predikátu $P$ v $I^*$
    nasledovne:
    $P^{(n)}(h_1,\dots,h_n)$ je pravdivý v $I^*$ práve vtedy keď je
    $P^{(n)}(d_1,\dots,d_n)$ pravdivý v $I$.
\end{definicia}

\begin{lema}
    \label{lema:h-interpretacia}
    Majme interpretáciu $I$ na oblasti $D$. Nech táto interpretácia
    vyhovuje množine klauzúl $S$. Potom ľubovoľná H-interpretácia $I^*$,
    ktorá je priradená (zodpovedá) $I$, taktiež vyhovuje množine klauzúl $S$.
\end{lema}

\begin{dokaz}
    Majme množinu klauzúl $S=\{C^{(1)}, C^{(2)}, \ldots , C^{(n)}\}$.
    Nech klauzula $C^{(i)}$ je nasledujúca disjunkcia literálov
    $C^{(i)} = L^{(i)}_1 \lor L^{(i)}_2 \lor \ldots \lor
        L^{(i)}_{r_i}$ pre $i=1,\ldots,n$.
    Ľubovoľný literál $L^{(i)}_j$ je tvaru atomická formula alebo
    jej negácia. Predpokladajme že interpretácia $I$ na univeze $D$
    vyhovuje množine klauzúl $S$. Potom musí byť pravdivý aspoň jeden
    literál v každej klauzule $C^{(i)}$.
    Lenže pre tieto literály
    $L^{(i)}_j$ platí
    $L^{(i)}_j = P^{(n)}(d_1,\dots,d_n)$ je pravdivé v $I$ a z toho
    dostávame $P^{(n)}(h_1, \dots, h_n)$ je pravdivé v $I^*$.
    Teda, ak $I$ vyhovuje množine klauzúl $S$, bude jej vyhovovať aj
    $I^*$.
\end{dokaz}

\begin{veta}
    Množina klauzúl $S$ nie je splniteľná práve vtedy,
    keď $S$ je nepravdivá pri všetkých H-interpretáciach $S$.
\end{veta}


\begin{dokaz}
    \noindent
    \begin{itemize}
    \item[$\Rightarrow:$] Nech množina klauzúl $S$ nie je splniteľná.
        Potom je nepravdivá pre ľubovoľnú interpretáciu na ľubovoľnej
        oblasti.
        Teda aj pre ľubovoľnú H-interpretáciu na H-univerze.

    \item[$\Leftarrow:$] Nech množina klauzúl $S$ je nepravdivá pre
        ľubovoľnú H-interpretáciu množiny klauzúl $S$.
        Pre spor predpokladajme, že existuje interpretácia $I$ 
        s oblasťou $D$ pre množinu klauzúl $S$, ktorá
        vyhovuje $S$.
        Uvažujme $I^*$, ktorá je priradená (zodpovedá) interpretácii
        $I$ pre množinu klauzúl $S$.
        Podľa lemy \ref{lema:h-interpretacia} ale vieme,
        že ak vyhovuje $I$, vyhovuje aj $I^*$, čo je spor.
    \end{itemize}
\end{dokaz}

Podarilo sa nám teda objaviť také univerzum (H-univerzum), pre ktoré
stačí vyšetriť splniteľnosť množiny klauzúl $S$ na všetkých
interpretáciach a budeme vedieť povedať (ne)splniteľnosť formuly na
ľubovoľnej interpretácii v ľubovoľnom univerze. Posunuli sme sa o krok
bližšie k rozhodovaniu platnosti $S$.

\begin{poznamka}
    V nasledujúcom texte budeme používať iba H-interpretácie.
    Preto sa dohodneme (na skrátenie a zjednodušenie zápisu), že
    ich budeme nazývať iba interpretácie a označovať ako $I$.
\end{poznamka}

\begin{poznamka}
    \noindent
    \begin{enumerate}
	\item Majme klauzulu (disjunkciu literálov) $C$
            a nech $C'$ je jej základná inštancia, t.j. 
            každá premenná bola nahradená prvkom H-univerza.
            $C'$ je splniteľná v interpretácii $I$
	    práve vtedy, keď existuje základný literál $L' \in I$,
            ktorý je splniteľný.
            Teda $C'$ je splniteľná $\iff$ $C' \intersect I \ne 0$.            

	\item Klauzula $C$ je splnená v interpretácii $I$ $\iff$ každá jej
	základná inštancia $C'$ je splnená v $I$.

	\item Klauzula $C$ je odmietnutá (vyvrátená) v $I$ $\iff$ existuje
	aspoň jedna taká základná inštancia $C'$, ktorá 
        nie je splniteľná (teda je vyvrátená) v $I$.

	\item Množina klauzúl $S$ nie je splniteľná $\iff$ 
        pre každú interpretáciu $I$ existuje aspoň jedna základná inštancia 
        $C'$ klauzuly $C\in S$, ktorá nie je splniteľná v $I$.
    \end{enumerate}
\end{poznamka}

\begin{priklad}
    Uvažujme klauzulu $C=\neg P(x) \lor Q(f(x))$ a
    interpretácie $I_1,I_2,I_3$ definované nasledovne:
    \begin{align*}
        I_1 &= \{ \neg P(a),\ \neg Q(a),\ \neg P(f(a)),\ 
            \neg Q(f(a)),\ \neg P(f(f(a))),\ \neg Q(f(f(a))),\ \ldots \} \\
        I_2 &= \{ P(a),\ Q(a),\ P(f(a)),\ Q(f(a)),\ P(f(f(a))),\ 
            Q(f(f(a))),\ \ldots \} \\
        I_3 &= \{ P(a),\ \neg Q(a),\ P(f(a)),\ \neg Q(f(a)),\ 
            P(f(f(a))),\ \neg Q(f(f(a))),\ \ldots \} \\
    \end{align*}
    Všímajme si klauzulu $C$ a jednotlivé interpretácie:
    $I_1$ vyhovuje $C$ (inak povedané $C$ je splnená) a zabezpečujú to
    $\neg P(a),\ \neg P(f(a)),\ \neg P(f(f(a))), \dots$.
    Podobne, $C$ je splnená v $I_2$ a zabezpečujú to
    $Q(a),\ Q(f(a)),\ Q(f(f(a))), \dots$. Naopak, $C$ nie je
    splniteľná v $I_3$.
\end{priklad}

\begin{priklad}
    Uvažujme množinu klauzúl $S=\{P(x),\ \neg P(x)\}$ a
    interpretácie $I_1 = \{ P(x) \}, I_2 = \{ \neg P(x) \}$.

    Množina $S$ nie je splnená ani jednou interpretáciou.
\end{priklad}


\section{Sémantické stromy}

Už sme povedali, že Herbrandovské interpretácie sú to pravé orechové,
čo chceme overovať. Ostáva nám ale vyriešiť problém, ako ich nejakým
spôsobom postupne preberať. A práve na to nám budú slúžiť sémantické
stromy.

\begin{definicia}[Kontrárna dvojica]
    Majme nejaký literál (elementárna formulu) $A$. Dvojicu
    $\langle A, \neg A \rangle$ budeme nazývať kontrárnou dvojicou.

    Ak klauzula obsahuje kontrárnu dvojicu, potom je vždy platná a
    teda je to tautológia.
\end{definicia}

\begin{definicia}[Sémantický strom]
    Nech $S$ je množina klauzúl a $A$ je Herbrandovská báza pre
    množinu $S$. Pod sémantickým stromom pre množinu klauzúl $S$
    budeme rozumieť zakorenený dolu rastúci strom
    v ktorom je každej hrane pripísaná množina atómov alebo negácií
    atómov (teda vlastne množina literálov) z Herbrandovskej bázy,
    pričom platí:
    \begin{enumerate}
        \item Z každého vrchola stromu vychádza konečný počet hrán.
            Ooznačme ich ich $l_1, l_2, \dots, l_n$.
            Nech $Q_i$ je konjunkcia všetkých literálov
            pripísaných hrane $l_i$, potom požadujeme aby 
            $Q_1 \lor Q_2 \lor \cdots \lor Q_n$ bola tautológia.

        \item Označme pre vrchov $v$ symbolom $I(v)$
            zjednotenie všetkých množín literálov pripísaných hranám cesty,
            ktorá vedie z koreňa do vrcholu $v$. Potom $I(v)$
            nesmie obsahovať kontrárne dvojice.
    \end{enumerate}
\end{definicia}

\begin{definicia}[Úplný sémantický strom]
    Nech $A=\{ A_1, A_2, \ldots A_n \}$ je Herbrandovská báza pre množinu
    klauzúl $S$.
    Hovoríme, že sémantický strom prislúchajúci $S$ je úplný,
    ak pre každý koncový vrchol(list) $v$ platí:
    $I(v)$ obsahuje $A_i$ alebo\footnote{Toto je exkluzívne alebo. Nemôže
    totiž platiť, že by obsahovala obe formuly} $\neg A_i$ pre každé $i$.
\end{definicia}

\begin{priklad}
    \label{prikl:semantic-tree-example1}
    Uvažujme množinu klauzúl $S = \{P,Q\}$. Potom úplný sémantický strom
    pre túto množinu môže vyzerať napríklad ako na obrázku
    \ref{fig:semantic-tree-example1}.

    \begin{figure}[h]
        \centering
        \includegraphics{img/09stromy/stromy.11.mps}
        \caption{Jeden z možných sémantických stromov pre príklad
                \ref{prikl:semantic-tree-example1}}
        \label{fig:semantic-tree-example1}
    \end{figure}
\end{priklad}

\begin{priklad}
    \label{prikl:semantic-tree-example2}
    K tej istej množine klauzúl môžeme mať viacero sémantických stromov,
    dokonca aj viacero úplných sémantických stromov.
    Uvažujme napríklad množinu klauzúl $S$, ktorá H-bázu $A: \{ P, Q, R \}$.
    Potom 2 úplné sémantické stromy pre túto množinu sú naznačené na
    obrázku
    \ref{fig:semantic-tree-example2}.

    \begin{figure}[h]
        \centering
        \subfigure{
            \includegraphics{img/09stromy/stromy.21.mps}
        }
        \subfigure{
            \includegraphics{img/09stromy/stromy.22.mps}
        }
        \caption{Sémantické stromy pre príklad
                \ref{prikl:semantic-tree-example2}}
        \label{fig:semantic-tree-example2}
    \end{figure}
\end{priklad}


\begin{priklad}
    \label{prikl:semantic-tree-example3}
    Veľmi malý strom môžeme dostať v prípade, ak $S=\{P(x),\ P(a)\}$. Vtedy
    je báza $A:\{P(a)\}$ (nemáme žiadne funkčné symboly a teda si vystačíme
    s konštantou) a strom vyzerá nasledovne (obr.
    \ref{fig:semantic-tree-example3}):

    \begin{figure}[h]
        \centering
        \includegraphics{img/09stromy/stromy.31.mps}
        \caption{Jeden z možných sémantických stromov pre príklad
                \ref{prikl:semantic-tree-example3}}
        \label{fig:semantic-tree-example3}
    \end{figure}
\end{priklad}

\begin{priklad}
    \label{prikl:semantic-tree-example4}
    Sémantický strom zďaleka nemusí byť konečný. Uvažujme napríklad množinu
    klauzúl $S=\{P(x), Q(f(x)) \}$.
    Vtedy je herbrandovská báza nekonečná: 
    $\{ P(a),\ Q(a),\ P(f(a)),\ Q(f(a)),\ \ldots \}$.
    A pretože báza je nekonečná, musí byť nekonečný aj strom, ak chceme aby
    bol úplný. Jeden taký strom je znázornený na obrázku
    \ref{fig:semantic-tree-example4}):

    \begin{figure}[h]
        \centering
        \includegraphics[scale=0.9]{img/09stromy/stromy.41.mps}
        \caption{Nekonečný sémantický strom pre príklad
                \ref{prikl:semantic-tree-example4}}
        \label{fig:semantic-tree-example4}
    \end{figure}
\end{priklad}

\begin{poznamka}
    Radi by sme upozornili na istú drobnosť -- definícia úplného stromu má
    isté problémy v príapde nekonečných stromov, pretože v nich nemáme
    koncové vrcholy. Preto definíciu upravíme nasledovne:
    Strom je úplný ak pre každú (nekonečnú) cestu platí, že obsahuje $A_i$
    pre všetky $i$.
\end{poznamka}

\begin{poznamka}
    \noindent
    \begin{itemize}
    \item Na úplný sémantický strom sa teda môžeme pozerať ako na organizované
        preberanie všetkých možných interpretácii.
    \item Navyše, na množinu $I(v)$ pre vrchol $v$ sa môžeme pozerať ako na
        čiastočnú interpretáciu (interpretuje len niektoré prvky bázy).
    \item Už vieme, že množina klauzúl $S$ je nesplniteľná $\iff$
        je nepravdivá pre všetky herbrandovské interpretácie. Otázka znie,
        ako sa to prejaví na sémantickom strome. Vieme, že každá klauzula
        je konečný objekt. Preto sa musí nájsť nejaká konečná čiastočná
        interpretácia zamietajúca túto klauzulu.

        V prípade, ak množina základných inštancií $I'(v)$ prvkov $I(v)$
        je odmietnutá, na tomto mieste môžeme prehľadávanie stromu ukončiť,
        pretože žiadna interpretácia nevyhovuje. Hovoríme tiež,
        že na tomto mieste môžeme strom \emph{odrezať}.
    \end{itemize}
\end{poznamka}

\begin{definicia}[Odmietajúci vrchol]
    Vrchol $v$ sémantického stromu pre množinu klauzúl $S$ sa
    nazýva odmietajúcim, ak  $I(v)$ odmieta niektorú základú inštanciu
    klauzuly z množiny $S$, no ľubovoľný vrchol $v'$  na ceste z koreňa 
    do vrcholu $v$
    neodmieta žiadnu základnú inštanciu klauzúl z $S$.
\end{definicia}

\begin{definicia}[Uzavretý sémantický strom]
    Hovoríme, že sémantický strom $T$ pre množinu klauzúl $S$
    je uzavretý, ak každá vetva $T$ končí odmietajúcim vrcholom.
\end{definicia}

\begin{definicia}[Akceptujúci vrchol]
    Vrchol $v$ sémantického stromu $T$ pre  množinu klauzúl $S$ 
    nazývame akceptujúcim, ak všetky nasledujúce
    vrcholy vrchola $v$ sú odmietajúce.
\end{definicia}

\begin{priklad}
    \label{prikl:semantic-tree-example5}
    Uvažujme množinu klauzúl
    $S= \{ P,\ Q \lor R,\ \neg P \lor \neg Q,\ \neg R \lor \neg P \}$.
    Herbrandovskú báza je $\{P, Q, R\}$. Na obrázku
    \ref{fig:semantic-tree-example5}.

    \begin{figure}[h]
        \centering
        \subfigure[Pôvodný strom]{
            \includegraphics{img/09stromy/stromy.51.mps}
        }
        \subfigure[Strom vzniknutý odrezaním odmietajúcich vrcholov]{
            \includegraphics{img/09stromy/stromy.52.mps}
        }
        \caption{Odrezanie sémantického stromu z príkladu
                \ref{prikl:semantic-tree-example5}}
        \label{fig:semantic-tree-example5}
    \end{figure}
\end{priklad}



\begin{priklad}
    \label{prikl:semantic-tree-example6}
    Uvažujme množinu klauzúl $S=\{ P(x),\ \neg P(x) \lor Q(f(x)),\ \neg Q(f(x)) \}$.\footnote{
        Na tomto mieste má Toman v poznámkach $P \lor Q(f(x))$ ale na
        prednáške to opravil}
    Herbrandovská báza pre množinu klauzúl $S$ je
    $\{ P(a),\ Q(a),\ P(f(a)),\ Q(f(a)),\ \ldots \}$. Orezaný sémantický
    strom pre túto množinu klauzúl možno nájsť na obrázku
    \ref{fig:semantic-tree-example6}.

    \begin{figure}[h]
        \centering
        \includegraphics{img/09stromy/stromy.61.mps}
        \caption{Odrezaný sémantický strom z príkladu
                \ref{prikl:semantic-tree-example6}}
        \label{fig:semantic-tree-example6}
    \end{figure}
\end{priklad}



\section{Herbrandova veta}
\startFIXME

\paragraph{Poznámka} Dirichletov princíp:
$$T_1:$$
$$X, Y -- \mbox{konečné množiny}$$
$$ |X| < |Y| $$
$$ f: X \rightarrow Y$$
$$ \exists y \in Y_1\qquad x_1,x_2 \in X, x_1 \neq x_2 \qquad f(x_1) = f(x_2) =
y$$
Sporom. Ak by také $x_1$, $x_2$ neexistovali, potom pre každé $x_1, x_2 \in X$,
$x_1 \neq x_2$ a $f(x_1) \neq f(x_2)$, potom $f$ je injektívne zobrazenie, z
čoho vyplýva, že $|X| \leq |Y|$.
$$T_2: $$
$$ f: X \rightarrow Y,\qquad X\mbox{-- nekonečná}, Y\mbox{ -- konečná} $$
$$T_2: \exists y\in Y, \qquad y=\{x|x\in X, f(x)=y\}$$

\paragraph{Dôkaz} Nech $|Y|=n$. Množinu $X$ môžeme napísať ako $X=\Cup_{y\in
Y}A_y$ takých, že ak $y_1\neq y_2$, potom $A_{y_1} \cap A_{y_2} = \emptyset$.

$$ |X| = \left|\Cup_{y\in Y} A_y \right|$$

Nech $k_0$ je maximálna mohutnosť $|A_y|$, $y\in Y$, množina $X$ ale nemá
konečnú mohutnosť -- spor.

\paragraph{Definícia} Strom chápeme ako usporiadanú dvojicu $(T,\leq)$ -- $T$ a
relácia, ktorá čiastočne usporadúva $T$:
\begin{enumerate}
	\item $T(n)=\{v\in T, v<n\}$ je dobre usporiadaná.
	\item $T$ má najmenší vrchol (koreň)
\end{enumerate}

Množina je dobre usporiadaná, ak každá jej neprázdna podmnožina má najmenší
prvok. $T$ je lineárne usporiadanie -- ľubovoľné dva prvky sú porovnateľné:
$a \neq b$, $a,b\in T$, pre $\{a, b\}$ musí platiť $a<b$ alebo $a>b$.

\par Ak mám prvky $u$ a $v$, prvok $v$ nasleduje bezprostredne po prvku $u$, ak
neexistuje $z$ také, že $u < z < v$.

\paragraph{Lema (K\"onig)} (je dôležitá na pochopenie dôkazu Herbrandovej vety)
Nech každý vrchol stromu s koreňom má konečné vetvenie (t.j. konečný stupeň
vetvenia) a strom $T$ je nekonečný, tak potom v ňom existuje nekonečne dlhá
vetva.

\paragraph{Dôkaz} $(T_1,\leq)$, $v \in T$. Označme $A_v = \{ u \in T | v <u
\}$ -- $A_v$ je množina tých vrcholov, ktoré ležia nad $v$. Majme vrchol $v$ a
nech $v_1, v_2, \ldots v_n$ sú bezprostrední nasledovníci $v$ (obr.
\ref{fig:nasledovnici}). $A_v = A_{v_1} \cup A_{v_2} \cup \ldots \cup A_{v_n}
\cup \{v_1, v_2, \ldots, v_n\}$. $A_v$ je nekonečná. Na základe uvedeného
tvrdenia aspoň jedna z množín $A_{v_i}$ musí byť nekonečná.

\begin{figure}[h]
	\centering\includegraphics{img/10/nasledovnici.1.mps}
	\caption{Nasledovníci $v$}
	\label{fig:nasledovnici}
\end{figure}

\par Vyberieme si $x_0$ ako koreň (najmenší vrchol stromu $T$). $A_{x_0} =
T \backslash \{x_0\}$, $A_{x_0}$ je nekonečná. Vezmime postupne prvky $\{x_0,
x_1, \ldots , x_n, x_\{n+1\}\}$. Ku každému prvku viem vybrať nasledovný prvok a
počet nasledovníkov je nekonečný. \todo{lepšia formulácia}

\paragraph{Veta (Herbrandova, variant 1)} Množina klauzúl $S$ nie je splniteľná
práve vtedy, keď ľubovoľnému úplnému sémantickému stromu pre množinu klauzúl
$S$ zodpovedá konečný uzavretý sémantický strom, t.j. ľubovoľná vetva úplného
stromu vedie do odmietajúceho vrchola.

\paragraph{Dôkaz} Predpokladajme, že množina klauzúl $S$ nie je splniteľná, nech
$T$ je úplný sémantický strom prislúchajúci $S$. $I_V$ je množina všetkých
literálov pripísaných vetve $V$ stromu $T$. Potom $I_V$ je interpretácia
množiny klauzúl $S$. Predpokladajme, že $S$ je nesplniteľná (teda nesplniteľná v
každej interpretácii). To znamená, že existuje nejaká základná inštancia klauzúl
$C'$ (pre klauzulu $C$), ktorá je v interpretácii $I_V$ odmietnutá. (základná
inštancia neobsahuje premenná, klauzuly sú konečné objekty). 
%Je odmietnutá vo vrchole, ktorý je konečne vzdialený od koreňa stromu.
\par Teraz potrebujeme zabezpečiť, aby strom bol konečný -- to musí, lebo ak by
existovala nekonečná vetva, bola by to interpretácia, ktorá by neodmietala
niektorú zo základných inštancií.

\par Obrátená implikácia. Ak množina klauzúl nie je splniteľná, existuje k
úplnému sémantickému stromu uzavretý konečný sémantický strom. Teraz
predpokladajme, že k úplnému sémantickému stromu pre množinu klauzúl
$S$ existuje konečný uzavretý sémantický strom (každá vetva stromu $T$ končí v
zamietajúcom vrchole), potom interpretácia $I_V$ odmieta každú základnú
inštanciu.


\paragraph{Veta (Herbrandova)}: Množina klauzúl $S$ nie je splniteľná $\iff$ keď
existuje konečná množina $S'$ základných inštancií klauzúl z $S$, ktorá nie je
splniteľná. \fixme{}

\paragraph{Dôkaz:} Predpokladajme, že množina klauzúl $S$ nie je splniteľná,
existuje $T'$ -- konečný uzavretý strom priradený ku stromu $T$.

\paragraph{Obrátene}: Predpoklajme, že existuije konečná $S'$ množina základných
inštancií klauzúl $S$ nie je splniteľná, $I'$, $S'$. Potom $S$ nie je
splniteľná, $I$ pre $S$. 

\par
Interpretácia $I$ obsahuje $I'$, teda keď $I'$ odmieta $S'$, potom $I$ odmieta
$S$. Každá klauzula je disjunkcia literálov. Ak existuje množina základných
inštancií, ktorá nie je splniteľná, potom celá množina nie je splniteľná.


\paragraph{Príklad} Majme množinu klauzúl $S=\{P(x), \neg P(f(a))\}$. Vezmime
množinu $S' = \{ P(f(a)), \neg P(f(a))\}$.

\paragraph{Príklad} Majme množinu klauzúl $S=\{\neg P(x) \lor Q(f(x),x),
P(g(b)), \neg Q(y,z) \}$. Táto množina nie je splniteľná -- opäť vezmime $$S' =
\{ \neg P(g(b)) \lor Q(f(g(b)), g(b)), P(g(b)), \neg Q(f(g(b)), g(b)) \}$$.

\paragraph{Príklad} Majme množinu klauzúl $S$:
$$
\begin{array}{l}
	\neg P(x,y,u) \lor \neg P(y,z,v) \lor \neg P(x,v,w) \lor P(u,z,w) \\
	\neg P(x,y,u) \lor \neg P(z,y,v) \lor \neg P(u,z,w) \lor P(x,v,w) \\
	P(g(x,y),x,y), P(x,h(x,y),y), P(x,y,f(x,y),\neg P(f(v),x,f(x)) \\
\end{array}
$$
\subparagraph{Návod} Zostrojte konečný sémantický strom.

\par Algoritmické metódy na zisťovanie nesplniteľnosti pracovali iba pre menšie
množiny klauzúl -- problém bol, že $S$ je množina disjunkcií, pokiaľ tam
dosadíme prvky z univerza, dostaneme konjunktívnu normálnu formu (prípadne
transformujeme na disjunktívnu normálnu formu a tú preverujeme). Ak máme $10$
formúl, potrebujeme preverovať $2^{10}$ konjunkcií -- zložitosť exponenciálne
rastie.

\paragraph{Príklad}
$$S=\{P(x), \neg P(a)\}, H_0 = \{a\}$$
$$S'=P(a) \land \neg P(a) = \square$$
$S$ nie je splniteľná.

\paragraph{Príklad}
$$S=\{P(a), \neg P(x) \lor Q(f(x)), \neg Q(f(a)) \}, H_0 = \{a\}$$
$ S_0'=P(a) \land (\neg P(a)\lor Q(f(a))) \land \neg Q(f(a)) = \\
P(a)\land \neg P(a) \land \neg Q(f(a)) \lor P(a) \land Q(f(a)) \land \neg Q(f(a)) = 
\\ \square \lor \square = \square$.

\paragraph{Pravidlá}
Nech $S$ je množina klauzúl.

\subparagraph{Pravidlo tautológie.} Vynecháme z $S$ všetky tautologické klauzuly.
Množina $S'$, ktorá nám zostane, nie je splniteľná práve vtedy, keď $S$ nie je
splniteľná.

\subparagraph{Pravidlo jednoliterálnych klauzúl.} Nech $L$ je nejaký literál.
Vynechajme z $S$ všetky klauzuly, ktoré obsahujú literál $L$. Nech $S'$ sú
klauzuly, ktoré nám zostanú po vynechaní. Môžu nastať dva prípady:

\begin{enumerate}
	\item $S' = \emptyset$, tak potom množina klauzúl $S$ je splniteľná --
	Stačí zobrať model, ktorý obsahuje $L$.
	\item $S' \neq \emptyset$. Vezmem si literál $\neg L$ a vynechám z
	množiny $S'$ všetky klauzuly, ktoré obsahujú $\neg L$, pričom dostanem
	množinu klauzúl $S''$. Ak sa $S'$ nachádza $\neg L$, po vynechaní
	namiesto nej zostane $\square$.
\end{enumerate}

$S$ nie je splniteľná $\iff$ $S''$ nie je splniteľná.

\subparagraph{Pravidlo čistých literálov} Literál $L$ základnej klauzuly z $S$
budeme nazývať \emph{čistým}, ak literál $\neg L$ sa nevyskytuje v žiadnej
základnej klauzule $S$. Vezmem si z $S$ literál $L$, ktorý je čistý, a
vynecháme z $S$ všetky základne klauzuly obsahujúce $L$. $S'$ je množina, ktorá
nám zostala. $S$ nie je splniteľná $\iff$ $S'$ nie je splniteľná.

\subparagraph{Pravidlo rezu} Ak množinu $S$ vieme vyjadriť v tvare $(A_1 \lor L)
\land \ldots \land (A_m \lor L) \land (B_1 \lor \neg L) \land \ldots \land (B_n
\lor \neg L) \land R$, pričom v $A_i$, $B_j$, sa $R$, $L$, $\neg L$, $i=1..m$,
$j=1..n$ už nevyskytujú.

$$S_1 = A_1 \land A_2 \land \ldots \land A_m \land R$$
$$S_2 = B_1 \land B_2 \land \ldots \land _n \land R$$

$S$ nie je splniteľná $\iff$ $S_1 \lor S_2$ nie je splniteteľná.

\paragraph{Dôkaz}
\begin{enumerate}
	\item Keď $S$ nie je splniteľné, tak ani $S_2$ nie je splniteľné --
	tautológie sú splnené pre ľubovoľné interpretácie. To, čo zostane,
	hovorí, či množina je alebo nie je splniteľná.
	\item Zostáva nám $S'$, ktoré vzniklo vynechaním $L$ z $S$. Ak
	$S'=\emptyset$, každá klauzula z $S$ obsahuje $L$ (je tvaru $L \lor
	\square$), stačí nám ohodnotenie, kedy $L$ je pravdivé, čo zabezpečí
	pravdivosť celého výroku. 
	\par
	Keď $S'$ nie je prázdna, vytvoríme $S''$ tak,
	že vynecháme všetky klauzuly obsahujúce $\neg L$. $S'$ nie je splniteľná
	práve vtedy, keď $S$ nie je splniteľná. Predpokladajme, že $S''$ nie je
	splniteľná a $S$ je splniteľná. Model $\mathcal{M}$ musí obsahovať $L$ a
	$S$ platí na $S''$.
	\todo{???}
	V konjunktívnej normálnej forme musia byť platné doplnky $\neg L \lor
	\square$. Potom $S''$ je splniteľná, čo je spor.

	\par Obrátene, nech $S$ nie je splniteľná. Z toho vyplýva, že $S''$ nie
	je splniteľná. Nech $S''$ je splniteľná. Potom existuje model
	$\mathcal{M}''$ pre $S''$. Ak tomuto modelu pridám $L$, potom
	$\mathcal{M}'' \cup L$ je model pre $S$, čo je spor.

	\item Pravidlo čistých literálov -- Máme čistý literál $L$, všetky
	klauzuly, ktoré ho obsahujú, vyhodíme z $S$ a dostaneme $S'$. $S$ nie je
	splniteľná práve vtedy, keď $S'$ nie je splniteľná. Predpokladajme, že
	$S'$ nie je splniteľná, teda pre žiadnu interpretáciu $I'$ neplatí. $S\
	\subseteq S$. Musí platiť, že $I' \subseteq I$.

	\par Obrátene -- predpokladajme, že $S$ nie je splniteľná. Potom aj $S'$
	nie je splniteľná. Predpokladajme, že $S'$ je splniteľná. Potom musí mať
	model $\mathcal{M}'$, ktorý jej vyhovuje. Keď si vytvorím model
	$\mathcal{M}=\mathcal{M}'\cup L$, potom $\mathcal{M}$ je modelom v $S$.

	\item Pravidlo rezu. Mám množinu $S$, popísaným spôsobom som zostrojil
	$S_1$ a $S_2$. Chceme ukázať, že $S$ nie je splniteľná $\iff$ $S_1 \lor
	S_2$. Predpokladajme, že $S$ je splniteľná. Potom $S_1 \lor S_2$ je
	splniteľná, teda aspoň jeden člen disjunkcie je splniteľný. Keď pridám
	modelu $S_1$ negované $B$, dostávam model pre $S$. Taktiež, keď je
	splnená $S_2$, má nejaký model $\mathcal{M_2}$. Keď k nemu pridám $B$,
	dostávam model pre $S$.

	\par Obrátene, predpokladajme, že $S_1$ alebo $S_2$ nie je splniteľná. Z
	toho vyplýva, že $S$ nie je splniteľná. Pre spor predpokladajme, že $S$
	je splniteľná. Pozrime sa, ako vyzerá $S$ -- všetky konjunktívne členy
	musia byť splnené a teda buď platí $\neg L$ alebo $L$. Vždy dospievam do
	sporu.
\end{enumerate}

\paragraph{Záver} Najvšeobecnejšie je pravidlo rezu, ktoré pokrýva aj všetky
ostatné prípady.

\paragraph{Príklad} Majmne množinu $S=(P \lor Q \lor \neg R) \land (P \lor \neg
Q) \land \neg P \land R \land U$. Ukážte, že množina klauzúl nie je splniteľná.
Použijeme pravidlo 2 s $\neg P$. Dostávam $Q\land\neg R) \land \neg  \land R
\land U$. Na túto množinu klauzúl použijem pravidlo 2 s $\neg Q$. Dostávam $\neg
R \land R \land U$, čo je $\square \land U \square$.




\paragraph{Príklad} (7) $S=(P \lor Q) \land \neg Q \land ( \neg P \lor Q \lor \neg
R)$. Ukážte, že množina klauzúl $S$ je splniteľná.
\par Pravidlo 2 s $\neg Q$: $(P \lor Q) \land ( \neg P \lor Q \lor \neg R)$ \\
$P \land (\neg P \lor \neg R)$ \\
$(P \land \neg P) \lor (P \land \neg R)$ (distributívnosť) \\
$P \land \neg L, I=\{P,\neg Q, \neg R\}$

\paragraph{Príklad} (8) $S=(P\lor \neg Q) \land (\neg P \lor Q) \land (Q \lor
\neg R) \land (\neg Q \lor \neg R)$. Zistite, či množina je alebo nie je
splniteľná.
\par vytvoríme si mnonžiny podľa pravidla 4: \\
$$(P\land \neg Q) \land (Q \lor \neg R)  \land (\neg Q \lor \neg R)$$
$$(\neg P \lor Q) \land (Q \lor \neg R) \land (\neg Q \lor \neg R)$$

\par
$$S_1 = \neg Q \land (Q \lor \neg R) \land (\neg Q \lor \neg R)$$
$$S_2 = Q \land (Q \lor \neg R) \land (\neg Q \lor \neg R)$$

($S_1$ a $S_2$ sú množiny rezu)
Na množiny môžem použiť pravidlo 2 -- s $Q$ a $\neg Q$. Dostávam $\neg R \lor
R$, čo je splniteľná klauzula.

\paragraph{Príklad} (9) $S=(P \lor Q) \land (P \lor \neg Q) \land (R \lor Q)
\land (R \lor \neg Q)$. $P$ je čistý literál; použijeme pravidlo 3:
$S': (R \lor Q) \land (R \lor \neg Q)$. \\
$\fbox{P,R}$ \\

\section{Metóda rezolvent pre výrokovú logiku}
\startFIXME

\paragraph{Definícia} Majme $C_1$ a $C_2$ -- ľubovoľné klauzuly, $L_1$ je
literál z $C_1$ a je kontrárny literálu $L_2$, ktorý sa vyskytuje v $C_2$. Ak z
$C_1$ vynecháme $L_1$ a z $C_2$ vynecháme $L_2$. Ak z tých častí, ktoré
zostávajú, zostrojíme disjunkciu $C_1 \lor C_2$, táto sa nazýva
\emph{rezolventa} $C_1$ a $C_2$.

\paragraph{Príklad}
$$
\begin{array}{ll}
C_1: P \lor R, & C_2: \neg P \lor Q \\
P \in C_1 & C_1': R \\
P \in C_2 & C_2': Q \\
\end{array}
$$

\paragraph{Príklad}
$$C_1: \neg P \lor Q\lor R$$
$$C_2: \neg Q \lor S$$
Kontrárna dvojica $\neg Q$, $Q$; rezolventa je teda:
$$C_1, C_2: \neg P \lor R \lor S$$

\paragraph{Príklad}

$$C_1: \neg P \lor Q$$
$$C_2: \neg P \lor R$$

\paragraph{Veta} Nech $C_1$ a $C_2$ sú klauzuly, $C$ je rezolventa klauzúl
$C_1$, $C_2$. Potom $C$ je logickým dôsledkom klauzúl $C_1$ a $C_2$
$$A_1, A_2, \ldots, A_n, B$$
$$
\begin{array}{ll}
A_1 \land A_2 \land \ldots \land A_n \implies B & \equiv 1	\\
\provable A_1 \land A_2 \land \ldots \land A_n \implies B & \\
A_1 \land A_2 \land \ldots \land A_n \land \neg B & \equiv 0	\\
\end{array}
$$
\paragraph{Dôkaz} $C_1$, $C_2$, $C$


$$
\begin{array}{lll}
C_1 &= &L \lor C_1' \\
C_2 &=& \neg L \lor C_2' \\
C   &=& C_1' \lor C_2'	\\
\end{array}
$$
$C_1$ a $C_2$ sú pravdivé a klauzula $C$ je logickým dôsledkom $C_1$ a $C_2$.

\stopFIXME

\begin{poznamka}
    Nech $C_1$, $C_2$ sú jednotkové klauzuly. Ak $C_1,C_2$ majú
    rezolventu, potom musia tvoriť kontrárnu dvojicu $L, \neg L$
    a rezolventou je prázdna klauzula $C \equiv \eps$.
\end{poznamka}

Našim cieľom bude zovše obecniť predchádzajúcu poznámku do
nasledujúceho variantu: Nech $S$ je množina klauzúl. Potom $S$ je
nesplniteľná práve vtedy keď z nej vieme nejakým spôsobom pomocou
rezolvent získať prázdnu klauzulu.
Teraz si to formálne rozpíšeme.

\begin{definicia}[Rezolvenčné pravidlo]
    Nech $S$ je množina klauzúl.
    Rezolvenčným odvodením klauzuly $C$ z množiny klauzúl $S$
    nazývame konečnú postupnosť klauzúl $C_1, C_2, \ldots, C_n \equiv C$ takú,
    že pre každé $i \in \{1,2,\dots,n\}$ platí:
    $C_i$ je buď z $S$ alebo $C_i$ je rezolventa niektorých klauzúl
    $C_j, C_k$ pre $j, k < i$.

    Ak $C$ je prázdna klauzula, potom takémuto odvodeniu hovoríme 
    zamietnutie odvodenia alebo tiež dôkaz nesplniteľnosti $S$.
\end{definicia}

\begin{definicia}
    Majme množinu klauzúl $S$ a klauzulu $C$.
    Hovoríme, že $C$ môžeme získať z $S$,
    ak existuje (rezolvenčné) odvodenie $C_1, \dots, C_m$ z množiny $S$ také, že
    $C_m \equiv C$.
\end{definicia}

\begin{priklad}
    Uvažujme množinu klauzúl $S=\{\neg P \lor Q,\ \neg Q,\ P\}$.
    Uvažujme nasledujúce rezolvenčné odvodenie
    \begin{itemize} 
	\item[1:] $ \neg P \lor Q $  -- z $S$
	\item[2:] $ \neg Q $ -- z $S$
	\item[3:] $ \neg P $ -- pravidlo rezolventy na 1,2
	\item[4:] $ P $ -- z $S$.
        \item[5:] $\eps$ -- pravidlo rezolventy na 3,4
    \end{itemize}
    Dostali sme prázdnu klauzulu a teda množina klauzúl $S$ nie je
    splniteľná.
\end{priklad}

\begin{priklad}
    \label{prikl:rezolvencne_odvodenie}
    Nech $S=\{P \lor Q,\ \neg P \lor Q,\ 
            P \lor \neg Q,\ \neg P \lor \neg Q\}$.
    Opäť vieme dokázať nesplniteľnosť množiny klauzúl $S$:

    \begin{itemize}
	\item[1:] $P \lor Q$ -- z $S$
	\item[2:] $\neg P \lor Q$ -- z $S$
	\item[3:] $P \lor \neg Q$ -- z $S$
	\item[4:] $\neg P \lor \neg Q$ -- z $S$
	\item[5:] $Q$ -- rezolventa 1,2
	\item[6:] $\neg Q$ -- rezolventa 3,4
        \item[7:] $\eps$ -- rezolventa 5,6
    \end{itemize}

    K tomuto odvodeniu môžeme navyše nakresliť aj jeho strom (obr.
    \ref{fig:strom_odvodenia}.

    \begin{figure}[h]
	\centering\includegraphics{img/11/odvodenie.1.mps}
	\caption{Strom rezolvenčného odvodenia z príkladu 
            \ref{prikl:rezolvencne_odvodenie}}
        \label{fig:strom_odvodenia}
    \end{figure}
\end{priklad}

\begin{poznamka}
    Pravidlom rezolventy smerujeme k tomu, že sa snažíme získať
    prázdnu klauzulu. Dôležité je, že je to silné pravidlo,
    t.j., že ak je množina $S$ nie je splniteľná, vieme prázdnu
    klauzulu naozaj odvodiť iba pomocou tohoto pravidla. Na začiatok
    začneme ekvivalenciou pravidla rezolventy a pravidla modus ponens.
\end{poznamka}

\begin{lema}
    Pravidlo modus ponens je ekvivalentné s pravidlom rezolventy.
\end{lema}

\begin{dokaz}
    Najskôr si ukážeme, že pomocou MP vieme dokázať pravidlo rezolventy:
    Chceme ukázať $A \implies B, \neg A \implies C \provable \neg B
    \implies C$.
    Postupujme nasledovne:
    \begin{itemize}
        \item[1.] $\provable (A \implies B) \implies (\neg B \implies
                \neg A)$ -- tvrdenie z výrokovej logiky
        \item[2.] $A \implies B \provable A \implies B$
        \item[3.] $A \implies B \provable \neg B \implies \neg A$ --
                MP 1,2
        \item[4.] $\provable (\neg B \implies A) \implies
            ((\neg A \implies C) \implies (\neg B \implies C))$ --
                tvrdenie z výrokovej logiky
        \item[5.] $A \implies B \provable
            (\neg A \implies C) \implies (\neg B \implies C)$ -- MP
            3,4
        \item[6.] $\neg A \implies C \implies \neg A \implies C$
        \item[7.] $A \implies B, \neg A \implies C \provable
            \neg B \implies C$ -- MP 5,6
    \end{itemize}
    Poznamenajme pritom, že obidve využité tvrdenia výrokovej logiky
    sa dajú dokázať z axióm iba pomocou pravidla modus ponens.

    Naopak, uvažujme, že chceme pravidlo MP simulovať pomocou pravidla
    rezolventy. Teda chceme ukázať $A, A\implies B \provable B$.
    Použijeme nasledujúci trik.
    \begin{itemize}
        \item[1.] $A \equiv A \lor \eps$
        \item[2.] $A \implies B \equiv \neg A \lor B$
        \item[3.] $A \lor \eps, \neg A \lor B \provable \eps \lor B$
            -- použime pravidlo rezolventy na ekvivalentné zápisy
            formúl $A, A\implies B$
        \item[4.] $B \equiv \eps \lor B$ a teda
        \item[5.] $A,A\implies B \provable B$.
    \end{itemize}
\end{dokaz}


\section{Substitúcia a unifikácia}

Vo výrokovej logike nebol problém hľadať kontrárne dvojice.
Zložitejšia situácia ale nastáva v prípade predikátovej logiky
prvého rádu. A práve tomu sa budeme venovať v tejto kapitole.
Uvažujme napríklad dve klauzuly
$C_1=P(x) \lor Q(x)$, $C_2 = \neg P(f(x)) \lor R(x)$.
V nich neexistuje žiadna kontrárna dvojica. Ak však nahradím premennú
$x$ za term $f(a)$ v prvej klauzule a za term $a$ v druhej klauzule,
dostaneme základné inštancie
$C_1'=P(f(a)) \lor Q(f(a))$, $C_2'=\neg P(f(a)) \lor R(a)$.
Teraz môžeme definovať rezolventu; bude to $Q(f(a)) \lor R(a)$.
Mohli by sme postupovať aj všeobecnejšie -- nahraďme $x$ za $f(x)$ v
prvej klauzule a dostávame
$ C_1^*= P(f(x)) \lor Q(f(x))$, $C_2^*= \neg P(f(x)) \lor R(x)$.

Rezolventa potom bude $C^*= Q(f(x)) \lor R(x)$.
Vidíme teda, že sme získali 2 rôzne rezolventy, jednu viac všeobecnú
ako druhú. No a práve v ďalšom texte si formálne zadefinujeme toto
dosadzovanie hodnôt a popíšeme spôsob, ako hľadať najvšeobecnejšie
rezolventy.

\begin{definicia}[Substitúcia]
    Pod substitúciou rozumieme konečnú množinu tvaru:
    $\{t_1/v_1, \ldots, t_n/v_n\}$, kde každé $v_i$ je premenná a
    $t_i$ je term.
    Ďalej požadujeme, aby všetky
    $v_i$ boli navzájom rôzne ($i \in\{1,\dots,n\}$)
    a aby term $t_i$ bol rôzny od $v_i$. Touto množinuo budeme
    popisovať činnosť ``naraz nahraď každú premennú $v_i$ termom $t_i$''.

    Ak $t_1, \ldots, t_n$ sú základné inštancie (teda termy bez premenných),
    tak substitúciu nazývame tiež základná substitúcia.
\end{definicia}

\begin{poznamka}
    Na označovanie substitúcií budeme používať grécke písmená.
    Špeciálne, prázdnu substitúciu označíme ako $\eps$.\footnote{
        Pozn.: Vzniká nám tu kolízia označenia s prázdnou nesplniteľnou
        klauzulou. Z kontextu však bude jasné, o ktorý prípad pôjde.
    }
\end{poznamka}

\begin{poznamka}
    Je dôležité si všimnúť, že poradie prvkov v definícii substitúcie je
    čisto antiintuitívne -- človek by očakával ``premenná/výraz ktorým ju
    máme nahradiť'' a nie ``výraz/premenná''.
\end{poznamka}

\begin{priklad}
    Jedna substitúcia môže byť napr.
    $\alpha = \{ f(z)/x,\ y/z\}$, teda $x$ nahrádzame za $f(z)$ a
    $z$ za $y$.
    Ďalšia môže byť $\beta = \{ a/x,\ g(y)/y,\ f(g(y))/z\}$.
\end{priklad}

\begin{definicia}
    Nech $\theta$ je ľubovoľná substitúcia a $E$ je nejaký výraz.
    Nech $\theta = \{ t_1/v_1, \dots, t_n/v_n\}$.
    Potom $E\theta$ označuje výraz, ktorý vznikne tak,
    že súčasne vo výraze $E$ nahradíme každý výskyt premenej $v_i$ termom $t_i$
    pre $i \in \{1,\dots,n\}$. Takýto výraz nazveme inštanciou $E$.
\end{definicia}

\begin{priklad}
    Majme substitúciu $\theta=\{a/x,\ f(b)/y,\ c/z\}$
    a výraz $E = P(x, y, z)$.
    Potom $E\theta = P(a, f(b), c)$.
\end{priklad}

\medskip
\noindent
Ďalšia operácia, ktorú budeme potrebovať, je operácia skladania
substitúcií.

\begin{definicia}[Kompozícia substitúcii]
    Majme substitúcie $\theta = \{t_1/x_1, \ldots t_n/x_n\}$ a
    $\lambda = \{ u_1/y_1, \ldots u_m/y_m \}$.
    Zloženie (kompozícia) $\theta \circ \lambda$ substitúcií
    $\theta,\lambda$ je definované ako množina
    \begin{equation*}
        \{t_1 \lambda/x_1, \dots, t_n \lambda/x_n, u_1/y_1m \dots u_m/y_m \}
    \end{equation*}
    z ktorej navyše vyradíme všetky členy $t_i\lambda/x_i$,
    pre ktoré platí, že $t_i \lambda = x_i$ (aby sme nesubstituovali
    identititami)
    a tiež vyradíme všetky $u_i/y_i$,
    pre ktoré $y_i \in \{x_1, x_2, \dots, x_n\}$ (lebo by sme mali
    dvojité správanie sa substitúcie na $x_i$).
\end{definicia}

\begin{poznamka}
    Kompozícia $\theta \circ \lambda$ sa správa rovnako ako postupné
    aplikovanie $\theta, \lambda$. Čiže
    $E(\theta \circ \lambda) = (E\theta)\lambda$.
\end{poznamka}

\begin{priklad}
    Majme substitúcie
    \begin{align*}
        \theta = \{t_1/x_1,\ t_2/x_2\} = \{ f(y)/x,\ z/y\} \\
        \lambda = \{u_1/y_1,\ u_2/y_2,\ u_3/y_3\} = \{ a/x,\ b/y,\ y/z\}
    \end{align*}
    Potom
    \begin{equation*}
    \begin{split}
        \theta \circ \lambda &= 
            \{ t_1 \lambda / x_1,\ t_2\lambda/x_2,\
            u_1/y_1,\ u_2/y_2,\ u_3/y_3\} - \{\dots\} \\ 
        &= \{f(b)/x,\ y/y,\ a/x,\ b/y,\ y/z\} - \{y/y,\ a/x,\ b/y\} \\
        &= \{f(b)/x,\ y/z\}
    \end{split}
    \end{equation*}
\end{priklad}

\begin{poznamka}
    Skladanie substitúcií je asociatávna operácia, teda ak zoberieme 
    $\theta, \lambda, \mu$, potom platí 
    $\theta \circ(\lambda \circ \mu) = (\theta \circ \lambda) \circ \mu$.

    Tiež platí, že $\varepsilon \circ \theta = \theta = 
            \theta \circ \varepsilon$.
    To znamená, že množina substitúcii s operáciou skladania je
    pologrupa (monoid) s jednotkou.
\end{poznamka}

\begin{definicia}[Unifikátor]
    Substitúciu $\theta$ nazveme unifikátorom
    množiny výrazov $E_1, E_2, \dots, E_n$,
    ak platí $E_1\theta = E_2\theta = \cdots = E_n\theta$.
    Množinu nazveme unifikovateľnou, ak pre ňu existuje
    unifikátor, ktorý je zjednocuje.
\end{definicia}

\begin{priklad}
    Majme množinu $\{P(a,y),\ P(x,f(b)\}$.
    Potom jeden z možných unifikátorov je napríklad
    $\theta = \{a/x,\ f(b)/y\}$.
\end{priklad}

\begin{poznamka}
    Nie každá množina má unifikátor. Naopak, množina môže mať aj
    viacej unifikátorov. Vtedy má medzi nimi význačné miesto takzvaný
    najvšeobecnejší unifikátor.
\end{poznamka}

\begin{definicia}[Najvšeobecnejší unifikátor]
    Majme množinu výrazov $S=\{ E_1, E_2, \ldots, E_n\}$.
    Unifikátor $\sigma$ pre množinu výrazov $S$ nazveme 
    najvšeobecnejší unifikátor, ak pre ľubovoľný unifikátor $\theta$
    množiny $S$ platí, že existuje substitúcia 
    $\lambda_{\theta}$ taká, že $\theta = \sigma \circ \lambda_{\theta}$.
\end{definicia}

\medskip
Pri hľadaní unifikátorov budeme pozerať na rozdiely vo výrazoch.
Uvažujme napríklad výrazy $P(a)$, $P(x)$.
Pozerajme sa na ne ako na konečnú postupnosť symbolov -- 
odlišujú sa akurát v treťom symbole. Toto je prvá odlišnosť/diferencia.
Vo všeobecnosti môže byť týchto výrazov viacej a preto si zadefinujeme
diferenčnú množinu.

\begin{definicia}[Diferenčná množina]
Nech $W$ je neprázdna množina výrazov.
Diferenčnú množinu pre množinu výrazov $W$ dostávame tak, 
že na výrazy sa pozrieme ako na postupnosti symbolov, nájdeme prvú pozíciu
(zľava), na ktorej sa líšia a tieto rozdielne výrazy vypíšeme.
\end{definicia}

\begin{poznamka}[Nebolo na prednáške]
    Iný (a podľa mňa lepší) pohľad na to, ako získať diferenčnú množinu je
    pozrieť sa na stromy daných výrazov a začať ich naraz rekurzívne
    prehľadávať zľava doprava až narazíme na vrchol, ktorý je v niektorom
    výraze iný. Vtedy do diferenčnej množiny zoberieme pre každý výraz
    podstrom zakorenený v dotyčnom vrchole.
\end{poznamka}

\begin{priklad}
    \label{prikl:dif1}
    Majme množinu $W = \{P(x,f(y,z),\ P(x,a),\ P(x,g(h(k(x))))\}$.
    Nájdeme prvý pozíciu na ktorej sa líšia:
    $\{P(x,\underline{f(y,z)},\ P(x,\underline{\phantom{(}\!a}),\
    P(x,\underline{g(h(k(x)))}) \}$.
    Diferenčnou množinou bude množina líšiacich sa podvýrazov, teda
    $D= \{ f(y,z),\ a,\ f(h(k(x)))\}$. Jej grafická konštrukcia je na
    obrázku \ref{fig:dif1}.
    \begin{figure}[h]
        \centering
        \subfigure[$P(x,f(y,z))$]{
            \imageontop{
                \includegraphics[scale=0.8]{img/12/diferencia.1.mps}
            }
        }
        \subfigure[$P(x,a)$]{
            \imageontop{
                \includegraphics[scale=0.8]{img/12/diferencia.2.mps}
            }
        }
        \subfigure[$P(x,g(h(k(x))))$]{
            \imageontop{
                \includegraphics[scale=0.8]{img/12/diferencia.3.mps}
            }
        }

        \caption{Ukážka diferenčnej množiny z príkladu \ref{prikl:dif1}}
        \label{fig:dif1}
    \end{figure}
\end{priklad}

\begin{priklad}[Nebol na prednáške]
    \label{prikl:dif2}
    Uvažujme $W=\{P(a,f(a),f(g(y))),\ P(a,f(a),f(u))\}$.
    Diferenčná množina je $D=\{g(y),u\}$ a grafické znázornenie je na
    obrázku \ref{fig:dif2}.
    \begin{figure}[h]
        \centering
        \subfigure[$P(a,f(a),f(g(y)))$]{
            \imageontop{
                \includegraphics[scale=0.8]{img/12/diferencia.11.mps}
            }
        }
        \subfigure[$P(a,f(a),f(u))$]{
            \imageontop{
                \includegraphics[scale=0.8]{img/12/diferencia.12.mps}
            }
        }

        \caption{Ukážka diferenčnej množiny z príkladu \ref{prikl:dif2}}
        \label{fig:dif2}
    \end{figure}
\end{priklad}

\begin{poznamka}
    Do diferenčnej množiny zakaždým vyberáme iba jednu (prvú) nezhodu.
    Napríklad pre $W=\{P(x,y,z),\ P(y,f(a),g(x,y))\}$ je diferenčná množina
    iba $D=\{x,y\}$.
\end{poznamka}

\subsection{Unifikačný algoritmus}

Teraz si ukážeme jeden z algoritmov používaných na unifikáciu množiny. Bude
dookola opakovať nasledujúce kroky:
\begin{enumerate}
    \item na začiatku polož kolo $k=0$, množinu na unifikovanie 
        $W_0 = W$ a počiatočnú substitúciu  $\sigma_0 = \eps$.

    \item Ak $W_k$ obsahuje jedinú klauzulu,\footnote{Na prednáške
        to bolo ``obsahuje jednotkovú klauzulu'' ale toto označenie je
        mätúce.} algortimus zakončí svoju činnosť
        a $\sigma_k$ je najvšeobecnejší unifikátor.
        V opačnom prípade nájdeme diferenčnú množinu $D_k$ pre $W_k$.

    \item Ak existujú také elementy $v_k,t_k \in D_k$, že $v_k$ je
        premenná, ktorá sa nevyskytuje v terme $t_k$, tak pokračujeme ďalším
        krokom.
        V opačnom prípade algoritmus zakončuje svoju činnosť
        s výsledkom, že množina $W$ nie je unifikovateľná.

    \item Položme $W_{k+1} = W_k \{t_k/v_k\}$ a
        $\sigma_{k+1} = \sigma_k \circ \{t_k/v_k\}$. 

    \item pokračujeme krokom 2.
\end{enumerate}

\begin{poznamka}
    Ak je množina unifikovateľná, vždy existuje najvšeobecnejší unifikátor.
\end{poznamka}

\begin{priklad}
    Nájdite najvšeobecnejší unifikátor pre množinu
    \begin{equation*}
        W=\{ P(a,x,f(g(y))),\ P(z,f(z),f(u)) \}
    \end{equation*}

    Algoritmus bude pracovať nasledovne:
    \begin{enumerate}
        \item $\sigma_0 = \eps, W_0 = W$.

        \item Pretože $W_0$ obsahuje viac klauzúl klauzula,
            $\sigma_0$ nie je najvšeobecnejší unifikátor a pokračujeme
            vo výpočte.

        \item Zostrojíme diferenčnú množinu $D_0 = \{a, z\}$.
            Existuje premenná $v_0 = z$, 
            ktorá nie je obsiahnutá v terme $t_0 = a$.

        \item \itemMath{
            \begin{align*}
            \sigma_1 &= \sigma_0 \circ \{ t_0/v_0 \} = 
                \eps \circ \{a/z\} = \{a/z\}\\[5pt]
            \begin{split}
                W_1 &= W_0 \{ t_0/v_0 \} = 
                    \{P(a,x,f(g(y))),\ P(z,f(z),f(u))\} \{a/z\} \\
                    &= \{P(a,x,f(g(y))),\ P(a,f(a),f(u))\}
            \end{split}
            \end{align*}
            }

        \item $W_1$ neobsahuje jedinú klauzulu. Pokračujeme vo výpočte

        \item Zostrojíme diferenčnú množinu $D_1 = \{x, f(a)\}$.

        \item V $D_1$ máme premennú $v_1 = x$ a term $t_1 = f(a)$.

        \item \itemMath{
            \begin{align*}
            \sigma_2 &= \sigma_1 \circ \{ t_1/v_1\}= 
                \{a/z\} \circ \{ f(a)/x\} = \{a/z,\ f(a)/x \} \\[5pt]
            \begin{split}
                W_2 &= W_1 \{t_1/v_1\} = 
                    \{P(a,x,f(g(y))),\ P(a,f(a),f(u))\} \{f(a)/x\} = \\
                    &= \{ P(a,f(a),f(g(y))),\ P(a,f(a),f(u)) \}
            \end{split}
            \end{align*}
            }

        \item $W_2$ nie je jednotková klauzula -- vytvárame diferenčnú množinu
            $D_2 = \{ g(y), u \}$.

        \item $v_2 = u$, $t_2 = g(y)$.

        \item \itemMath{ 
            \begin{align*}
            \sigma_3 &= \sigma_2 \circ \{t_2/v_2\}
                = \{a/z, f(a)/x\} \circ \{g(y)/u\}
                = \{a/z,\ f(a)/x,\ g(y)/u\} \\[5pt]
            \begin{split}
                W_3 &= W_2 \{ t_x/v_2\} = 
                    \{ P(a,f(a),f(g(y))),\ P(a,f(a),f(u))\} \{g(y)/u\} = \\
                    &= \{ P(a,f(a),f(g(y))),\ f(a,f(a),f(g(y)) \}
            \end{split}
            \end{align*}
            }

        \item $W_3$ obsahuje iba jedinú klauzulu a teda
            $\sigma_3$ je najvšeobecnejší unifikátor pre množinu klauzúl $W$.
    \end{enumerate}
\end{priklad}


\begin{priklad}
    Zistite, či je unifikovateľná množina 
    \begin{equation*}
        W=\{Q(f(a),g(x)),\ Q(y,y)\}
    \end{equation*}

    \begin{enumerate}
    \item $\sigma_0 = \eps$, $W_0 = W$.

    \item $W_0$ obsahuje viac klauzúl.
        Nájdeme diferenčnú množinu $D_0 = \{f(a), y \}$.

    \item $v_1=y$, $t_1=f(a)$.

    \item $\sigma_1 = \sigma_0 \circ \{ t_0/v_0\} = 
                \eps \circ  \{f(a)/y\} = \{f(a)/y\}$.

    \item $W_1 = W_0 \{ t_0/\sigma_0\} = \{ Q(f(a),g(x)),\
        Q(f(a),f(a))\}$.

    \item $W_1$ obsahuje 2 klauzuly,
        zostrojujeme $D_1 = \{g(x), f(a)\}$.

    \item V $D_1$ nemáme prvok, ktorý by bol premennou.
        Algoritmus ukončí svoju činnosť s výsledkom, že
        $W$ nie je unifikovateľná.
    \end{enumerate}
\end{priklad}

\begin{poznamka}
    Pri zisťovaní unifikovateľnosti vždy vytvárame množiny $W_i$ tvaru
    \begin{equation*}
        W\sigma_0, W\sigma_1, W\sigma_2, \dots
    \end{equation*}
    pričom v každom kroku sa zmenší počet premenných aspoň o 1.
    Po konečnom počte krokov sa teda algoritmus musí zastaviť.
\end{poznamka}

\begin{veta}[Unifikačná]
    Ak $W$ je konečná neprádzna unifikovateľná množina výrazov,
    tak unifikačný algoritmus vždy zakončuje svoju činnosť na druhom kroku
    a posledné $\sigma_k$ je najvšeobecnejší unifikátor.
\end{veta}

\begin{dokaz}
    Nech $W$ je unifikovateľná množina a nech $\Theta$ označuje jej
    ľubovoľný unifikátor.
    Označme si počet kôl algoritmu ako $n$.
    
    Indukciou ukážeme, že pre každé kolo $k$ počas výpočtu programu existuje
    taká substitácia $\lambda_k$, že $\Theta = \sigma_k \circ \lambda_k$.

    \begin{itemize}
    \item Báza indukcie: Nech $k = 0$. Máme ukázať, že existuje
    $\lambda_0$, pre ktorú platí $\Theta = \sigma_0 \circ \lambda_0$.
    V tomto prípade $\sigma_0 = \varepsilon$ a teda $\lambda_0 = \Theta$.

    \item Indukčný krok: Predpokladáme, že existuje $\lambda_k$, 
        pre ktoré platí $\Theta = \sigma_k \circ \lambda_k$.
        Pozrime sa na množinu $W_{k+1}= W\sigma_{k+1}$.

        Ak $W_k$ je jednotková klauzula, tak algoritmus zakončuje 
        svoju činnosť na druhom kroku a $\sigma_k$ je
        najvšeobecnejší unifikátor pre $W$. 

        Nech teda $W_k$ nie je jednotková množina.
        Potom hľadáme diferenčnú množinu $D_k$ pre množinu $W_k$. 
        $D_k$ je diferenčná množina pre $W_k$ a 
        vo $W_k$ musí existovať premenná -- označme ju $v_k$. 
        Ďalej musí existovať term $t_k$ rôzny od $v_k$.
        Ak by toto neplatilo, tak množina by nemohla byť
        unifikovateľná.\footnote{Náhľad prečo: Vieme, že
            $\Theta = \sigma_k \circ \lambda_k$ a pre $\lambda_k$ musí
            nutne unifikovať diferenčnú množinu $D_k$. Tým pádom aspoň
            jedna z ``jednotkových substitúcii'' v $\lambda_k$ musí
            obsahovať elementy z $D_k$.
        }

        Vieme, že diferenčnú množinu $D_k$ unifikuje substitúcia $\lambda_k$.
        Teda $v_k \lambda_k = t_k \lambda_k$.

        Ďalej budeme potrebovať, že $v_k$ nie je obsiahnuté v $t_k$.
        Ak by totiž premenná $v_k$ bola obsiahnutá v $t_k$, dôjdeme k sporu.
        Platilo by aj ``$v_k \lambda_k$ je obsiahnutá v 
        $t_k \lambda_k$''. Lenže vieme, že tam platí rovnosť a preto by
        musela platiť aj rovnosť $v_k=t_k$. Spor.

        Vypočítame $\sigma_{k+1}=\sigma_k \circ \{t_k/v_k\}$.
        Potrebovali by sme nájsť $\lambda_{k+1}$. Môžeme si ho
        napríklad ``tipnúť'' ako 
        $\lambda_{k+1} = \lambda_k - \{t_k\lambda_k/v_k\}$. 

        Pretože $v_k$ sa nevyskytuje v $t_k$, platí
        $t_k \{ t_k\lambda_k / v_k\} = \eps$ a tým pádom
        \begin{equation*}
            t_k \lambda_{k+1} = t_k (\lambda_k - \{t_k\lambda_k / v_k\}) 
            =t_k \lambda_k
        \end{equation*}
        A teda dostávame
        \begin{equation*}
        \begin{split}
            \{ t_k / v_k\} \circ \lambda_{k+1} &= 
            \{t_k \lambda_{k+1}/v_k \} \union \lambda_{k+1} \\
            & = \{t_k \lambda_k/v_k \} \union \lambda_{k+1} \\
            & = \{t_k \lambda_k/v_k\} \union 
                ( \lambda_k - \{ t_k\lambda_k/v_k\}) \\ 
            &= \lambda_k
        \end{split}
        \end{equation*}

        Výsledok je teda, že 
        $\lambda_{k} = \{ t_k / v_k \} \circ \lambda_{k+1}$,
        čiže $\Theta = \sigma_k \circ \lambda_k
            = \sigma_k \circ \{t_k/v_k\} \circ \lambda_{k+1}
            = \sigma_{k+1} \circ \lambda_{k+1}$. A to sme chceli
            ukázať.
    \end{itemize}
\end{dokaz}

\section {Metóda rezolvent pre logiku 1. rádu}

\begin{definicia}[Spojenie]
    Nech $C$ je klauzula, ktorá obsahuje dva alebo viac literálov 
    (a tie pozostávajú z rovnakého predikátu len s inými parametrami).
    Ak tieto literály majú najvšeobecnejší unifikátor $\sigma$, 
    tak $C\sigma$ sa nazývame spojením $C$. 
    
    Ak $C\sigma$ je jednotková klauzula, tak $C\sigma$ nazývame tiež
    jednotkovým spojením $C$.
\end{definicia}

\begin{priklad}
    Uvažujme klauzulu $C$, ktorá vyzerá nasledovne: 
    $C = \{ P(x) \lor P(f(y)) \lor \neg Q(x)\}$. Zoberme literály
    $P(x)$ a $P(f(y))$. Ich najvšeobecnejší unifikátor je
    $\sigma=\{ f(y) / x \}$. Potom spojenie je 
    $C\sigma = \{P(f(y)) \lor \neg Q(x) \}$.
\end{priklad}

\begin{definicia}[binárna rezolventa]
    Nech $C_1$ a $C_2$ sú dve klauzuly (budeme ich nazývať predpoklady), 
    ktoré nemajú spoločné premenné. Nech $L_1 \in C_1$ a $L_2 \in C_2$ 
    sú dva literály. Ak $L_1$ a $\neg L_2$ majú 
    najvšeobecnejší unifikátor $\sigma$, tak výraz
    \begin{equation*}
        (C_1\sigma - L_1\sigma) \union (C_2\sigma - L_2\sigma)
    \end{equation*}
    sa nazýva binárnou rezolventou.\footnote{Za pozornosť stojí fakt, že
    vo všeobecnosti $(C\sigma - L\sigma) \ne (C-L)\sigma$.} 
    Literály $L_1$ a $L_2$ môžeme vynechať a nazývame ich nadbytočné.
\end{definicia}

\begin{priklad}
    Majme $C_1 = P(x) \lor Q(x)$ a $C_2 = \neg P(a) \lor R(x)$, 
    čo budú predpoklady. Na to, aby sme mohli previesť operáciu
    binárnej rezolventy, musíme najskôr premenovať premenné v druhom
    výraze, aby boli rôzne od tých v prvom. Máme teda
    $C_2' = \neg P(a) \lor R(y)$.

    Uvažujme teraz klauzuly $L_1 = P(x)$ a $L_2 = \neg P(a)$.
    Ich najvšeobecnejší unifikátor je $\sigma = \{a/x\}$.

    Binárna rezolventa $C_1$ a $C_2$ je
    \begin{equation*}
    \begin{split}
        (C_1\sigma - L_1\sigma) \union (C_2\sigma - L_2\sigma) 
        &= (\{P(a),\ Q(a))\} - \{P(a)\}) \union
            (\{\neg P(a),\ R(y)\}-\{\neg P(a)\}) \\
        &= Q(a)\lor R(y)
    \end{split}
    \end{equation*}
    Nadbytočné literály sú $P(x), \neg P(a)$.
\end{priklad}

\begin{definicia}[Rezolventa logiky 1. rádu] 
    Rezolventou z predpokladov $C_1$ a $C_2$ definujeme ako jednu z
    nasledujúcich binárnych rezolvent:
    \begin{enumerate}
        \item Binárna rezolventa $C_1$ a $C_2$
        \item Binárna rezolventa $C_1$ a spojenia $C_2$
        \item Binárna rezolventa spojenia $C_1$ a $C_2$
        \item Binárna rezolventa spojenia $C_1$ a spojenia $C_2$
    \end{enumerate}
\end{definicia}

\startFIXME


\paragraph{Príklad} $$C_1 = P(x) \lor P(f(y))\lor Rg(y))$$
$$ C_2 = \neg P(f(g(a)) \lor Q(b)$$
Spojenie pre $C_1$ vyzerá ako: $C_1': P(f(y)) \lor R(g(y))$.
Binárna rezolventa $C_1' a C_2$ bude vyzerať takto: $R(g(g(a))) \lor Q(b)$ --
rezolventa $C_1$ a $C_2$.


\section{Opakovanie}
	Ak množina klauzúl nie je splniteľná, potom metódou rezolvent z nej vždy
	môžeme dostať prázdnu klauzulu (a ak sa táto dostane množiny klauzúl,
	tak formula nie je splniteľná v žiadnej interpretácii). Ak máme nejakú
	klauzulu $C$, $C\sigma$ sme nazývali spojením klauzuly $C$. Definícia
	binárnej rezolventy. 

\par  Metódu rezolvent zaviedol roku 1965 Robinson, je efektívnejšia ako obe
varianty Herbrandovej metódy. 

\par \{ sleep...\}

\paragraph{Úplnosť metódy rezolvent}

\paragraph{Príklad} Majem množinu klauzúl $S$:
\begin{enumerate}
	\item $P$
	\item $\neg P\lor Q$
	\item $\neg P \lor \neg Q$
\end{enumerate}
Tejto množine klauzúl zodpoveda uzavretý sémantický strom.
Prislúchajúca herbrandovská báza je $\{P, Q\}$ (na tabuľu sa kreslí sémantický
strom pre $P$ a $Q$, usilovný čitateľ si ho isto domyslí). Každá vetva sa končí
odmietajúcim vrcholom, žiadna z tých interpretácií, ktoré končia v listoch, nie
je splniteľná. Tomuto stromu môžeme priradiť uzavretý podstrom (označíme ho
$T'$,  má odseknuté vetvy na miestach, kde sú podstromy odmietajúce)


\begin{verbatim}
            T
            /\
         P /  \ \neg P
       Q /\ nQ Q/\ nQ 
\end{verbatim}

\begin{verbatim}
            T'
            (1)
            /\
      (2)P /  \ \neg P
   (4) Q /\ nQ x (5)
       x   x
\end{verbatim}

\par $\neg P$ -- rezolventa $(4)$, $(5)$, $\neg P \cup S$.  $S\cup \{ \neg P
\}\cup \{ \square \}$.

\par Vznikli nám teda klauzuly.
$$(4) \neg P\qquad (2) (3)$$
$$(5) \Box\qquad (4)(1)$$

\par Strom sa po každej aplikácii pravidla postupne skracuje. 

\paragraph{Lema} Nech $C_1'$ a $C_2'$ sú inštancie $C_1$ resp. $C_2$ (v uvedenom
poradí). Ak $C'$ je rezolventa $C_1'$ a $C_2'$, tak potom existuje rezolventa
$C$ klauzúl $C_1$ a $C_2$, že $C'$ je inštancia $C$. 

\paragraph{Dôkaz} Ak je treba, premenujeme premenné $C_1$ a $C_2$. Nech $L_1'$ a
$L_2'$ sú literály, ktoré môžeme vynechať (sú nadbytočné). ďalej nech platí:
$$C' = (C_1' \nu - L_1'\nu) \cup ( C_2'\nu - L_2'\nu)$$
Pričom $\nu$ je najvšeobecnejší unifikátor pre $L_1'$ a $\neg L_2'$. $C_1'$,
$C_2'$ sú inštancie $C_1$ a $C_2$, a teda existuje substitúcia $\Theta$ taká, že
platí:

$$ C_1' = C_1 \Theta $$
$$ C_2' = C_2 \Theta $$

(Pozn.: $C_1$ a $C_2$ nemajú spoločné premenné). $L^1_i, L^2_i, \ldots, L^{r_i}_i,
i=1,2$ sú literály, ktoré v $C_1$ zodpovedajú $L_i'$, teda $L^1_i \Theta = L^2_i
\Theta = \cdots = L^{r_i}_i\Theta = L'_i (i=1,2)$. $D_i > 1$ dostaneme
najvšeobecnejší kvantifikátor $\lambda_i (i=1,2)$ pre $\{ L^1_i, L^2_i, \ldots,
L^{r_i}_i\}, L_i = L^1_i \lambda_i (i=1,2)$. $\lambda_i$ je najvšeobecnejší
unifikátor, tak pre vhodnú substitúciu $\xi$ platí:
$$ L_i' = L^1_i \Theta = L^1_i (lambda_i \circ \xi) = (L^1_i\lambda_i)\xi =
L_i \xi$$

$$L_i\xi = L_i'$$

$L_i$ .. spojení $C_i\lambda_i$, pre $C_i$, ak $r_i = 1$, ak $r_i = 1$, potom
$\lambda_i = \Sigma$, $L_i = L_i^1\lambda_i$.

$$ \lambda = \lambda_1 \cup \lambda_2$$
$$L_i' = \mbox{...} L_i$$

$L'_i, \neg L_2'$ -- unifikovateľné.
$L-1, \neg L_2$ -- unifikovateľné.

Označme $\sigma$ najvšeobecnejší unifikátor pre $L_1'$ a $\neg L_2'$.

$C= ((C_1\lambda)\sigma = L_1\sigma) \cup
((C_2\lambda_2)\sigma-L_2\sigma) = ((c_1\lambda)\sigma - (\{L^1_+, L^2_1,
\ldots, L^{r_i}_1 ...
= C_1(\lambda\circ \sigma) = \{ L^1_1, \ldots L^{r_1}_1 \} (\lambda\circ\sigma)
\cup C_2)(\lambda\circ\sigma) - \{L^1_2, \ldots
L^{r_i}_1\}(\lambda\circ\sigma)$

\par $C$ -- rezolventa $C_1$ a$C_2$, $C'$ je substitúcia $C$:
$C = (C_1' \nu = L_1' \nu) \cup (C_2'\nu = L_2'\nu) = (C_1\Theta)\nu -
(\{L^1_1, \ldots L^{r_i}_1\}\Theta)\nu)\cup ((C_2\Theta)\nu - \{L^1_2, L^2_2,
\ldots, L^{r_i}_2\}\Theta )\nu) = 
C_1(\Theta\circ\nu) - \{L^1_1, \ldots, L^{r_i}_1\} \Theta\circ\nu) \cup
(C_2(\Theta\circ\nu) - \{L^1_2, \ldots L^{r_i}_2\} \Theta\circ\nu)
$

$\lambda \circ \sigma$ jke všeobecnejšia ako $\theta \circ \nu$.


\paragraph{Veta (úplnosť metódy rezolvent)} Množina klauzúl $S$ nie je
splniteľná práve vtedy, keď existuje odvodenie prázdnej klauzuly $\Box$ z $S$.

\paragraph{Dôkaz} Predpokladajme, že z $S$ existuje odvodenie prázdnej klauzuly
$\square$. $R_1, R_2, \ldots R_n$ sú všetky rezolventy v odvodení (medzi nimi
niekde bude aj $\square$). Zoberiem $C_1$, $C_2$ -- ľubovoľné klauzuly z $S$ a
$C_1$ a $C_2$ bude príslúchať nejaká rezolventa. Ak sú klauzuly splniteľné, je
splniteľná aj rezolventa. To znamená, že $C_1$ a $C_2$ nemôžu byt splniteľné a
teda nemôže byť množina klauzúl (klauzuly, ktorých rezolventou je prázdna
klauzula, nebudú splniteľné nikdy). 

\subparagraph{Obrátené tvrdenie} Prepokladajme, že množina klauzúl $S$ nie je
splniteľná (máme ukázať, že ako rezolventa sa tam ukáže prázdna klauzula). Ak
predpokladáme, že $S$ nie je splniteľná, potom podľa Herbrandovej vety (1.
variant), nie je splniteľná práve vtedy, keď je možné jej priradiť konečný
uzavretý sémantický strom. 

\par Môže as stať, že strom $T$ pozostáva jedine z koreňa -- odmieta jedinú
klauzulu a v tomto prípade veta platí. Teraz predpokladajme, že je konečný a má
viac ako 1 vrchol. V tomto prípade, tak má aspoň jeden akceptujúci vrchol. Potom
$i_v$ je čiastočná interpretácia končiaca v tom vrchole. Ďalej, každý
nasledovník je odmietajúci. Vrchol je akceptujúci, ak čiastočná interpretácia v
ňom existuje, a každý nasledujúci vrchol je odmietajúci.

\par Predpokladajme, že by tento strom nema akceptujúci vrchol. Potom každý
vrchol obsahuje nasledovníka, ktorý nie je odmietajúci. Týmto pádom by sme
vytvorili nekonečne dlhú vetvu, čo je spor (strom je konečný).

\par Ideme pracovať s akceptujúcim vrhcholom. Nech $v$ je akceptujúci vrchol
stromu $T$ a $v_1$, $v_2$ sú odmietajúci nasledovníci $v$. $I(v)$ (čiastočná
interpretácia končiaca vo vrchole $v$) vyzerá nasledovne:

\begin{align*}
    I(v)    &= \{ m_1, m_2, \ldots, m_n \} \\
    I(v_1)  &= \{ m_1, m_2, \ldots, m_n, m_{n+1} \}  \\
    I(v_2)  &= \{ m_1, m_2, \ldots, m_n, \neg m_{n+1} \} 
\end{align*}

$C_1'$ a $C_2'$ sú dve základne inštancie klauzúl $C_1$ a $C_2$ -- $C_1'$ a
$C_2'$ neplatia v $I(v_1)$ a $I(v_2)$. $C_1'$ a $C_2'$ sa neodmietajú v $I(v)$.
$C_1'$ musí obsahovať $\neg m_{n+1}$ a $C_2'$ musí obsahovať $m_{n+1}$. $L_1' =
\neg m_{n+1}$ a $L_2' = m_{n+1}$. AAk vynecháme $L_1'$ a $L_2'$, dostaneme
rezolventu. $C'$ je rezolventa $C_1$ a $C_2$. $C' = (C_1' - L_1') \cup (C_2' =
L_2')$. $C'$ -- musí byť nepravdivá v $I(v)$. Podľa predchádzajúcej lemy musí
existovať rezolventa $C$ taká, že $C'$ je základná inštancia $C$.

\par Vezmime si $T''$ -- uzavretý sémantický strom, $C \cup \{C\}$. ... (niečo
ďalej?)



\paragraph{Príklad} Majme množinu formúl $F_1: (\forall x) (C(x) \implies (W(x)
\land R(x))$, $F_2: (\exists x)(C(x) \land Q(x))$ $G: (\exists x) (Q(x) \land
R(x))$. Ukážte, že $G$ je logickým dôsledkom $F_1$ a $F_2$.

\paragraph{Riešenie} Pre $F_1$, $F_2$ a $\neg G$ vytvoríme štandardné formy.
Dostávame nasledujúcich 5 klauzúl:
\begin{enumerate}
	\item $(\forall x) (C(x) \implies (W(x) \land R(x)) \iff (\forall
	x)(\neg C(x) \lor (W(x)\land R(x)) \iff (\forall x) ((\neg C(x) \lor
	W(x)) \land (\neg C(x) \lor R(x)))$
	\par (1) $\neg C(x) \lor W(x)$ -- $F_1$
	\par (2) $\neg C(x) \lor R(x)$ -- $F_1$
	\par $C(a)$ -- $F_2$
	\par $Q(a)$ -- $F_2$
\end{enumerate}
$$\neg G \iff \neg (\exists x)(Q(x)\land R(x)) \iff (\forall x) (\neg Q(x) \lor
\neg R(x)) $$. Štandardná formula pre túto formulu je:
\par (5) $\neg Q(x) \lor \neg R(x)$ -- $G$.

\par Rezolventy: 
\par (6) $R(a)$ -- rezolventa (2), (3)
\par (7) $\neg R(a)$ ($\sigma = \{a / x \})$ -- rezolventa (5), (4)
\par (8) $\square$ -- rezolventa (6), (7)

\par Záver: $G$ je logickýkm dôsledkom $F_1$ a $F_2$


\section{Stratégia vymazávania}

Na základe vety o úplnosti vieme, že ak máme nejakú množinu klauzúl $S$,
tak viem z nej postupne vytvárať rezolventy a ak nie je splniteľná, 
po konečnom počte krokov dostávam prázdnu klauzulu $\eps$.
Za účelom dôkazu teda musím zaradom prehľadávať všetky rezolventy, ktoré
môže vzniknúť zo všetkých možných dvojíc klauzúl.

\begin{priklad}
    \label{prikl:vymazavanie}
    Majme množinu klauzúl $S=\{P\lor Q,\ \neg P\lor Q,\ 
        P \lor \neg Q,\ \neg P \lor \neg Q\}$. 
    Metódou rezolvent ukážte, že $S$ nie je splniteľná.

    \paragraph{Riešenie:}
    Použijeme takzvanú metódu nasýtenia -- budeme postupne 
    v každom kroku generovať najväčšiu možnú množinu rezolvent.
    Definujme si postupnosť $\{S^i\}_{i=0}$ množín klauzúl nasledovne:
    $S^0 = S$ a
    \begin{equation*}
        S^{n+1} = \{ \mbox{rezolventa klauzúl} C_1, C_2 | 
            C_1 \in S^0 \union S^1 \union \dots \union S^{n},\
            C_2 \in S^n\},\quad n=1, 2, \dots
    \end{equation*}
    Takýmto spôsobom by sme po $39$ krokoch dostali prázdnu klauzulu.
    Problém tejto metódy je, že niektoré klauzuly sa vyskytnú v
    popísanom prístupe viackrát. Prípadne sa tam môžu vyskytnúť tautológie.
\end{priklad}

Videli sme, že predchádzajúca metóda nie je úplne optimálna. Preto za
účelom rýchlejšieho prehľadávania budeme niektoré evidentne nadbytočné
rezolventy zahadzovať (a algoritmus nazveme stratégiou vymazávania).
Najskôr si zadefinujeme, ako si budeme predstavovať nadbytočné rezolventy.

\begin{definicia}[podklauzula]
    Klauzula $C$ je podklauzulou klauzuly $D$ (alebo tiež $C$ pohlcuje
    $D$) práve vtedy, keď existuje substitúcia $\sigma$ taká, že platí $C\sigma
    \subseteq D$. Klauzulu $D$ vtedy tiež nazývame nadklauzulou klauzuly $C$.
\end{definicia}

\begin{priklad}
    Majme klauzuly $C = P(x)$ a $D = P(A) \lor Q(a)$.
    Ak budeme uvažovať substitúciu $\sigma = \{a/x\}$, dostaneme
    $C\sigma = P(a)$ a teda $C\sigma \subseteq D$.
    Čiže $C$ je podklauzula $D$.
\end{priklad}

\begin{poznamka}
    Ak $D$ je identicky rovná $C$ alebo ak klauzula $D$ je inštancia $C$, 
    potom $D$ je nadklauzula $C$.
\end{poznamka}


Ako sme už teda povedali, stratégia vymazávania spočíva vo vylepšení metódy
nasýtenia o zahadzovanie zbytočných výsledkov.
Čiže opäť konštruujeme postupnosť $\{S^i\}_{i=0}$.
Do $S^{n+1}$ ale teraz vyberieme iba tie rezolventy $C_1, C_2$ 
(opäť $C_1 \in (S^0 \union S^1 \union \dots \union S^n)$ a $C_2 \in S^n$),
ktoré nie sú tautológie a ani nadklauzuly niektorej z klauzúl dosiahnutej
doteraz.

\begin{priklad}[Revízia \ref{prikl:vymazavanie}]
    Opäť máme množinu klauzúl $S=\{P\lor Q,\ \neg P\lor Q,\ 
        P \lor \neg Q,\ \neg P \lor \neg Q\}$.

    \begin{itemize}
    %%% {{{
    \item[$S^0:$]
        \begin{itemize}
            \item[1] $P\lor Q$
            \item[2] $\neg P \lor Q$
            \item[3] $P\lor \neg Q$
            \item[4] $\neg P \lor \neg Q$
        \end{itemize}

    \item[$S^1:$]
        \begin{itemize}
        \item[5] $Q$ -- rezolventa 1, 2
        \item[6] $P$ -- rezolventa 1, 3
        \item[7] $\neg P$ -- rezolventa 2, 4
        \item[8] $\neg Q$ -- rezolventa 3, 4
        \end{itemize}

    \item[$S^2:$]
        \begin{itemize}
            \item[$\vdots$] -- zahodíme veľa rezolvent, napr. rezolventa
                1,8 je $P \lor \eps \equiv P$ a tú už máme.
            \item[9] $\eps$ -- rezolventa 5,8
        \end{itemize}
    %%% }}}
    \end{itemize}
\end{priklad}

\subsection{Algoritmus  pohltenia}
Jediný problém, ktorý nám ostáva vyriešiť je detekcia či je nejaká
klauzula tautológia alebo či je podformulou inej klauzuly. Prvý prípad
sa rieši jednoducho -- tautológiu máme práve vtedy, ak sa vo formule
vyskytuje kontrárna dvojica. S podformulou to bude horšie.
Skúsme sa na to pozrieť takto:

Nech $C$, $D$ sú klauzuly. Označme si substitúciu premenných v D za
nové konštanty (nevyskytujúce sa v $C$, $D$) ako
\begin{equation*}
    \Theta = \{ a_1 / x_1,\ a_2 / x_2,\ \ldots,\ a_n / x_n \}
\end{equation*}
%
Ak je klauzula $D$ tvaru
\begin{equation*}
    D = L_1 \lor L_2 \lor \ldots \lor L_m
\end{equation*}
dostávame, že $D\Theta$ je základná klauzula (neobsahujúca premenné).
%
\begin{equation*}
    D \Theta = L_1\Theta \lor L_2 \Theta \lor \ldots \lor L_m \Theta
\end{equation*}
%
Nás bude zaujímať jej negácia
\begin{equation*}
\neg D\Theta = \neg L_1 \Theta \land \neg L_2 \Theta 
                \land \ldots \land \neg L_m \Theta
\end{equation*}
%
Algoritums, ktorý preveruje, či klauzula $C$ je podklauzulou $D$ bude
vyzerať nasledovne:

\begin{enumerate}
    \item Nech 
        $W = \{ \neg L_1 \Theta, \neg L_2 \Theta, \ldots, \neg L_m \Theta \}$
        a nech $k=0$ a $U^0 = \{ C \}$

    \item Ak $U^k$ obsahuje $\eps$, tak algoritmus skončí
        s výsledkom, že $C$ je pod $D$.

    \item V opačnom prípade kladieme 
        \begin{equation*}
            U^{k+1} = \{ \mbox{rezolventa } C_1 \mbox{ a } C_2 | 
            C_1 \in U^{k} \land C_2 \in U\}
        \end{equation*}

    \item Ak $U^{k+1}$ je $\emptyset$, tak algoritmus skončí s
        výsledkom, že  $C$ nie je podklauzula $D$. 

    \item V opačnom prípade kladieme $k=k+1$ a opakujeme krok 2
\end{enumerate}

\startFIXME

\begin{poznamka}
    $\mathcal{U}^k, \mathcal{U}^{k+1}$, klauzuly z
    $\mathcal{U}^{k}$ sú konečné. $\mathcal{U}^0, \mathcal{U}^1, \ldots \square$.
\end{poznamka}

\begin{dokaz}
    Predpokladajme, že $C$ je podklauzula $D$. Na základe našej
    definície existuje substitúcia $\sigma$, že $C\sigma \subseteq D$. Teda
    $C(\sigma \circ \Theta) \subseteq D\Theta$. Literály z $C\sigma \circ \Theta$
    môžeme vynechať pomocou jednotkových klauzúl z $W$. ... Algoritmus skončí svoju
    činnosť.
    \par
    Obrátené tvrdenie: predpokladajme, že algoritmus zakončuje prácu na treťom
    kroku. Odmietnutie môžeme znázorniť nasledujúcim obrázkom:

    \todo{obrazok}

    $$C_0, N_1 ,\ldots B_r \in W$$
    $$C(\sigma_0 \circ \sigma_1 \circ \sigma \circ \sigma_r) = \{ \neg B_0, \neg
    B_1, \ldots \neg B_r\} \subseteq D\Theta$$
    $$\lambda = \sigma_0 \circ \sigma_1 \circ \sigma_2 \ldots \circ \sigma_r \implies
    C \lambda \subseteq D\Theta$$

    $\sigma$, ktorá dostaneme z $\lambda$ tak, že v každom komponente $\lambda$
    nahradíme  konštantu $a_i$ premennou $x_i$, $i=1, 2, 3, \ldots$. $C\sigma
    \subseteq D$. $C$ je pod $D$.
\end{dokaz}

\begin{priklad}
    $C = \neg P(x) \lor Q(f(x), a)$. $D = \neg P(h(y)) \lor
    Q(f(h(y)),a) \lor P(z)$. Zistite, či klauzula $C$ je podklauzulou $D$.

    \par $y$ a $z$ sú premenné v $D$. $\Theta = \{ b/y, c/z\}$. Konštanty $b$, $c$
    nevystupujú v $C$, $I$. najprv vypočítame $D\Theta \neg P(h(b)) \lor
    Q(f(h(b)),a) \lor \neg P(c)$

    $$\neg D \Theta = P(h(b)) \land \neg Q(f(h(b)),a) \ lor P(c)$$
    $$W = \{P(h(b)), \neg Q(f(h(b)),a), P(c) \}$$
    $$\mathcal{U}^0 = C = \neg P(x) \lor Q(f(x),a)\}$$
    $\mathcal{U}^0$ neobsahuje $\square$, musíme vytvoriť $\mathcal{U}^1$. Urobíme
    príslušnú substitúciu v množine $\mathcal{U}^0$. Dostávam nasledovné rezolventy:
    $$\mathcal{U}^1 = \{ Q(f(h(b)),a), \neg P(h(b)), Q(f(b),a)\}$$. 
    \par
    $\mathcal{U}^1$
    nie je prádzna a neobsahuje prádznu klauzulu -- musím vytvoriť $\mathcal{U}^2$.
    V tomto sa už vyskytne prádzna klauzula, čo znamená, že $C$ pohlcuje klauzulu
    $D$.
\end{priklad}

\begin{priklad}
    $C=P(x,x)$ a $D=P(f(x),y) \lor P(y,f(x))$. Zistite, či $C$
    je podklauzula $D$.

    \paragraph{Riešenie} (1) $x$, $y$ sú premenné v $D$. $a$ a $b$ sú konštanty,
    ktoré sa nevyskytujú $C$, $D$. $\Theta = \{ a/x, b/y\}$. $D\Theta = P(f(a),b), \lor P(b,
    f(a))$.

    $$\neg D\Theta = \neg P(f(a),b) \lor \neg O(b,f(a))$$
    $$W = \{ \neg P(f(a),b), \neg P(b,f(a))\}$$
    $$\mathcal{U}^0 = P(x,x)$$


    \par (2) $\mathcal{U}^0$ neobsahuje $\square$, tak sa môže zistiť
    $\mathcal{U}^1$
    \par (3) $\mathcal{U}^1 = \emptyset$. Záver: $C$ nie je podklauzula $D$.
\end{priklad}

\begin{priklad}
    Majme formuly:

    \begin{enumerate}
        \item $P\implies S$
        \item $S \implies U$
        \item $P$
        \item $U$
    \end{enumerate}

    Dokážte, že formula 4 vyplýva z formúl 1, 2 a 3. 

    \paragraph{Riešenie} Prepíšeme si formuly do správneho tvaru, aby sme mohli
    použiť pravidlo rezolventy:
    \begin{enumerate}
            \item $\neg P \lor S$
            \item $\neg S\lor U$
            \item $P$
            \item $U$
    \end{enumerate}
    Snažíme sa nájsť negáciu -- chceme ukázať, že 
    \begin{enumerate}
            \item $\neg P \lor S$
            \item $\neg S\lor U$
            \item $P$
            \item $\neg U$
    \end{enumerate}

    nie je splniteľná. Zoberiem si rezolventu 1 a 3, dostávam $S$ (5). Keď zoberiem 
    2 a 4, dostávam $\neg S$ (6). Zoberiem 5 a 6, dostávam $\square$ (7).
\end{priklad}


\begin{priklad}
    Predpoklad: Študenti sú občania. Záver: Hlasy študentov sú
    hlasy občanov.

    \paragraph{Riešenie} 
    \begin{itemize}
            \item $S(x)$ označuje \uv{$x$ je študent}.
            \item $C(x)$ označuje \uv{$x$ je občan}.
            \item $V(x,y)$ označuje \uv{$x$ je hlas $y$}.
    \end{itemize}
    Predpoklad: $(\forall y) (S(y)\implies C(y))$. Študenti sú občania.
    Záver: $(\forall x) ((\exists y) (S(y) \land V(x,y)) \implies (\exists z)(C(z)
    \implies V(x,z)))$. Hlasy študentov sú hlasy občanov.

    Aká bude štandardná forma pre vyjadrenie predpokladu?
    \begin{enumerate}
            \item $\neg S(y) \lor C(y)$
                \par $\neg ((\forall x) ((\exists y)(S(y)\land V(x,y)) \implies
                (\exists x)(C(z) \land V(x,z)))) \iff 
                \neg ((\forall x)(\forall y)(\neg S(y)\lor \neg V(x,y)) \lor
                (\exists z)(C(z)\land V(x,z))) \iff 
                \neg ((\forall x)(\forall y)(\exists x)(\neg S(y) \lor
                \neg U(x,y)\lor (C(z) \land V(x,z)))) \iff
                (\exists x)(\exists y)(\forall z) (S(y) \land V(x,y)) \land
                (\neg C(z) \lor \neg V(x,z))$
                Teraz potrebujeme Skolemov normálny tvar:
                $(\forall z)(S(b) \land U(a,b)) \land(\neg C(z) \lor \neg
                V(a,z)))$
                \par Pre negáciu záver dostávane nasledujúce klauzuly:
            \item $S(b)$
            \item $V(a,b)$
            \item $\neg C(z) \lor \neg V(a,z)$.

            \item $C(b)$ z (1) a (2) (miesto $y$ dosadíme $b$)
            \item $\neg V(a,b)$ zo (4) a (5)
            \item $\square$ z (3)  a (6)
    \end{enumerate}

    \par Predpokladajme, že $b$ je študent, $a$ je hlas študenta $b$ a nie je hlas
    žiadneho občana. Pretože $b$ je študent, $b$ je občan. Okrem toho $a$ nemôže byž
    hlas $b$, pretože $b$ je občan a to nie je možné.
\end{priklad}


\chapter{Neodprednášané v šk. roku 09/10}
Nasledujúca kapitola obsahuje veci, ktoré neboli odprednášané doc.
Tomanom v roku, keď som dôkladne prerábal tieto poznámky. Evidentne
ale dané veci boli odprednášané rok predtým, odkiaľ sú aj nasledujúce
poznámky. Na začiatku by to chcelo snáď upozornenie, že dané poznámky
sú neoverené a z predchádzajúcich skúseností môžu byť chaotické a
neúplné (i keď základný obsah majú zachytený dobre).

Poprosil by som preto, ak máš poznámky k tejto časti, bolo by
najideálnejšie, ak by si vygeneroval/-a pripomienky k skriptám
a buď doručil/-a mne osobne alebo rovno prepísala zdrojové kódy
skrípt. Myslím, že Ti za to budú vďačné budúce generácie.

PPershing

\startFIXME

\section{Rozširovanie teórie}
\paragraph{Veta} Nech $A$ je formula teórie $T$ s jazykom $L$, nech všetky voľné
premenné formuly $A$ sú $x_1, x_2, \ldots x_n$, ďalej nech $L'$ je rozšírenie
$L$ pridaním nového predikátového symbolu $P$ a nech $T'$ vznikne z $T$ pridaním
axiomy $P(x_1, \ldots x_n) \iff A$ (označme ju *). 
\par Potom teória $T'$ s jazykom $L'$ je konzervatívne rozšírenie teórie $T$ s
jazykom $L$. Ďalej, pre každú formulu $B$ jazyka $L'$ existuje formula $B^*$ z
jazyka $L$, že platí $T \models  B \iff B^*$. 

\par Najprv budeme konštruovať $B^*$. Dôkaz: Nech $B$ je formula na jazyku $L'$,
nech $A'$ je variant  formuly $A$ (definujúcej predikát $P$), že žiadna
premennáz formuly $B$ nie je viazaná vo formule $A'$.

\par Nech $B^*$ vznikne z $B$ tak, že každú podformulu $P(a_1, \ldots a_n)$
nahradíme podformulou $A'(a_1, \ldots a_n)$ a podľa vety o variantoch nám platí,
že:
$$ T' \models P(a_1, \ldots, a_n) \iff A'(a_1, \ldots a_n)$$
$$ T' \models B \iff B^*$$.

\par Ukážeme, že $T'$ v jazyku $L'$ je konzervatívne rozšírenie teórie $T$ v
$L$ (teda pre ľubovoľnú formulu na pôvodnom jazyku, ktorá je dokázateľná v
teórii $L'$, je dokázateľná aj v pôvodnej teórii --  nevzniknú žiadne nové
teorémy). Nech $C$ je ľubovoľné formula na jazyku $L'$, ďalej nech platí $T'
\models C$. Nám stačí dokázať, že $T \models C^*$. Ak je potom formula $C$ na
jazyku $L$, potom $C^*$ je $C$.
\par Uvažujme $C_1, C_2, \ldots C_n$ je odvodenie formuly $C$ v teórii $T'$. Nám
stačí ukázať, že $C_i^*$ sú dokázateľné (odvoditeľné) v $T$.

Pri odvodení $C$ sa môže stať nasledovné:
\begin{enumerate}
	\item $C_i$ je axioma predikátovej logiky $L$, potom $C_i^*$ je axioma
	rovnakého druhu
	\item $C_i$ je axioma z $T$, potom $C_i^*$ je $C_i$ a teda je
	dokázateľná z $T$.
	\item Môže sa stať, že $C_i$ je axioma (x) $P(x_1,  \ldots, x_n) \iff
	A(x_1, \ldots, x_n)$. \par $C_i^*: A_i \iff A_i$, je dokázateľná pomocou
	vety o variantoch.
	\item Ak $C_i$ je odvodená z $C_j$ a $C_k$, použijeme pravidlo modus
	ponens. $C_i^*$ je odvodené z $C_j^*$ a $C_k^*$ tým istým pravidlom.
	\item $C_i$ je odvodené z $C_k$ pravidlom zovšeobecnenia ($k<i$). Potom
	$C_i^*$ je odvodené z $C_k^*$ pomocou toho istého pravidla.
\end{enumerate}



Ak máme teóriu $T$ a každá jej $A \in T$ je otvorená (každá premenná je voľná),
hovoríme o otvorenej teórii.

\par Majme danú teóriu $T$ s jazykom $L$. Máme k nej zostrojiť teóriu $T_H$
(henkinova teória) s jazykom $L(C)$, kde $C$ je zjednotenie konštánt všetkých
rádov. Dôležité pre nás budú konštanty $c_A, c_{\neg A}$.

\begin{itemize}
	\item $A_{c_{\neg A}} \implies (\forall x) A$ (*)
	\item $(\exists ) A \implies A(c_A)$ (**)
\end{itemize}

$$A \iff \neg \neg A$$
$$c_{\neg A} \iff \neg A(x)$$
$$\neg A(c_{\neg\neg A}) \implies (\forall x)\neg A$$
$$(\exists x) A \implies A(c_{\neg\neg A})$$
(Použili sme prenexnú operáciu).

\par Ak $c_{\neg A}$ je konštanta z axiomy s (*), tak hovoríme, že prislúcha
(patrí) formule A. Keď zoberieme formulu $A$.

$$(\forall x) A \implies A(t)$$ -- $t$ je term bez premenných v $L(C)$

$$A_{[c_{\neg A}]} \implies (\forall x) A \mbox{.......} c_{\neg A}$$

$\delta(T)$ množina na jazyku $L(C)$ prislchajúu konštante $C$ má ... axiomy
identity a ... a formúl z $T$.


\paragraph{Lema 1} Nech $A$ je formula na jazyku $L$ a ďalej, nech $A'$ je
uzavretá inštancia formuly $A$ na jazyku $L(C)$. Ak platí, že formula $A$ je
dokázateľná v pôvodnej teórii, potom $A'$ je tautologickým dôsledkom konečne
mnoho formúl z $\delta(T)$

\paragraph{Dôkaz} Nech $A_1, A_2, \ldots, A_n$ je dôkaz formuly $A$ v teórii
$T$. Indukciou podľa dĺžky dôkazu ukážeme, že ľubovoľná uzavretá formula v
jazyku $L(C)$ inštancia $A_j'$ formuly $A_j$ je tautologickým dôsledkom konečne
veľa formúl z $\delta(T)$.

\par Môžu nastať tieto prípady:
\begin{itemize}
	\item $A_j$ je axioma výrokovej logiky. Potom $A_j'$ je opäť tautológia.
	$A$ je tautologickým dôsledkom prázdnej množiny predpokladov.
	\item $A_j$ je axioma tvaru $(\forall x) B \implies B_x[t]$. Axioma
	špecifikácie. $A_j: (\forall x) B' \implies B_x'[t]]$ -- leží v
	$\delta T$. $t'$ je term bez premenných a je v $L(C)$. 
	\item $A_j$ je axioma $(\forall x) (C \implies D) \implies (C \implies
	(\forall x) D)$, $x$ nie je voľná v $C$.
	\par Uvažujme formulu $(\forall x) (C \implies D) \implies (C \implies
	(\forall x) D_x[c_{\neg D}])$. Táto formula patrí to $\delta(T)$, lebo
	je to inštancia axiomy špecifikácie.
	\par $ D_x[c_{\neg D}] \implies (\forall x) D$ -- opäť patrí do
	$\delta(T)$ (opäť (*))
	\par Tvrdíme, že formula $A_j$ je tautologický dôsledok horeuvedených 
	formúl. Budeme uvažovať takúto formulu:
	t -- $(A_1 \implies (B_1 \implies D_1)) \implies ((D_1 \implies C_1)
	\implies (A_1 \implies (B_1 \implies C_1)))$ Tvrdíme, že táto formula
	je teoréma.

	\par Dôkaz formuly: Za predpoklady si zoberiem formuly $A_1, (A_1
	\implies (B_1 \implies D_1))$, $B_1$, $(D_1 \implies C_1)$. Z týchto
	predpokladov dokážem odvodiť $C_1$.

	\par $A_1$ bude formula $(\forall x)(C \implies D)$, $B_1: C$, $C_1:
	(\forall x) D$, $D_1: D_x[c_{\neg D}]$

	\item $A_j$ je axioma z $T$, potom $A_j' \in \delta(T)$.
	\item $A_j$ je axioma identity alebo axioma rovnosti, potom $A_j'$ patrí
	do $\delta(T)$.

	\item $A_j$ dostaneme aplikáciou pravidla modu
	ponens z formúl $A_k$ a $A_l$, pričom $k, l < i$. $A_j'$ dostávame z
	$A_k'$ a $A_l'$ pomocou pravidla modus ponens.
	\item $A_j$ je odvodená z $A_k$ pravidlom zovšeobecnenia, pričom
	predpoklad je $k<j$. Teda $A_j: (\forall x) C$, teda $A_k: C$, x --
	premenná. Inštancia $A_j': (\forall x) C'$. Uvažujme inštanciu $A_k'$
	formuly $A_k$ v takomto tvare: $A_k:  C_k'[c_{\neg C'}]$. Podľa
	indukčného predpokladu je táto formula tautologickým dôsledkom konečne
	veľa formúl z $\delta (T)$. Uvažujme formulu $$C_x'[c_{\neg C'}]
	\implies (\forall x) C' \qquad \mbox{(***)}$$
	$A_j'$ je tautologickým dôsledkom formuly z $\delta(T)$. Dokázali sme
	formulu $A'$ tak, že sme nepoužili pravidlo zovšeobecnenia.
\end{itemize} 

\paragraph{Definícia} Hovoríme, že formula $A$ je \emph{kvázitautológia}, ak je
tautologickým dôsledkom inštancií axiom identity a rovnosti.

\paragraph{Veta (Hilbert-Ackermann)} Otvorená teória $T$ v jazyku $L$ (s
rovnosťou) je sporná práve vtedy, keď existuje (kvázi-)tautológia, ktorá je
disjunkciou negácií inštancií axiom z $T$.

\paragraph{Dôkaz} Najprv ľahšia implikácia: Predpokladajme, že sú splnené
podmienky vety, tak potom $T$ je sporná. Zoberme si nejakú formulu $A$. Berieme
$\neg A_1 \lor \neg A_2 \lor \ldots \lor \neg A_n$, kde $A_i$ sú inštancie axiom
z $T$. Pomocou de Morganovho zákona dostaneme $A_1 \land A_2 \land \ldots \land
A_n$. Ak platí $T \models A_i$, potom $T \models A_1 \land \ldots \land A_m$, a
teda $T$ je sporná, lebo je z nej dokázateľná ľubovoľná formula $B$, pretože
$\models A \implies (\neg A \implies B)$ a $T \models B$.

\par Naopak: Predpokladajme, že otvorená teória $T$ je sporná. $x \neq x$, $T
\models x \neq x$. Z toho vyplýva, že ak si zoberieme ľubovoľnú konštantu $r \in
C$, tak potom dostávame inštanciu formuly: $r \neq r$. To je uzavretá inštancia
teorémy vety z $T$, čiže podľa  Lemy 1 eistuje $A_1, A_2, \ldots A_k$  z $\delta
(T)$ také, že $r \neq r$ je tautologickým dôsledkom $A_1$ až $A_k$, teda z
predpokladov inštancií je dokázateľné $A_1, A_2, \ldots, A_k \models r \neq r$.
Platí $A_1 \implies A_2 \implies \ldots \implies A_k \implies r \neq r$ --
tautológia. $p \implies q \iff \neg p \lor q$. $\neg A_1 \lor \neg A_2 \lor
\ldots .. \lor \neg A_k \lor \neg (r=r)$ je tautológia. $A_i$ sú inštancie z
$\delta (T)$. Posledná formula je tautológia, ktorá je disjunkciou negácií
inštancií z $\delta (T)$.

\paragraph{Definícia} Postupnosť formúl $A_1, A_2, \ldots A_n$ nazveme
\emph{špeciálnou}, ak $\neg A_1\lor \neg A_2 \lor \ldots \lor \neg A_n$ je
tautológia. (Vieme) Ak $T$ je sporná teória, potom existuje špeciálna postupnosť
z $\delta (T)$



(pokračovanie)
\paragraph{Definícia} \emph{Stupeň konštanty} $c_{\neg A}$. $(\forall x) A$
uzavretá formula na jazyku $L(C)$. Hovoríme, že konštanta $c_{\neg A}$ spojená
axiomou $$A_{c_{\neg A}} \implies (\forall x) A$$ s formulou $(\forall x) A$ je
stupňa $n$, ak formula $(\forall x) A$ obsahuje $n$ výskytov kvantifikátorov
$\forall$ alebo $\exists$. 

\par Množina formúl $\delta_n(T)$ vznikne z $\delta(T)$ vynechaním všetkých
formúl stupňa $>n$ pre konštantu $c_{\neg A} \in C$. Treba si uvedomiť, že
stupeň konštanty $c_{\neg A}$ je vždy aspoň $1$. $\delta_0(T)$ -- uzavreté
inštancie, axiomy z množiny $T$ a axiom identity a rovnosti.

\par (Dokázali sme) Ak teória $T$ je sporná, existuje špeciálna postupnosť,
ktorá patrí do $\delta(T)$. Nech $n$ je najmenšie také $n$, že špeciálna
postupnosť z $\delta(T)$ je obsiahnutá v $\delta_n(T)$:

$n=0$. Musíme nájsť špeciálnu postupnosť prislúchajúcu do $T$.
Predpokladáme, že pre $n=0$ máme postupnosť $A_1, A_2, \ldots A_k$ --
špeciálna a patrí do $\delta_O(T)$. $c_A, c_{\neg A}$ nahradíme pomocou
premenných $A_1', A_2', \ldots A_k'$ (opäť špeciálna postupnosť). $A_1',
A_2', \ldots, A_n'$ sú všetky formuly, ktoré patria do $T$. $A_1', A_2',
\ldots A_k'$ budú inštancie axiom identity a rovnosti. $\neg A_1' \lor
\neg A_2' \lor \ldots \lor \neg A_n' \leftarrow A_{n+1} \leftarrow
\ldots \leftarrow A_k'$ -- kvázitautológia (vyplýva z toho, že táto
postupnosť je špeciálna). $A_1', A_2', \ldots A_n' \in T$.

\paragraph{Lema} Ak $n>0$ a existuje špeciálna postupnosť z $\delta_n(T)$, tak
potom existuje špeciálna postupnosť aj z $\delta_{n-1}(T)$.

\paragraph{Dôkaz} (pokračovanie dôkazu) Hilbert-Ackermanovej vety.
$\delta_0(T)$. Vieme vytvoriť postupnosť $B_1, B_2, \ldots, B_q \in T$.

\paragraph{Poznámka} Elementárna aritmetika je otvorená teória s konečným počtom
axiom. Hilbertova arigmetika, Presburgerova (??) aritmetika (aritmetika so
symbolmi $0, S, +$. Peanova aritmetika (1931).

\paragraph{Definícia} 
\begin{enumerate}
\item
Hovoríme, že formula $A$ je existenčná, ak $A$ je v
prenexnom tvare a všetky kvantifikátory v prefixe sú existenčné .
\item Hovoríme, že formula $A$ je univerzálna, ak $A$ je v prenexnom tvare a
všetky kvantifikátory sú univerzálne.

\end{enumerate}

\paragraph{Lema 3} Uzavretá existenčná formula $A$ je dokázateľná v predikátovej
logike (s rovnosťou) práve vtedy, keď istá disjunkcia otvoreného jadra formuly
$A$ je kvázitautológia.

\paragraph{Dôkaz} $A: (\exists x_1)(\exists x_2) \ldots (\exists x_n)B$
--formula v prenexnom tvare, $B$ je otvorené jadro. 

\begin{enumerate}
	\item Formulá $A$ je dokázateľná $\iff$ teória s jedinou špeciálnou
	axiomou $\neg A$ je sporná.
	\item Ak použijeme prenexné operácie, dostávame $\models \neg A \iff
	(\forall x_1)\ldots (\forall x_n) \neg B$. $\neg A \iff (\forall x_1)
	\ldots  (_n) \neg B$.

	\item Formula $A$ je dokázateľná $\iff$ keď teória $T$ s jedinou
	špeciálnou axiomou $\neg B$ je sporná. $\neg B$ je otvorená. Keď
	použijeme Hilbert-Ackermannovu vetu, dostávame $\neg B_1, \neg B_2
	\ldots, \neg B_m$ -- nad $\neg B$. Platí $\neg \neg B_1 \lor \neg \neg
	B_2 \lor \ldots \lor \neg \neg B_m$, čo je to isté ako $B_1 \lor B_2
	\lor \ldots \lor B_m$, čo je to isté, ako Hilbert-Ackermannova veta.
\end{enumerate}


\par Teóriu $T$ rozširujeme do $T_H$ s jazykom $L(C)$ a tú zase do $T_R$ s
$L(C)$. $A_x[C_{\neg A}] \implies (\forall x) A$ -- toto môžeme aj obrátiť:
$$(\forall x) A \implies A_x[c_{\neg A}]$$ -- vďaka axiome špecifikácie.
\par Ak teraz vezmeme formulu $B$ takú, že $A\iff B$, potom platí $B_x[c_{\neg
B}] \implies (\forall x) B$. 
$$
\begin{array}{lll}
T_H &\models& A_x[c_{\neg A}] \iff B_x[c_{\neg B}] \\
T_H &\models& A_x [c_{\neg A}] \iff (\forall x) A \\
T_H &\models& A \iff B \\
T_H &\models& (\forall x) A \iff (\forall x) B \\
T_H &\models& B_x[c_{\neg B}] \iff (\forall x) B \\
\end{array}
$$


$$(\forall x)(A\iff B) \implies c_{\neg A} = c_{\neg B}$$

\paragraph{Lema 4} Pre ľubovoľnú teóriu $T$ je $T_R$ konzervatívne rozšírenie.

\par Majme formulu $A$, ktorá je uzavretá a napísaná v prenexnom tvare.
Indukciou podľa počtu všeobecných kvantifikátorov budeme formulu transformovať
do tzv. Herbrandovho variantu. Ak formula $A$ má tvar:
$$ A: (\exists x_1) \ldots (\exists x_n) (\forall y) B, \qquad n \geq 0$$
$$ A^*: (\exists x_1) \ldots (\exists x_n) B_y[f(x\ldots x_n)]$$, $A^*$ je $A_H$

$A^{**}, L \cup \{ f,g,h, \ldots \}$.

\paragraph{Veta} (Herbrandova) Uzavretá formula v prenenom tvare je dokázateľnáv
predikátovej logike (s rovnosťou) práve vtedy, keď istá disjunkcia inštancií
otvoreného jadra je kvázitautológia.

\paragraph{Dôkaz} $A_H$ -- je existenčná formula. Je dokázateľná práve vtedy,
keď vezmeme do úvahy lemu 3 a platí formulácia našej vety. $L'$ .... ??? . Stačí
ukázať, že:
$$\models_L A \iff \models_{L'} A_H$$.
Implikácia zľava doprava je ľahšia -- vieme ukázať, že $\models_{L'} A \implies
\models_{L'} A^*$. Premennú $y$ sme nahradili funkciu $n$ premenných. Všimnime
si:

$$ \models_{L'} (\forall y) B \implies B_y[f(x_1\ldots x_n)]$$; axioma
špecifikácie. $(\exists x_1)(\exists x_2), \ldots (\exists x_n)$. Potom platí aj
$\models A \implies B$, $\models (\exists x) A \implies (\exists x) B$



\paragraph{Veta} (o zavedení funkčného symbolu) Nech formula $(\exists y) A$ je
dokázateľná v teórii $T$ s jazykom $L$. Nech $x_1, x_2, \ldots, x_n$ sú všetky
voľné premenné, ktoré sa vyskytujú vo formuli $(\exists y) A$. Nech $T'$ vznikne
z $T$ pridaním nového $n$-árneho funkčného symbolu $f$ a pridaním axiomy
$A_y[f(x_1, \ldots, x_n)]$. Potom $T'$ je konzervatívne rozšírenie teórie $T$.

\paragraph{Dôkaz} Nech $B$ je uzavretá formula teórie $T$ (s jazykom $L$).
Predpokladáme, že $B$ je teorémou (vetou teórie) $T'$. $T' \models B$. Máme
ukázať, že aj v $T \models B$. Predpokladajme, že formula $B$ má dôkaz v $T'$ a
že v tom dôkaze vystupujú $a_1, a_2, \ldots, a_n$ -- špeciálne axiomy teórie
$T$, prípadne axioma $(*)$. Na základe tohto predpokladu je v predikátovej
logike dokázateľná nasledovnaá formula.

$$(1) \models (\forall x_1) (\forall x_2) \ldots (\forall _n) a_y [f(x_1 \ldots
x_n)] \implies B_1 \implies B_2 \implies \ldots \implies B_n \implies B$$

$B_1, \ldots, B_n$ sú ...ny funkcií $A_1, A_2, \ldots, A_n$. 

$$ A_y[f(x_1, \ldots x_n)], A_1, \ldots, A_n \models B$$

Označme si $C$ -- prenexný tvar nasledovnej formuly:
$$A \implies B_1 \implies \ldots \implies B_n \implies B$$
Potom:

$$ (2) (\exists x_1) (\exists x_2) \ldots (\exists x_n) C_y[f(x_1, \ldots,
_n)]$$ je prenexný tvar $(1)$

Uvažujme $D: (\exists x_1) \ldots (\exists x_n) (\forall y) C$. Potom z tohto
vyplýva $D$ je formula jazyka $L$ a neobsahuje novo zavedený symbol $f$. Ak
konštruujeme herbrandovský variant tej podformuly ($D_H$) (meníme veľké
kvantifikátory) a na prvom kroku dostávame $D^*: (\exists x_1) \ldots (\exists
x_n) C_y [f(x_1, \ldots x_n)]$ -- prenexný tvar $(1)$. Z Herbrandovej vety
dostávame, že $D^*$ je dokázateľné, z tadiaľ $(D^*)_H$ je dokázateľné a z toho 
$(4) T \models D$

\par Prenexnými operáciami dostávame  $$(5) (\exists x_1) (\exists x_2) \ldots
(\exists x_n) (\forall y) (A \implies B_1 \implies \ldots \implies B_n \implies
B)$$

$$(6) T \models (\forall x_1 ) (\forall x_2) \ldots (\forall x_n) (\exists y)
A \implies B_1 \implies B_2 \implies \ldots \implies B_n \implies B$$

\par Každý z týchto predpokladov je dokázateľný v $T$, dostávame $T \models B$ a
teda $T'$ je konzervatívne rozšírenie $B$.

Mám formulu $A$, ktorá je uzavretá, idem priradiť skolemov tvar (skolemov
variant).
\begin{enumerate}
	\item Ak formula $A$ je univerzálna, potom $A_S$ je formula $A$
	\item Ak $A$ je tvaru $(\forall x_1) \ldots (\forall x_n) (\exists y) B,
	n\geq 0$, $f$ je funkčný symbol, kladieme $A_S: (\forall x_1) \ldots
	(\forall x_n) B_y[f(x_1,\ldots, x_n)]$, $(A')C ...$
\end{enumerate}

(1*) $A' \models A$ a ... $A_S \models A$
$ \models B_y[f(x_1, \ldots, x_n)] \implies (\exists y) B$ -- duálny tvar
axiomy špecifikácie. Teraz môžeme sformulovať Skolemovu vetu:

\paragraph{Veta} (Skolemova) V ľubovoľnej teórii $T$ môžeme zostrojiť otvorenú
teóriu $T'$, ktorá je konzervatívnym rozšírením teórie $T$.
\paragraph{Dôkaz} Mám teóriu $T$ s jazykom $L$. Zostrojím si teóriu $T_1$ s tým
istým jazykom a tá teória $T_1$ sú uzávery prenexných tvarov axiom z $T$. Z vety
o uzávere a vety o prenexnom tvare platí, že $T_1$ je konzervatívne rozšírenie
$T$ a naopak, $T$ je konzervatívne rozšírenie $T_1$. $T \equiv T_1$.

\par Teória $T_2$ vznikne z teórie $T_1$ tak, že ľubovoľnej formuli $A \in T_1$
zostrojíme skolemov variant $A_S$. V $T_2$ je konzervatívne rozšírenie $T_1$
podľa vety o zavedení funkčného symbolu. 

\par Ideme vytvárať teóriu $T_3$ -- vznikne z $T_2$ vynechaním všetkých axiom z
$T_1$. Podľa $(1^*)$ dostávame, že $T_2 \equiv T_3$.

\par Ďalej vytvárame teóriu $T_4$ -- pozostáva z otvorených jadier $T_3$, teda
$T_3 \equiv T_4$. Keď to zhrnieme, dostávame: $T \equiv T_1$, $T_2$ je
konzervatívne rozšírenie $T_1$. $T_2 \equiv T_3 \equiv T_4$, a z toho $T_4$ je
konzervatívne rozšírenie $T$.

\paragraph{Veta} (o zavedení funkcie pomocou definície) Majme teóriu $T$ s
jazykom $L$ a nech $x_1, \ldots, x_n, y$ sú navzájom rôzne premenné, ktoré sa
vyskytujú voľne vo formule $D$. Nech platí:
\begin{enumerate}
	\item $T\models  (\exists y) D$
	\item $T\models D \implies (D_y[y] \implies y = t)$
\end{enumerate}
Nech $L'$ vznikne z $L$ pridaním nového $n$-árneho funkčného symbolu $f$ a $T'$
z $T$ pridaním axiomy $$(3) y=f(x_1, \ldots x_n) \iff D$$ (definícia axiomy). Potom
$T'$ je konzervatívne rozšírenie $T$ a ku každej formule $A$ na jazyku $L'$
existuje formula $A^*$ na jazyku $L$ taká, že platí $$(4) T' \models A \iff A^*$$

\paragraph{Dôkaz} Najprv ukážeme, ako ku formuli $A$ priradiť formulu $A^*$, a
potom ukážeme, že $T'$ je konzervatívne rozšírenie $T$.

\par Zostrojíme $A^*$ z $A$ tak, aby platilo $(4)$. Formulu $A$ máme na jazyku
$L'$, problematický je symbol $f$ -- vyskytuje sa v atomických podformuliach.
Nech funkčný symbol $f$ sa vyskytuje vo formuli $A$ a nech je to ten
najvnútornejší výskyt. Je to nejaký $f(t_1, \ldots t_n)$, pričom $t_1, \ldots,
t_n$ už neobsahujú $f$. 

$$A: B_z[f(t_1, \ldots, t_b)]$$, a naše $z$ sa nevyskytuje vo formuli $A$, a ani
v definujúcej formuli $D$. Položme $A^*$  tvare:
$$(5) (\exists z) D'_{x_1, \ldots, x_n, y}[t_1, \ldots, t_n, z] \land B^*)$$,
pričom $D'$ je variant $D$, v ktorej nie je viazaná žiadna premenná, ktorá sa
vyskytuje vo formuli $A$. Z vety o variantoch a definujúcej axiomy $(3)$
dostávame, že na jazyku $L'$ je dokázané $ (6) \models_{L'} z = f(t_1, \ldots, t_n) \iff D'[t_1,
\ldots t_n, z]$. Z vety o ekvivalencii dostávame, že platí $$(7) T' \models
(\exists z) z = f(t_1, \ldots, t_n) \land B^*) \iff f^*$$. Z indukčného
predpokladu dostávame $$(8) T' \models B \iff B'$$

\par $(1) T' \models (\exists z) (z = f (t_1, \ldots, t_n) \land B) \iff A^*$ z
vedy o .... ($\models (\exists x) (x=t \land A) \iff A_x[t]$)
$$T' \models B_z[t_1, \ldots, t_n] \iff A^*$$, na ľavej strane je formula $A:
B_z[t_1, \ldots t_n]$.

\par Použijeme vetu o zavedení funkčného symbolu. Uvažujme $S$ -- teóriu z
jazyka $L'$, ktorá vznikne z $T$ pridaním axiomy:
$$(10) D_y[f(x_1, \ldots, x_n)]$$.

Ukážeme, že $T'$ a $S$ sú ekvivalentné a platí $T' \equiv S$. Potrebujeme
ukázať, že $(10)$ je teorémou $T'$ a $(3) y=f(x_1, \ldots, x_n) \iff D$ je
teorémou (vetou) $S$. V $T' \models f(x_1, \ldots x_n) = f(x_1, \ldots, _n)
\implies D_y[f(x_1, \ldots _n)]$, formula je inštancia axiomy $(3)$. Ak
použijeme inštanciu na axiomu identity $(3)$ $T' \models D_y[f(_1, \ldots,
x_n)]$. Potrebujeme ukázať, že $3$ je dokázateľná v $S$. Vyjdeme z tvaru $(11)
\models_{L'} y = f(x_1, \ldots x_n) \implies (D \iff D_y[f(x_1, \ldots, x_n)])$.

$$(12) \models_{L'} D_y[f'(x_1, \ldots, x_n)] \implies (y = f(x_1, \ldots _n
\implies D) $$

$$(13) S \models (y = f(x_1, \ldots, x_n) \implies D) \mbox{ z $(10)$}$$

$$(14) \models_{L'} D_y[f(x_1, \ldots, x_n) ] \implies (D \implies y = f(x_1,
\ldots x_n)] [T \models D \implies [D_y[t] \implies y = t) (2)$$


$$S \models D \implies y = f(x_1, \ldots x_n)$$


\chapter{Skúška}
\section{Písomná časť na konci semestra}

Na konci semestra existuje k predmetu písomka.
Táto písomka by sa mala skladať z 5 príkladov.
\begin{enumerate}
    \item Preniesť formulu do prenexného tvaru. Ako podotázka môže byť
        dôkaz niektorej z prenexných operácii.

    \item Napíšte axiómy nejakej teórie s rovnosťou, jej jazyk,
        špeciálne symboly a model (napríklad Teória telies alebo grúp)

    \item Dokážte Lindenbaumovy/Henkinovu/inú z ľahších viet počas
        semestra. Prípadne niečo na štýl 
        ``definujte konzervatívne rozšírenie + príklad''.
        Táto otázka je teoretická otázka.

    \item Daná je množina $S$. Nájdite unifikátor, Skolemov tvar.
        Proste, nejaká úloha z automatického dokazovania.

    \item Metódou rezolvent ukážte nesplniteľnosť zadanej množiny
        klauzúl $S$.
\end{enumerate}

\section{Samotná skúška}

\startFIXME
\begin{itemize}
	\item Dokázať $4$ vety
	\item je daná formula, nájsť prenexný tvar, skolemov tvar, metódou
	rezolvent ukázať, či platí alebo neplatí, napíšte teóriu s rovnosťou
	\item 3 alebo 4 definície
\end{itemize}
\stopFIXME

\end{document}
