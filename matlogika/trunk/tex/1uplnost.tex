\section{Veta o úplnosti}

G\"odelova veta, ktorú si teraz vyslovíme a dokážeme, má 2 varianty.

\begin{veta}[G\"odel, 1]
    Nech $T$ je teória v jazyku $L$ a nech $A$ je
    ľubovoľná formula jazyka $L$. Potom $T \provable A \iff T \models A$,
    čiže $A$ je dokázateľná práve vtedy keď
    je splnená v každom modeli teórie $T$.
\end{veta}

\begin{veta}[G\"odel, 2]
    Teória $T$ je bezosporná práve vtedy, ked $T$ má model.
\end{veta}

\begin{poznamka}
    Varianta 1 G\"odelovej vety vyplýva z varianty 2.
\end{poznamka}

\begin{dokaz}[Poznámky]
    Veta o dedukcii mala nasledovný dôsledok:

    Ak máme teóriu $T$, $A$ je jej formula a $A'$ je uzáver formuly $A$,
    tak $T \provable A \iff T \union \{ \neg A' \}$ je sporná teória,
    t.j. podľa vety 2 nemá $T \union \{ \neg A' \}$ model.

    Toto znamená, že v každom modeli teórie $T$ je pravdivý uzáver $A'$.
    Z toho dostávame $T \models A' \Rightarrow T \models A$
    (ak zoberieme ľubovoľný model $\mathcal{M}$ teórie $T$, formula
    $A'$ je v ňom splnená ale to nutne znamená, že ak formula $A$ v
    ňom musí byť splnená)
    a teda z platnosti vety 2 vyplýva veta 1.
\end{dokaz}

\begin{dokaz}[2. G\"odelovej vety]
    Budeme sa snažiť zostrojiť model pre teóriu, ktorá je bezosporná.
    Majme bezospornú teóriu v jazyku $L$. Potrebujeme v prvom rade $M$ --
    univerzum. K dispozícii máme len syntaktické prostriedky teórie.
    Preto ako kandidát na $M$ prichádza do úvahy
    množina termov bez premenných.
    Tieto termy majú jednoznačnú realizáciu (sami sebe budú realizáciou).
    Teda, všetky objekty teórie budú charakterizované termami.

    Ďalšou otázkou je, ako definovať splniteľnosť. Malo by platiť, že
    formula je splniteľná práve vtedy keď je dokázateľná, čize
    \begin{equation*}
    A[e] \iff T \provable A
    \end{equation*}

    Pri konštrukcii modelu sa nám pritrafia isté nepríjemnosti, ktoré bude 
    treba riešiť:
    \begin{enumerate}
    \item \label{en:model_problem_konst}
        Jazyk $L$ neobsahuje žiadne konštanty (a teda žiadne termy bez
        premenných).

    \item \label{en:model_problem_rovnost}
        Ak jazyk $L$ bude jazyk s rovnosťou, môže sa stať, že v teórii $T$
        bude $T \provable t=s$, ale $t$ a $s$ sú rôzne termy bez
        premenných (rôzne konštanty).

    \item \label{en:model_problem_neg}
        Nech $m$ je ľubovoľná realizácia jazyka $L$ a $A$ je uzavretá
        formula jazyka $L$. Potom práve jedna z formúl $A$, $\neg A$ je
        pravdivá, ale žiadna z nich nemusí byť dokázateľná v $T$.

    \item \label{en:model_problem_korektnost}
        Môže sa stať, že uzavretá formula $(\exists x)B$ je dokázateľná v teórii
        $T$, ale pre žiaden term $t$ bez premenných formula $B_x[t]$ nie
        je dokázateľná v $T$. To znamená, že
        podľa Tarského definície pravdivosti je $(\exists x)B$
        nepravdivá, čo je spor s vetou o korektnosti.
    \end{enumerate}

    Ako odstránime tieto nedostatky?
    Odstránenie bodu \ref{en:model_problem_rovnost} je jednoduché --
    riešime vhodnou faktorizáciou, čiže zavedieme si množinu $\tau$,
    čo je množina všetkých termov bez premenných
    a na nej zavedieme reláciu ekvivalencie.

    Body č. \ref{en:model_problem_neg}, \ref{en:model_problem_konst} a 
    \ref{en:model_problem_korektnost} sa riešia tzv.
    úplným konzervatívnym rošírením teórie (Henkinovským).
    Budú to tzv.  konzervatívne teórie (na pôvodnom jazyku
    nezískame žiadne nové teorémy a ani nestratíme žiadne).
    Nachvíľu teda opustíme dôkaz G\"odelovej vety, aby sme si mohli niečo
    porozprávať o Henkinovej teórii. K dôkazu sa vrátime, keď na to budeme
    mať pripravenú pôdu.
\end{dokaz}

\begin{definicia}[Úplná teória]
%%% {{{
    Hovoríme, že teória $T$ s jazykom $L$ je \emph{úplná}, ak $T$ je
    bezosporná teória a pre ľubovoľnú uzavretú formulu $A$ na jazyku
    $L$ buď $A$ alebo $\neg A$ je dokázateľná v $T$.
%%% }}}
\end{definicia}

\begin{poznamka}
    Pojem úplnosti teda zodpovedá nasledujúcej konštrukcii:
    Majme teóriu $T$ nad jazykom $L$ a jej model $\mathcal{M}$.
    Model $\mathcal{M}$ rozhoduje o pravdivosti každej uzavretej
    formuly.
    Označme ako $T_h(\mathcal{M})$ množinu všetkých
    pravdivých uzavretých formúl $T$.
    Potom platí, že $T_h(\mathcal{M})$ je úplná.

    Je dôležité si uvedomiť, že neberieme otvorené formuly --
    napr. formula $x=0$ v elementárnej aritmetike nemusí byť pradivá, 
    pretože závisí od ohodnotenia $x$.
\end{poznamka}

\begin{poznamka}
    V úplnej teórii (a teda špeciálne v $T_h$) nemôže nastať problém 
    \ref{en:model_problem_neg}, ktorý sme spomínali.
\end{poznamka}

\begin{definicia}[Henkinova teória]
%%% {{{ Henkinova teoria
    Hovoríme, že teória $T$ s jazykom $L$ je \emph{Henkinova}, ak pre
    ľubovoľnú uzavretú formulu $(\exists x)B$ jazyka $L$ platí
    \begin{equation*}
        T \provable (\exists x)B \implies B_x[c]
    \end{equation*}
    pre nejakú konštantu $c$.
%%% }}}
\end{definicia}

\begin{poznamka}
    Ak je teória Henkinova, tak sme vyriešili problémy
    \ref{en:model_problem_konst}, \ref{en:model_problem_korektnost}.
\end{poznamka}

\begin{lema}
    Ak $T$ je úplná a Henkinova teória, tak potom $T$ má model.
    \label{lema:uplna_henkinova}
\end{lema}
\begin{dokaz}
    Nech $L$ je jazyk teórie $T$, $\tau$ je množina všetkých
    termov jazyka $L$ bez premenných. Na množine $\tau$ definujem reláciu
    ekvivalencie nasledovne:
    \begin{equation*}
        \forall t_1, t_2 \in \tau :
            t_1 \equiv t_2 \iff T \provable t_1 = t_2
    \end{equation*}
    Rovnosť je reflexívna, symetrická, tranzitívna, teda týmto
    spôsobom definovaná relácia je relácia ekvivalencie a rozdeľuje
    nám množinu $\tau$ na triedy ekvivalencie:
    \begin{equation*}
        [t] = \{ s \in \tau: t \equiv s\}
    \end{equation*}
    Týmto sme vyriešili problém \ref{en:model_problem_rovnost}.

    Zadefinujme si univerzum $M$ tak, že ho budú tvoriť vyššie popísané
    triedy ekvivalencie.
    Nech $f$ je ľubovoľný $n$-árny funkčný symbol a nech
     $[t_1], [t_2], \ldots, [t_n] \in M$. Definujeme funkciu $f$ v relačnej
     štruktúre $M$ nasledovne:
     \begin{equation*}
      f_\mathcal{M}([t_1], \dots, [t_n]) = [f(t_1, \dots t_n)]
     \end{equation*}
     \fixme{Ma tam byt $m,M,\mathcal{M}$ ?}
    Bolo by potrebné ukázať, že táto definícia je konzistentná. Teda
    $f_\mathcal{M}([t_1], \dots, [t_n])=f_\mathcal{M}([s_1], \dots, [s_n])$
    ak $s_i \equiv t_i$, čiže nezáleží na výbere reprezentantov.
    Taktiež je dobré si uvedomiť, že
    $[t_{x_1,\dots,x_n}[t_1,\dots,t_n]]=t[e]$, ak platí $e(x_i/t_i)$ resp.
    $e(x_i)=t_i$.

    Podobne, nech $P$ je $n$-árny predikátový symbol rôzny od $=$
    (predikát $=$ sme si už zaviedli). V tom prípade definujeme
    \begin{equation*}
     ([t_1], \ldots [t_n]) \in P_\mathcal{M} \iff
        T \provable P(t_1, \ldots, t_n)
    \end{equation*}
    Zostáva nám ukázať, že nami definované $\mathcal{M}$ je
    naozaj modelom teórie $T$, čo bude predmetom nasledujúcej vety.
\end{dokaz}

\begin{veta}[O kanonickej štruktúre]
    Nech $T$ je úplná Henkinovská teória a nech $\mathcal{M}$ je
    definované podľa dôkazu lemy \ref{lema:uplna_henkinova}.
    Potom $T \provable A \iff \mathcal{M} \models A$.
\end{veta}

\startFIXME
\begin{dokaz}
    Budeme postupovať matematickou indukciou vzhľadom na zložitosť
    formuly:
    \begin{itemize}
    \item[1:]
        \begin{itemize}
        \item Formula $A$ je tvaru $P(t_1,\dots,t_n)$. Pritom $t_1,
            \dots t_n$ sú termy bez premenných (inak by formula $A$
            nebola uzavretá). Podľa definície splniteľnosti platí, že
            $A$ je pravdivá v $T$ práve vtedy keď je dokázateľná v
            $T$. Zeberme si teraz ohodnotenie $e$ také, že
            $t_i[e] = t_i$ (Čiže každý term realizujeme sám sebou).
            Potom $A$ je pravdivá práve vtedy, keď je splnená aspoň
            pre jedno ohodnotenie $e$.
            Teda
            $\mathcal{M} \models A \iff
                ([t_1],[t_2],\dots,[t_n]) \in P_\mathcal{M} \iff
                T \provable P(t_1,t_n, \dots, t_n)$.
            \fixme{O co islo v tomto bode?}
        \item Formula môže byť tvaru $A:t_1=t_2$. V tom prípade
            $\mathcal{M} \models t_1 = t_2 \iff [t_1]=[t_2] \iff
                T \provable t_1=t_2$
        \end{itemize}
    \item[2:]   Máme niekoľko možností, ako mohla formula $A$ vzniknúť:
        \begin{itemize}
        \item $A:\neg B$. Na formulu $B$ sa vzťahuje IP, keďže $B$ je
            uzavretá formula. Teda
            $\mathcal{M} \models A \iff \mathcal{M} \not \models B$.
            Vieme, že teória $T$ je úplná, že buď $T \provable A$
            alebo $T \provable B$. Keďže je bezosporná, platí práve
            jedna možnosť. Tým sme ale ukázali, že
            $\mathcal{M} \models A \iff T \provable A$.
        \item $A: B \implies C$
        \item $A: (\forall x) B$

        \end{itemize}
    \end{itemize}
\end{dokaz}
% \begin{unknown}
$A$ -- ľubovoľná uzavretá formula z $L$, potom $m \models A \iff T \provable A$.
$P(t_1, t_2, \ldots t_n)$ -- atomická formula,  uzavretá, $t_1. \ldots t_n$ sú
termy bez premenných.

$m \models A \iff ([t_1], \ldots, [t_n]) \in P_m \iff T \provable P(t_1,\ldots
t_n)$ pre ľubovoľné ohodnotenie.

\par $A: t_1 = t_2$. $m \models t_1 = t_2 \iff [t_1] = [t_2] \iff T \provable t_1 =
t_2$.

\par $A: \neg B$. Indukčný predpoklad: pre $B$ bolo tvrdenie dokázané ($m\models
B \iff T \provable B)$. Formula $A$ je pravdivá v $m$ práve vtedy, keď formula $B$
nie je pravdivá $B$ nie je pravdivá, $T \unprovable B, T \vdash \neg B$.

\par $A. B \implies C$, pričom $B$ a $Cc$ sú dokázané podľa indukčného
predpokladu. Ak platí $m \models B \implies C$, potom $m \models \neg B \lor m
\models C$. $T \provable \neg B \lor T \vdash C$.

\begin{align*}
    T \provable B \implies C \\
    \provable C \implies (B \implies C)
\end{align*}

Opačná implikácia:
\begin{equation*}
    T\provable B\implies C\Rightarrow m \models B \implies C
\end{equation*}

Teória $T$ je úplná teória a $B \implies C$ je uzavretá. Pre každú uzavretú
formulu platí, jedna z možností $T \provable \neg B$ (potom $m \models \neg B$)
alebo $T \provable B$, a z toho vyplýva, že $T \vdash C$, a teda $m \models C$.
Tvrdenie je teda dokázané.

\begin{equation}
    m \models A \iff T \provable A
\end{equation}

\par Predpokladajme, že formula $A$ je v tvare $A: (\forall x) B$. Pre každú
inštanciu formuly $B$ tvrdenie platí, $\provable (\forall x) B \implies B_x[t]$
(axióma špecifikácie). Potom aj $T \provable B_x[t]$ (lebo teória je Henkinova). $m \models B[e(x/t)]$.

\par Ak je pravdivá negácia, potom $T \provable \neg A$, čo znamená, že $(\exists
x) \neg B$. $T \provable (\exists x) \neg B \implies B_x[c]$, a to znamená, že $T
\provable \neg B_x[c]$.

\par Nech $A$ je ľubovoľná formula z teórie $T$. Pre ňu platí, že $T \provable A$
(predpoklad). Potrebujem dokázať, že $T \provable A \iff m \models A$. Zoberiem si
$A'$ -- uzáver formuly $A$. Na $A'$ sa vzťahuje tvrdenie, ktoré sme už dokázali,
a teda $m \models A' \iff T \provable A'$. Podľa vety o uzávere:
$m \models A \iff  m\models A' T \provable A' \iff T \vdash A$

\par Pre ľubovoľnú teóriu, ktorá je idealizovaná (Henkinova a úplná), viem
zostrojiť model.

\stopFIXME
