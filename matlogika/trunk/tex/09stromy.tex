\section{Sémantické stromy}

Už sme povedali, že Herbrandovské interpretácie sú to pravé orechové,
čo chceme overovať. Ostáva nám ale vyriešiť problém, ako ich nejakým
spôsobom postupne preberať. A práve na to nám budú slúžiť sémantické
stromy.

\begin{definicia}[Kontrárna dvojica]
    Majme nejaký (elementárna formulu) $A$. Dvojicu
    $\langle A, \neg A \rangle$ nazývame kontrárnou dvojicou.

    Ak klauzula obsahuje kontrárnu dvojicu, potom je tautológia.
\end{definicia}

\begin{definicia}[Sémantický strom]
    Nech $S$ je množina klauzúl a $A$ je Herbrandovské báza pre
    množinu $S$. Pod sémantickým stromom pre množinu klauzúl $S$
    budeme rozumieť zakorenený dolu rastúci strom
    v ktorom je každej hrane pripísaná množina atómov alebo negácií
    atómov (teda vlastne množina literálov) z Herbrandovskej bázy,
    pričom platí:
    \begin{enumerate}
	\item Z každého vrchola stromu vychádza konečný počet hrán.
            Ooznačme ich ich $l_1, l_2, \ldots l_n$.
            Nech $Q_i$ je konjunkcia všetkých literálov
	    pripísaných hrane $l_i$, potom požadujeme aby 
            $Q_1 \lor Q_2 \lor \ldots \lor Q_n$ bola tautológia.

	\item Označme pre vrchov $v$ symbolom $I(v)$
            zjednotenie všetkých množín literálov pripísaných hranám cesty,
            ktorá vedie z koreňa do vrcholu $v$. Potom $I(v)$
            nesmie obsahovať kontrárne dvojice.
    \end{enumerate}
\end{definicia}

\startFIXME

\paragraph{Definícia} Majme $A=\{ A_1, A_2, \ldots A_n \}$ je Herbrandovská báza
pre množinu $S$. Hovoríme, že sémantický strom prislúchajúci množine klauzúl $S$
je úplný, ak pre každý vrchol $v$ platí, že pre každé $i$ končiace vo vrchole,
$I(v)$ obsahuje $A_i$ alebo $\neg A_i$ (ale nie obe naraz).
\todo{check}

\paragraph{Príklad} $A: \{ P, Q, R \}$, viď obrázok
\ref{fig-semantic_tree_example}.
\begin{figure}[h]
	\caption{Príklad sémantického stromu}
	\centering\includegraphics{img/xx/2009-03-24.1.mps}
	\label{fig-semantic_tree_example}
\end{figure}



\begin{figure}[h]
	\centering\includegraphics{img/xx/2009-03-26.1.mps}
	\caption{Príklad stromu}
	\label{0326_tree1}
\end{figure}
\paragraph{Príklad}
$S=\{P(x), ... \}$, H-báza je $\{P($ ... \todo{}

\paragraph{Príklad} $S=\{P(x), Q(f(x)) \}$, herbrandovská báza pre množinu
klauzúl je $\{ P(a), Q(a), P(f(a)), Q(f(a)), \ldots \}$ -- je nekonečná, teda aj
prislúchajúci sémantický storm je nekonečný (obr. \ref{0326_tree2}). %pic2
\begin{figure}[h]
	\centering\includegraphics{img/xx/2009-03-26.2.mps}
	\caption{Ďalší sémantický strom}
	\label{0326_tree2}
\end{figure}

Nech $v$ je vrchol a $I(v)$ je \todo{}. Jednotlivé vetvy sémantického stromu nám
preberajú všetky interpretácie sémantickej bázy. Nech $I'(v)$ je množina
základných inštancií odmietnutá (úsek zamieta množinu klauzúl), potom na tomto
mieste môžeme strom \emph{odrezať}.

\paragraph{Definícia} Vrchol $v$ sémantického stromu pre množinu klauzúl $S$ sa
nazýva \emph{odmietajúcim}, ak  $I(v)$ odmieta niektorú základú inštanciu
klauzuly z množiny $S$, no ľubovoľný predchádzajúci vrchol $v'$ ($v'$ predchádza
$v$), $I(v')$ neodmieta žiadnu základnú inštanciu klauzúl z $S$.

\paragraph{Definícia}  Hovoríme, že sémantický strom $T$ pre množinu klauzúl $S$
je \emph{uzavretý}, ak každá vetva $T$ končí odmietajúcim vrcholom.

\paragraph{Definícia} Vrchol $v$ sémantického \todo{alebo uzavretého???} stromu
$T$ pre  množinu klauzúl $S$ nazývame \emph{akceptujúcim}, ak všetky nasledujúce
vrcholy vrchola $v$ sú odmietajúce.

\paragraph{Príklad} $S= \{ P, Q \lor R, \neg P \lor \neg Q, \neg R \lor \neg P
\}$. Herbrandovskú báza množinu $S$ vyzerá nasledovne: $\{P, Q, R\}$ (obr.
\ref{0326_tree3}, \ref{0326_tree_cut}).

\begin{figure}[h]
	\centering\includegraphics{img/xx/2009-03-26.3.mps}
	\caption{A ďalší sémantický strom\ldots}
	\label{0326_tree3}
\end{figure}
%pic3+4

\begin{figure}[h]
	\centering\includegraphics{img/xx/2009-03-26.4.mps}
	\caption{Orezaný sémantický strom}
	\label{0326_tree_cut}
\end{figure}


\paragraph{Príklad} $S=\{ P(x), P(x) \lor Q(f(x)), \neg Q(f(x)) \}$.
Herbrandovská báza pre množinu klauzúl $S$ vyzerá: $\{ P(a), Q(a), P(f(a)),
Q(f(a)), \ldots \}$ (obr. \ref{0326_tree_cut2}). %pic5


\begin{figure}[h]
	\centering\includegraphics{img/xx/2009-03-26.5.mps}
	\caption{Ďalší orezaný sémantický strom}
	\label{0326_tree_cut2}
\end{figure}
