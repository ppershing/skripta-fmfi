\section{Herbrandovské univerzum}

\begin{poznamka}
    V nasledujúcom texte budeme používať a medzi sebou zamieňať
    nasledujúce výrazy s rovnakým významom:
    ``nie je splniteľná'', 
    ``je sporná'' a
    ``je protirečivá''. Taktiež, občas budeme zamieňať ich negácie
    medzi sebou.
\end{poznamka}

\begin{definicia}[Herbrandovské univerzum množiny klauzúl]
    Nech $H_0$ je množina konštánt, ktoré sa vyskytujú v $S$.
    Ak $S$ neobsahuje žiadnu konštantu, tak položíme $H_0=\{ a \}$,
    kde $a$ je nejaká konštanta.
    Ďalej definujme $H_{i+1}$ ako zjednotenie $H_{i}$ a množiny všetkých termov
    tvaru $f^{(n)}(t_1,\dots, t_n)$, kde $f^{(n)} \in S$ a
    $t_1, \dots, t_n \in H_i$.
    Množinu $H_i$ nazývame
    \emph{herbrandovské univerzum $i$-tej úrovne}.
    Herbrandovské univerzum množiny klauzúl definujeme ako zjednotenie
    cez všetky úrovne:
    \begin{equation*}
        H=\bigcup_{i=0}^{\infty} H_i
    \end{equation*}
\end{definicia}

\begin{priklad}
    Majme množinu klauzúl
    $S= \{ P(x),\ \neg P(x) \lor \neg P(f(x))\}$.
    Potom herbrandovské univerzá jednotilivých úrovní sú
    \begin{align*}
        H_0& = \{ a \} \\
        H_1& = \{ a,\ f(a) \} \\
        H_2& = \{ a,\ f(a),\ f(f(a)) \} \\
         \vdots\ \ &\\
        H_{\phantom{0}} &= \{ a,\ f(a),\ f(f(a),\ f(f(f(a))),\ \ldots \}
    \end{align*}
\end{priklad}

\begin{priklad}
    Nech $S=\{P(x) \lor R(x),\ R(z), T(y) \lor \neg W(y) \}$, teda
    množina $S$ neobsahuje žiadnu konštantu.
    Preto kladieme $H_0 = \{ a \}$. Dostávame
    $H_0=H_1=\dots=H=\{a\}$.
\end{priklad}

\begin{priklad}
    Uvažujme množinu klauzúl $S=\{P(f(x),a g(y), b)\}$. Potom
    \begin{align*}
        H_0 &= \{ a,\ b\} \\
        H_1 &= \{ a,\ b,\ f(a),\ g(a),\ f(b),\ g(b) \} \\
        H_2 &= \{ a,\ b,\ f(a),\ g(a),\ f(b),\ g(b),\
                f(f(a)),\ f(g(a)),\ f(f(b)),\ f(g(b)),\
                g(f(a)),\ g(f(b)),\ g(g(a)),\ g(g(b)) \}
    \end{align*}
\end{priklad}

\begin{definicia}[výraz]
    Pod pojmom výraz budeme chápať 
    term, množinu termov,
    klauzulu, množinu klauzúl,
    atóm, množinu atómov,
    literál, množinu literálov.
\end{definicia}

\begin{definicia}[podvýraz]
    Podvýraz výrazu $F$ je ľubovoľný výraz, ktorý sa vyskytuje v $F$.
\end{definicia}

\begin{definicia}[základný výraz]
    Ľubovoľný výraz, ktorý neobsahuje premenné, sa nazýva základný výraz
    (základný term, základný atóm, základná klauzula, základný literál, ...).
\end{definicia}

\begin{definicia}[základná inštancia]
    Základnou inštanciou klauzuly $C$ z množiny klauzúl $S$ je klauzula,
    ktorú dostaneme zámenou všetkých premenných
    za prvky Herbrandovského univerza.
\end{definicia}

\begin{definicia}[Herbrandovská báza]
    Nech $S$ je množina formúl. Potom množina atómov
    tvaru $p^{(n)}(t_1, \ldots, t_n)$ pre všetky $n$-árne predikáty,
    ktoré sa vyskytujú v $S$ a všetky termy $t_i$ z Herbrandovského univerza
    nazývame Herbrandovskou bázou $S$.
    Sú to atómy takého tvaru, že sa v nich nevyskytuje žiadna
    premenná.
\end{definicia}

\begin{priklad}
    Majme množinu $S = \{P(x),\ Q(f(x)) \lor R(y) \}$ a majme klauzulu
    $C: P(x)$. 
    Herbrandovské univerzum $H$ je
    $H = \{ a,\ f(a),\ f(f(a)),\ f(f(f(a))),\ \dots \}$.
    Základné inštancie pre $C$ sú $P(a), P(f(a)), P(f(f(a))), \dots$.
\end{priklad}

Položme si otázku, čo znamená interpretovať množinu $S$ na
Herbrandovskom univerze $H$. Musíme poznať kodnoty konštánt,
interpretáciu funkčných a predikátových symbolov. Budeme uvažovať
špeciálnu interpretáciu, takzvanú $H$-interpretáciu, pri ktorej
nebudeme interpretovať predikátové symboly (necháme si to ako keby na
neskôr).

\begin{definicia}[H-interpretácia]
    Nech $S$ je množina klauzúl. Ďalej nech $H$ je herbrandovské
    univerzum pre množinu klauzúl $S$ a $I$ je interpretácia v množine
    klauzúl $S$ nad $H$.
    Hovoríme, že interpretácia $I$ je herbrandovská interpretácia 
    (alebo tiež H-interpretácia), ak platí: 

    \begin{enumerate}
	\item Interpretácia $I$ zobrazuje všetky konštanty na samé
            seba, t.j. konštante $a \in S$ priradí tú istú konštantu $a \in H$.

	\item Ak $f^{(n)}$ je $n$-árny funkčný symbol a
            $h_1, \dots, h_n$ sú prvky herbrandovského univerza $H$.
            Potom funkciu $f$ budeme v $I$ realizovať ako funkciu,
            ktorá zobrazuje $(h_1, \dots, h_n) \in H^n$ na element
            $f^{(n)}(h_1,\dots,h_n) \in H$.
    \end{enumerate}
\end{definicia}

\begin{poznamka}
    Ako sme už spomínali, nekladieme žiadne obmedzenia 
    na interpretáciu predikátov.

    Uvažujme ako príklad množinu $A = \{ A_1, A_2, \dots \}$.
    Nech je to herbrandovská báza pre množinu klauzúl $S$.
    Herbrandovskú interpretáciu určíme tak, že $I$ zadáme ako
    $I=\{ m_1, m_2, \ldots \}$, kde
    $m_i$ bude buď $A_j$ (ak $A_j$ je pravdivé) alebo
    $\neg A_j$ (ak $A_j$ je nepravdivé).
\end{poznamka}

\begin{priklad}
    Nech $S=\{ P(x) \lor Q(x),\ R(f(y)) \}$.
    Herbrandovské univerzum je
    $H=\{a,\ f(a),\ f(f(a)),\ \ldots \}$.
    V $S$ sa vyskytujú unárne predikáty $P,Q,R$.
    Herbrandovská báza je potom
    \begin{equation*}
        A: \{ P(a),\ Q(a),\ R(a),\ 
            P(f(a)),\ Q(f(a)),\ R(f(a)),\ \dots\}.\footnote{
            Všimnime si, že v danej množine je aj $R(a)$, hoci
            by sme možno očakávali, že to musí začínať $R(f(a))$
            }
    \end{equation*}
    Môžeme mať nasledovné interpretácie:
    \begin{equation*}
        I_1: \{ P(a),\ Q(a),\ R(a),\ 
        P(f(a)),\ Q(f(a)),\ R(f(a)),\ \dots \}
    \end{equation*}
    teda, predikáty sú vždy pravdivé. Alebo
    \begin{equation*}
        I_2: \{ \neg P(a),\ \neg Q(a),\ \neg R(a),\ 
            \neg P(f(a)),\ \neg Q(f(a)),\ \neg R(f(a)),\ \dots \}
    \end{equation*}
    čiže predikáty nie sú nikdy pravdivé.
    Ďalšia možná interpretácia je
    \begin{equation*}
        I_3: \{ P(a),\ Q(a),\ \neg R(a),\ 
            P(f(a)),\ Q(f(a)),\ \neg R(f(a)),\ \dots \}
    \end{equation*}
    V zásade pre každú možnú kombináciu si vieme vytvoriť
    interpretáciu.
\end{priklad}


\medskip
Ďalšou úlohou bude k ľubovoľnej interpretácii nad ľubovoľným univerzom
priradiť Herbrandovskú interpretáciu a zaviesť príslušné tvrdenia a definície.

\paragraph{Príklad} $S=\{ P(x) \lor Q(x). R(f(y)) \}$.  Jedna z možných
Herbrandovských interpretácií je nasledovná:
$$ I^* = \{ \neg P(x), \neg Q(a), \neg R(a), \neg P(f(a)), \neg Q(f(a)), \neg
R(f(a)), \ldots \}$$
$$ I_2^* = \{ P(a), Q(a), \neg R(a), P(f(a)), Q(f(a)), \neg R(f(a)), \ldots \}$$

\paragraph{Príklad} Majme množinu $S = \{ P(x), Q(y, f(y,a)) \}$. Oblasť
interpretácie $D = \{1,2\}$.

\begin{tabular}{|l|c|c|c|c|}
\hline
a & f(1,1) & f(1,2) & f(2,1) & f(2,2) \\
\hline
2 & 1 & 2 & 2 & 1 \\
\hline
\end{tabular}

\begin{tabular}{|l|c|c|c|c|c|}
\hline
P(1) & P(2) & Q(1,1) & Q(1,2) & Q(2,1) & Q(2,2) \\
\hline
T & F & F & T & F & T \\
\hline
\end{tabular}

$I^*$, H-im ??
Herbrandovská báza: $$A=\{ P(a), Q(a,a), P(f(a,a)), Q(a,f(a,a)), Q(f(a,a), a,
\ldots \}$$.

Hodnoty pre príslušné predikáty určíme pomocou zadaných tabuliek:
$$
\begin{array}{lll}
P(a) &=& P(2) \mbox{neplatí} \\
Q(a,a) &=& Q(2,2) \mbox{platí} \\
P(f(a,a)) &=& P(f(2,2)) = P(1) \mbox{platí} \\
Q(f(a,a),a) &=& Q(f(2,2),2) = Q(1,2) \mbox{platí} \\
Q(a,f(a,a)) &=& Q(1,f(2,2)) = Q(2,1) \mbox{platí} \\
Q(f(a,a),f(a,a)) &=& Q(f(2,2), f(2,2)) = Q(1,1) \mbox{neplatí} \\
\end{array}
$$

$$I^* = \{ \neg P(a), Q(a,a), P(f(a,a)), \neg Q(a,f(a,a)), Q(f(a,a),a), \ldots
\}$$

\par
Ďalšia vec, čo sa môže stať -- majme množinu klauzúl, ktorá neobsahuje
konštantu: $S=\{P(x), Q(y,f(y)) \}$. Máme danú interpretáciu $I$ s oblasťou
$D=\{1,2\}$, oblasti priradím dve interpretácie $I_1^*$ a $I_2^*$, pričom v
prvej bude $a$ interpretované ako $1$ a v druhej ako $2$.

\paragraph{Definícia} Majme interpretáciu $I$, množinu klauzúl $S$ na oblasti
interpretácií $D$. $I^*$ je H-interpretácia priradená ku $I$, vytvorená
nasledovne: Nech $h_1, h_2, \ldots h_n$ patria do Herbrandovského univerza $H$,
$I(h_1) \mapsto d_i \in , i=1 \ldots n$. $P^{(n)}$ patrí do množiny klauzúl $S$.
Ak $P^{(n)}(d_1, \ldots d_n)$ je splnený a pravdivý v $I$, potom
$P^{(n)}(h_1,\ldots,h_n)$ je splnený a pravdivý v $I^*$.

\paragraph{Lema} Majme interpretáciu $I$ na oblasti $D$. Nech táto interpretácia
vyhovuje množine klauzúl $S$. Potom ľubovoľná H-interpretácia $I^*$, ktorá je
priradená (zodpovedá) $I$ taktiež vyhovuje množine klauzúl $S$.

\paragraph{Dôkaz} Majme množinu klauzúl $S=\{C_1, C_2, \ldots , C_n\}$. Nech
klauzula (disjunkcia literálov) $C_i = L_{i_1} \lor L_{i_2} \lor \ldots \lor
L_{i_{r_i}}$, $i=1,\ldots,n$. Ľubovoľný literál $L_{i_j}$ je tvaru $P^{(n)}(d_1,
\ldots, d_n)$. Množina klauzúl $C_1 \land C_2 \land \ldots \land C_n$ je
pravdivá práve vtedy, keď je pravdivá každá klauzula.
$$I(h_i) = d_i$$
$$L^*_{i_j}: P^{(n)}(h_1, \ldots, h_n)$$

\paragraph{Veta} Množina klauzúl $S$ nie je splniteľná práve vtedy, keď $S$ je
nepravdivá pri všetkých H-interpretáciách $S$. 

\paragraph{Dôkaz} Jedna implikácia (zľava doprava) je očividná:

\subparagraph{Zľava doprava} Nech $S$ nie je splniteľná, potom je nepravdivá pre ľubovoľnú interpretáciu
na ľubovoľnej oblasti, a teda aj pre ľubovoľnú H-interpretáciu na H-univerze.

\subparagraph{Obrátená implikácia} Nech množina klauzúl $S$ je nepravdivá pre
ľubovoľnú H-interpretáciu množiny klauzúl $S$. Pre spor predpokladajme, že
existuje interpretácia $I$ s oblasťou $D$ pre množinu klauzúl $S$, ktorá
vyhovuje $S$. Uvažujme $I^*$, ktorá je priradená (zodpovedá) interpretácii pre
množinu klauzúl $S$. Podľa lemy, ktorú sme dokázali, ak vyhovuje $I$, vyhovuje
aj $I^*$, čo je spor. $\Box$

\paragraph{Poznámka}
\begin{enumerate}
	\item Základná inštancia $C'$ (neobsahuje premenné), $C$ je splniteľná v interpretácii $I$
	práve vtedy, keď existuje základný literál $L'$:
	$$ L' \in C'$$
	$$ C' \cap I \neq \emptyset, L' \in I = \{m_1, m_2, \ldots \} $$

	\item Klauzula $C$ je splnená v interpretácii $I$ $\iff$ každá jej
	základná inštancia $C$ je splnená v $I$.

	\item Klauzula $C$ je odmietnutá (vyvrátená) v $I$ $\iff$ keď existuje
	aspoň jedna taká základná inštancia $C'$, ktorá je odmietnutá v $I$
	(vyvrátená).

	\item Množina klauzúl $S$ nie je splniteľná v $I$, ak existuje aspoň
	jedna základná inštancia $C'$ klauzuly $C$, ktorá nie je splniteľná v
	$I$. $$C \in S$$
\end{enumerate}


\paragraph{Príklad} Uvažujme klauzulu $C=\neg P(x) \lor Q(f(x))$ a
interpretáciu:
$$
\begin{array}{lll}
I_1 &=& \{ \neg P(a), \neg Q(a), \neg P(f(a)), \neg Q(f(a)), \neg P(f(f(a))),
\neg Q(f(f(a))), \ldots \} \\
I_2 &=& \{ P(a), Q(a), P(f(a)), Q(f(a)), P(f(f(a))), Q(f(f(a))), \ldots \} \\
I_3 &=& \{ P(a), \neg Q(a), P(f(a)), \neg Q(f(a)), P(f(f(a))), \neg Q(f(f(a))),
\ldots \} \\
\end{array}
$$

\paragraph{Príklad} Uvažujme množinu klauzúl $S=\{P(x), \neg P(x)\}$ a
interpretácie:
$$
\begin{array}{lll}
I_1 &=& \{ P(x) \} \\
I_2 &=& \{ \neg P(x) \} \\
\end{array}
$$

Množina nie je splnená ani jednou interpretáciou.

