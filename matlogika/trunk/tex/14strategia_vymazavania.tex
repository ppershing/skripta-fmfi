\section{Stratégia vymazávania}
\startFIXME
Na základe vety o úplnosti, majme nejakú množinu klauzúl $S$, postupne si z nej
vytváram rezolventy. Ak nie je splniteľná, po konečnom počte krokov dostávam
prázdnu klauzulu.

\begin{priklad}
    Majme množinu klauzúl $S=\{P\lor Q, \neg P\lor Q, P \lor
    \neg Q, \neg P \lor \neg Q\}$. Metódou rezolvent ukážte, že $S$ nie je
    splniteľná.

    \par nejakým spôsobom zostrojujeme postupnosť klauzúl, nakoniec niekde dostanem
    prázdnu klauzulu. 
    $S^0 = S$, $S^n = \{ \mbox{rezolventy} C_1, C_2 | C_1 \in S^0 \cup S^1,
    \ldots \cup S^{n-1} \land C_2 \in S^{n-1}\}, n=1, 2, \ldots$. Takýmto spôsobom
    by sme po $39$ krokoch dostali odpoveď. Niektoré klauzuly sa vyskytnú v
    popísanom prístupe dvakrát. Môžu sa tam vyskytnúť tautológie.
\end{priklad}

\begin{definicia}
    Klauzula $C$ je podklauzulou klauzuly $D$ (alebo pohlcuje
    $D$) práve vtedy, keď existuje substitúcia $\sigma$ taká, že platí $C\sigma
    \subseteq D$. $D$ nazývame nadklauzulou $C$.
\end{definicia}

\begin{priklad}
    $C = P(x)$, $D = P(A) \lor Q(a)$, $\sigma = \{a/x\}$,
    $C\sigma = P(a)$, $C\sigma \subseteq D$.
\end{priklad}

\paragraph{Poznámka} $D$ je identicky rovná $C$. (??)
Ak klauzula $D$ je inštancia $C$, tak $D$ je nadklauzula $C$. (??). 

\par
Možno to sformulovať takto: Majme $(S^0 \cup S^1 \cup \ldots \cup
S^{n-1}\}$, berieme z nej klauzulu $C_1$ a $C_2$. V prípade, že ako rezolventa
$C_1$ a $C_2$ nevznikne tautológia a nebude to ani nadklauzula niektorej z tých,
ktoré sú už vypísané, vypíšem ju. Takto sa zbavujem tautológií a nadklauzúl.

\begin{priklad}
    $S=S^0$ a :
    \begin{itemize}
        \item $P\lor Q$
        \item $\neg P \lor Q$
        \item $P\lor \neg Q$
        \item $\neg P \lor \neg Q$
    \end{itemize}
    $$
    S_1:
    \begin{array}{ll}
    (5)& Q (1) (2)\\
    (6)& P (1) (3)\\
    (7)& \neg P (2) (4)\\
    (8)& \square (6) (7)\\
    \end{array}
    $$
\end{priklad}

\subsection{Algoritmus  pohltenia}

$$\theta = \{ a_1 / x_1, a_2 / x_2, \ldots, a_n / x_n \}$$. $x_1, x_2, \ldots
x_n$ sú premenné, ktoré sa vyskytujú v $D$. $a_1, a_2, \ldots a_n$ sú nové
konštanty, kotére sa nevyskytujú v $C$, $D$.

$D\Theta$ je základná... .

$$D = L_1 \lor L_2 \lor \ldots \lor L_m$$
$$D \theta = L_1\Theta \lor L_2 \Theta \lor \ldots \lor L_m \Theta$$
$$\neg D\Theta = \neg L_1 \Theta \land \neg L_2 \Theta \land \ldots \land \neg
L_m \Theta$$
Algoritums preveruje, či klauzula $C$ je podklauzulou $D$.

\begin{enumerate}
    \item Nech $W = \{ \neg L_1 \Theta, \neg L_2 \Theta, \ldots, \neg L_m
    \Theta \}$

    \item Kladieme $k=0$ a $\mathcal{U}^0 = \{ C \}$
    \item Ak $\mathcal{U}^k$ obsahuje $\square$, tak koniec $C$ je pod D. V
    opačnom prípade kladieme $\mathcal{U}^{k+1} = \{ \mbox{rezolventa} C_1 a
    C_2 | C_1 \in \mathcal{U}^{k} \land C_2 \in U\}$. 
    \item Ak $\mathcal{U}^{k+1}$ je $\emptyset$, tak koniec, $C$ nie je
    podklauzula $D$. V opačnom prípade kladieme $k=k+1$ a prejdeme ku kroku

\end{enumerate}

\begin{poznamka}
    $\mathcal{U}^k, \mathcal{U}^{k+1}$, klauzuly z
    $\mathcal{U}^{k}$ sú konečné. $\mathcal{U}^0, \mathcal{U}^1, \ldots \square$.
\end{poznamka}

\begin{dokaz}
    Predpokladajme, že $C$ je podklauzula $D$. Na základe našej
    definície existuje substitúcia $\sigma$, že $C\sigma \subseteq D$. Teda
    $C(\sigma \circ \Theta) \subseteq D\Theta$. Literály z $C\sigma \circ \Theta$
    môžeme vynechať pomocou jednotkových klauzúl z $W$. ... Algoritmus skončí svoju
    činnosť.
    \par
    Obrátené tvrdenie: predpokladajme, že algoritmus zakončuje prácu na treťom
    kroku. Odmietnutie môžeme znázorniť nasledujúcim obrázkom:

    \todo{obrazok}

    $$C_0, N_1 ,\ldots B_r \in W$$
    $$C(\sigma_0 \circ \sigma_1 \circ \sigma \circ \sigma_r) = \{ \neg B_0, \neg
    B_1, \ldots \neg B_r\} \subseteq D\Theta$$
    $$\lambda = \sigma_0 \circ \sigma_1 \circ \sigma_2 \ldots \circ \sigma_r \implies
    C \lambda \subseteq D\Theta$$

    $\sigma$, ktorá dostaneme z $\lambda$ tak, že v každom komponente $\lambda$
    nahradíme  konštantu $a_i$ premennou $x_i$, $i=1, 2, 3, \ldots$. $C\sigma
    \subseteq D$. $C$ je pod $D$.
\end{dokaz}

\begin{priklad}
    $C = \neg P(x) \lor Q(f(x), a)$. $D = \neg P(h(y)) \lor
    Q(f(h(y)),a) \lor P(z)$. Zistite, či klauzula $C$ je podklauzulou $D$.

    \par $y$ a $z$ sú premenné v $D$. $\Theta = \{ b/y, c/z\}$. Konštanty $b$, $c$
    nevystupujú v $C$, $I$. najprv vypočítame $D\Theta \neg P(h(b)) \lor
    Q(f(h(b)),a) \lor \neg P(c)$

    $$\neg D \Theta = P(h(b)) \land \neg Q(f(h(b)),a) \ lor P(c)$$
    $$W = \{P(h(b)), \neg Q(f(h(b)),a), P(c) \}$$
    $$\mathcal{U}^0 = C = \neg P(x) \lor Q(f(x),a)\}$$
    $\mathcal{U}^0$ neobsahuje $\square$, musíme vytvoriť $\mathcal{U}^1$. Urobíme
    príslušnú substitúciu v množine $\mathcal{U}^0$. Dostávam nasledovné rezolventy:
    $$\mathcal{U}^1 = \{ Q(f(h(b)),a), \neg P(h(b)), Q(f(b),a)\}$$. 
    \par
    $\mathcal{U}^1$
    nie je prádzna a neobsahuje prádznu klauzulu -- musím vytvoriť $\mathcal{U}^2$.
    V tomto sa už vyskytne prádzna klauzula, čo znamená, že $C$ pohlcuje klauzulu
    $D$.
\end{priklad}

\begin{priklad}
    $C=P(x,x)$ a $D=P(f(x),y) \lor P(y,f(x))$. Zistite, či $C$
    je podklauzula $D$.

    \paragraph{Riešenie} (1) $x$, $y$ sú premenné v $D$. $a$ a $b$ sú konštanty,
    ktoré sa nevyskytujú $C$, $D$. $\Theta = \{ a/x, b/y\}$. $D\Theta = P(f(a),b), \lor P(b,
    f(a))$.

    $$\neg D\Theta = \neg P(f(a),b) \lor \neg O(b,f(a))$$
    $$W = \{ \neg P(f(a),b), \neg P(b,f(a))\}$$
    $$\mathcal{U}^0 = P(x,x)$$


    \par (2) $\mathcal{U}^0$ neobsahuje $\square$, tak sa môže zistiť
    $\mathcal{U}^1$
    \par (3) $\mathcal{U}^1 = \emptyset$. Záver: $C$ nie je podklauzula $D$.
\end{priklad}

\begin{priklad}
    Majme formuly:

    \begin{enumerate}
        \item $P\implies S$
        \item $S \implies U$
        \item $P$
        \item $U$
    \end{enumerate}

    Dokážte, že formula 4 vyplýva z formúl 1, 2 a 3. 

    \paragraph{Riešenie} Prepíšeme si formuly do správneho tvaru, aby sme mohli
    použiť pravidlo rezolventy:
    \begin{enumerate}
            \item $\neg P \lor S$
            \item $\neg S\lor U$
            \item $P$
            \item $U$
    \end{enumerate}
    Snažíme sa nájsť negáciu -- chceme ukázať, že 
    \begin{enumerate}
            \item $\neg P \lor S$
            \item $\neg S\lor U$
            \item $P$
            \item $\neg U$
    \end{enumerate}

    nie je splniteľná. Zoberiem si rezolventu 1 a 3, dostávam $S$ (5). Keď zoberiem 
    2 a 4, dostávam $\neg S$ (6). Zoberiem 5 a 6, dostávam $\square$ (7).
\end{priklad}


\begin{priklad}
    Predpoklad: Študenti sú občania. Záver: Hlasy študentov sú
    hlasy občanov.

    \paragraph{Riešenie} 
    \begin{itemize}
            \item $S(x)$ označuje \uv{$x$ je študent}.
            \item $C(x)$ označuje \uv{$x$ je občan}.
            \item $V(x,y)$ označuje \uv{$x$ je hlas $y$}.
    \end{itemize}
    Predpoklad: $(\forall y) (S(y)\implies C(y))$. Študenti sú občania.
    Záver: $(\forall x) ((\exists y) (S(y) \land V(x,y)) \implies (\exists z)(C(z)
    \implies V(x,z)))$. Hlasy študentov sú hlasy občanov.

    Aká bude štandardná forma pre vyjadrenie predpokladu?
    \begin{enumerate}
            \item $\neg S(y) \lor C(y)$
                \par $\neg ((\forall x) ((\exists y)(S(y)\land V(x,y)) \implies
                (\exists x)(C(z) \land V(x,z)))) \iff 
                \neg ((\forall x)(\forall y)(\neg S(y)\lor \neg V(x,y)) \lor
                (\exists z)(C(z)\land V(x,z))) \iff 
                \neg ((\forall x)(\forall y)(\exists x)(\neg S(y) \lor
                \neg U(x,y)\lor (C(z) \land V(x,z)))) \iff
                (\exists x)(\exists y)(\forall z) (S(y) \land V(x,y)) \land
                (\neg C(z) \lor \neg V(x,z))$
                Teraz potrebujeme Skolemov normálny tvar:
                $(\forall z)(S(b) \land U(a,b)) \land(\neg C(z) \lor \neg
                V(a,z)))$
                \par Pre negáciu záver dostávane nasledujúce klauzuly:
            \item $S(b)$
            \item $V(a,b)$
            \item $\neg C(z) \lor \neg V(a,z)$.

            \item $C(b)$ z (1) a (2) (miesto $y$ dosadíme $b$)
            \item $\neg V(a,b)$ zo (4) a (5)
            \item $\square$ z (3)  a (6)
    \end{enumerate}

    \par Predpokladajme, že $b$ je študent, $a$ je hlas študenta $b$ a nie je hlas
    žiadneho občana. Pretože $b$ je študent, $b$ je občan. Okrem toho $a$ nemôže byž
    hlas $b$, pretože $b$ je občan a to nie je možné.
\end{priklad}
