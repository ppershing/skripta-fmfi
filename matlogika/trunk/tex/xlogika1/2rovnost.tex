\section{Predikátová logika s rovnosťou}

Predikátovú logiku môžeme rozšíriť o nové axiómy, ktoré budú
hovoriť o predikáte ``=''.

\par{Axiómy rovnosti:}
\noindent
\begin{itemize}
%%% {{{
    \item[R1:] ak $x$ je premenná, potom formula $x=x$ je axióma
    \item[R2:] ak $x_1,\dots,x_k, y_1, \dots, y_k$ sú premenné a 
        $f$ je $k$-árny funkčný symbol, potom je axiómou formula
        \begin{equation*}
            (x_1 = y_1) \implies ( (x_2 = y_2) \implies ( \dots
                \implies ((x_k = y_k) \implies
                    [f(x_1,\dots,x_k) = f(y_1,\dots,y_k)] \dots )))
        \end{equation*}
    \item[R3:] ak $x_1,\dots,x_k, y_1, \dots y_k$ sú premenné a 
        $P$ je $k$-árny predikátový symbol, potom je axiómou formula
        \begin{equation*}
            (x_1 = y_1) \implies ( (x_2 = y_2) \implies ( \dots
                \implies ((x_k = y_k) \implies
                    [P(x_1,\dots,x_k) \implies P(y_1,\dots,y_k)] \dots )))
        \end{equation*}
%%% }}}        
\end{itemize}

\begin{priklad}[Teória neostrého čiastočného usporiadania $\le$]
\noindent
%%% {{{
    \begin{itemize}
        \item[1.] $(\forall x) \langle x,x\rangle \in \varphi$ -- Identita

        \item[2.] $(\forall x) (\forall y) [
            (\langle x,y \rangle \in \varphi \land
             \langle y,x \rangle \in \varphi) \implies (x=y)]$ -- 
             Antisymetrickosť

        \item[3.] $(\forall x) (\forall y) (\forall z)
            [( \langle x,y \rangle \in \varphi \land 
               \langle y,z \rangle \in \varphi) \implies
               \langle x,z \rangle \in \varphi]$. Tranzitívnosť
    \end{itemize}
    Ak pridáme trichotomickosť, dostaneme teóriu neostrého
    usporiadania:
    \begin{itemize}
        \item[4.] $(\forall x) (\forall y) [x \not=y \implies
            (\langle x,y \rangle \in \varphi \lor 
             \langle y,x \rangle \in \varphi)]$

        \item[4'] $(\forall x) (\forall y) [x = y \lor
            (\langle x,y \rangle \in \varphi \lor 
             \langle y,x \rangle \in \varphi)]$
    \end{itemize}
%%% }}}    
\end{priklad}

\begin{lema}
Rovnosť je symetrická a tranzitívna.
    \begin{itemize}
        \item[1] $\provable (x=y) \implies (y=x)$ -- symetria
        \item[2] $\provable (x=y) \implies ((y=z) \implies (x=z))$ --
            tranzitívnosť
    \end{itemize}
\end{lema}
\begin{dokaz}
%%% {{{
\noindent
\begin{itemize}
    \item Symetria:
        \begin{itemize}
        \item[1] $\provable (x=y) \implies ((x=x) \implies ((x=x)
                \implies (y=x)))$
            pretože \\
                $\provable (x_1=y_1) \implies ((x_2=y_2) \implies
                    ((x_1=x_2) \implies (y_1=y_2)))$ je inštancia R3
        \item[2] $\provable A \implies (B \implies (B \implies C))$.
        \item[3] $\provable B \implies (B \implies (A \implies C))$ --
            2x použité pravidlo zámeny predpokladov + veta o dedukcii
        \item[4] $\provable (x=x) \implies (x=x) \implies (x=y)
                    \implies (y=x)$
        \item[5] $\provable x=x$ -- R1
        \item[6] $\provable (x=y) \implies (y=x)$ -- 2x MP na 5,4
        \end{itemize}

    \item Tranzitívnosť:
        \begin{itemize}
            \item[1] $\provable (y=x) \implies ((z=z) \implies
                ((y=z) \implies (x=z)))$, pretože \\
                $\provable (x_1 = y_1) \implies ((x_2=y_2) \implies
                 ((x_1=x_2) \implies (y_1=y_2)))$ je je R3.
            \item[2] $\provable A \implies (B \implies (C \implies D))$
            \item[3] $\provable B \implies (A \implies (C \implies D))$
                -- pravidlo zámeny predpokladov
            \item[4] $\provable (z=z) \implies ((y=x) \implies ((y=z) \implies
                (x=z)))$
            \item[5] $\provable z=z$ -- R1
            \item[6] $\provable (y=x) \implies ((y=z) \implies (x=z))$
                -- MP 4,5
            \item[7] $\provable (A \implies B) \implies ((B \implies
                C)) \implies (A \implies C))$ -- JS
            \item[8] $\provable ((x=y) \implies (y=x)) \implies [
                ((y=x) \implies ((y=z) \implies (x=z))) \implies
                ((x=y) \implies ((y=z) \implies (x=z)))]$ -- inštancia 7
            \item[9] $\provable (x=y) \implies (y=x)$ -- Symetria
            \item[10] $\provable
                ((y=x) \implies ((y=z) \implies (x=z))) \implies
                ((x=y) \implies ((y=z) \implies (x=z)))$ -- MP 9,8
            \item[11] $\provable ((x=y) \implies ((y=z) \implies (x=z)))$ -- MP 6,10
        \end{itemize}
\end{itemize}
%%% }}}
\end{dokaz}

\begin{veta}
    Nech $t_1,\ldots,t_n,s_1,\ldots,s_n$ sú termy, pričom platí
        $\forall i \in \{1,\dots,n\}:\; \provable t_i = s_i$.
    Potom
    \begin{itemize}
    \item[i)] Ak $t$ je term, ktorý vznikne z termu $s$ nahradením
        niektorých výskytov termov $s_i$ za $t_i$. Potom 
        $\provable t=s$.
    \item[ii)] Ak $A'$ je formula, ktorá vznikne z formuly $A$
    dosadením $t_i$ za niektoré termy $s_i$, okrem prípadov, keď
    $t_i$ je premenná $x$ v $(\forall x)$ resp. $(\exists x)$. Potom
    $\provable A \leftrightarrow A'$.
    \end{itemize}
\end{veta}

\begin{dokaz}
%%% {{{
\noindent
\begin{itemize}
    \item[i)] Dôkaz matematickou indukciou vzhľadom na zložitosť termu $t$.
        \begin{itemize}
        \item Ak $t$ je premenná alebo $t$ je $s_i$ pre nejaké $i$.
            Potom zjavne $\provable t=s_i$.

        \item Nech $t$ je $f(r_1,\dots,r_k)$, $s$ je $f(r_1',\dots,r_k')$.
            Pre $r_1,\dots,r_k$ platí IP $\provable r_i = r_i'$.
            Potom $\provable (r_1=r_1') \implies  \dots \implies
                (r_k = r_k') \implies (f(r_1,\dots,r_k) = f(r_1',
                    \dots, r_k'))$, čo $k$-násobným použitím MP na 
                    indukčný predpoklad vedie k
                    $\provable f(r_1,\dots,r_k) = f(r_1', \dots,
                    r_k')$.
                
        \end{itemize}
    \item[ii)]
    \todo{dopisat}
\end{itemize}
%%% }}}
\end{dokaz}
