\section{Predikátová logika s rovnosťou}

\par{Axiómy rovnosti:}
~
\begin{itemize}
%%% {{{
    \item[R1] ak $x$ je premenná, potom formula $x=x$ je axióma
    \item[R2] ak $x_1,\dots,x_k, y_1, \dots y_k$ sú premenné a 
        $f$ je $k$-árny funkčný symbol, potom je axiómou formula
        \begin{equation*}
            (x_1 = y_1) \implies ( (x_2 = y_2) \implies ( \dots
                \implies ((x_k = y_k) \implies
                    (f(x_1,\dots,x_k) = f(y_1,\dots,y_k)) \dots )))
        \end{equation*}
    \item[R3] ak $x_1,\dots,x_k, y_1, \dots y_k$ sú premenné a 
        $P$ je $k$-árny predikátový symbol, potom je axiómou formula
        \begin{equation*}
            (x_1 = y_1) \implies ( (x_2 = y_2) \implies ( \dots
                \implies ((x_k = y_k) \implies
                    (P(x_1,\dots,x_k) \implies P(y_1,\dots,y_k)) \dots )))
        \end{equation*}
%%% }}}        
\end{itemize}

\begin{priklad}[Teória neostrého čiastočného usporiadania $\le$]
~
%%% {{{
    \begin{itemize}
        \item[1.] $(\forall x) (x,x) \in \phi$ -- Identita
        \item[2.] $(\forall x) (\forall y) (((x,y) \in \phi \land
            (y,x) \in \phi) \implies (x=y))$ -- Antisymetrickosť
        \item[3.] $(\forall x) (\forall y) (\forall z)
            (((x,y) \in \phi \land (y,z) \in phi) \implies
                (x,z) \in \phi)$. Tranzitívnosť
    \end{itemize}
    Ak pridáme trichotomickosť, dostaneme teóriu neostrého
    usporiadania:
    \begin{itemize}
        \item[4.] $(\forall x) (\forall y) (x \not=y \implies
            ((x,y) \in \phi \lor (y,x) \in \phi))$
        \item[4'] $(\forall x) (\forall y) (x = y \lor
            ((x,y) \in \phi \lor (y,x) \in \phi))$
    \end{itemize}
%%% }}}    
\end{priklad}

\begin{lema}
~
    \begin{itemize}
        \item[1] $\provable x=y \implies y=x$ -- symetria
        \item[2] $\provable (x=y) \implies ((y=z) \implies (x=z))$ --
            tranzitívnosť
    \end{itemize}
\end{lema}
\begin{dokaz}
~
%%% {{{
\begin{itemize}
    \item Symetria:
        \begin{itemize}
        \item[1] $\provable (x=y) \implies (x=x) \implies (x=x) \implies (y=x)$
            pretože \\
                $\provable (x_1=y_1) \implies (x_2=y_2) \implies
                    (x_1=x_2) \implies (y_1=y_2)$ je inštancia R3
        \item[2] $\provable A \implies (B \implies (B \implies C))$.
        \item[3] $\provable B \implies (B \implies (A \implies C))$ --
            2x použité pravidlo zámeny predpokladov + veta o dedukcii
        \item[4] $\provable (x=x) \implies (x=x) \implies (x=y)
                    \implies (y=x)$
        \item[5] $\provable x=x$ -- R1
        \item[6] $\provable (x=y) \implies (y=x)$ -- 2x MP na 5,4
        \end{itemize}

    \item Tranzitívnosť:
        \begin{itemize}
            \item[1] $\provable (y=x) \implies ((z=z) \implies
                ((y=z) \implies (x=z)))$, pretože \\
                $\provable (x_1 = y_1) \implies ((x_2=y_2) \implies
                 ((x_1=x_2) \implies (y_1=y_2)))$ je je R3.
            \item[2] $\provable A \implies (B \implies (C \implies D))$
            \item[3] $\provable B \implies (A \implies (C \implies D))$
                -- pravidlo zámeny predpokladov
            \item[4] $\provable (z=z) \implies ((y=x) \implies ((y=z) \implies
                (x=z)))$
            \item[5] $\provable x=x$ -- R1
            \item[6] $\provable (y=x) \implies ((y=z) \implies (x=z))$
                -- MP 4,5
            \item[7] $\provable ((A \implies B) \implies (B \implies
                C)) \implies (A \implies C))$ -- JS
            \item[8] $\provable (x=y \implies y=x) \implies [
                (y=x \implies (y=z \implies x=z)) \implies
                (x=y \implies (y=z \implies x=z))]$ -- inštancia 7
            \item[9] $\provable (x=y) \implies (y=x)$ -- Symetria
            \item[10] $\provable
                (y=x \implies (y=z \implies x=z)) \implies
                (x=y \implies (y=z \implies x=z))$ -- MP 9,8
            \item[11] $\provable (x=y \implies (y=z \implies x=z))$ -- MP 6,10
        \end{itemize}
\end{itemize}
%%% }}}
\end{dokaz}

\begin{veta}
    Nech $t_1,\ldots,t_n,s_1,\ldots,s_n$ sú termy, pričom platí
        $\forall i \in \{1,\dots,n\}: \provable t_i = s_i$.
    Potom
    \begin{itemize}
    \item[i] Ak $t$ je term, ktorý vznikne z termu $s$ nahradením
        niektorých výskytov termov $s_i$ za $t_i$, potom 
        $\provable t=s$.
    \item[ii] Ak $A'$ je formula, ktorá vznikne z formuly $A$
    dosadením $t_i$ za niektoré termy $s_i$, okrem prípadov, keď
    $t_i$ je \fixme{}.
    $\provable A \leftrightarrow A'$.
    \end{itemize}
\end{veta}
