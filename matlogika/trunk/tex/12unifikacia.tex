\section{Substitúcia a unifikácia}

Vo výrokovej logike nebol problém hľadať kontrárne dvojice.
Zložitejšia situácia ale nastáva v prípade predikátovej logiky
prvého rádu. A práve tomu sa budeme venovať v tejto kapitole.
Uvažujme napríklad dve klauzuly
$C_1=P(x) \lor Q(x), C_2 = \neg P(f(x)) \lor R(x)$.
V nich neexistuje žiadna kontrárna dvojica. Ak však nahradím premennú
$x$ za $f(a)$ v prvej klauzule a za $a$ v druhej,
dostaneme základné inštancie
$C_1'=P(f(a)) \lor Q(f(a)), C_2'=\neg P(f(a)) \lor R(a)$.
Teraz môžeme definovať rezolventu; bude to $Q(f(a)) \lor R(a)$.
Mohli by sme postupovať aj všeobecnejšie -- nahraďme $x$ za $f(x)$ v
prvej klauzule a dostávame
$ C_1^*= P(f(x)) \lor Q(f(x)), C_2^*= \neg P(f(x)) \lor R(x)$.

Rezolventa potom bude $C^*= Q(f(x)) \lor R(x)$.
Vidíme teda, že sme získali 2 rôzne rezolventy, jednu viac všeobecnú
ako druhú. No a práve v ďalšom texte si formálne zadefinuujeme toto
dosadzovanie hodnôt a popíšeme spôsob, ako hľadať najvšeobecnejšie
rezolventy.

\begin{definicia}[Substitúcia]
    Pod substitúciou rozumieme konečnú množinu tvaru:
    $\{t_1/v_1, \ldots, t_n/v_n\}$, kde každé $v_i$ je premenná a všetky
    $v_i$ sú navzájom rôzne ($i \in\{1,\dots,n\}$).
    Ďalej požadujeme, že pre každé $i$ je term  $t_i$ rôzny od $v_i$.

    Ak $t_1, \ldots, t_n$ sú základné inštancie (teda termy bez premenných),
    tak substitúciu nazývame tiež základná substitúcia.
\end{definicia}

\begin{poznamka}
    Na označovanie substitúcií budeme používať grécke písmená.
    Špeciálne, prázdnu substitúciu označíme ako $\eps$.
\end{poznamka}

\begin{poznamka}
    Je dôležité si všimnúť, že poradie prvkov v definícii substitúcie je
    čisto antiintuitívne -- človek by očakával ``premenná/výraz ktorým ju
    máme nahradiť'' a nie ``výraz/premenná''.
\end{poznamka}

\begin{priklad}
    Jedna substitúcia môže byť napr.
    $\alpha = \{ f(z)/x,\ y/z\}$, teda $x$ nahrádzame za $f(z)$ a
    $z$ za $y$.
    Ďalšia môže byť $\beta = \{ a/x,\ g(y)/y,\ f(g(y))/z\}$.
\end{priklad}

\begin{definicia}
    Nech $\theta$ je ľubovoľná substitúcia a $E$ je nejaký výraz.
    Nech $\theta = \{ t_1/v_1, \dots, t_n/v_n\}$.
    Potom $E\theta$ označuje výraz, ktorý vznikne tak,
    že súčasne vo výraze $E$ nahradíme každý výskyt premenej $v_i$ termom $t_i$
    pre $i \in \{1,\dots,n\}$.
\end{definicia}

\begin{priklad}
    Majme substitúciu $\theta=\{a/x,\ f(b)/y,\ c/z\}$
    a výraz $E = P(x, y, z)$.
    Potom $E\theta = P(a, f(b), c)$.
\end{priklad}

\medskip
\noindent
Ďalšia operácia, ktorú budeme potrebovať, je operácia skladania
substitúcií.

\begin{definicia}[kompozícia substitúcii]
    Majme substitúciu $\theta = \{t_1/x_1, \ldots t_n/x_n\}$ a
    $\lambda = \{ u_1/y_1, \ldots u_m/y_m \}$.
    Zloženie (kompozícia) $\theta \circ \lambda$ substitúcií
    $\theta,\lambda$ je definovaná ako množina
    \begin{equation*}
        \{t_1 \lambda/x_1, \dots, t_n \lambda/x_n, u_1/y_1m \dots u_m/y_m \}
    \end{equation*}
    z ktorej navyše vyradíme vyradíme všetky členy $t_i\lambda/x_i$,
    pre ktoré platí, že $t_i \lambda = x_i$ (aby sme nesubstituovali
    identititami)
    a tiež vyradíme všetky $u_i/y_i$,
    pre ktoré $y_i \in \{x_1, x_2, \dots, x_n\}$ (lebo by sme mali
    predefinovanie správania sa substitúcie na $x_i$).
\end{definicia}

\begin{poznamka}
    Kompozícia $\theta \circ \lambda$ sa správa rovnako ako postupné
    aplikovanie $\theta, \lambda$. Čiže
    $E(\theta \circ \lambda) = (E\theta)\lambda$.
\end{poznamka}

\begin{priklad}
    Majme substitúcie
    \begin{align*}
        \theta = \{t_1/x_1,\ t_2/x_2\} = \{ f(y)/x,\ z/y\} \\
        \lambda = \{u_1/y_1,\ u_2/y_2,\ u_3/y_3\} = \{ a/x,\ b/y,\ y/z\}
    \end{align*}
    Potom
    \begin{equation*}
    \begin{split}
        \theta \circ \lambda &= 
            \{ t_1 \lambda / x_1,\ t_2\lambda/x_2,\
            u_1/y_1,\ u_2/y_2,\ u_3/y_3\} - \{\dots\} \\ 
        &= \{f(b)/x,\ y/y,\ a/x,\ b/y,\ y/z\} - \{y/y,\ a/x,\ b/y\} \\
        &= \{f(b)/x,\ y/z\}
    \end{split}
    \end{equation*}
\end{priklad}

\begin{poznamka}
    Skladanie substitúcií je asociatávna operácia, teda ak zoberieme 
    $\theta, \lambda, \mu$, potom platí 
    $\theta \circ(\lambda \circ \mu) = (\theta \circ \lambda) \circ \mu$.

    Tiež platí, že $\varepsilon \circ \theta = \theta = 
            \theta \circ \varepsilon$.
    To znamená, že množina substitúcii s operáciou skladanie je
    pologrupa (monoid) s jednotkou.
\end{poznamka}

\begin{definicia}[unifikátor]
    Substitúciu $\theta$ nazveme unifikátorom pre
    množinu výrazov $E_1, E_2, \dots, E_n$,
    ak platí $E_1\theta = E_2\theta = \cdots = E_n\theta$.
    Množinu nazveme unifikovateľnou, ak pre ňu existuje
    unifikátor, ktorý je zjednocuje.
\end{definicia}

\begin{priklad}
    Majme množinu $\{P(a,y),\ P(x,f(b)\}$.
    Potom jeden z možných unifikátorov je napríklad
    $\theta = \{a/x,\ f(b)/y\}$.
\end{priklad}

\begin{poznamka}
    Nie každá množina má unifikátor. Naopak, množina môže mať aj
    viacej unifikátorov. Vtedy má medzi nimi význačné miesto takzvaný
    najvšeobecnejší unifikátor.
\end{poznamka}

\begin{definicia}[Najvšeobecnejší unifikátor]
    Majme množinu výrazov $S=\{ E_1, E_2, \ldots, E_n\}$.
    Unifikátor $\sigma$ pre množinu výrazov $S$ nazveme 
    najvšeobecnejší unifikátor, ak pre ľubovoľnú unifikátor $\theta$
    množiny $S$ platí, že existuje substitúcia 
    $\lambda_{\theta}$ taká, že $\theta = \sigma \circ \lambda_{\theta}$.
\end{definicia}

\medskip
Pri hľadaní unifikátorov budeme pozerať na rozdiely vo výrazoch.
Uvažujme napríklad výrazy $P(a)$, $P(x)$.
Pozerajme sa na ne ako na konečnú postupnosť symbolov -- 
odlišujú sa akurát v treťom symbole. Toto je prvá odlišnosť/diferencia.
Vo všeobecnosti môže byť týchto výrazov viacej a preto si zadefinujeme
diferenú množinu.

\begin{definicia}[Diferenčná množina]
Nech $W$ je neprázdna množina výrazov.
Diferenčnú množinu pre množinu výrazov $W$ dostávame tak, 
že na výrazy sa pozrieme ako na postupnosti symbolov, nájdeme prvú pozíciu
(zľava), na ktorej sa líšia a tieto rozdielne výrazy vypíšeme.
\end{definicia}

\begin{poznamka}[Nebolo na prednáške]
    Iný (a podľa mňa lepší) pohľad na to, ako získať diferenčnú množinu je
    pozrieť sa na stromy daných výrazov a začať ich naraz rekurzívne
    prehľadávať zľava doprava až narazíme na vrchol, ktorý je v niektorom
    výraze iný. Vtedy do diferenčnej množiny zoberieme podstromy daných
    vrcholov.
\end{poznamka}

\begin{priklad}
    \label{prikl:dif1}
    Majme množinu $W = \{P(x,f(y,z),\ P(x,a),\ P(x,g(h(k(x))))\}$.
    Nájdeme prvý pozíciu na ktorej sa líšia:
    $\{P(x,\underline{f(y,z)},\ P(x,\underline{a}),\
    P(x,\underline{g(h(k(x)))}) \}$.
    Diferenčnou množinou bude množina líšiacich sa podvýrazov, teda
    $D= \{ f(y,z),\ a,\ f(h(k(x)))\}$. Jej grafická konštrukcia je na
    obrázku \ref{fig:dif1}.
    \begin{figure}[h]
        \subfigure[$P(x,f(y,z))$]{
            \includegraphics{img/12/diferencia.1.mps}
        }
        \subfigure[$P(x,a)$]{
            \includegraphics{img/12/diferencia.2.mps}
        }
        \subfigure[$P(x,g(h(k(x))))$]{
            \includegraphics{img/12/diferencia.3.mps}
        }

        \caption{Ukážka diferenčnej množiny z príkladu \ref{prikl:dif1}}
        \label{fig:dif1}
    \end{figure}
\end{priklad}

\begin{priklad}[nebol na prednáške]
    \label{prikl:dif2}
    Uvažujme $W=\{P(a,f(a),f(g(y))),\ P(a,f(a),f(u))\}$.
    Diferenčná množina je $D=\{g(y),u\}$ a grafické znázornenie je na
    obrázku \ref{fig:dif2}.
    \begin{figure}[h]
        \subfigure[$P(a,f(a),f(g(y)))$]{
            \includegraphics{img/12/diferencia.11.mps}
        }
        \subfigure[$P(a,f(a),f(u))$]{
            \includegraphics{img/12/diferencia.12.mps}
        }

        \caption{Ukážka diferenčnej množiny z príkladu \ref{prikl:dif2}}
        \label{fig:dif2}
    \end{figure}
\end{priklad}

\begin{poznamka}
    Ďo diferenčnej množiny zakaždým vyberáme iba jednu (prvú) nezhodu.
    Napríklad pre $W=\{P(x,y,z),\ P(y,f(a),g(x,y))\}$ je diferenčná množina
    iba $D=\{x,y\}$.
\end{poznamka}

\subsection{Unifikačný algoritmus}
\startFIXME

Teraz si ukážeme jeden z algoritmov používaných na unifikáciu množiny. Bude
dookola opakovať nasledujúce kroky:
\begin{enumerate}
    \item na začiatku polož kolo $k=0$, množinu na unifikovanie 
        $W_0 = W$ a počiatočnú substitúciu  $\sigma_0 = \eps$.

    \item Ak $W_k$ obsahuje jedinú klauzulu, algortimus zakončí svoju činnosť
        a $\sigma_k$ je najvšeobecnejší unifikátor.
        V opačnom prípade nájdeme diferenčnú množinu $D_k$ pre $W_k$.

    \item Ak existujú také elementy $v_k,t_k \in D_k$, že $v_k$ je
        premenná, ktorá sa nevyskytuje v $t_k$, tak pokračujeme ďalším
        krokom.
        V opačnom prípade algoritmus zakončuje svoji činnosť
        s výsledkom, že množina $W$ nie je unifikovateľná.

    \item Nech $W_{k+1} = W_k \{t_k/v_k\}$, $\sigma_{k+1} = \sigma_k \circ
    \{t_k/v_k\}$ ($W_{k+1} = W_k \sigma_{k+1}$)

    \item Vypíšeme hodnoty pre $k+1$ a prejdeme ku kroku 2.
\end{enumerate}


Ak je množina unifikovateľná, vždy existuje najvšeobecnejší unifikátor.

\paragraph{Príklad}
Nájdite najvšeobecnejší unifikátor pre množinu: $$W=\{ P(a,x,f(g(y))),
P(z,f(z),f(u)), \}$$

\begin{enumerate}
	\item $\sigma_0 = \varepsilon, W_0 = W$. Pretože $W_0$ nie je jednotková
	klauzula, $\sigma_0$ nie je najvšeobecnejší unifikátor.
	\item Musíme zostrojiť diferenčnú množinu $D_0 = \{a, z\}$. Existuje
	premenná $v_0 = z$, ktorá nie je obsiahnutá v terme $t$. $t_0 = a$.
	\item $$\sigma_1 = \sigma_0 \circ \{ t_0/v_0 \} = \epsilon \{a/z\} =
	\{a/z\}$$ $W_1 = W_0\{ t_0/v_0 \} = \{P(a,x,f(g(y)),
	P(z,f(z),f(u))\}\{a/z\} =  \\
	= \{P(a,x,f(g(y))), P(a, f(a), f(u))\}$

	\item $W_1$ nie je jednotková klauzula. Zostrojíme diferenčnú množinu
	$D_1$ pre $W_1$. $D_1 = \{x_1, f(a)\}$

	\item Z $D_1$ dostávame $v_1 = x$ a $t_1 = f(a)$.
	\item $$\sigma_2 = \sigma_1 \circ \{ t_1/v_1\}= \{a/z\} \circ \{ f(a)/x\}
	= \{a/z, f(a)/z \}$$
	$ W_2 = W_1 \{t_1/v_1\} = \{P(a,x,f(g(y))), P(a,f(a),f(u)\} \{ f(a)/x
	\} = \\
	= \{ P(a,f(a), f(g(y))), P(a,f(a), f(u), \}$. $W_2$ opäť nie je
	jednotková klauzula -- musíme vytvoriť diferenčnú množinu.

	\item $W_2$ nie je jednotková klauzula -- vytvárame diferenčnú množinu
	$D_2$ pre $W_2$. $D_2 = \{ g(y), u \}$. $v_2 = u$, $t_2 = g(y)$.

	\item $$\sigma_3 = \sigma_2 \circ \{t_2/v_2\} = \{a/z, f(a)/x\} \circ
	\{g(y)/u\}$$.
	$W_3 = W_2 \{ t_x/v_2\} = \{ P(a,f(a), f(g(y))), P(a,f(a),
	f(u))\}\{g(y)/u\} = \\
	= \{ P(a, f(a), f(g(y))), f(a, f(a), f(g(y)) \}$.
	$W_3$ je jednotková klauzula.
	$$ \sigma_3 = \{ a/z, f(a)/x, g(y)/u\}$$. $\sigma_3$ je najvšeobecnejší
	unifikátor pre množinu klauzúl $W$.
\end{enumerate}

\paragraph{Príklad}

\paragraph{Príklad} Zistitte, či je unifikovateľná množina $W=\{Q(f(x),g(x)),
Q(y,y)\}..$


	$$y\circ \sigma_0 \varepsilon \qquad \mbox{a} \qquad W_0 = W$$
$W_0$ nie je jednotková klauzula. Nájdeme diferenčnú množinu $D_0$ pre $W_0$:
$$D_0 = \{ f(a)/y \}$$
$$v_0, t_0 = f(a)$$

Nech $\sigma_1 = \sigma_0 \circ \{ t_1/\sigma_0\} = \varepsilon \circ  \{
f(a)y\}$

$W_1 = W_0 \{ t_0/\sigma_0\} = \{ Q(f(x),g(x)), Q(f(a),f(a))\}$

$W_1$ nie je jednotková klauzula, zostrojujeme $D_1$, diferenčnú množinu
pre $W_1$.


$$D_1 = \{ g(x), f(a)\}$$ nemáme prvok, ktorý by bol premennou. Algoritmus
ukončí svoju činnosť a $W$ nie je unifikovateľná.

\par Pri zisťovaní unifikovateľnosti vždy vytvárame množiny $W$ tvaru:
$$W\sigma_0, W\sigma_1, \ldots, W\sigma_k, \ldots$$, pričom v každom kroku sa
zmenší počet premenných o 1. Po konečnom počte krokov sa musí zastaviť.


\paragraph{Veta (unifikačná):} Ak $W$ je konečná neprádzna unifikačná množina
výrazov, tak unifikačný algoritmus vždy zakončuje svoju činnosť na druhom kroku
a posledné $\sigma_k$ bude najvšeobecnejší unifikátor.

\paragraph{Dôkaz:} $W$ je unifikovateľná množina, $\Theta$ označuje jej
ľubovoľný unifikátor. Indukciou ukážeme, že pre každé $k$ existuje taká
substitácia $\lambda_k$, že $\Theta = \sigma_k \circ \lambda_k$

\subparagraph{Báza indukcie} Nech $k = 0$. Máme ukázať, že $\Theta = \sigma_0
\circ \lambda_0$. V tomto prípade $\sigma_0 = \varepsilon$ a $\lambda_0 =
\Theta$.

\subparagraph{Indukčný krok} Indukčný predpoklad: $\Theta = \sigma_k \circ
\lambda_k$ $(0 \leq k \leq n)$. $W_n = W\sigma_n$. Ak $W_n$ je jednotková
klauzula, tak algoritmus zakončuje svoju činnosť na druhom kroku a $\sigma_n$ je
najvšeobecnejší unifikátor pre $W$. 
\par Nech $W_n$ nie je jednotková množina. Potom hľadám diferenčnú množinu $D_n
$ pre množinu $W_n$. Keď $D_n$ je diferenčná množina pre $W_n$, tak vo $W_n$
musí existovať premenná -- označme ju $v_n$. Ďalej nech $t_n$ je ľubovoľný výraz
z $D_n$ rôzny od $v_n$.

$$\Theta = \sigma_n \circ \lambda_n$$ 
$\Theta$ je unifikátor. Diferenčnú množinu $D_n$ unifikuje substitúcia
$\lambda_n$. $v_n \lambda_n = t_n \lambda_n$.

\par Ak by premenná $v_n$ bola obsiahnutá v $t_n$. Predpokladali sme, že $v_n$ a
$t_n$ sú rôzne a teda $v_n$ sa nemôže vyskytovať v $t_n$.

\par Prejdeme ku kroku 4 -- množine $W\sigma_{n+1}$ (množina $W_{n+1}$). Platí,
že $\sigma_{n+1} = \sigma-n \circ \{ t_n/v_n \}$. $\Theta = \sigma_{n+1} \circ
\lambda-{n+1}$.

\par Nech $\lambda_{n+1} = \lambda_n - \{t_n\lambda_n/v_n\}$. 

$$t_n\lambda_{n+1} = t_n (\lambda_n - \{t_n\lambda_n / v_n\}) -  t_n\lambda_n$$
$$\{ t_n / v_n\} \circ \lambda_{n-1} = \{t_n \lambda_{n+1}\}/v_n \cup
\lambda_{n+1} = \{t_n\lambda_n /v_n\} \cup \lambda_{n+1} = \{t_n\lambda_n/v_n\}
\cup \{ \lambda_n - \{ t_n\lambda_n/v_n\} = ??? \{ t_n/v_n \} \circ
\lambda_{n+1} = \lambda_n$$

$$\Theta = \sigma_n \circ \lambda_n$$.

