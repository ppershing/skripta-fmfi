\section{Substitúcia a unifikácia}

Vo výrokovej logike nebol problém hľadať kontrárne dvojice.
Zložitejšia situácia ale nastáva v prípade predikátovej logiky
prvého rádu. A práve tomu sa budeme venovať v tejto kapitole.
Uvažujme napríklad dve klauzuly
$C_1=P(x) \lor Q(x), C_2 = \neg P(f(x)) \lor R(x)$.
V nich neexistuje žiadna kontrárna dvojica. Ak však nahradím premennú
$x$ za $f(a)$ v prvej klauzule a za $a$ v druhej,
dostaneme základné inštancie
$C_1'=P(f(a)) \lor Q(f(a)), C_2'=\neg P(f(a)) \lor R(a)$.
Teraz môžeme definovať rezolventu; bude to $Q(f(a)) \lor R(a)$.
Mohli by sme postupovať aj všeobecnejšie -- nahraďme $x$ za $f(x)$ v
prvej klauzule a dostávame
$ C_1^*= P(f(x)) \lor Q(f(x)), C_2^*= \neg P(f(x)) \lor R(x)$.

Rezolventa potom bude $C^*= Q(f(x)) \lor R(x)$.
Vidíme teda, že sme získali 2 rôzne rezolventy, jednu viac všeobecnú
ako druhú. No a práve v ďalšom texte si formálne zadefinuujeme toto
dosadzovanie hodnôt a popíšeme spôsob, ako hľadať najvšeobecnejšie
rezolventy.

\begin{definicia}[Substitúcia]
    Pod substitúciou rozumieme konečnú množinu tvaru:
    $\{t_1/v_1, \ldots, t_n/v_n\}$, kde každé $v_i$ je premenná a všetky
    $v_i$ sú navzájom rôzne ($i \in\{1,\dots,n\}$).
    Ďalej požadujeme, že pre každé $i$ je term  $t_i$ rôzny od $v_i$.

    Ak $t_1, \ldots, t_n$ sú základné inštancie (teda termy bez premenných),
    tak substitúciu nazývame tiež základná substitúcia.
\end{definicia}

\begin{poznamka}
    Na označovanie substitúcií budeme používať grécke písmená.
    Špeciálne, prázdnu substitúciu označíme ako $\eps$.
\end{poznamka}
\startFIXME

\paragraph{Príklad} $\{ f(z)/x; y/z\}$, $\{ a/x, g(y)/y, f(g(y))/z\}$.

\paragraph{Definícia} Nech $\theta$ je ľubovoľná substitúcia a $E$ je nejaký
výraz. Nech $\theta = \{ t_1/v_1, \ldots t_n/v_n\}$. Potom $E\theta$ označuje
výraz, ktorý vznikne tak, že súčasne vo výraze $E$ nahradíme premennú $v_i$
termom $t_i, i=1..n$.

\paragraph{Príklad} Majme $\theta=\{a/x, f(b)/y, c/z\}$, $E = P(x, y, z)$. Potom
$E\theta = P(P(a, f(b), c)$.

\par Ďalšia operácia, ktorú budeme potrebovať, je operácia skladania
substitúcií.

\paragraph{Definícia} Majme $\theta = \{t_1/x_1, \ldots t_n/x_n\}$ a $\lambda =
\{ u_1/y_1, \ldots u_m/y_m \}$. Zloženie (kompozícia) substitúcií, označíme
$\theta \circ \lambda$:
$$\{t_1 \lambda/x_1, \ldots t_n \lambda/x_n, u_1/y_1m \ldots u_m/y_m \}$$ také,
že vyradí všetky členy $t_i\lambda/x_i$, pre ktoré platí, že $t_i \lambda = x_1$,
všetky $u_i/y_1$, pre ktoré $y_i \in \{x_1, x_2, \ldots x_n\}$.

\paragraph{Príklad} 
$$\theta \{t_1/x_1, t_2/x_2\} = \{ f(y)/x, z/y\}$$
$$\lambda = \{ u_1/y_1, u_2/y_2, u_3/y_3\} = \{ a/x, b/y, y/z\}$$

$$ \theta \circ \lambda = \{ t_1 \lambda / x_1, t_2\lambda/x_2, u_1/y_1,
u_2/y_2, u_3/y_3\} = \\ 
\{ f(b)x, y/y, a/x, b/y, y/z\}$$
$$ \theta \circ \lambda = \{ f(b)/x, y/z\}$$

\paragraph{Úloha} Dokážte, že skladanie substitúcií je asociatávna operácia.
Tiež dokážte, že $\varepsilon \circ \theta = \theta \circ \varepsilon$.
\par Zoberieme $\theta, \lambda, \mu$, potom platí $ \theta \circ(\lambda \circ
\mu) = (\theta \circ \lambda) \circ \mu$.

\paragraph{Definícia} Substitúciu $\theta$ nazveme \emph{unifikátorom} pre
množinu výrazov $E_1, E_2, \ldots, E_n$, ak platí $E_1\theta = E_2\theta =
\cdots = E_n\theta$. Množinu nazveme \emph{unifikovateľnou}, ak pre ňu existuje
unifikátor.

\paragraph{Príklad} Majme množinu $\{ P(a,y), P(x,f(b)\}$, $\theta = \{a/x,
f(b)/y\}$ -- unifikátor pre túto množinu. 

\paragraph{Definícia} Majme množinu výrazov $\{ E_1, E_2, \ldots, E_n\}$,
$\theta$ je ľubovoľný unifikátor pre množinu uvedených výrazov. Unifikátor
$\sigma$ pre množinu výrazov $\{E_1, E_2, \ldots, E_n\}$ nazveme
\emph{najvšeobecnejší unifikátor}, ak pre $\theta$ platí $\theta = \sigma \circ
\lambda$, kde $\lambda$ je vhodná substitúcia.

\par Uvažujme výrazy $P(a)$, $P(x)$ Pozerajme sa na ne ako na konečnú
postupnosť symbolov -- odlišujú sa akurát v treťom symbole. Toto je prvá
diferencia. Vo všeobecnosti môže byť týchto výrazov $n$.

\paragraph{Definácia} Nech $W$ je neprázdna množina výrazov. \emph{Diferenčnú
množinu}
pre množinu výrazov $W$ dostávame tak, že na prvú pozíciu (zľava) vypíšem všetky
tie výrazy, ktoré sa líšia. 

\paragraph{Príklad} Majme množinu $W = \{P(x,\underline{f(y,z)},
P(x,\underline{a}), P(x,\underline{g(h(k(x)))}) \}$
Diferenčnou množinou bude $D= \{ f(y,z), a, f(h(k(x)))\}$. 

\subsection{Unifikačný algoritmus}

Kroky:
\begin{enumerate}
	\item $k=0, W_0 = W, \sigma_0 = \Sigma$
	\item Ak $W_k$ je jednotková klauzula, algortimus zakončí svoju činnosť
	$\sigma_k$ je najvšeobecnejší unifikátor. V opačnom prípade nájdeme $D_k$
	(diferenčnú množinu) pre $W_k$.
	\item Ak existujú také premenné $v_k$ a $t_k$ v $D_k$, že $v_k$ je
	premenná, ktorá sa nevyskytuje v $t_k$, tak prejdeme ku kroku 4. V
	opačnom prípade algoritmus zakončuje svoji činnosť. Množina $W$ nie je
	unifikovateľná.
	\item Nech $W_{k+1} = W_k \{t_k/v_k\}$, $\sigma_{k+1} = \sigma_k \circ
	\{t_k/v_k\}$ ($W_{k+1} = W_k \sigma_{k+1}$)

	\item Vypíšeme hodnoty pre $k+1$ a prejdeme ku kroku 2.
\end{enumerate}


Ak je množina unifikovateľná, vždy existuje najvšeobecnejší unifikátor.

\paragraph{Príklad}
Nájdite najvšeobecnejší unifikátor pre množinu: $$W=\{ P(a,x,f(g(y))),
P(z,f(z),f(u)), \}$$

\begin{enumerate}
	\item $\sigma_0 = \varepsilon, W_0 = W$. Pretože $W_0$ nie je jednotková
	klauzula, $\sigma_0$ nie je najvšeobecnejší unifikátor.
	\item Musíme zostrojiť diferenčnú množinu $D_0 = \{a, z\}$. Existuje
	premenná $v_0 = z$, ktorá nie je obsiahnutá v terme $t$. $t_0 = a$.
	\item $$\sigma_1 = \sigma_0 \circ \{ t_0/v_0 \} = \epsilon \{a/z\} =
	\{a/z\}$$ $W_1 = W_0\{ t_0/v_0 \} = \{P(a,x,f(g(y)),
	P(z,f(z),f(u))\}\{a/z\} =  \\
	= \{P(a,x,f(g(y))), P(a, f(a), f(u))\}$

	\item $W_1$ nie je jednotková klauzula. Zostrojíme diferenčnú množinu
	$D_1$ pre $W_1$. $D_1 = \{x_1, f(a)\}$

	\item Z $D_1$ dostávame $v_1 = x$ a $t_1 = f(a)$.
	\item $$\sigma_2 = \sigma_1 \circ \{ t_1/v_1\}= \{a/z\} \circ \{ f(a)/x\}
	= \{a/z, f(a)/z \}$$
	$ W_2 = W_1 \{t_1/v_1\} = \{P(a,x,f(g(y))), P(a,f(a),f(u)\} \{ f(a)/x
	\} = \\
	= \{ P(a,f(a), f(g(y))), P(a,f(a), f(u), \}$. $W_2$ opäť nie je
	jednotková klauzula -- musíme vytvoriť diferenčnú množinu.

	\item $W_2$ nie je jednotková klauzula -- vytvárame diferenčnú množinu
	$D_2$ pre $W_2$. $D_2 = \{ g(y), u \}$. $v_2 = u$, $t_2 = g(y)$.

	\item $$\sigma_3 = \sigma_2 \circ \{t_2/v_2\} = \{a/z, f(a)/x\} \circ
	\{g(y)/u\}$$.
	$W_3 = W_2 \{ t_x/v_2\} = \{ P(a,f(a), f(g(y))), P(a,f(a),
	f(u))\}\{g(y)/u\} = \\
	= \{ P(a, f(a), f(g(y))), f(a, f(a), f(g(y)) \}$.
	$W_3$ je jednotková klauzula.
	$$ \sigma_3 = \{ a/z, f(a)/x, g(y)/u\}$$. $\sigma_3$ je najvšeobecnejší
	unifikátor pre množinu klauzúl $W$.
\end{enumerate}

\paragraph{Príklad}

\paragraph{Príklad} Zistitte, či je unifikovateľná množina $W=\{Q(f(x),g(x)),
Q(y,y)\}..$


	$$y\circ \sigma_0 \varepsilon \qquad \mbox{a} \qquad W_0 = W$$
$W_0$ nie je jednotková klauzula. Nájdeme diferenčnú množinu $D_0$ pre $W_0$:
$$D_0 = \{ f(a)/y \}$$
$$v_0, t_0 = f(a)$$

Nech $\sigma_1 = \sigma_0 \circ \{ t_1/\sigma_0\} = \varepsilon \circ  \{
f(a)y\}$

$W_1 = W_0 \{ t_0/\sigma_0\} = \{ Q(f(x),g(x)), Q(f(a),f(a))\}$

$W_1$ nie je jednotková klauzula, zostrojujeme $D_1$, diferenčnú množinu
pre $W_1$.


$$D_1 = \{ g(x), f(a)\}$$ nemáme prvok, ktorý by bol premennou. Algoritmus
ukončí svoju činnosť a $W$ nie je unifikovateľná.

\par Pri zisťovaní unifikovateľnosti vždy vytvárame množiny $W$ tvaru:
$$W\sigma_0, W\sigma_1, \ldots, W\sigma_k, \ldots$$, pričom v každom kroku sa
zmenší počet premenných o 1. Po konečnom počte krokov sa musí zastaviť.


\paragraph{Veta (unifikačná):} Ak $W$ je konečná neprádzna unifikačná množina
výrazov, tak unifikačný algoritmus vždy zakončuje svoju činnosť na druhom kroku
a posledné $\sigma_k$ bude najvšeobecnejší unifikátor.

\paragraph{Dôkaz:} $W$ je unifikovateľná množina, $\Theta$ označuje jej
ľubovoľný unifikátor. Indukciou ukážeme, že pre každé $k$ existuje taká
substitácia $\lambda_k$, že $\Theta = \sigma_k \circ \lambda_k$

\subparagraph{Báza indukcie} Nech $k = 0$. Máme ukázať, že $\Theta = \sigma_0
\circ \lambda_0$. V tomto prípade $\sigma_0 = \varepsilon$ a $\lambda_0 =
\Theta$.

\subparagraph{Indukčný krok} Indukčný predpoklad: $\Theta = \sigma_k \circ
\lambda_k$ $(0 \leq k \leq n)$. $W_n = W\sigma_n$. Ak $W_n$ je jednotková
klauzula, tak algoritmus zakončuje svoju činnosť na druhom kroku a $\sigma_n$ je
najvšeobecnejší unifikátor pre $W$. 
\par Nech $W_n$ nie je jednotková množina. Potom hľadám diferenčnú množinu $D_n
$ pre množinu $W_n$. Keď $D_n$ je diferenčná množina pre $W_n$, tak vo $W_n$
musí existovať premenná -- označme ju $v_n$. Ďalej nech $t_n$ je ľubovoľný výraz
z $D_n$ rôzny od $v_n$.

$$\Theta = \sigma_n \circ \lambda_n$$ 
$\Theta$ je unifikátor. Diferenčnú množinu $D_n$ unifikuje substitúcia
$\lambda_n$. $v_n \lambda_n = t_n \lambda_n$.

\par Ak by premenná $v_n$ bola obsiahnutá v $t_n$. Predpokladali sme, že $v_n$ a
$t_n$ sú rôzne a teda $v_n$ sa nemôže vyskytovať v $t_n$.

\par Prejdeme ku kroku 4 -- množine $W\sigma_{n+1}$ (množina $W_{n+1}$). Platí,
že $\sigma_{n+1} = \sigma-n \circ \{ t_n/v_n \}$. $\Theta = \sigma_{n+1} \circ
\lambda-{n+1}$.

\par Nech $\lambda_{n+1} = \lambda_n - \{t_n\lambda_n/v_n\}$. 

$$t_n\lambda_{n+1} = t_n (\lambda_n - \{t_n\lambda_n / v_n\}) -  t_n\lambda_n$$
$$\{ t_n / v_n\} \circ \lambda_{n-1} = \{t_n \lambda_{n+1}\}/v_n \cup
\lambda_{n+1} = \{t_n\lambda_n /v_n\} \cup \lambda_{n+1} = \{t_n\lambda_n/v_n\}
\cup \{ \lambda_n - \{ t_n\lambda_n/v_n\} = ??? \{ t_n/v_n \} \circ
\lambda_{n+1} = \lambda_n$$

$$\Theta = \sigma_n \circ \lambda_n$$.

