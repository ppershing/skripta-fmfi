\section{Herbrandova veta}
\startFIXME

\paragraph{Poznámka} Dirichletov princíp:
$$T_1:$$
$$X, Y -- \mbox{konečné množiny}$$
$$ |X| < |Y| $$
$$ f: X \rightarrow Y$$
$$ \exists y \in Y_1\qquad x_1,x_2 \in X, x_1 \neq x_2 \qquad f(x_1) = f(x_2) =
y$$
Sporom. Ak by také $x_1$, $x_2$ neexistovali, potom pre každé $x_1, x_2 \in X$,
$x_1 \neq x_2$ a $f(x_1) \neq f(x_2)$, potom $f$ je injektívne zobrazenie, z
čoho vyplýva, že $|X| \leq |Y|$.
$$T_2: $$
$$ f: X \rightarrow Y,\qquad X\mbox{-- nekonečná}, Y\mbox{ -- konečná} $$
$$T_2: \exists y\in Y, \qquad y=\{x|x\in X, f(x)=y\}$$

\paragraph{Dôkaz} Nech $|Y|=n$. Množinu $X$ môžeme napísať ako $X=\Cup_{y\in
Y}A_y$ takých, že ak $y_1\neq y_2$, potom $A_{y_1} \cap A_{y_2} = \emptyset$.

$$ |X| = \left|\Cup_{y\in Y} A_y \right|$$

Nech $k_0$ je maximálna mohutnosť $|A_y|$, $y\in Y$, množina $X$ ale nemá
konečnú mohutnosť -- spor.

\paragraph{Definícia} Strom chápeme ako usporiadanú dvojicu $(T,\leq)$ -- $T$ a
relácia, ktorá čiastočne usporadúva $T$:
\begin{enumerate}
	\item $T(n)=\{v\in T, v<n\}$ je dobre usporiadaná.
	\item $T$ má najmenší vrchol (koreň)
\end{enumerate}

Množina je dobre usporiadaná, ak každá jej neprázdna podmnožina má najmenší
prvok. $T$ je lineárne usporiadanie -- ľubovoľné dva prvky sú porovnateľné:
$a \neq b$, $a,b\in T$, pre $\{a, b\}$ musí platiť $a<b$ alebo $a>b$.

\par Ak mám prvky $u$ a $v$, prvok $v$ nasleduje bezprostredne po prvku $u$, ak
neexistuje $z$ také, že $u < z < v$.

\paragraph{Lema (K\"onig)} (je dôležitá na pochopenie dôkazu Herbrandovej vety)
Nech každý vrchol stromu s koreňom má konečné vetvenie (t.j. konečný stupeň
vetvenia) a strom $T$ je nekonečný, tak potom v ňom existuje nekonečne dlhá
vetva.

\paragraph{Dôkaz} $(T_1,\leq)$, $v \in T$. Označme $A_v = \{ u \in T | v <u
\}$ -- $A_v$ je množina tých vrcholov, ktoré ležia nad $v$. Majme vrchol $v$ a
nech $v_1, v_2, \ldots v_n$ sú bezprostrední nasledovníci $v$ (obr.
\ref{fig:nasledovnici}). $A_v = A_{v_1} \cup A_{v_2} \cup \ldots \cup A_{v_n}
\cup \{v_1, v_2, \ldots, v_n\}$. $A_v$ je nekonečná. Na základe uvedeného
tvrdenia aspoň jedna z množín $A_{v_i}$ musí byť nekonečná.

\begin{figure}[h]
	\centering\includegraphics{img/10/nasledovnici.1.mps}
	\caption{Nasledovníci $v$}
	\label{fig:nasledovnici}
\end{figure}

\par Vyberieme si $x_0$ ako koreň (najmenší vrchol stromu $T$). $A_{x_0} =
T \backslash \{x_0\}$, $A_{x_0}$ je nekonečná. Vezmime postupne prvky $\{x_0,
x_1, \ldots , x_n, x_\{n+1\}\}$. Ku každému prvku viem vybrať nasledovný prvok a
počet nasledovníkov je nekonečný. \todo{lepšia formulácia}

\paragraph{Veta (Herbrandova, variant 1)} Množina klauzúl $S$ nie je splniteľná
práve vtedy, keď ľubovoľnému úplnému sémantickému stromu pre množinu klauzúl
$S$ zodpovedá konečný uzavretý sémantický strom, t.j. ľubovoľná vetva úplného
stromu vedie do odmietajúceho vrchola.

\paragraph{Dôkaz} Predpokladajme, že množina klauzúl $S$ nie je splniteľná, nech
$T$ je úplný sémantický strom prislúchajúci $S$. $I_V$ je množina všetkých
literálov pripísaných vetve $V$ stromu $T$. Potom $I_V$ je interpretácia
množiny klauzúl $S$. Predpokladajme, že $S$ je nesplniteľná (teda nesplniteľná v
každej interpretácii). To znamená, že existuje nejaká základná inštancia klauzúl
$C'$ (pre klauzulu $C$), ktorá je v interpretácii $I_V$ odmietnutá. (základná
inštancia neobsahuje premenná, klauzuly sú konečné objekty). 
%Je odmietnutá vo vrchole, ktorý je konečne vzdialený od koreňa stromu.
\par Teraz potrebujeme zabezpečiť, aby strom bol konečný -- to musí, lebo ak by
existovala nekonečná vetva, bola by to interpretácia, ktorá by neodmietala
niektorú zo základných inštancií.

\par Obrátená implikácia. Ak množina klauzúl nie je splniteľná, existuje k
úplnému sémantickému stromu uzavretý konečný sémantický strom. Teraz
predpokladajme, že k úplnému sémantickému stromu pre množinu klauzúl
$S$ existuje konečný uzavretý sémantický strom (každá vetva stromu $T$ končí v
zamietajúcom vrchole), potom interpretácia $I_V$ odmieta každú základnú
inštanciu.


\paragraph{Veta (Herbrandova)}: Množina klauzúl $S$ nie je splniteľná $\iff$ keď
existuje konečná množina $S'$ základných inštancií klauzúl z $S$, ktorá nie je
splniteľná. \fixme{}

\paragraph{Dôkaz:} Predpokladajme, že množina klauzúl $S$ nie je splniteľná,
existuje $T'$ -- konečný uzavretý strom priradený ku stromu $T$.

\paragraph{Obrátene}: Predpoklajme, že existuije konečná $S'$ množina základných
inštancií klauzúl $S$ nie je splniteľná, $I'$, $S'$. Potom $S$ nie je
splniteľná, $I$ pre $S$. 

\par
Interpretácia $I$ obsahuje $I'$, teda keď $I'$ odmieta $S'$, potom $I$ odmieta
$S$. Každá klauzula je disjunkcia literálov. Ak existuje množina základných
inštancií, ktorá nie je splniteľná, potom celá množina nie je splniteľná.


\paragraph{Príklad} Majme množinu klauzúl $S=\{P(x), \neg P(f(a))\}$. Vezmime
množinu $S' = \{ P(f(a)), \neg P(f(a))\}$.

\paragraph{Príklad} Majme množinu klauzúl $S=\{\neg P(x) \lor Q(f(x),x),
P(g(b)), \neg Q(y,z) \}$. Táto množina nie je splniteľná -- opäť vezmime $$S' =
\{ \neg P(g(b)) \lor Q(f(g(b)), g(b)), P(g(b)), \neg Q(f(g(b)), g(b)) \}$$.

\paragraph{Príklad} Majme množinu klauzúl $S$:
$$
\begin{array}{l}
	\neg P(x,y,u) \lor \neg P(y,z,v) \lor \neg P(x,v,w) \lor P(u,z,w) \\
	\neg P(x,y,u) \lor \neg P(z,y,v) \lor \neg P(u,z,w) \lor P(x,v,w) \\
	P(g(x,y),x,y), P(x,h(x,y),y), P(x,y,f(x,y),\neg P(f(v),x,f(x)) \\
\end{array}
$$
\subparagraph{Návod} Zostrojte konečný sémantický strom.

\par Algoritmické metódy na zisťovanie nesplniteľnosti pracovali iba pre menšie
množiny klauzúl -- problém bol, že $S$ je množina disjunkcií, pokiaľ tam
dosadíme prvky z univerza, dostaneme konjunktívnu normálnu formu (prípadne
transformujeme na disjunktívnu normálnu formu a tú preverujeme). Ak máme $10$
formúl, potrebujeme preverovať $2^{10}$ konjunkcií -- zložitosť exponenciálne
rastie.

\paragraph{Príklad}
$$S=\{P(x), \neg P(a)\}, H_0 = \{a\}$$
$$S'=P(a) \land \neg P(a) = \square$$
$S$ nie je splniteľná.

\paragraph{Príklad}
$$S=\{P(a), \neg P(x) \lor Q(f(x)), \neg Q(f(a)) \}, H_0 = \{a\}$$
$ S_0'=P(a) \land (\neg P(a)\lor Q(f(a))) \land \neg Q(f(a)) = \\
P(a)\land \neg P(a) \land \neg Q(f(a)) \lor P(a) \land Q(f(a)) \land \neg Q(f(a)) = 
\\ \square \lor \square = \square$.

\paragraph{Pravidlá}
Nech $S$ je množina klauzúl.

\subparagraph{Pravidlo tautológie.} Vynecháme z $S$ všetky tautologické klauzuly.
Množina $S'$, ktorá nám zostane, nie je splniteľná práve vtedy, keď $S$ nie je
splniteľná.

\subparagraph{Pravidlo jednoliterálnych klauzúl.} Nech $L$ je nejaký literál.
Vynechajme z $S$ všetky klauzuly, ktoré obsahujú literál $L$. Nech $S'$ sú
klauzuly, ktoré nám zostanú po vynechaní. Môžu nastať dva prípady:

\begin{enumerate}
	\item $S' = \emptyset$, tak potom množina klauzúl $S$ je splniteľná --
	Stačí zobrať model, ktorý obsahuje $L$.
	\item $S' \neq \emptyset$. Vezmem si literál $\neg L$ a vynechám z
	množiny $S'$ všetky klauzuly, ktoré obsahujú $\neg L$, pričom dostanem
	množinu klauzúl $S''$. Ak sa $S'$ nachádza $\neg L$, po vynechaní
	namiesto nej zostane $\square$.
\end{enumerate}

$S$ nie je splniteľná $\iff$ $S''$ nie je splniteľná.

\subparagraph{Pravidlo čistých literálov} Literál $L$ základnej klauzuly z $S$
budeme nazývať \emph{čistým}, ak literál $\neg L$ sa nevyskytuje v žiadnej
základnej klauzule $S$. Vezmem si z $S$ literál $L$, ktorý je čistý, a
vynecháme z $S$ všetky základne klauzuly obsahujúce $L$. $S'$ je množina, ktorá
nám zostala. $S$ nie je splniteľná $\iff$ $S'$ nie je splniteľná.

\subparagraph{Pravidlo rezu} Ak množinu $S$ vieme vyjadriť v tvare $(A_1 \lor L)
\land \ldots \land (A_m \lor L) \land (B_1 \lor \neg L) \land \ldots \land (B_n
\lor \neg L) \land R$, pričom v $A_i$, $B_j$, sa $R$, $L$, $\neg L$, $i=1..m$,
$j=1..n$ už nevyskytujú.

$$S_1 = A_1 \land A_2 \land \ldots \land A_m \land R$$
$$S_2 = B_1 \land B_2 \land \ldots \land _n \land R$$

$S$ nie je splniteľná $\iff$ $S_1 \lor S_2$ nie je splniteteľná.

\paragraph{Dôkaz}
\begin{enumerate}
	\item Keď $S$ nie je splniteľné, tak ani $S_2$ nie je splniteľné --
	tautológie sú splnené pre ľubovoľné interpretácie. To, čo zostane,
	hovorí, či množina je alebo nie je splniteľná.
	\item Zostáva nám $S'$, ktoré vzniklo vynechaním $L$ z $S$. Ak
	$S'=\emptyset$, každá klauzula z $S$ obsahuje $L$ (je tvaru $L \lor
	\square$), stačí nám ohodnotenie, kedy $L$ je pravdivé, čo zabezpečí
	pravdivosť celého výroku. 
	\par
	Keď $S'$ nie je prázdna, vytvoríme $S''$ tak,
	že vynecháme všetky klauzuly obsahujúce $\neg L$. $S'$ nie je splniteľná
	práve vtedy, keď $S$ nie je splniteľná. Predpokladajme, že $S''$ nie je
	splniteľná a $S$ je splniteľná. Model $\mathcal{M}$ musí obsahovať $L$ a
	$S$ platí na $S''$.
	\todo{???}
	V konjunktívnej normálnej forme musia byť platné doplnky $\neg L \lor
	\square$. Potom $S''$ je splniteľná, čo je spor.

	\par Obrátene, nech $S$ nie je splniteľná. Z toho vyplýva, že $S''$ nie
	je splniteľná. Nech $S''$ je splniteľná. Potom existuje model
	$\mathcal{M}''$ pre $S''$. Ak tomuto modelu pridám $L$, potom
	$\mathcal{M}'' \cup L$ je model pre $S$, čo je spor.

	\item Pravidlo čistých literálov -- Máme čistý literál $L$, všetky
	klauzuly, ktoré ho obsahujú, vyhodíme z $S$ a dostaneme $S'$. $S$ nie je
	splniteľná práve vtedy, keď $S'$ nie je splniteľná. Predpokladajme, že
	$S'$ nie je splniteľná, teda pre žiadnu interpretáciu $I'$ neplatí. $S\
	\subseteq S$. Musí platiť, že $I' \subseteq I$.

	\par Obrátene -- predpokladajme, že $S$ nie je splniteľná. Potom aj $S'$
	nie je splniteľná. Predpokladajme, že $S'$ je splniteľná. Potom musí mať
	model $\mathcal{M}'$, ktorý jej vyhovuje. Keď si vytvorím model
	$\mathcal{M}=\mathcal{M}'\cup L$, potom $\mathcal{M}$ je modelom v $S$.

	\item Pravidlo rezu. Mám množinu $S$, popísaným spôsobom som zostrojil
	$S_1$ a $S_2$. Chceme ukázať, že $S$ nie je splniteľná $\iff$ $S_1 \lor
	S_2$. Predpokladajme, že $S$ je splniteľná. Potom $S_1 \lor S_2$ je
	splniteľná, teda aspoň jeden člen disjunkcie je splniteľný. Keď pridám
	modelu $S_1$ negované $B$, dostávam model pre $S$. Taktiež, keď je
	splnená $S_2$, má nejaký model $\mathcal{M_2}$. Keď k nemu pridám $B$,
	dostávam model pre $S$.

	\par Obrátene, predpokladajme, že $S_1$ alebo $S_2$ nie je splniteľná. Z
	toho vyplýva, že $S$ nie je splniteľná. Pre spor predpokladajme, že $S$
	je splniteľná. Pozrime sa, ako vyzerá $S$ -- všetky konjunktívne členy
	musia byť splnené a teda buď platí $\neg L$ alebo $L$. Vždy dospievam do
	sporu.
\end{enumerate}

\paragraph{Záver} Najvšeobecnejšie je pravidlo rezu, ktoré pokrýva aj všetky
ostatné prípady.

\paragraph{Príklad} Majmne množinu $S=(P \lor Q \lor \neg R) \land (P \lor \neg
Q) \land \neg P \land R \land U$. Ukážte, že množina klauzúl nie je splniteľná.
Použijeme pravidlo 2 s $\neg P$. Dostávam $Q\land\neg R) \land \neg  \land R
\land U$. Na túto množinu klauzúl použijem pravidlo 2 s $\neg Q$. Dostávam $\neg
R \land R \land U$, čo je $\square \land U \square$.




\paragraph{Príklad} (7) $S=(P \lor Q) \land \neg Q \land ( \neg P \lor Q \lor \neg
R)$. Ukážte, že množina klauzúl $S$ je splniteľná.
\par Pravidlo 2 s $\neg Q$: $(P \lor Q) \land ( \neg P \lor Q \lor \neg R)$ \\
$P \land (\neg P \lor \neg R)$ \\
$(P \land \neg P) \lor (P \land \neg R)$ (distributívnosť) \\
$P \land \neg L, I=\{P,\neg Q, \neg R\}$

\paragraph{Príklad} (8) $S=(P\lor \neg Q) \land (\neg P \lor Q) \land (Q \lor
\neg R) \land (\neg Q \lor \neg R)$. Zistite, či množina je alebo nie je
splniteľná.
\par vytvoríme si mnonžiny podľa pravidla 4: \\
$$(P\land \neg Q) \land (Q \lor \neg R)  \land (\neg Q \lor \neg R)$$
$$(\neg P \lor Q) \land (Q \lor \neg R) \land (\neg Q \lor \neg R)$$

\par
$$S_1 = \neg Q \land (Q \lor \neg R) \land (\neg Q \lor \neg R)$$
$$S_2 = Q \land (Q \lor \neg R) \land (\neg Q \lor \neg R)$$

($S_1$ a $S_2$ sú množiny rezu)
Na množiny môžem použiť pravidlo 2 -- s $Q$ a $\neg Q$. Dostávam $\neg R \lor
R$, čo je splniteľná klauzula.

\paragraph{Príklad} (9) $S=(P \lor Q) \land (P \lor \neg Q) \land (R \lor Q)
\land (R \lor \neg Q)$. $P$ je čistý literál; použijeme pravidlo 3:
$S': (R \lor Q) \land (R \lor \neg Q)$. \\
$\fbox{P,R}$ \\
