% vim: set fdm=marker: sw=2: sts=2: ts=2:

\documentclass[a4paper]{report}
%\documentclass[a5paper,12pt]{report}
\usepackage{ulem}
\usepackage{slovak}
\usepackage[utf8]{inputenc}
\usepackage{a4wide}
\usepackage{tabularx}
\usepackage{amsfonts}
\usepackage{amssymb}
\usepackage{amsmath}
\usepackage{epsfig}
\usepackage[usenames,dvipsnames]{color}
\usepackage{mathrsfs}
\usepackage{verbatim}
\usepackage{hyperref}
\usepackage{ifthen}
\usepackage{subfigure}
\usepackage[margin=2cm]{geometry}


% vim: set fdm=marker:
%% Original by Michal Forisek

\ifx \RestyleAlgo \undefined
    \def\RestyleAlgo#1{\restylealgo{#1}}
\fi

\RestyleAlgo{boxed} %% nastav styl algorithm2e

%% zakladne definicie
\newcommand{\quoteme}[1]{\clqq#1\crqq}
\def\todo#1{[{\color{red} TODO:} {\bf  #1}]}
\def\fixme#1{[{\color{red} FIXME:} {\bf  #1}]}
\def\verify#1{\todo{verify: #1}}

\def\xor{\oplus}
\def\concat{\|}
%\def\inr{\in_{R}}
\def\toa #1 {\overset{#1}{\rightarrow}}
\def\inr{\overset{\$}{\leftarrow}}
\def\assign{\leftarrow}
\def\send{\rightarrow}
\def\isomorph{\cong}
\def\nsd{NSD}
\def\union{\cup}
\newcommand{\unit}[1]{\ensuremath{\, \mathrm{#1}}}
\DeclareMathOperator{\dlog}{dlog}

\def\compactlist{
  \setlength{\itemsep}{1pt}
  \setlength{\parskip}{0pt}
  \setlength{\parsep}{0pt}
}
\def\mod{\,{\rm mod}\,}

%%% original od Misofa:
%% {{{

\catcode`\@=11

\def\R{{\cal R}}
\def\cent{{c\kern-0.3em|\kern0.1em}}
\def\N{{N}} % FIXME FIXME 

\let\eps=\varepsilon

\def\relupdown#1#2#3{\mathrel{\mathop{#1}\limits^{#2}_{#3}} }

\let\then=\Rightarrow
\let\neht=\Leftarrow

\def\krok#1{\relupdown{\Longrightarrow}{}{#1}}
\def\thenrm{\relupdown{\Longrightarrow}{}{rm}}

\def\bicik{\upharpoonright}
\def\B{{\mathbf B}}
\def\kodTS#1{{\tt <}#1{\tt >}}

\newtheorem{definicia}{Definícia}[section]
\newtheorem{HLPpoznamka}{Poznámka}[section]
\newtheorem{HLPpriklad}{Príklad}[section]
\newtheorem{HLPcvicenie}[HLPpriklad]{Cvičenie}
\newtheorem{zadanie}{Úloha}[section]
\newenvironment{poznamka}{\begin{HLPpoznamka}\rm}{\end{HLPpoznamka}}
\newenvironment{priklad}{\begin{HLPpriklad}\rm}{\end{HLPpriklad}}
\newenvironment{cvicenie}{\begin{HLPcvicenie}\rm}{\end{HLPcvicenie}}
\newtheorem{veta}{Veta}[section]
\newtheorem{lema}[veta]{Lema}
\newtheorem{dosledok}[veta]{Dôsledok}
\newtheorem{teza}[veta]{Téza}
% \newtheorem{dokaz}{Dôkaz}[section]

\long\def\odsadene#1{
\leftskip=\parindent
\parindent=0pt
\vskip-5pt

\parskip=5pt
#1
\parskip=0pt

\parindent=\leftskip
\leftskip=0pt

} % end \odsadene




%%%%%%%%%%% PROSTREDIE PRE PISANIE KOMENTAROV

%\newenvironment{komentar}{%
%\vskip\baselineskip
%\tabularx{0.95\textwidth}{|X|}
%\sl
%}
%{\endtabularx
%\vskip\baselineskip
%}

\newenvironment{komentar}{%
\vskip\baselineskip\noindent
\tabularx{\textwidth}{>{\hsize=.2\hsize}X>{\hsize=1.8\hsize}X}
\sl ~ & \sl
}
{\endtabularx
\vskip\baselineskip
}

%\newenvironment{komentar}{%
%\vskip\baselineskip
%\trivlist\vspace{-4pt}\raggedleft\item\relax\tabularx{0.9\textwidth}{X}\sl}
%{\endtabularx\vspace{-4pt}\endtrivlist
%\vskip\baselineskip
%}

\newenvironment{dokaz}{\trivlist
  \item[\hskip \labelsep{\bfseries Dôkaz.}]}{\endtrivlist}
  
%\newenvironment{dokaz}{%
%\vskip\baselineskip\noindent
%\tabularx{\textwidth}{||X||}
%\sl
%}
%{\endtabularx
%\vskip\baselineskip
%}

%%%%%%%%%%% PROSTREDIE PRE MOJE ITEMIZE 

\newenvironment{myitemize}{%
\begin{itemize}
\itemsep-3pt
}
{\end{itemize}
}

%%%%%%%%%%% MATICKE MAKRA

\font\tenrm=csr10

\def\eps{\varepsilon}
% \def\R{{\mathbb R}}
\def\lvec#1{\overrightarrow{#1}}
\def\uhol{{\measuredangle}}
\def\then{\Rightarrow}
% \def\lg{{\rm lg}}
\def\lg{\log_2}
%\def\div{\mathbin{\rm div}}
\def\div{{\rm div}}

%%%%%%%%%%% PDF

\newif\ifpdf
\ifx\pdfoutput\undefined
  \pdffalse
\else
  \pdfoutput=1 \pdftrue
\fi

%%%%%%%%%%% OBRAZKY 

\newcommand{\myincludegraphics}[2][]{\includegraphics[#1]{images/#2}}

%%%%%%%%%%% SLOVNICEK

\openout2=\jobname.slo

\newcommand{\definuj}[3][]{%
\def\tmpvoid{}\def\tmpfirst{#1}%
\ifx\tmpvoid\tmpfirst%
  {\sl #2}\label{definicia:#2}\write2{#2 & #3 & \pageref{definicia:#2} \cr}%
\else%
  {\sl #2}\label{definicia:#2}\write2{#1 & #3 & \pageref{definicia:#2} \cr}%
\fi}

\newcommand{\definujsilent}[2]{%
\label{definicia:#1}\write2{#1 & #2 & \pageref{definicia:#1} \cr}%
}

\newcommand\myglossary{
  \immediate\closeout2 
  %\if@twocolumn\@restonecoltrue\onecolumn\else\@restonecolfalse\fi
  \chapter{Slovníček pojmov}
  \begin{tabular}{|l|l|r|}
  \hline
  {\bfseries slovenský pojem} & {\bfseries anglický preklad} & {\bfseries str.} \\ 
  \hline
  \InputIfFileExists{\jobname.srs}{}{~ & ~ & ~ \\}
  \hline
  \end{tabular}
  %\if@restonecol\twocolumn\fi
}

%%%%%%%%%%% UVODZOVKY

\catcode`\"=13
\def "{\begingroup\clqq\def "{\endgroup\crqq}}
\def\dospecials{\do\ \do\\\do\{\do\}\do\$\do\&%
  \do\#\do\^\do\^^K\do\_\do\^^A\do\%\do\~\do\"}

%%%%%%%%%%% DANGER BENDS 

\font\manual=manfnt % font used for the METAFONT logo, etc.
\def\dbend{{\manual\char127}} % dangerous bend sign

\newlength{\bendwidth}   \settowidth{\bendwidth}{\dbend}    \newlength{\hangwidth}

\def\hangone{%
  \hangwidth=\bendwidth%
  \advance\hangwidth 5pt%
  \hangindent\hangwidth%
}
\def\hangtwo{%
  \hangwidth=\bendwidth%
  \multiply\hangwidth 2%
  \advance\hangwidth 6pt% 
  \hangindent\hangwidth%
}

\def\medbreak{\par\ifdim\lastskip<\medskipamount \removelastskip\penalty-100\medskip\fi}
\let\endgraf=\par

\def\d@nger{\medbreak\begingroup\clubpenalty=10000
%\def\d@nger{\begingroup\clubpenalty=10000
%  \def\par{\endgraf\endgroup\medbreak} \noindent\hangone\hangafter=-2
  \def\par{\endgraf\endgroup} \noindent\hangone\hangafter=-2
  \hbox to0pt{\hskip-\hangindent\dbend\hfill}}
\outer\def\danger{\d@nger}

\def\dd@nger{\medbreak\begingroup\clubpenalty=10000
%  \def\par{\endgraf\endgroup\medbreak} \noindent\hangtwo\hangafter=-2
  \def\par{\endgraf\endgroup} \noindent\hangtwo\hangafter=-2
  \hbox to0pt{\hskip-\hangindent\dbend\kern1pt\dbend\hfill}}
\outer\def\ddanger{\dd@nger}

\def\enddanger{\endgraf\endgroup} % omits the \medbreak
\def\enddangerhop{\endgraf\endgroup\medbreak}




\def\@nakedcite#1#2{{#1\if@tempswa , #2\fi}}
\DeclareRobustCommand\nakedcite{%
  \@ifnextchar [{\@tempswatrue\@nakedcitex}{\@tempswafalse\@nakedcitex[]}}
\def\@nakedcitex[#1]#2{%
  \let\@citea\@empty
  \@nakedcite{\@for\@citeb:=#2\do
    {\@citea\def\@citea{,\penalty\@m\ }%
     \edef\@citeb{\expandafter\@firstofone\@citeb\@empty}%
     \if@filesw\immediate\write\@auxout{\string\citation{\@citeb}}\fi
     \@ifundefined{b@\@citeb}{\mbox{\reset@font\bfseries ?}%
       \G@refundefinedtrue
       \@latex@warning
         {Citation `\@citeb' on page \thepage \space undefined}}%
       {\hbox{\csname b@\@citeb\endcsname}} }}{#1}}

\long\def\FIXME#1{
  \begin{center}
  \begin{minipage}{0.8\textwidth}
  {\bf FIXME:~}\sl #1
  \end{minipage}
  \end{center}
}


\catcode`\@=12
%% }}}



\newenvironment{dokaz}{\smallskip \par \noindent{\bfseries D:}}{}
\newenvironment{definicia}{\smallskip \par \noindent{\bfseries Def:}}{}
\newenvironment{priklad}{\smallskip \par \noindent{\bfseries Príkl:}}{}
\newenvironment{veta}{\smallskip \par \noindent{\bfseries Veta:}}{}
\newenvironment{lema}{\smallskip \par \noindent{\bfseries Lema:}}{}
\newenvironment{dosledok}{\smallskip \par \noindent{\bfseries Dôsl:}}{}
\newenvironment{poznamka}{\smallskip \par \noindent{\bfseries Pozn:}}{}
\newenvironment{postup}{\smallskip \par \noindent{\bfseries Postup:}}{}

%%%%%%%%%%%%%%%%%%%%%%%%%%%%%%%%%%%%%%
\begin{document}

%%%%%%%%%%%
\subsection{Prenexné tvary formúl}
%%% {{{

\begin{definicia}[Prenexný tvar]
\end{definicia}

\begin{priklad}
    Formula elementárnej aritmetiky:
    $ (\forall x) (\forall y) (\exists z) (x+y=z)$
\end{priklad}

\begin{veta}
 Nech $A$ je ľubovoľná formula predikátovej logiky. Potom existuje
 formula $A'$ v prenexnom tvare taká, že
 $\provable A \lequiv A'$.
 \label{veta:prenex}
\end{veta}

\begin{definicia}[Zoznam prenexných operácií]
\end{definicia}

\begin{lema}
    Prenexnými operáciami dostaneme ekvivalentné formuly
\end{lema}
\begin{dokaz}
\end{dokaz}

\begin{dokaz}[Dôkaz vety \ref{veta:prenex} o prexexných tvaroch]
\end{dokaz}

\begin{priklad}
    Formula $A: B \lequiv (\forall x) C$ kde $x$ nie je voľná
    v $B$ a $y$ sa nevyskytuje v $B,C$.
\end{priklad}

\begin{priklad}
    Formula elementárnej aritmetiky:
    $(\exists x) (x=y) \implies (\exists x)((x=0) \lor
                    \neg (\exists y)(y<0))$
\end{priklad}
%%% }}}

%%%%%%%%%%%
\subsection{Skolemov tvar formuly}
%%% {{{

\begin{definicia}[Skolemov normálny tvar]
\end{definicia}

\begin{veta}
    Nech $A$ je formula predikátovej logiky. Potom k nej môžeme
    zostrojiť formulu $A'$ v Skolemovom normálnom tvare, pričom platí
    $\provable A \iff \provable A'$.
    \label{veta:skolem}
\end{veta}

\begin{definicia}[Hodnosť formuly]
\end{definicia}

\begin{priklad}[Príklad hodnosti formuly]
\end{priklad}

\begin{dokaz}[vety \ref{veta:skolem}]
\end{dokaz}
%%% }}}

%%%%%%%%%%%
\subsection{Predikátová logika s rovnosťou}
%%% {{{

\begin{definicia}[Zoznam axióm rovnosti]
\end{definicia}

\begin{priklad}[Teória neostrého čiastočného usporiadania $\le$]
\end{priklad}

\begin{lema}[Rovnosť je symetrická a tranzitívna]
\end{lema}
\begin{dokaz}
\end{dokaz}

\begin{veta}
    Nech $t_1,\ldots,t_n,s_1,\ldots,s_n$ sú termy, pričom platí
        $\forall i \in \{1,\dots,n\}:\; \provable t_i = s_i$.
    Potom
    \begin{itemize}
    \item[i)] Ak $t$ je term, ktorý vznikne z termu $s$ nahradením
        niektorých výskytov termov $s_i$ za $t_i$, potom 
        $\provable t=s$.
    \item[ii)] Ak $A'$ je formula, ktorá vznikne z formuly $A$
    dosadením $t_i$ za niektoré termy $s_i$, okrem prípadov, keď
    $t_i$ je premenná $x$ v kvantifikácii $(\forall x)$ 
    resp. $(\exists x)$. Potom
    $\provable A \lequiv A'$.
    \end{itemize}
\end{veta}

\begin{dokaz}
\end{dokaz}

\begin{veta}
    Majme term $t$, termy $t_1,\dots,t_n, s_1,\dots, s_n$ a formulu
    $A$.
    Potom platí
    \begin{itemize}
        \item[i)] $\provable t_1=s_2 \implies t_2=s_2 \implies
            t_n=s_n \implies (t[t_1,\dots,t_n] = t[s_1,\dots,s_n])$.
        \item[ii)] $\provable t_1=s_2 \implies t_2=s_2 \implies
            t_n=s_n \implies (A[t_1,\dots,t_n] \lequiv 
                              A[s_1,\dots,s_n])$.
    \end{itemize}
    Ak navyše $x$ je premenná, ktorá nie je obsiahnutá v terme $t$,
    potom platí
    \begin{itemize}
        \item[iii)] $\provable A_x[t] \lequiv 
            (\forall x)((x=t) \implies A)$
        \item[iv)] $\provable A_x[t] \lequiv 
            (\exists x)((x=t) \land A)$
    \end{itemize}
\end{veta}
\begin{dokaz}
\end{dokaz}
%%% }}}

%%%%%%%%%%%
\subsection{Pravdivosť a dokázateľnosť}
%%% {{{

\begin{definicia}[Logická platnosť formuly]
\end{definicia}

\begin{definicia}[Teória]
\end{definicia}

\begin{definicia}[Model teórie]
\end{definicia}

\begin{definicia}[Tautologický dôsledok]
\end{definicia}

\begin{priklad}[Teória ostrého usporiadania]
\end{priklad}

\begin{priklad}[Elementárna aritmetika]
\end{priklad}

\begin{poznamka}[Model verzus realizácia]
\end{poznamka}

\begin{priklad}[Teória grúp]
\end{priklad}

\begin{veta}[O korektnosti]
    Ak $T$ je teória v jazyku $L$ a ak formula $A$ je taká,
    že $T \provable A$, potom $T \models A$.
\end{veta}

\begin{dokaz}
\end{dokaz}


\begin{priklad}[Existencia nedokázateľnej formuly a jej negácie v elementárnej aritmerike]
\end{priklad}


\begin{poznamka}[Neexistencia vety o dedukcii v predikátovej logike]
\end{poznamka}

\begin{dosledok}[Vety o korektnosti]
    Ak teória $T$ v jazyku $L$ má model $\mathcal{M}$, potom $T$ je
    bezosporná.
\end{dosledok}
\begin{dokaz}
\end{dokaz}

\begin{poznamka}[Syntaktické vs. sémantické dôkazy]
\end{poznamka}

\begin{priklad}[Bezospornosť predikátovej logiky]
\end{priklad}
%%% }}}

%%%%%%%%%%%
\subsection{Veta o úplnosti}
%%% {{{


\begin{veta}[G\"odel, 1. variant]
    Nech $T$ je teória v jazyku $L$ a nech $A$ je
    ľubovoľná formula jazyka $L$. Potom $T \provable A \iff T \models A$,
    čiže $A$ je dokázateľná práve vtedy keď
    je splnená v každom modeli teórie $T$.
\end{veta}

\begin{veta}[G\"odel, 2. variant]
    Teória $T$ je bezosporná práve vtedy, ked $T$ má model.
\end{veta}

\begin{poznamka}
    Varianta 1 G\"odelovej vety vyplýva z variantu 2.
\end{poznamka}

\begin{dokaz}[Poznámky]
\end{dokaz}

\begin{dokaz}[2. variantu G\"odelovej vety]
\end{dokaz}

\begin{definicia}[Úplná teória]
\end{definicia}

\begin{poznamka}
    Pojem úplnosti teda zodpovedá nasledujúcej konštrukcii: ...
\end{poznamka}

\begin{poznamka}
    V úplnej teórii (a teda špeciálne v $T_h$) nemôže nastať problém 
    \ref{en:model_problem_neg}, ktorý sme spomínali.
\end{poznamka}

\begin{definicia}[Henkinova teória]
\end{definicia}

\begin{poznamka}
    Ak je teória Henkinova, tak sme vyriešili problémy
    \ref{en:model_problem_konst}, \ref{en:model_problem_korektnost}.
\end{poznamka}

\begin{lema}
    Ak $T$ je úplná a Henkinova teória, tak potom $T$ má model.
    \label{lema:uplna_henkinova}
\end{lema}
\begin{dokaz}
\end{dokaz}

\begin{veta}[O kanonickej štruktúre]
    Nech $T$ je úplná Henkinovská teória a nech $\mathcal{M}$ je
    definované podľa dôkazu lemy \ref{lema:uplna_henkinova}.
    Potom $T \provable A \iff \mathcal{M} \models A$.
\end{veta}

\begin{dokaz}
\end{dokaz}

%%% }}}

%%%%%%%%%%%
\subsection{Rozšírenia teórie}
%%% {{{

\begin{definicia}[Rozšírenie jazyka]
\end{definicia}

\begin{priklad}[Na rozšírenie jazyka]
\end{priklad}    

\begin{definicia}[Rozšírenie teórie]
\end{definicia}

\begin{priklad}[Na rozšírenie teórie -- teória grúp, okruhov, ...]
\end{priklad}

\begin{poznamka}
    Je dôležité si uvedomiť, že nie všetky axiómy pôvodnej teórie $T$
    sa musia nachádzať aj v rozšírenej teórii $T'$ -- 
    stačí, ak sa dajú odvodiť v $T'$.
\end{poznamka}

\begin{poznamka}
 Ak teória $T'$ nad jazykom $L'$ je rozšírením teórie $T$ nad jazykom
 $L$ a vieme že $T'$ je bezosporná, potom aj $T$ je bezosporná.
\end{poznamka}
\begin{dokaz}
\end{dokaz}

\begin{definicia}[Konzervatívne rozšírenie]
\end{definicia}

\begin{poznamka}
    Spojením posledných dvoch definícii dostávame, že pre
    konzervatívne rozšírenie $T',L'$ teórie $T,L$ platí
    $ \forall A \in L:\quad\quad T \provable A \iff T' \provable A$
\end{poznamka}

\begin{poznamka}
    Nech $T',L'$ je konzervatívne rozšírenie $T,L$. Potom platí (z
    predchádzajúcej poznámky)
    $T$ je bezosporná $\iff$ $T'$ je bezosporná.
\end{poznamka}

\begin{veta}[Henkinova] 
    K ľubovoľnej teórii $T$ môžeme zostrojiť Henkinovu
    teóriu $T_H$, ktorá je konzervatívnym rozšírením teórie $T$.
\end{veta}    

\begin{dokaz}
\end{dokaz}

\begin{veta}[Lindenbaum]
    Ak $T$ je bezosporná teória s jazykom $L$, potom existuje úplné
    rozšírenie $T'$ teórie $T$ s rovnakým jazykom $L$.
\end{veta}

\begin{dokaz}
\end{dokaz}

\begin{definicia}[Redukcia]
\end{definicia}

\begin{poznamka}
    Nech $\mathcal{M}$ je redukcia $\mathcal{M}'$ na jazyk $L$
    a nech $A$ je formula jazyka $L$.
    Potom platí $\mathcal{M} \models A \iff \mathcal{M}' \models A$.
\end{poznamka}
\begin{dokaz}
\end{dokaz}

\begin{veta}
    Nech $T'$ je rozšírenie teórie $T$ s jazykom $L$.
    Ak $\mathcal{M}'$ je model $T'$ a 
    $\mathcal{M} =\mathcal{M}'\triangle L$, tak $\mathcal{M}$ je model $T$.
\end{veta}
\begin{dokaz}
\end{dokaz}

\begin{dokaz}[Dokončenie 2. variantu G\"odelovej vety]
\end{dokaz}
%%% }}}

%%%%%%%%%%%
\subsection{Veta o kompaktnosti}
%%% {{{
\begin{veta}[O kompaktnosti]
    Nech $T$ je množina formúl jazyka $L$, $A$ je formula $L$. Potom
    $T \models A \iff \exists 
        \mbox{ konečná podmnožina }T'\subseteq T: T' \models A$.
\end{veta}

\begin{dokaz}
\end{dokaz}

\begin{priklad}
    Na základe vety o kompaktnosti ukážeme, že teóriu telies
    charakteristiky 0 nedokážeme v teórii prvého rádu zapísať
    pomocou konečného počtu axióm.
\end{priklad}

\begin{poznamka}[slabá logika druhého rádu]
\end{poznamka}

\begin{veta}[2. tvar vety o kompaktnosti]
    Nech $T$ je množina formúl jazyka $L$. 
    Model teórie $T$ existuje práve vtedy, keď každá konečná podmnožina
    $T' \subseteq T$ má model.
\end{veta}

\begin{dokaz}
\end{dokaz}

\begin{priklad}
    Pomocou vety o kompaktnosti tohto tvaru dokážeme zaručiť,
    že pre Peanovu aritmetiku existujú aj neštandardné modely.
\end{priklad}
%%% }}}

%%%%%%%%%%%
\subsection{Metóda rezolvent}
%%% {{{
Prenexné operácie:
\begin{align*}
    (Qx)A(x) \lor B &\equiv (Qx)(A(x) \lor B) \tag{1a}\\
    (Qx)A(x) \land B &\equiv (Qx)(A(x) \land B) \tag{1b}\\
    \neg (\forall x) A(x) &\equiv (\exists x) \neg A(x) \tag{2a}\\
    \neg (\exists x) A(x) &\equiv (\forall x) \neg A(x) \tag{2b}\\
    (\forall x) A(x) \land (\forall x) B(x) &\equiv 
        (\forall x) (A(x) \land B(x)) \tag{3a}\\
    (\exists x) A(x) \lor (\exists x) B(x) &\equiv
        (\exists x) (A(x) \lor B(x)) \tag{3b}
\end{align*}
\begin{poznamka}
    S predchádzajúcimi operáciami 3a,3b treba pracovať pozorne. Im veľmi
    podobné úpravy totiž neplatia:
\end{poznamka}

Pretože môžeme substituovať za premenné --
    $(\forall x) B(x) \equiv (\forall z) B(z)$, môžeme predchádzajúce
prenexné operácie zhrnúť do nasledujúcich všeobecných transformácií.
\begin{align*}
    (Q_1 x)A(x) \lor (Q_2 x)B(x) \equiv 
        (Q_1 x)(Q_2 z)(A(x) \lor B(z)) \tag{4a} \\
    (Q_3 x)A(x) \land (Q_4 x)B(x) \equiv
        (Q_3 x)(Q_4 z)(A(x) \land B(z)) \tag{4b}
\end{align*} 
kde v oboch prípadoch $z$ je premenná, ktorá sa nevyskytuje voľne v $A$
(a ani v pôvodnom $B(x)$).
Tento najvšeobecnejší tvar ale zbytočne pridáva premenné a preto je
vhodný iba v prípadoch, keď nefungujú operácie 1 až 3.

%%%%%
\subsection{Algoritmus na zostrojenie prenexného tvaru}

\begin{enumerate}
\item Odstránenie implikácií a ekvivalencií:

\item Odstránenie dvojitej negácie a presun negácie k formule.

\item  Premenovanie viazaných premenných, ak je to potrebné.

\item Použijeme zákony 1,2,3,4:
\end{enumerate}

\begin{priklad}
        $(\forall x)(\forall y) \left[ (\exists z) \left(P(x,z) \land
            P(y,z)\right) \implies (\exists u) Q(x,y,u) \right]$
\end{priklad}
%%% }}}

%%%%%%%%%%%
\subsection{Skolemovské štandardné formuly}
%%% {{{

\begin{postup}[Ako prevádzať do štandardnej skolemovskej formy]
\end{postup}

\begin{priklad}
    Nájdite štandardnú formu pre formuly:
    $(\exists x) (\forall y) (\forall z) (\exists u) (\forall v) (\exists w)
        P(x,y,z,u,v,w) $
\end{priklad}

\begin{priklad}
    Nájdite štandardnú formu pre formulu:
    $ (\forall x)(\exists y) (\exists z) 
        \left[ (\neg P(x,y) \land Q(x,z))\lor R(x,y,z)\right] $
\end{priklad}

\begin{definicia}[Literál]
\end{definicia}

\begin{definicia}[Klauzula]
\end{definicia}

\begin{definicia}[Prázdna klauzula]
\end{definicia}

\begin{priklad}[Množinový zápis]
    Uvažujme klauzulu $P \land Q \land \neg R$.
\end{priklad}

\begin{veta}
    Nech $\mathscr{S}$ je množina klauzúl, ktoré tvoria štandardnú
    formu formuly $A$. Potom $A$ nie je splniteľná $\iff$ $\mathscr{S}$
    nie je splniteľná.
\end{veta}

\begin{dokaz}
\end{dokaz}

\begin{poznamka}
    Ak $A$ \emph{nie je splniteľná}, tak splniteľnosť $A$ je
    ekvivalentná splniteľnosti množinovej verzie štandardnej formy
    $\mathscr{S}$, teda môžeme písať
    $A \equiv \mathscr{S}$.
\end{poznamka}

\begin{poznamka}
    Ak $A$ je splniteľná, tak tvrdenie neplatí.
\end{poznamka}

\begin{poznamka}
    Z predchádzajúcej poznámky sa ukazuje, že metódy dokazovania
    formúl budú založené na vyvrátení negácie.
\end{poznamka}

\begin{poznamka}[Štandardná forma konjunkcie]
\end{poznamka}

\begin{priklad}[Z teórie grúp]
    Pokúsime sa zapísať pomocou množiny
    klauzúl tvrdenie z teórie grúp. 
    Chceme ukázať, že ak pre všetky $x \in G$ platí $x \circ x \in G$,
    potom je grupa komutatívna.
\end{priklad}
%%% }}}

%%%%%%%%%%%
\subsection{Herbrandovské univerzum}
%%% {{{

\begin{definicia}[Herbrandovské univerzum množiny klauzúl]
\end{definicia}

\begin{priklad}
    Majme množinu klauzúl
    $S= \{ P(x),\ \neg P(x) \lor \neg P(f(x))\}$.
\end{priklad}

\begin{priklad}
    Nech $S=\{P(x) \lor R(x),\ R(z),\ T(y) \lor \neg W(y) \}$, teda
    množina $S$ neobsahuje žiadnu konštantu.
\end{priklad}

\begin{priklad}
    Uvažujme množinu klauzúl $S=\{P(f(x),a,g(y), b)\}$.
\end{priklad}

\begin{definicia}[výraz]
\end{definicia}

\begin{definicia}[podvýraz]
\end{definicia}

\begin{definicia}[základný výraz]
\end{definicia}

\begin{definicia}[základná inštancia]
\end{definicia}

\begin{priklad}
    Majme množinu $S = \{P(x),\ Q(f(x)) \lor R(y) \}$ a majme klauzulu
    $C: P(x)$. 
\end{priklad}

\begin{definicia}[Herbrandovská báza]
\end{definicia}

\begin{definicia}[H-interpretácia]
\end{definicia}

\begin{poznamka}
    Ako sme už spomínali, nekladieme žiadne obmedzenia 
    na interpretáciu predikátov.
\end{poznamka}

\begin{priklad}[Na rôzne interpretácie]
    Majme množinu klauzúl $S=\{ P(x) \lor Q(x),\ R(f(y)) \}$.
\end{priklad}


\medskip
Ďalšou úlohou bude k ľubovoľnej interpretácii nad ľubovoľným univerzom
priradiť Herbrandovskú interpretáciu a zaviesť príslušné tvrdenia a definície.

\begin{priklad}
    \label{prikl:interpretacia}
    Majme množinu klauzúl $S = \{ P(x),\ Q(y, f(y,a)) \}$.
    Nech je oblasť interpretácie $D = \{1,2\}$ a interpretujme
    konštanty, funkčné symboly a predikátové symboly podľa tabuľky ...
    \begin{table}[h]
        %%% {{{
        \centering
        \begin{tabular}{|r||c|c|c|c|c|}
            \hline
            symbol & a & f(1,1) & f(1,2) & f(2,1) & f(2,2) \\
            \hline
            hodnota & 2 & 1 & 2 & 2 & 1 \\
            \hline
        \end{tabular}
        
        \medskip
        \begin{tabular}{|r||c|c|c|c|c|c|}
            \hline
            predikát & P(1) & P(2) & Q(1,1) & Q(1,2) & Q(2,1) & Q(2,2) \\
            \hline
            hodnota & true & false & false & true & false & true \\
            \hline
        \end{tabular}
        \caption{Interpretácia $I$ z príkladu
          \ref{prikl:interpretacia}}
        \label{tab:priklad-interpretacia}
        %%% }}}
    \end{table}
\end{priklad}

\begin{priklad}
    \label{prikl:interpretacia2}
    Môže nastať aj iný prípad -- majme množinu klauzúl $S$, ktorá neobsahuje
    konštantu: $S=\{P(x),\ Q(y,f(y,z)) \}$.
\end{priklad}

\begin{definicia}[Zodpovedajúca H-interpretácia]
\end{definicia}

\begin{lema}
    Majme interpretáciu $I$ na oblasti $D$. Nech táto interpretácia
    vyhovuje množine klauzúl $S$. Potom ľubovoľná H-interpretácia $I^*$,
    ktorá je priradená (zodpovedá) $I$, taktiež vyhovuje množine klauzúl $S$.
\end{lema}

\begin{dokaz}
\end{dokaz}

\begin{veta}
    Množina klauzúl $S$ nie je splniteľná práve vtedy,
    keď $S$ je nepravdivá pri všetkých H-interpretáciach $S$.
\end{veta}


\begin{dokaz}
\end{dokaz}

\begin{poznamka}
    Zhrňme si niekoľko základných pozorovaní:
    \begin{enumerate}
	\item Majme klauzulu (disjunkciu literálov) $C$
            a nech $C'$ je jej základná inštancia, t.j. 
            každá premenná bola nahradená prvkom H-univerza.
            $C'$ je splniteľná v (H-)interpretácii $I$
	    práve vtedy, keď existuje základný literál $L' \in I$,
            ktorý je splniteľný.
            Teda $C'$ je splniteľná $\iff$ $C' \intersect I \ne 0$.            

	\item Klauzula $C$ je splnená v interpretácii $I$ $\iff$ každá jej
	    základná inštancia $C'$ je splnená v $I$.

	\item Klauzula $C$ je odmietnutá (vyvrátená) v $I$ $\iff$ existuje
	    aspoň jedna taká základná inštancia $C'$, ktorá 
            nie je splniteľná (teda je vyvrátená) v $I$.

	\item Množina klauzúl $S$ nie je splniteľná $\iff$ 
            pre každú interpretáciu $I$ existuje aspoň jedna klauzula
            $C\in S$ a jej základná inštancia $C'$, 
            ktorá nie je splniteľná v $I$.
    \end{enumerate}
\end{poznamka}

\begin{priklad}
    Uvažujme klauzulu $C=\neg P(x) \lor Q(f(x))$ a
    interpretácie $I_1,I_2,I_3$ definované nasledovne ...
\end{priklad}

\begin{priklad}
    Uvažujme množinu klauzúl $S=\{P(x),\ \neg P(x)\}$ a
    interpretácie $I_1 = \{ P(x) \}, I_2 = \{ \neg P(x) \}$.

    Množina $S$ nie je splnená ani jednou interpretáciou.
\end{priklad}
%%% }}}

%%%%%%%%%%%
\subsection{Sémantické stromy}
%%% {{{

\begin{definicia}[Kontrárna dvojica]
\end{definicia}

\begin{definicia}[Sémantický strom]
\end{definicia}

\begin{definicia}[Úplný sémantický strom]
\end{definicia}

\begin{priklad}
    Uvažujme množinu klauzúl $S = \{P,Q\}$. Potom úplný sémantický strom
    pre túto množinu môže byzerať napríklad ako na obrázku...
\end{priklad}

\begin{priklad}
    K tej istej množine klauzúl môžeme mať viacero sémantických
    stromov.
\end{priklad}


\begin{priklad}
    Veľmi malý strom môžeme dostať v prípade, ak $S=\{P(x),\ P(a)\}$. 
\end{priklad}

\begin{priklad}
    Sémantický strom zďaleka nemusí byť konečný. Uvažujme napríklad množinu
    klauzúl $S=\{P(x), Q(f(x)) \}$.
\end{priklad}

\begin{poznamka}
    Strom = organizované preberanie interpretácii,\\
    $I(v)$ ako čiastočná interpretácia,\\
    nesplniteľnosť a odrezávanie stromu
\end{poznamka}

\begin{definicia}[Odmietajúci vrchol]
\end{definicia}

\begin{definicia}[Uzavretý sémantický strom]
\end{definicia}

\begin{definicia}[Akceptujúci vrchol]
\end{definicia}

\begin{priklad}
    Uvažujme množinu klauzúl
    $S= \{ P,\ Q \lor R,\ \neg P \lor \neg Q,\ \neg R \lor \neg P \}$.
    Obrázok: Pôvodný strom/Strom vzniknutý odrezaním odmietajúcich
    vrcholov.
\end{priklad}



\begin{priklad}
    Uvažujme množinu klauzúl $S=\{ P(x),\ \neg P(x) \lor Q(f(x)),\ \neg Q(f(x)) \}$.
\end{priklad}
%%% }}}

%%%%%%%%%%%
\subsection{Herbrandova veta}
%%% {{{
\begin{definicia}[Usporiadanie stromu]
\end{definicia}

\begin{lema}[K\"onig] 
    Nech každý vrchol stromu s koreňom má konečné vetvenie
    (t.j. konečný stupeň vetvenia) a strom $T$ je nekonečný.
    Potom v ňom existuje nekonečne dlhá vetva.
\end{lema}
\begin{dokaz} 
\end{dokaz}

\begin{veta}[Herbrandova, 1. variant]
    Množina klauzúl $S$ nie je splniteľná práve vtedy,
    keď ľubovoľnému úplnému sémantickému stromu pre množinu klauzúl
    $S$ zodpovedá konečný uzavretý sémantický strom,
    t.j. ľubovoľná vetva úplného stromu vedie do odmietajúceho vrchola.
\end{veta}

\begin{dokaz}
\end{dokaz}

\begin{veta}[Herbrandova, 2. variant]
    Množina klauzúl $S$ nie je splniteľná $\iff$
    existuje konečná podmnožina $S'$ základných inštancií klauzúl z $S$,
    ktorá nie je splniteľná.
\end{veta}

\begin{dokaz}
\end{dokaz}

\begin{priklad}
    Majme množinu klauzúl $S=\{P(x),\ \neg P(f(a))\}$.
    Chceme ukázať, že $S$ nie je splniteľná.
\end{priklad}

\begin{priklad}
    Majme množinu klauzúl $S=\{\neg P(x) \lor Q(f(x),x),\
        P(g(b)),\ \neg Q(y,z) \}$.
    Táto množina nie je splniteľná.
\end{priklad}

\begin{priklad}
    Uvažujme množinu klauzúl $
        S = \{ \neg P(x,y,u) \lor \neg P(y,z,v) \lor 
                    \neg P(x,v,w) \lor P(u,z,w),\ 
	        \neg P(x,y,u) \lor \neg P(z,y,v) \lor
                    \neg P(u,z,w) \lor P(x,v,w),\
	        P(g(x,y),x,y),\
                P(x,h(x,y),y),\
                P(x,y,f(x,y),\
                \neg P(k(v),x,k(x))\}$.
    Chceme ukázať, že $S$ nie je splniteľná.
\end{priklad}

\begin{poznamka}[Gilmore, 1960]
\end{poznamka}

\begin{priklad}
    Majme množinu klauzúl $S=\{P(x),\ \neg P(a)\}$ a
    univerzum nech je $H_0 = \{a\}$.
\end{priklad}


\begin{priklad}
    Majme $S=\{P(a),\ \neg P(x) \lor Q(f(x)),\ \neg Q(f(a)) \}$ a
    univerzum $H_0 = \{a\}$.
\end{priklad}

\subsubsection{Dokazovacie pravidlá}

\begin{definicia}[Pravidlo tautológie]
\end{definicia}
\begin{dokaz}
\end{dokaz}

\begin{definicia}[Pravidlo jednoliterálnych klauzúl]
\end{definicia}
\begin{dokaz}
\end{dokaz}

\begin{definicia}[Čistý literál]
\end{definicia}

\begin{definicia}[Pravidlo čistých literálov]
\end{definicia}
\begin{dokaz}
\end{dokaz}


\begin{definicia}[Pravidlo rezu]
\end{definicia}
\begin{dokaz}
\end{dokaz}

\begin{priklad}
    Majme množinu $S$ vyjadrenú v tvare
    $S=(P \lor Q \lor \neg R) \land (P \lor \neg Q) 
        \land \neg P \land R \land U$.
    Ukážeme, že množina klauzúl nie je splniteľná.
\end{priklad}

\begin{priklad}
    Uvažujme $S=(P \lor Q) \land \neg Q \land ( \neg P \lor Q \lor \neg R)$.
    Ukážte, že množina klauzúl $S$ je splniteľná.
\end{priklad}

\begin{priklad}
    Nech je $S=(P \lor Q) \land (P \lor \neg Q) \land
        (R \lor Q) \land (R \lor \neg Q)$.
    Ukážte, že $S$ je splniteľná.
\end{priklad}

\begin{priklad}
    Uvažujme $S=(P \lor \neg Q) \land (\neg P \lor Q) \land 
                (Q \lor \neg R) \land (\neg Q \lor \neg R)$.
    Zistite, či množina je alebo nie je splniteľná.
\end{priklad}

\begin{priklad}
    Majme $S=(P \lor Q) \land (P \lor \neg Q) \land 
        (R \lor Q) \land (R \lor \neg Q)$. 
\end{priklad}
%%% }}}

%%%%%%%%%%%
\subsection{Metóda rezolvent pre výrokovú logiku}
%%% {{{
\begin{definicia}[Rezolventa]
\end{definicia}

\begin{priklad}
    Majme $C_1 = P \lor R$ a $C_2 = \neg P \lor Q$. Uvažujme kontrárnu dvojicu
    $P,\neg P$. Potom $C_1' = R$, $C_2'=Q$. Rezolventa je $R \lor Q$.
\end{priklad}

\begin{priklad}
    Majme $C_1= \neg P \lor Q\lor R$, $C_2 = \neg Q \lor S$.
    Kontrárna dvojica je $\neg Q$, $Q$ a rezolventa $\neg P \lor R \lor S$.
\end{priklad}

\begin{priklad}
    Majme klauzuly $C_1: \neg P \lor Q$ a  $C_2: \neg P \lor R$.
    Tieto klauzuly nemajú rezolventu.
\end{priklad}

\begin{veta}
    Nech $C_1$ a $C_2$ sú klauzuly a nech $C$ je ich rezolventa.
    Potom $C$ je logickým dôsledkom klauzúl $C_1$ a $C_2$.
\end{veta}
\begin{dokaz}
\end{dokaz}

\begin{poznamka}
    Nech $C_1$, $C_2$ sú jednotkové klauzuly. Ak $C_1,C_2$ majú
    rezolventu, potom musia tvoriť kontrárnu dvojicu $L, \neg L$
    a rezolventou je prázdna (nesplniteľná) klauzula $C \equiv \eps$.
\end{poznamka}

\begin{definicia}[Rezolvenčné odvodenie]
\end{definicia}

\begin{definicia}
    Majme množinu klauzúl $S$ a klauzulu $C$.
    Hovoríme, že $C$ môžeme získať z $S$,
    ak existuje (rezolvenčné) odvodenie $C_1, \dots, C_m$ z množiny $S$ také, že
    $C_m \equiv C$.
\end{definicia}

\begin{priklad}
    Uvažujme množinu klauzúl $S=\{\neg P \lor Q,\ \neg Q,\ P\}$.
    Dokážte, že množina klauzúl $S$ nie je
    splniteľná.
\end{priklad}

\begin{priklad}
    Nech $S=\{P \lor Q,\ \neg P \lor Q,\ 
            P \lor \neg Q,\ \neg P \lor \neg Q\}$.
    Opäť vieme dokázať nesplniteľnosť množiny klauzúl $S$...
\end{priklad}

\begin{lema}
    Pravidlo modus ponens je ekvivalentné s pravidlom rezolventy.
\end{lema}

\begin{dokaz}
\end{dokaz}
%%% }}}

%%%%%%%%%%%
\subsection{Substitúcia a unifikácia}
%%% {{{
\begin{definicia}[Substitúcia]
\end{definicia}

\begin{priklad}
\end{priklad}

\begin{definicia}[Inštancia výrazu]
\end{definicia}

\begin{priklad}
\end{priklad}

\begin{definicia}[Kompozícia substitúcii]
\end{definicia}

\begin{priklad}
    Majme substitúcie
    $ \theta = \{t_1/x_1,\ t_2/x_2\} = \{ f(y)/x,\ z/y\},
        \lambda = \{u_1/y_1,\ u_2/y_2,\ u_3/y_3\} = \{ a/x,\ b/y,\
        y/z\} $
\end{priklad}

\begin{poznamka}
    To znamená, že množina substitúcii s operáciou skladania je
    pologrupa (monoid) s jednotkou.
\end{poznamka}

\begin{definicia}[Unifikátor]
\end{definicia}

\begin{priklad}
    Majme množinu $\{P(a,y),\ P(x,f(b)\}$.
\end{priklad}

\begin{definicia}[Najvšeobecnejší unifikátor]
\end{definicia}

\begin{definicia}[Diferenčná množina]
\end{definicia}

\begin{priklad}
    Majme množinu $W = \{P(x,f(y,z),\ P(x,a),\ P(x,g(h(k(x))))\}$.
    Nájdite jej diferenčnú množinu.
\end{priklad}

\begin{priklad}[Nebol na prednáške]
    Uvažujme $W=\{P(a,f(a),f(g(y))),\ P(a,f(a),f(u))\}$.
    Diferenčná množina je ...
\end{priklad}

\begin{postup}[Unifikačný algoritmus]
\end{postup}

\begin{poznamka}
    Ak je množina unifikovateľná, vždy existuje najvšeobecnejší unifikátor.
\end{poznamka}

\begin{priklad}
    Nájdite najvšeobecnejší unifikátor pre množinu
    $ W=\{ P(a,x,f(g(y))),\ P(z,f(z),f(u)) \}$.
\end{priklad}


\begin{priklad}
    Zistite, či je unifikovateľná množina 
    $ W=\{Q(f(a),g(x)),\ Q(y,y)\}$.
\end{priklad}

\begin{poznamka}[Konečnosť algoritmu]
\end{poznamka}

\begin{veta}[Unifikačná]
    Ak $W$ je konečná neprádzna unifikovateľná množina výrazov,
    tak unifikačný algoritmus vždy zakončuje svoju činnosť na druhom kroku
    a posledné $\sigma_k$ je najvšeobecnejší unifikátor.
\end{veta}

\begin{dokaz}
\end{dokaz}
%%% }}}

%%%%%%%%%%%
\subsection {Metóda rezolvent pre logiku 1. rádu}
%%% {{{
\begin{definicia}[Spojenie]
\end{definicia}

\begin{priklad}
    Uvažujme klauzulu $C$, ktorá vyzerá nasledovne: 
    $C = \{ P(x) \lor P(f(y)) \lor \neg Q(x)\}$.
\end{priklad}

\begin{definicia}[binárna rezolventa]
\end{definicia}

\begin{priklad}
    Majme $C_1 = P(x) \lor Q(x)$ a $C_2 = \neg P(a) \lor R(x)$, 
    čo budú predpoklady.
\end{priklad}

\begin{definicia}[Rezolventa logiky 1. rádu] 
\end{definicia}

\begin{priklad}
    Uvažujme $C_1 = P(x) \lor P(f(y))\lor R(g(y))$,
             $C_2 = \neg P(f(g(a)) \lor Q(b)$.
\end{priklad}

\subsubsection{Úplnosť metódy rezolvent}

\begin{lema}
    Nech $C_1'$ a $C_2'$ sú inštancie klauzúl $C_1$ resp. $C_2$.
    Ak $C'$ je rezolventa $C_1'$ a $C_2'$,
    tak potom existuje rezolventa
    $C$ klauzúl $C_1$ a $C_2$, 
    že $C'$ je inštancia $C$. 
\end{lema}

\begin{dokaz}
\end{dokaz}

\begin{veta}[Úplnosť metódy rezolvent] 
    Množina klauzúl $S$ nie je splniteľná $\iff$ 
    existuje odvodenie prázdnej klauzuly $\eps$ z $S$.
\end{veta}

\begin{dokaz}
\end{dokaz}

\begin{priklad}
    Majme nasledujúce formuly:
    $ F_1: (\forall x) (C(x) \implies (W(x) \land R(x)),
        F_2: (\exists x) (C(x) \land Q(x)),
        G:   (\exists x) (Q(x) \land R(x))$.
    Ukážte, že $G$ je logickým dôsledkom $F_1$ a $F_2$.
\end{priklad}
%%% }}}

%%%%%%%%%%%
\subsection{Stratégia vymazávania}
%%%% {{{
\begin{priklad}
    Majme množinu klauzúl $S=\{P\lor Q,\ \neg P\lor Q,\ 
        P \lor \neg Q,\ \neg P \lor \neg Q\}$. 
    Metódou rezolvent ukážte, že $S$ nie je splniteľná.
\end{priklad}


\begin{definicia}[podklauzula]
\end{definicia}

\begin{priklad}
    Majme klauzuly $C = P(x)$ a $D = P(A) \lor Q(a)$.
\end{priklad}

\begin{poznamka}
    Ak $D$ je identicky rovná $C$ alebo ak klauzula $D$ je inštancia $C$,
    potom $D$ je nadklauzula $C$.
\end{poznamka}

\begin{postup}[Algoritmus pohltenia]
\end{postup}

\begin{poznamka}
    $\mathcal{U}^k, \mathcal{U}^{k+1}$, klauzuly z
    $\mathcal{U}^{k}$ sú konečné. $\mathcal{U}^0, \mathcal{U}^1, \ldots \square$.
\end{poznamka}

\begin{dokaz}
\end{dokaz}

\begin{priklad}
    $C = \neg P(x) \lor Q(f(x), a)$. $D = \neg P(h(y)) \lor
    Q(f(h(y)),a) \lor P(z)$. Zistite, či klauzula $C$ je podklauzulou $D$.

\end{priklad}

\begin{priklad}
    $C=P(x,x)$ a $D=P(f(x),y) \lor P(y,f(x))$. Zistite, či $C$
    je podklauzula $D$.

\end{priklad}

\begin{priklad}
    Majme formuly: $P\implies S, S \implies U,P,U$.
    Dokážte, že formula 4 vyplýva z formúl 1, 2 a 3. 
\end{priklad}

%%% }}}

\end{document}
