\chapter{Dokončenie úvodu do matematickej logiky}

\section{Prenexné tvary formúl}

Ako sme mali vo výrokovej logike isté normálne tvary - konjunktívnu a
disjunktívnu normálnu formu, budeme mať aj v predikátovej logike isté
špeciálne tvary. Zaujímavé sú najmä prexenxná forma a ešte Skolemov
normálny tvar, čo je špeciálny prípad prexenej formy.
V prípade prexenxného tvaru ide o preskupenie kvantifikátorov na
začiatok formuly - tvar vzniká aplikovaním kvantifikátorov na otvorenú
formulu.


\begin{definicia}
    Formula $A$ je v prenexnom tvare, ak $A$ je v nasledujúcom tvare:
     $(Q_1 x_1) (Q_2 x_2) (Q_3 x_3) \dots (Q_n x_n) B$, kde
     $x_1, \dots , x_n$ sú navzájom rôzne premenné, $Q_i$ sú
     kvantifikátory $(\forall, \exists)$ a 
     $B$ je bez kvantifikátorov.
    Formulu $B$ nazveme otvoreným jadrom formuly $A$, 
    formulu $(Q_1 x_1) (Q_2 x_2) (Q_3 x_3) \dots (Q_n x_n)$ nazývame
    prefixom formuly $A$.
\end{definicia}

\begin{priklad}
    Formula elementárnej aritmetiky:
    \begin{equation*}
        (\forall x) (\forall y) (\exists z) (x+y=z)
    \end{equation*}
\end{priklad}

\fixme{Pozn.}
\begin{itemize}
 \item $x_1, \dots, x_n$ sú navzájom rôzne pre vylúčenie viacnásobných
    kvantifikácií.
 \item ak $n=0$, tak $A$ je otvorená a nemá prefix
 \item $B$ je najväčšia otvorená podformula formuly $A$.
\end{itemize}


\begin{veta}
 Nech $A$ je ľubovoľná formula predikátovej logiky. Potom existuje
 formula $A'$ v prenexnom tvare také, že
 $\provable A \leftrightarrow A'$.
\end{veta}

\todo{dopisat}

\section{Skolemov tvar}
\todo{}
